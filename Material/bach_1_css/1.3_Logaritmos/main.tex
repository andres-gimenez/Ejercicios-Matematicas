\documentclass[addpoints,spanish, 12pt,a4paper,cancelspace]{./include/gexercises}

 %%%%%%%%%%%%%%%%%%%%%%%%%%%
\renewcommand{\documentName} { Logaritmos }
\renewcommand{\documentContent} { Propiedades y ejercicios } 
\renewcommand{\waterMark} {  } 
%%%%%%%%%%%%%%%%%%%%%%%%%%%

% Configuración del documento.
\renewcommand{\schoolSubject} { Examen Matemáticas 2º ESO  }
\renewcommand{\school} { IES José de Churriguera  }
\renewcommand{\academicPeriod} { Curso 2022/2023 }

\renewcommand{\autor} { Andrés Giménez Muñoz }
\renewcommand{\emailAuthor} { andresprofemates@outlook.es }
\renewcommand{\autorSing}{ Profesor: Andrés }

% \renewcommand{\thepartno}{\arabic{partno}}

%%%%%%%%%%%%%%%%%%%%%%%%%%%
% Exam configuration
%\pointsdroppedatright   %% No mostrar la puntuación
\pointsinrightmargin{} % Para poner las puntuaciones a la derecha. Se puede cambiar. Si se comenta, sale a la izquierda.
\extrawidth{-1.5cm} %Un poquito más de margen por si ponemos textos largos.
\marginpointname{ \emph{\points}}

%% Si se comenta no aparecerán los espacios de la solución.
%\nocancelspace

%% Esto es de la clase exam. Si dejamos sin comentar \printanswers, se mostraran las soluciones. 
%% Si la comentamos y dejamos sin comentar \noprintanswers, pues no se muestran las soluciones.
% \printanswers
%\noprintanswers

%%%%%%%%%%%%%%%%%%%%%%%%%%%

\begin{document}

% \StudentData{}
% \GradeTableHeader{}

%% Para ampliar la hoja
% https://www.alfonsogonzalez.es/asignaturas/1_bach_ccss/ejercicios_propuestos_1_bach_ccss/3_logaritmos.pdf
% http://www.ieselescorial.org/wp-content/uploads/2016/11/D.MAT-Ejercicios-de-Logar%C3%ADtmos-y-soluciones.pdf

\justifying{}

\begin{center}
    \fbox{\parbox{6.5in}{
        \vspace{3mm}
            {\Large
                \centerline{\textbf{Definión de logaritmo}}
            }
        \vspace{5mm}

    Si \textbf{b} es un número positivo y distinto de 1, el \textbf{logaritmo en base a de un número positivo N}
    es el exponente al que hay que elevar la base a para obtener N. 
    
    \center
    $\log_b N =x \Leftrightarrow b^x=N$

    \vspace{5mm}
    }}
\end{center}

{\Large
    % \centerline{
        \textbf{Propiedades de los logaritmos:}
    % }
}

\vspace{5mm}

\begin{enumerate}

\item El logaritmo de la base es siempre igual a 1:

\begin{equation*}
    \log_b b = 1
\end{equation*}

\item El logaritmo de cualquier base de 1 es igual a 0:

\begin{equation*}
    \log_b 1 = 0
\end{equation*}

\item El logaritmo del producto es igual a la suma de los logaritmos:

\begin{equation*}
    \log_b \left(x \cdot y\right) =\log_b \left(x\right)+\log_b \left(y\right)
\end{equation*}

\item El logaritmo del cociente es igual a la diferencia de los logaritmos:

\begin{equation*}
    \log_b \left(\dfrac{x}{y}\right) = \log_b \left(x\right)-\log_b \left(y\right)
\end{equation*}

\item El logaritmo de una potencia es igual al producto del exponente por el logaritmo de la base:

\begin{equation*}
    \log_b \left(x^p\right)= p\log_b \left(x\right)
\end{equation*}

\item El logaritmo de un radical es igual al logaritmo de la base dividido entre el grado del radical:

\begin{equation*}
    \log_b \left(\sqrt[n]{x}\right) = \frac{\log_b \left(x\right)}{n}
\end{equation*}

\item La relación entre los logaritmos de un mismo número A en distintas bases viene dada por:

\begin{equation*}
    \log_a \left(x\right) =\dfrac{\log_b \left(x\right)}{\log_b \left(a\right)} =\dfrac {\ln \left(x\right)}{\ln \left(a\right)}
\end{equation*}

\item Si no se especifica la base de un logaritmo, se sobreentiende que es base 10.

\begin{equation*}
    \log_{10} \left(x\right) = \log \left(x\right)
\end{equation*}

\item El logaritmo neperiano o naturales, tiene como base el número $e = 2,718281828459045235360...$ y se indica como $ln$.

\begin{equation*}
    \log_e \left(x\right) = \ln \left(x\right)
\end{equation*}

\end{enumerate}

\newpage
\textbf{Ejercicios}

\begin{questions}
    \question
    Calcula los siguientes logaritmos
    \begin{multicols}{3}
        \begin{parts}
            \part $\log_2 8$
            \part $\log_2 32$
            \part $\log_2 \frac{1}{16}$
            \part $\log_3 81$
            \part $\log_3 729$
            \part $\log_3 \frac{1}{81}$
            \part $\log_4 (8) + \log_4(2)$ 
            \part $\log_4 4^4$
            \part $\log_5 (125) + \log_5(5)$    
            \part $\log_5 \left(100\right) + \log_5 \left(\frac{1}{4}\right)$ 
            \part $\log_6 36^2$
            \part $\log_{17} {17} $
            \part $\log_8 1$
            \part $\log_5 10$
            \part $\log 100000000$
            \part $\log \frac{1}{10000}$
            \part $\log 0,001$
        \end{parts}
    \end{multicols}

    \question
    Conociendo el valor de $\log 2 \approx 0,3010$, calcula los siguientes logaritmos:
    \begin{multicols}{3}
        \begin{parts}
            \part $\log 0,0002$
            \part $\log 16$
            \part $\log 32^4$
            \part $\log 4^5$
            \part $\log 0,0125$
            \part $\log \frac{0,64^3 \cdot 0,32}{80 \cdot 6,25}$
        \end{parts}
    \end{multicols}

    \question
    Conociendo el valor de $\log 2 \approx 0,3010$, y el de $\log 3 \approx 0,4771$, calcula:
    \begin{multicols}{3}
        \begin{parts}
            \part $\log 12$
            \part $\log (4,8)^2$
            \part $\log (0,6)^3$
            \part $\log 3,\wideparen{3}$
            \part $\log 40,5$
            \part $\log 5000$
            \part $\log \frac{\left(0,027\right)^3 \cdot 540}{96 \cdot 51,84}$
        \end{parts}
    \end{multicols}

    \question 
    Si $\log 8 \approx 0,9031$, halla:
    \begin{multicols}{3}
        \begin{parts}
            \part $\log 800$
            \part $\log 2$
            \part $\log 0,64$
            \part $\log 40$
            \part $\log 5$
            \part $\log 64^4$
        \end{parts}
    \end{multicols}

    \question 
    Utilizando la calculadora, calcula el valor de x en las siguientes expresiones:
    \begin{multicols}{3}
        \begin{parts}
            \part $3^x = 20$
            \part $5^x = 43$
            \part $5^{x + 2} = 18$
            \part $3^x = 0,2$
            \part $2^x= \frac{8}{27}$
            \part $10^x = 3,4$
        \end{parts}
    \end{multicols}

    \question
    Resuelve las siguientes ecuaciones
    \begin{multicols}{2}
        \begin{parts}
            \part $12^{x-2} = 4^x$
            \part $2^{2x+2} = 9 \cdot 2^x - 2$
            \part $\left(8^{x+2}\right)\left(4^{x-6}\right) = 16$

            \part $\log{\left(2x + 6\right)} = 2$
            \part $\log{\left(x + 3\right)} - \log{\left(2x -2\right)} = 1 - \log{5}$

            \part $\frac{3}{2} + \log_x{2} = 2$
            \part $\frac{3}{2} \cdot 10^{-2} = e^x$

            \part $2^{x+1} + 4 = 80$
            \part $2 \cdot 3^x - 3^{2x} + 3 = 0$
            \part $3^{2x-3} = 8^{x+1}$
            \part $3^{x+2} + 9^{x+1} = 810$
            \part $2^{x-3} = -3$
            \part $5^{x-1} = 2 + \frac{2}{5^{x-2}}$
            \part $2 e^{x-4} = 3$
            \part $100 \cdot 10^x = \sqrt[x]{1000^5}$

            \part $\log{\left(\log{x}\right)} = 1 $
        \end{parts}
    \end{multicols}

    \question
    Halla el valor de x en estas expresiones aplicando las propiedades de los logaritmos:
    \begin{multicols}{3}
        \begin{parts}
            \part $\ln{x} = \ln{8} + \ln{2}$
            \part $\ln{x} = \ln{3} + \ln{2} - \ln{6}$
            \part $\log{x} = \log{36} - \log{6}$
            \part $\log{x} = 4\log{2} - \frac{1}{2} \log{25}$
            \part $\ln{x} = 3 \ln{2}$
            \part $\log{x} = 3 \log{2} - \frac{1}{4} \log{16}$
        \end{parts}
    \end{multicols}

    \question
    Usando las propiedades de los logaritmos, calcular:
    \begin{parts}
            \part $\log_2 {28} - \log_2{\frac{8}{7}} + \log_2 {\frac{1}{49}} - 4 \log_2 {\frac{1}{\sqrt{2}}}$

            \part $\log_3 {\frac{\sqrt{27}}{5}} + \log_3{\frac{3}{125}} - \log_3{\frac{1}{\sqrt{75}}}$

            \part $\log_5{50} - \log_3{\frac{\sqrt{125}}{2}} + \log_5{\frac{25}{8}} - \log_5{\frac{1}{\sqrt{20}}}$

            \part $\log_3{63} - \log_3 {\frac{\sqrt[3]{3}}{7}} + \log_3{\frac{27}{49}} - \log_3 {\left(3 \sqrt[3]{9} \right)}$

            \part $\log_2{40} - \log_2 {\frac{\sqrt{8}}{5}} + \log_2 {\frac{2}{125}} - \log_2 {\frac{1}{\sqrt{5}}}$
    \end{parts}

    \question 
    La función que permite calcular en canto se convierte un capital \textbf{$C_0$} acumulado al cabo de \textbf{t} años es:
    \begin{equation*}
        C(t) = C_0 \cdot \left(1+\frac{i}{100}\right)^t
    \end{equation*}
    donde: \textbf{$C_0$} es el capital inicial e \textbf{i} el interés anual.
    \begin{parts}
        \part ¿Cuánto dinero tendremos al cabo de 6 años si colocamos a plazo fijo 20.000 \euro{} al 2\%?

        \part ¿Cuántos años deberemos mantener 100.000 \euro{} en una entidad bancaria a una tasa del 2,5\% si queremos duplicar el capital?
        ¿Es relevante el dato del capital inicial?

        \part Una persona que tiene depositada en una caja de ahorros 30.000\euro{} a una tasa de 3\% queire llegar a tener 40.000\euro
        ¿Cuánto tiempo deberá mantener intacto el capital?
    \end{parts}

    \question 
    La función que expresa el volumen de madera que tiene un bosque al cabo de \textbf{t} años es:
    \begin{equation*}
        M(t) = M_0 \cdot \left(1+I\right)^t
    \end{equation*}
    donde: \textbf{$M_0$} es el volumen inicial de madera, en $m^3$ e \textbf{I} es el crecimiento anual, en \%.
    \begin{parts}
        \part Se calcula que un bosque tiene $12.000 m^3$ de madera y que aumenta en $5\%$ cada año. ¿Cuánta madera tendrá al cabo de 10 años si sigue creciendo en estas condiciones?
        \part Cuánto tiempo tardará en duplicarse el bosque?
    \end{parts}

    \question 
    Algunos tipos de bacterias tienen un crecimiento de sus poblaciones muy rápido. La \textit{escherichia coli} puede duplicar su población cada hora.
    \begin{parts}
        \part Supongamos que hacemos un cultivo en el que inicialmente 
        hay 5000 bacterias de este tipo. Construir una tabla para razonar que la función que nos da el número de 
        bacterias al cabo de \textbf{t} horas es: 
        \begin{equation*}
            f(t) = 5000 \cdot 2^t
        \end{equation*}

        \part ¿Cuántas habrá al cabo de 16 horas?

        \part Dibujar una gráfica que represente el crecimiento en las 8 primeras horas.

        \part Si tenemos un cultivo de 100 bacterias y queremos conseguir un millón, ¿cuánto 
        tiempo ha de transcurrir?
    \end{parts}

    \question 
    Un coche que nos costo 12.000\euro{} pierde un 12\% de su valor cada año.
    \begin{parts}
        \part ¿Cuánto valdrá dentro de un año? ¿Y dentro de 3 años?
        \part Obtén la función que nos da el precio del coche según los años transcurridos.
        \part ¿Cuánto tiempo ha de transcurrir para que el valor del coche sea la mitad?
        \part ¿Cuánto tiempo ha de transcurrir para que el valor del coche sea un cuarto del valor inicial?
    \end{parts}

\end{questions}

\end{document}