 %%%%%%%%%%%%%%%%%%%%%%%%%%%
\newcommand{\documentName} { Ecuaciones }
\newcommand{\documentContent} { Ecuaciones de tercer y más grados } 
\newcommand{\waterMark} {  } 
%%%%%%%%%%%%%%%%%%%%%%%%%%%

% Configuración del documento.
\newcommand{\schoolSubject} { Matemáticas 3º ESO - Recuperación}
\newcommand{\school} { IES La Serna }
\newcommand{\academicPeriod} { Curso 2020/2021 }


\newcommand{\autor} { Andrés Giménez Muñoz }
\newcommand{\emailAuthor} { agimenezmunoz@ieslaserna.com }
\newcommand{\autorSing}{ Profesores: Andrés } 
\renewcommand{\schoolSubject} { Examen Matemáticas 2º ESO  }
\renewcommand{\school} { IES José de Churriguera  }
\renewcommand{\academicPeriod} { Curso 2022/2023 }

\renewcommand{\autor} { Andrés Giménez Muñoz }
\renewcommand{\emailAuthor} { andresprofemates@outlook.es }
\renewcommand{\autorSing}{ Profesor: Andrés } 

% \renewcommand{\thepartno}{\arabic{partno}}

\usepackage{yhmath}

\usepackage{amsthm}
\theoremstyle{definition}
\newtheorem*{theorem}{Theorem}
\newtheorem{definition}{Definición}
\newtheorem{property}{Propiedad}

%%%%%%%%%%%%%%%%%%%%%%%%%%%
% Exam configuration
%\pointsdroppedatright   %% No mostrar la puntuación
\pointsinrightmargin{} % Para poner las puntuaciones a la derecha. Se puede cambiar. Si se comenta, sale a la izquierda.
\extrawidth{-1.5cm} %Un poquito más de margen por si ponemos textos largos.
\marginpointname{ \emph{\points}}

%% Si se comenta no aparecerán los espacios de la solución.
%\nocancelspace

%% Esto es de la clase exam. Si dejamos sin comentar \printanswers, se mostraran las soluciones. 
%% Si la comentamos y dejamos sin comentar \noprintanswers, pues no se muestran las soluciones.
% \printanswers
%\noprintanswers

%%%%%%%%%%%%%%%%%%%%%%%%%%%

\begin{document}

% \StudentData{}
% \GradeTableHeader{}

\justifying{}

\begin{questions}
    \question
        Resuelve las siguientes ecuaciones:
        \begin{parts}
            \part
                $x^2-x-12 = 0$
            \part
                $x^3+x^2-17x+15 = 0$
            \part
                $x^3+7x^2+2x-40 = 0$
            \part
                $x^3+6x^2+x+6 = 0$
        \end{parts}

        \question
        Calcula todas las soluciones de las ecuaciones:
        \begin{parts}
            \part
                $4x^3-8x^2-x+2 = 0$
            \part
                $6x^4-17x^3+17x^2-7x+1 = 0$
        \end{parts}

        \question
        Resuelve las siguientes ecuaciones bicuadradas \\
        \scriptsize{*{Las ecuaciones de 4º grados, de la forma $ax^4+bx^2+c = 0$ se les llama ecuaciones bicuadradas y se pueden resolver haciendo el cambio de variable $t=x^2$.}}
        \normalsize
        \begin{parts}
            \part
                $-2x^4+58x^2-200=0$
            %    \begin{sample} Realizamos el cambio de variable $t=x^2$. \\ \\
            %         Resolvemos la ecuación $-2t^2+58t-200=0$.

            %         \begin{multline*}
            %             t = \frac{{ - 58 \pm \sqrt {58^2 - 4\cdot (-2) \cdot (-200)} }}{2\cdot (-2)} = 
            %             \frac{ - 58 \pm \sqrt {3364 - 1600}}{-4} = \dots \\
            %             \dots = \frac{ - 58 \pm \sqrt {1764}}{-4} = \frac{ - 58 \pm 42}{-4}
            %             % 2 P(x) \cdot 4 Q(x) = 2\left(2x-3x^2\right) \cdot 4 \left(2x^2 - 3\right) = \left(4x-6x^2\right) \left(8x^2-12\right) = \dots \\
            %             % 32x^3 -48x - 48x^4 + 72x^2 = \boxed{-48x^4+32x^3+72x^2-48x}
            %             %\boxed{3} \\
            %         \end{multline*}
            %         \begin{align*}
            %             t & = \frac{-58 + 42}{-4} = \frac{-16}{-4} = 4 \\
            %             t & = \frac{-58 - 42}{-4} = \frac{-100}{-4} = 25
            %         \end{align*}

            %         Deshacemos el cambio de variable:
            %         \begin{equation*}
            %         t=x^2 \Rightarrow x = \pm \sqrt{t}
            %         \end{equation*}
            %         \begin{align*}
            %             x = \pm \sqrt{4} = \pm 2 \\
            %             x = \pm \sqrt{5} = \pm 5 \\
            %         \end{align*}
            %         Tenemos cuatro soluciones de la ecuación.
            %         \begin{align*}
            %             \boxed{x = \phantom{-} 2} \\
            %             \boxed{x = -2} \\
            %             \boxed{x = \phantom{-} 5} \\
            %             \boxed{x = -5} \\
            %         \end{align*}
            %     \end{sample}

            \part
                $x^4-13x^2+36=0$
            \part
                $x^4-6x^2+8=0$
        \end{parts}

        \question
        Resuelve las siguientes ecuaciones.
        \begin{parts}
            \part
            %  Copiada del libro.
                $x^3-2x^2-x+2=0$
            \part
            %  Copiada del libro.
                $x^4+7x^3-x^2-7x=0$
        \end{parts}

        \question
        Resuelve las siguientes ecuaciones bicuadradas.
        \begin{parts}
            \part
            %  Copiada del libro.
                $x^4-5x^2+4=0$
                % \begin{sample} Calculamos el valor de $x^2$. \\
                %     \begin{equation*}
                %         x^2=\frac{5 \pm \sqrt{(-5)^2-4 \cdot4 }}{2} = \frac{5 \pm \sqrt{25-16}}{2} = \frac{5 \pm \sqrt{9}}{2} = \\
                %             = \frac{5 \pm 3}{2} 
                %     \end{equation*}

                %     $\begin{array}{ccc} 
                %         & & x_1 = \frac{7+5}{2}=1\\ 
                %         & \nearrow &\\ 
                %         x=\frac{7\pm \sqrt{7^2 - 4 \cdot 1 \cdot 6}}{2 \cdot1}= \frac{7\pm \sqrt{25}}{2}
                %         & &\\ 
                %         & \searrow &\\
                %         & &x_2 = \frac{7-5}{2}=6
                %     \end{array}$
                %  \end{sample}
            \part
            %  Copiada del libro.
                $x^4-5x^2+6=0$
        \end{parts}

\end{questions}

\end{document}