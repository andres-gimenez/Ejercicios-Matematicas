%%%%%%%%%%%%%%%%%%%%%%%%%%%
\newcommand{\documentName} { Multiplicaciones }
\newcommand{\documentContent} { \phantom{-}} 
\newcommand{\waterMark} {  } 
%%%%%%%%%%%%%%%%%%%%%%%%%%%

% Configuración del documento.
\newcommand{\schoolSubject} { Matemáticas 3º ESO - Recuperación}
\newcommand{\school} { IES La Serna }
\newcommand{\academicPeriod} { Curso 2020/2021 }


\newcommand{\autor} { Andrés Giménez Muñoz }
\newcommand{\emailAuthor} { agimenezmunoz@ieslaserna.com }
\newcommand{\autorSing}{ Profesores: Andrés } 
\renewcommand{\schoolSubject} { Examen Matemáticas 2º ESO  }
\renewcommand{\school} { IES José de Churriguera  }
\renewcommand{\academicPeriod} { Curso 2022/2023 }

\renewcommand{\autor} { Andrés Giménez Muñoz }
\renewcommand{\emailAuthor} { andresprofemates@outlook.es }
\renewcommand{\autorSing}{ Profesor: Andrés } 

% Imagen en background.
% \usepackage[pages=all]{background}

    % \backgroundsetup{
    % scale=1, %escala de la imagen, es recomendable que sea del mismo tamaño que el pdf
    % color=black, %fondo a usar para transparencia
    % opacity=0.2, %nivel de transparencia
    % angle=0, %en caso de querer una rotación
    % contents={%
    % includegraphics[width=paperwidth,height=paperheight]{SinCalculadora} %nombre de la imagen a utilizar como fondo
    % }

%%%%%%%%%%%%%%%%%%%%%%%%%%%
% Exam configuration
%\pointsdroppedatright   %% No mostrar la puntuación
\pointsinrightmargin{} % Para poner las puntuaciones a la derecha. Se puede cambiar. Si se comenta, sale a la izquierda.
\extrawidth{-1.5cm} %Un poquito más de margen por si ponemos textos largos.
\marginpointname{ \emph{\points}}

%% Si se comenta no aparecerán los espacios de la solución.
%\nocancelspace

%% Esto es de la clase exam. Si dejamos sin comentar \printanswers, se mostraran las soluciones. 
%% Si la comentamos y dejamos sin comentar \noprintanswers, pues no se muestran las soluciones.
% \printanswers
%\noprintanswers

%%%%%%%%%%%%%%%%%%%%%%%%%%%

\begin{document}

% \StudentData{}
% \GradeTableHeader{}

\justifying{}

% \makebox[0pt][l]{%
%   \raisebox{-\totalheight}[0pt][0pt]{%
%     \includegraphics[width=4in]{SinCalculadora.jpg}}}%

% \begin{tikzpicture}[remember picture, overlay]
%     \node[anchor=center] at (current page.center){%
%         \includegraphics[width=4in]{SinCalculadora.jpg}}%
% \end{tikzpicture}


\begin{questions}
    \question Realiza una operación diaria.
    \begin{multicols}{2}
        \begin{parts}
            \part 
            \begin{tabular}{lllllllll}
            &   &   & 4 & 3 & 2 & 1 & 2 & 4 \\
            &   & 2 & 1 & 1 & 2 & 3 & 2 &   \\
            $\times{}$ & 4 & 3 & 2 & 4 & 3 & 2 &   &  \\ \hline
            \end{tabular}
            \vspace{\stretch{1}}

            \part 
            \begin{tabular}{lllllllll}
            &   &   & 4 & 3 & 2 & 1 & 2 & 4 \\
            &   & 2 & 1 & 1 & 2 & 3 & 2 &   \\
            $\times{}$ & 4 & 3 & 2 & 4 & 3 & 2 &   &  \\ \hline
            \end{tabular}
            \vspace{\stretch{1}}

            \part 
            \begin{tabular}{lllllllll}
            &   &   & 4 & 3 & 2 & 1 & 2 & 4 \\
            &   & 2 & 1 & 1 & 2 & 3 & 2 &   \\
            $\times{}$ & 4 & 3 & 2 & 4 & 3 & 2 &   &  \\ \hline
            \end{tabular}
            \vspace{\stretch{1}}

            \part 
            \begin{tabular}{lllllllll}
            &   &   & 4 & 3 & 2 & 1 & 2 & 4 \\
            &   & 2 & 1 & 1 & 2 & 3 & 2 &   \\
            $\times{}$ & 4 & 3 & 2 & 4 & 3 & 2 &   &  \\ \hline
            \end{tabular}
            \vspace{\stretch{1}}

            \part 
            \begin{tabular}{lllllllll}
            &   &   & 4 & 3 & 2 & 1 & 2 & 4 \\
            &   & 2 & 1 & 1 & 2 & 3 & 2 &   \\
            $\times{}$ & 4 & 3 & 2 & 4 & 3 & 2 &   &  \\ \hline
            \end{tabular}
            \vspace{\stretch{1}}

            \part 
            \begin{tabular}{lllllllll}
            &   &   & 4 & 3 & 2 & 1 & 2 & 4 \\
            &   & 2 & 1 & 1 & 2 & 3 & 2 &   \\
            $\times{}$ & 4 & 3 & 2 & 4 & 3 & 2 &   &  \\ \hline
            \end{tabular}
            \vspace{\stretch{1}}
        \end{parts}
    \end{multicols}
    % \begin{parts}
    %     \part $18:\left(3 - 8\right)$
    %     \part $14-24:3+6:2$
    %     \part $9 : 3 + 5 \cdot 7$
    %     \part $17+ \left(4 \cdot 2 - 7\right) \cdot 3$
    %     \part $\left(22 - 5 \cdot 3 \right) \cdot 7$
    %     \part $7 \cdot 4 - 12 + 3 \cdot 6 - 2$
    %     \part $3 \cdot \left(14 + 12 - 20 \right) : 6 + 2$
    % \end{parts}
    
    % \question Realiza las siguientes operaciones combinadas con potencias y raíces:
    % \begin{parts}
    %     \part $4 \cdot 9 - 2^3 \cdot 3$
    %     \part $7 \cdot \left(6 + 3^2\right) - 4^2$
    %     \part $25 : \left(5^2 - 20\right) + 18$
    %     \part $5 - 3 \cdot \left(2^2 + 3\right) - 5$
    %     \part $7 + 4 \cdot \sqrt{25}$
    %     \part $9 - 2 \cdot \sqrt{4} + \sqrt{48}$
    % \end{parts}
\end{questions}

\end{document}