\documentclass[addpoints,spanish, 12pt,a4paper,cancelspace]{./include/gexercises}

 %%%%%%%%%%%%%%%%%%%%%%%%%%%
\renewcommand{\documentName} { Divisibilidad }
\renewcommand{\documentContent} { Máximo común divisor y mínimo común múltiplo } 
\renewcommand{\waterMark} {  } 

% Configuración del documento.
\renewcommand{\schoolSubject} { Examen Matemáticas 2º ESO  }
\renewcommand{\school} { IES José de Churriguera  }
\renewcommand{\academicPeriod} { Curso 2022/2023 }

\renewcommand{\autor} { Andrés Giménez Muñoz }
\renewcommand{\emailAuthor} { andresprofemates@outlook.es }
\renewcommand{\autorSing}{ Profesor: Andrés } 
%%%%%%%%%%%%%%%%%%%%%%%%%%%

\renewcommand{\thepartno}{\arabic{partno}}

%%%%%%%%%%%%%%%%%%%%%%%%%%%
% Exam configuration
%\pointsdroppedatright   %% No mostrar la puntuación
\pointsinrightmargin{} % Para poner las puntuaciones a la derecha. Se puede cambiar. Si se comenta, sale a la izquierda.
\extrawidth{-1.5cm} %Un poquito más de margen por si ponemos textos largos.
\marginpointname{ \emph{\points}}

%% Si se comenta no aparecerán los espacios de la solución.
%\nocancelspace

%% Esto es de la clase exam. Si dejamos sin comentar \printanswers, se mostraran las soluciones. 
%% Si la comentamos y dejamos sin comentar \noprintanswers, pues no se muestran las soluciones.
% \printanswers
%\noprintanswers

%%%%%%%%%%%%%%%%%%%%%%%%%%%

\begin{document}

% \StudentData{}
% \GradeTableHeader{}

\justifying{}

\begin{questions}
    \question
    En el ciclismo de persecución en pista, uno de los corredores da 
    una vuelta al circuito cada 54 segundos, y el otro cada 72 segundos. Parten 
    juntos de la línea de salida. ¿Cuánto tiempo tardarán en volverse a encontrar 
    por primera vez en la línea de salida? ¿Cuántas vueltas habrá dado cada 
    ciclista en ese tiempo?

    \begin{solution}
        Para calcular el tiempo que tardan en coincidir habrá que calcular el mínimo común múltiplo: 

        \begin{align*}
            54 &= 2 \cdot 3^3 \\
            72 &= 2^3 \cdot 3^2
        \end{align*}

        \begin{equation*}
            mcm(54,72) = 2^3 \cdot 3^3 = 8 \cdot 27 = 216
        \end{equation*}

        Volverán a coincidir en \textbf{216 segundos}, que son 3 minutos y 36 segundos. \\

        \begin{itemize}
            \item Primer ciclista: $ 216 : 54 = \textbf{4 vueltas} $
            \item Segundo ciclista: $ 216 : 72 = \textbf{3 vuletas} $
        \end{itemize}

    \end{solution}
    
    \question Un grupo de 60 niños, acompañados de 36 padres, acude a un 
    campamento en la montaña. Para dormir ocupan cabañas todas iguales, 
    pero sin mezclarse los padres con los niños. Cuantas menos cabañas ocupen, 
    menos pagan. ¿Cuántas personas dormirán en cada cabaña? 

    \begin{solution}
        Para calcular el mínimo número de personas que pueden entrar en cada cabaña, sin mezclarse, se busca el máximo común divisor: 

        \begin{align*}
            60 &= 2^2 \cdot 3 \cdot 5 \\
            36 &= 2 \cdot 3^2
        \end{align*}

        \begin{equation*}
            MCD(36,60) = 2^2 \cdot 3 = 4 \cdot 3 = 12
        \end{equation*}

        Dormirán \textbf{12 personas} en cada cabaña.\\

    \end{solution}

    \question Deseamos partir 2 cuerdas de 20 y 30 metros en trozos iguales lo 
    más grandes posible y sin desperdiciar ningún cabo. ¿Cuánto medirá cada 
    trozo? ¿Cuántos trozos habrá? 

    \begin{solution}
        Para calcular la longitud de cada trozo habrá que calcular el máximo común divisor: 

        \begin{align*}
            20 &= 2^2 \cdot 5 \\
            30 &= 2 \cdot 3 \cdot 5
        \end{align*}

        \begin{equation*}
            MCD(20,30) = 2 \cdot 5 = 10
        \end{equation*}

        Cada toroz medrirá \textbf{10 metros}. \\

        \begin{itemize}
            \item De la primera cuerda se obtendrán: $ 20 : 10 = 2 trozos $
            \item De la segunda cuerda se obtendrán: $ 30 : 10 = 3 trozos $
        \end{itemize}

        Por tanto se obtendrán en total \textbf{5 trozos}.

    \end{solution}

    \question Un panadero necesita envases para colocar 250 magdalenas y 75 
    mantecados en cajas, lo más grandes que sean posible, pero sin mezclar 
    ambos productos en la misma caja. ¿Cuántas cajas harán falta, y cuántos 
    bollos irán en cada caja? 

    \begin{solution}
        Para calcular los bollos que irán en cada caja hay que calcular el máximo común divisor: 

        \begin{align*}
            250 &= 2 \cdot 5^3 \\
            75 &= 3 \cdot 5^2
        \end{align*}

        \begin{equation*}
            MCD(250,75) = 5^2 = 25
        \end{equation*}

        Irán \textbf{25 bollos} en cada caja. \\

        \begin{itemize}
            \item Cajas de magdalenas: $ 250 : 25 = 10 cajas $
            \item Cajas de mantecados: $ 75 : 25 = 3 cajas $
        \end{itemize}

        Por tanto habrá un total de \textbf{13 cajas}.

    \end{solution}

    \question Por una vía ferroviaria pasa un tren con dirección a Zaragoza cada 30 minutos y otro con dirección a Gijón cada 18 minutos.
    Si se han cruzado los dos trenes a las 10 de la mañana, halla a qué hora volverán a cruzarse a lo largo de la mañana.

    \question El mayor de los 3 hijos de una familia visita a sus padres cada 15 
    días, el mediano cada 10, y la menor cada 12. La cena de Nochebuena se 
    reúne toda la familia. ¿Cuándo volverán a encontrarse los tres juntos?

    \begin{solution}
        Para calcular cuándo coinciden los tres juntos de nuevo habrá que calcular el mínimo común múltiplo:

        \begin{align*}
            15 &= 3 \cdot 5 \\
            10 &= 2 \cdot 5 \\
            12 &= 2^2 \cdot 3  
        \end{align*}

        \begin{equation*}
            mcm(10,12,15) = 2^2 \cdot 3 \cdot 5 = 4 \cdot 3 \cdot 5 = 60
        \end{equation*}

        Volverán a encontrarse en \textbf{60 días}. Es decir, Nochebuena, como es el 24 de diciembre, más 60 días, el 22 de febrero.

    \end{solution}

    \question Alan y Pedro comen en la misma taquería, pero Alan asiste cada 20 días y Pedro cada 38. ¿Cuánto volverán a encontrarse?

    \question David tiene 24 dulces para repartir y Fernando tiene 18. 
    Si desean regalar los dulces a sus respectivos familiares de modo que todos tengan la misma cantidad y que sea la mayor posible, 
    ¿cuántos dulces repartirán a cada persona? ¿A cuántos familiares regalará dulces cada uno de ellos?

    \question Andrés tiene una cuerda de 120 metros y otra de 96 metros.
    Desea cortarlas de modo que todos los trozos sean iguales pero lo más largos posibles.
    ¿Cuántos trozos de cuerda obtendrá?

    \question En un vecindario, un camión de helados pasa cada 8 días y un food truck pasa cada dos semanas.
    Se sabe que 15 días atrás ambos vehículos pasaron en el mismo días. \\

    Raúl cree que dentro de un mes los vehículos volverán a encontrarse y Oscar cree esto ocurrirá dentro de dos semanas.
    ¿Quién está en lo cierto?

    \question En una banda compuesta por un baterista, un guitarrista, un bajista y un saxofonista, el baterista toca en lapsos de 8 tiempos, 
    el guitarrista en 12 tiempos, el bajista en 6 tiempos y el saxofonista en 16 tiempos.
    Si todos empiezas al mismo tiempo, ¿en cuántos tiempos sus periodos volverán a iniciar al mismo tiempo?

    \question Un granjero ha recogido de sus gallinas 24 huevos morenos y 36 huevos blancos. 
    Quiere envasarlos en cajas con la mayor capacidad posible y con el mismo número de huevos (sin mezclar los blancos con los morenos).

    \question Realiza las siguientes operaciones combinadas:
    \begin{parts}
        \part ¿Cuántos huevos debe poner en cada caja?
        \part ¿Cuántas cajas se llenarán con huevos blancos y cuántas con huevos morenos?
    \end{parts}

    \question En un albergue coinciden tres grupos de excursión de 40, 56 y 72 personas cada grupo.
    El camarero quiere organizar el comedor de forma que en cada mesa haya igual número de comensales y se reúna el mayor 
    número de personas posibles sin mezclar los grupos.
    \begin{parts}
        \part ¿Cuántos comensales sentará en cada mesa?
        \part ¿Cuántas mesas habrá de cada grupo?
    \end{parts}

    \question El dependiente de una papelería tiene que organizar, en botes, 
    36 bolígrafos rojos, 60 bolígrafos azules y 48 bolígrafos negros. 
    Todos los botes han de ser iguales y con el mayor número de bolígrafos, sin mezclar los colores.
    \begin{parts}
        \part ¿Cuántos pondrá en cada bote?
        \part ¿Cuántos botes habrá de bolígrafos rojos? ¿Y de bolígrafos azules? ¿Y de negros?
    \end{parts}

    \question Un electricista tiene tres rollos de cable de 96, 120 y 144 metros de longitud.
    Desea cortarlos en trozos iguales de la mayor longitud posible, sin que quede ningún trozo sobrante.
    \begin{parts}
        \part ¿Qué longitud deberá tener cada trozo?
        \part ¿Cuántos trozos habrá en total?
    \end{parts}

    \question Una rana corre dando saltos de 30cm, perseguida por un gato que da saltos de 45 cm. 
    ¿Cada cuántos centímetros coinciden las huellas del gato y las de la rana?

    \question Un cometa es visible desde la Tierra cada 24 años y otro, cada 36 años.
    El último año que fueron visibles conjuntamente fue en 1944. 
    ¿En qué año volverán a coincidir?

    \question El autobús de la linea A pasa por cierta parada cada 12 minutos, 
    el de la linea B pasa cada 18 minutos y el de la linea C, cada 24 minutos.
    Si todos coinciden a las 10 de la mañana,
    ¿a qué hora vuelven a coincidir?

    \question Un carpintero dispone de tres listones de madera de 30, 45 y 60 cm de longitud, respectivamente.
    Desea dividirlos en trozos y de la mayor longitud posible sin desperdiciar nada.
    \begin{parts}
        \part ¿Qué longitud deberá tener cada trozo?
        \part ¿Cuántos trozos habrá en total?
    \end{parts}

\end{questions}

\end{document}