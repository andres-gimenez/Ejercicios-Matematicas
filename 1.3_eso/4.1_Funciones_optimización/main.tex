\documentclass[addpoints,spanish, 12pt,a4paper,cancelspace]{./include/gexercises} 
 
%%%%%%%%%%%%%%%%%%%%%%%%%%%
\renewcommand{\documentName} { Funciones }
\renewcommand{\documentContent} { Problemas con funciones lineales, cuadráticas y optimización} 
\renewcommand{\waterMark} {  } 
% Configuración del documento.
\renewcommand{\schoolSubject} { Examen Matemáticas 2º ESO  }
\renewcommand{\school} { IES José de Churriguera  }
\renewcommand{\academicPeriod} { Curso 2022/2023 }

\renewcommand{\autor} { Andrés Giménez Muñoz }
\renewcommand{\emailAuthor} { andresprofemates@outlook.es }
\renewcommand{\autorSing}{ Profesor: Andrés } 
%%%%%%%%%%%%%%%%%%%%%%%%%%%

% \renewcommand{\thepartno}{\arabic{partno}}

\theoremstyle{definition}
\newtheorem*{theorem}{Theorem}
\newtheorem{definition}{Definición}
\newtheorem{property}{Propiedad}

%%%%%%%%%%%%%%%%%%%%%%%%%%%
% Exam configuration
%\pointsdroppedatright   %% No mostrar la puntuación
\pointsinrightmargin{} % Para poner las puntuaciones a la derecha. Se puede cambiar. Si se comenta, sale a la izquierda.
\extrawidth{-1.5cm} %Un poquito más de margen por si ponemos textos largos.
\marginpointname{ \emph{\points}}

%% Si se comenta no aparecerán los espacios de la solución.
%\nocancelspace

%% Esto es de la clase exam. Si dejamos sin comentar \printanswers, se mostraran las soluciones. 
%% Si la comentamos y dejamos sin comentar \noprintanswers, pues no se muestran las soluciones.
% \printanswers
%\noprintanswers

%%%%%%%%%%%%%%%%%%%%%%%%%%%

% https://lasmatematicas.eu/docs/matematicas2bach/ejercicios/problemas_optimizacion_2.pdf

\begin{document}

% \StudentData{}
% \GradeTableHeader{}

\justifying{}

\begin{questions}
    \question Manuel trabaja como repartidor los fines de semana y recibe una cantidad fija de 40 \euro{} mensuales, 
    más 5 \euro{} por cada paquete que reparte.
    \begin{parts}
        \part Escribe la fórmula y representa la gráfica de la función que relaciona el número de paquetes repartidos con el dinero recibido al mes.
        \part ¿Cuántos paquetes tiene que repartir Manuel para cobrar en un mes 205 \euro{}.
    \end{parts}

    \question Jorge sale a correr con una velocidad constante de $10km/h$. 
    Una hora después, Marta también empieza a correr con una velocidad constante de $15km/h$.
    \begin{parts}
        \part Escribe las expresiones algebraicas de las funciones que describen la distancia recorrida por cada uno de los corredores.
        \part Determina el tiempo que transcurre hasta que Jorge y Marta se encuentran, y la distancia recorrida por cada uno hasta ese momento.
        \part Representa gráficamente estas funciones en un mismo sistema de coordenadas cartesianas
    \end{parts}

    \question 
        Nos queremos ir de vacaciones y hemos encontrado las siguientes ofertas:
        \begin{enumerate}
            \item Hotel La Laguna:  
            \begin{itemize}
                \item Precio por persona y día 70 \euro{}.
                \item Primer día gratis.
                \item Estancia mínima 5 días.
            \end{itemize}
            \item Hotel El Mar: 
            \begin{itemize}
                \item Precio por persona y día 60 \euro{}.
                \item Estancia mínima 2 días.
            \end{itemize}
        \end{enumerate}
        \begin{parts}
            \part Confecciona, para cada uno de los hoteles, la tabla de valores relativa los diez primeros días de estancia en el hotel.
            \part Expresa algebraicamente cómo varía el coste en cada uno de los hoteles al aumentar la estancia.
            \part Representa gráficamente las funciones obtenidas en el apartado anterior.
            \part ¿Cuánto pagará una persona después de 5 días en cada uno de los hoteles?
            \part ¿Al cabo de cuántos días resulta más económico el hotel El Mar?
            \part El coste de la estancia de una persona en el hotel es de 480 \euro{}.
            ¿En qué hotel se ha alojado? ¿Cuántos días?
        \end{parts}

    \question
       Queremos crear un recinto vallado, para lo cual disponemos de un muro ya construido y de 100 metros de malla metálica.
       ¿Qué dimensiones debe tener la valla para que el área del recinto sea máxima?

    \question 
        En un concurso se da a cada participante un alambre de dos metros de longitud para 
        que doblándolo convenientemente hagan con el mismo un cuadrilátero con los cuatro 
        ángulos rectos. Aquellos que lo logren reciben como premio tantos euros como 
        decímetros cuadrados tenga de superficie el cuadrilátero construido. Calcula 
        razonadamente la cuantía del máximo premio que se pueda obtener en este concurso. 

    \question     
        Un jardinero dispone de 160 metros de alambre que va a utilizar para cercar una 
        zona rectangular y dividirla en tres partes iguales. Las alambradas de las divisiones deben 
        quedar paralelas a uno de los lados del rectángulo. ¿Qué dimensiones debe tener la 
        zona cercada para que su área sea la mayor posible?

    \question 
        Un terreno de forma rectangular tiene $400 m^2$  y va a ser vallado. 
        El precio del metro lineal de valla es de 4 \euro{}.
        ¿Cuáles serán las dimensiones del solar que hacen que el costo de la valla sea mínimo?

    \question
        Supongamos que el solar del problema anterior tiene $200 m^2$ y un lado a lo largo del río requiere una valla más costosa de 5 \euro{} el metro lineal. 
        ¿Qué dimensiones darán el costo más bajo?
    
    \question 
        Se dispone de 400 metros de alambrada para vallar un solar rectangular. 
        ¿Qué dimensiones deberá tener el solar para que con esa alambrada se limite la mayor área posible? 

    \question Manuel tiene un negocio de alquiler de motos acuáticas, las cuales alquila a 80 \euro{} la media hora,
        pero decide ofrecer una oferta para grupo en la que descuenta 2 \euro{} por cada moto alquilada.
        ¿Cuántas motos ha de alquilar para que la ganancia sea máxima?
        ¿Cuál será el importe del alquiler de las motos?
        ¿Cuánto es el precio del alquiler de cada moto?

    \question Dos postes con longitudes de 6 y 8 metros respectivamente se colocan verticalmente 
    sobre el piso con sus bases separadas una distancia de 10 metros. 
    Calcule aproximadamente la longitud mínima de un cable que pueda ir desde la punta de uno de los
    postes hasta un punto en el suelo entre los postes y luego hasta la punta del otro poste.

    \question 
    Una ventana tiene la forma de un rectángulo coronado con un semicírculo. 
    Encuentre las dimensiones de la ventana que deja pasar más luz, si su perímetro mide 5 metros.
    \\
    \begin{minipage}{\linewidth}
        \centering
        \includegraphics[width=2cm]{Img02}
    \end{minipage}

    \question
        Las páginas de un libro deben medir cada una $600 cm^2$ de área. 
        Sus márgenes laterales y el inferior miden $2 cm$. y el superior mide $3 cm$. 
        Calcular las dimensiones de la página que permitan obtener la mayor área impresa posible. 

    \question 
        Una hoja de papel debe contener $18 cm^2$ de texto impreso. 
        Los márgenes superior e inferior han de tener $2 cm$ cada uno, y los laterales $1 cm$. 
        Halla las dimensiones de la hoja para que el gasto de papel sea mínimo. 

    \question
        Un pastor dispone de 1000 m de tela metálica para construir una cerca rectangular aprovechando una pared ya existente. 
        Halla las dimensiones de la cerca para que el área encerrada sea máxima. 

    \question 
        Se considera una ventana rectangular en la que el lado superior se ha sustituido por un triángulo equilátero.
        Sabiendo que el perímetro de la ventana es $6,6 m$, hallar sus dimensiones para que la superficie sea máxima. 
    
    \question 
        Dividir un segmento de $6 cm$ de longitud en dos partes, con la propiedad de que la suma de las áreas del cuadrado y
        del triángulo equilátero construidos sobre ellos sea máxima.

    \question 
        Se considera una ventana como la que se indica en la figura (la parte inferior es rectangular y la superior una semicircunferencia).
        El perímetro de la ventana mide 6 m. 
        Halla las dimensiones “x” e “y” del rectángulo para que la superficie de la ventana sea máxima (Expresa el resultado en función de $\pi$).
        \\
        \begin{minipage}{\linewidth}
            \centering
            \includegraphics[width=2cm]{Img02}
        \end{minipage}

    \question 
        Entre todos los rectángulos de perímetro $12 m$ ¿cuál es el que tiene la diagonal menor? 
        Razonar el proceso seguido.
    
    \question 
        Calcula el área máxima que puede tiene un triángulo rectángulo tal que la suma de la longitudes de sus dos catetos vale 4 cm. 

    \question
        Halla las dimensiones del rectángulo de área máxima inscrito en una circunferencia de $10 cm$ de radio.

    \question
    Un segmento de longitud de $5 cm$ apoya sus extremos en los semiejes positivos OX y OY, de tal manera que forma con éstos un triángulo.
    Halla las dimensiones del triángulo de área máxima así construido.

    \question 
    En un jardín con forma semicírculo de radio 10 m se va a instalar un parterre rectangular, 
    uno de cuyos lados está sobre el diámetro y el opuesto a él tiene sus extremos en la parte curva. 
    Calcula las dimensiones del parterre para que su área sea máxima.
    \\
        \begin{minipage}{\linewidth}
            \centering
            \includegraphics[width=6cm]{Img03}
        \end{minipage}

    \question 
    Calcule las dimensiones de tres campos cuadrados de modo que: el perímetro de uno de ellos sea triple del perímetro de otro, 
    se necesiten exactamente 1248 metros de valla para vallar los tres y la suma de las áreas de los tres campos sea la mínima posible.

    \question
    Una arquitecta quiere construir un jardín rectangular en un terreno circular de 100 metros de radio. 
    Halla las dimensiones de dicho jardín para que el área sea máxima.
    
    \question En un jardín rectangular se quiere reservar un espacio triangular para construir un parterre.
    \\
    \begin{minipage}{\linewidth}
        \centering
        \includegraphics[width=7cm]{Img01}
    \end{minipage}
    \begin{parts}
        \part Expresa el área del parterre $A(x)$ en función de $x$.
        \part ¿Qué valores puede tener $x$?
        \part Halla $A(2)$ y $A(4)$.
        \part Expresa el área de la parte del jardín que no tiene parterre, $B(x)$, en función de $x$.
        \part Representa gráficamente las funciones $A(x)$ y $B(x)$.
        \part ¿Puede estar situadas las gráficas de las funciones $A(x)$ y $B(x)$ en el segundo cuadrante?
    \end{parts}
    
    \question 
        La altura, $h$, a la que se encuentra en cada instante, $t$, 
        un objeto que lanzamos verticalmente hacia arriba con una velocidad de $20m/s$ es $h=20t - 5t^2$.
        \begin{parts}
            \part Representa gráficamente la función.
            \part Di cuál es su dominio de definición.
            \part ¿En qué momento alcanza la altura máxima? ¿Cuál es esa altura?
            \part ¿En qué momento el objeto llega al suelo?
            \part ¿En qué intervalo de tiempo el objeto está a una altura superior a 15 metros?
        \end{parts}

    \question 
        La temperatura, en grados centígrados, el día 28 de junio en Lisboa se puede expresar mediante 
        la función $f(x)=\frac{-9x^2+200x+1000}{100}$ siendo $x$ la hora del día comprendida en el intervalo $[0, 24)]$.
    \begin{parts}
        \part Calcula la temperatura que había al comienzo y al final del día.
        \part Calcula la hora a la que hubo mayor temperatura y su valor.
    \end{parts}

    \question 
    Si un cultivador valenciano planta 200 naranjos por hectárea, 
    el rendimiento promedio es de 300 naranjas por árbol. Por cada árbol adicional que siembre por hectárea, el cultivador obtendrá 15 naranjas menos por árbol. 
    ¿Cuántos árboles por hectárea darán la mejor cosecha?

    \question 
    El propietario de un edificio tiene alquilados los 40 pisos del mismo a un precio de 600 \euro{} cada uno. 
    Por cada 60\euro que el propietario aumenta el precio observa que pierde un inquilino. 
    ¿a qué precio le convienen alquilar los pisos para obtener la mayor ganancia posible?

    \question 
    Entre todos los triángulos isósceles (dos lados iguales) de perímetro 30 cm., ¿cuál es el de área máxima? 

\end{questions}

\end{document}