\documentclass[addpoints,spanish, 12pt,a4paper,cancelspace]{./include/gexercises} 
 
%%%%%%%%%%%%%%%%%%%%%%%%%%%
\renewcommand{\documentName} { Funciones }
\renewcommand{\documentContent} { Problemas con funciones lineales y cuadráticas } 
\renewcommand{\waterMark} {  } 
% Configuración del documento.
\renewcommand{\schoolSubject} { Examen Matemáticas 2º ESO  }
\renewcommand{\school} { IES José de Churriguera  }
\renewcommand{\academicPeriod} { Curso 2022/2023 }

\renewcommand{\autor} { Andrés Giménez Muñoz }
\renewcommand{\emailAuthor} { andresprofemates@outlook.es }
\renewcommand{\autorSing}{ Profesor: Andrés } 
%%%%%%%%%%%%%%%%%%%%%%%%%%%

% \renewcommand{\thepartno}{\arabic{partno}}

\theoremstyle{definition}
\newtheorem*{theorem}{Theorem}
\newtheorem{definition}{Definición}
\newtheorem{property}{Propiedad}

%%%%%%%%%%%%%%%%%%%%%%%%%%%
% Exam configuration
%\pointsdroppedatright   %% No mostrar la puntuación
\pointsinrightmargin{} % Para poner las puntuaciones a la derecha. Se puede cambiar. Si se comenta, sale a la izquierda.
\extrawidth{-1.5cm} %Un poquito más de margen por si ponemos textos largos.
\marginpointname{ \emph{\points}}

%% Si se comenta no aparecerán los espacios de la solución.
%\nocancelspace

%% Esto es de la clase exam. Si dejamos sin comentar \printanswers, se mostraran las soluciones. 
%% Si la comentamos y dejamos sin comentar \noprintanswers, pues no se muestran las soluciones.
% \printanswers
%\noprintanswers

%%%%%%%%%%%%%%%%%%%%%%%%%%%

\begin{document}

% \StudentData{}
% \GradeTableHeader{}

\justifying{}

\begin{questions}
    \question Manuel trabaja como repartidor los fines de semana y recibe una cantidad fija de 40 \euro{} mensuales, 
    más 5 \euro{} por cada paquete que reparte.
    \begin{parts}
        \part Escribe la fórmula y representa la gráfica de la función que relaciona el número de paquetes repartidos con el dinero recibido al mes.
        \part ¿Cuántos paquetes tiene que repartir Manuel para cobrar en un mes 205 \euro{}.
    \end{parts}

    \question Jorge sale a correr con una velocidad constante de $10km/h$. 
    Una hora después, Marta también empieza a correr con una velocidad constante de $15km/h$.
    \begin{parts}
        \part Escribe las expresiones algebraicas de las funciones que describen la distancia recorrida por cada uno de los corredores.
        \part Determina el tiempo que transcurre hasta que Jorge y Marta se encuentran, y la distancia recorrida por cada uno hasta ese momento.
        \part Representa gráficamente estas funciones en un mismo sistema de coordenadas cartesianas
    \end{parts}

    \question 
        Nos queremos ir de vacaciones y hemos encontrado las siguientes ofertas:
        \begin{enumerate}
            \item Hotel La Laguna:  
            \begin{itemize}
                \item Precio por persona y día 70 \euro{}.
                \item Primer día gratis.
                \item Estancia mínima 5 días.
            \end{itemize}
            \item Hotel El Mar: 
            \begin{itemize}
                \item Precio por persona y día 60 \euro{}.
                \item Estancia mínima 2 días.
            \end{itemize}
        \end{enumerate}
        \begin{parts}
            \part Confecciona, para cada uno de los hoteles, la tabla de valores relativa los diez primeros días de estancia en el hotel.
            \part Expresa algebraicamente cómo varía el coste en cada uno de los hoteles al aumentar la estancia.
            \part Representa gráficamente las funciones obtenidas en el apartado anterior.
            \part ¿Cuánto pagará una persona después de 5 días en cada uno de los hoteles?
            \part ¿Al cabo de cuántos días resulta más económico el hotel El Mar?
            \part El coste de la estancia de una persona en el hotel es de 480 \euro{}.
            ¿En qué hotel se ha alojado? ¿Cuántos días?
        \end{parts}

    \question
       Queremos crear un recinto vallado, para lo cual disponemos de un muro ya construido y de 100 metros de malla metálica.
       ¿Qué dimensiones debe tener la valla para que el área del recinto sea máxima? 

    \question Manuel tiene un negocio de alquiler de motos acuáticas, las cuales alquila a 80 \euro{} la media hora,
        pero decide ofrecer una oferta para grupo en la que descuenta 2 \euro{} por cada moto alquilada.
        ¿Cuántas motos ha de alquilar para que la ganancia sea máxima?
        ¿Cuál será el importe del alquiler de las motos?
        ¿Cuánto es el precio del alquiler de cada moto?

    \question En un jardín rectangular se quiere reservar un espacio triangular para construir un parterre.
    \\
    \begin{minipage}{\linewidth}
        \centering
        \includegraphics[width=7cm]{Img01}
    \end{minipage}
    \begin{parts}
        \part Expresa el área del parterre $A(x)$ en función de $x$.
        \part ¿Qué valores puede tener $x$?
        \part Halla $A(2)$ y $A(4)$.
        \part Expresa el área de la parte del jardín que no tiene parterre, $B(x)$, en función de $x$.
        \part Representa gráficamente las funciones $A(x)$ y $B(x)$.
        \part ¿Puede estar situadas las gráficas de las funciones $A(x)$ y $B(x)$ en el segundo cuadrante?
    \end{parts}
    
    \question 
        La altura, $h$, a la que se encuentra en cada instante, $t$, 
        un objeto que lanzamos verticalmente hacia arriba con una velocidad de $20m/s$ es $h=20t - 5t^2$.
        \begin{parts}
            \part Representa gráficamente la función.
            \part Di cuál es su dominio de definición.
            \part ¿En qué momento alcanza la altura máxima? ¿Cuál es esa altura?
            \part ¿En qué momento el objeto llega al suelo?
            \part ¿En qué intervalo de tiempo el objeto está a una altura superior a 15 metros?
        \end{parts}

    \question 
        La temperatura, en grados centígrados, el día 28 de junio en Lisboa se puede expresar mediante 
        la función $f(x)=\frac{-9x^2+200x+1000}{100}$ siendo $x$ la hora del día comprendida en el intervalo $[0, 24)]$.
    \begin{parts}
        \part Calcula la temperatura que había al comienzo y al final del día.
        \part Calcula la hora a la que hubo mayor temperatura y su valor.
    \end{parts}


\end{questions}

\end{document}