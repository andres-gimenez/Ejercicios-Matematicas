%%%%%%%%%%%%%%%%%%%%%%%%%%%
\newcommand{\documentName} { Lógica }
\newcommand{\documentContent} { \phantom{ } } 
\newcommand{\waterMark} {  } 
%%%%%%%%%%%%%%%%%%%%%%%%%%%

% Configuración del documento.
\newcommand{\schoolSubject} { Matemáticas 3º ESO - Recuperación}
\newcommand{\school} { IES La Serna }
\newcommand{\academicPeriod} { Curso 2020/2021 }


\newcommand{\autor} { Andrés Giménez Muñoz }
\newcommand{\emailAuthor} { agimenezmunoz@ieslaserna.com }
\newcommand{\autorSing}{ Profesores: Andrés } 
\renewcommand{\schoolSubject} { Examen Matemáticas 2º ESO  }
\renewcommand{\school} { IES José de Churriguera  }
\renewcommand{\academicPeriod} { Curso 2022/2023 }

\renewcommand{\autor} { Andrés Giménez Muñoz }
\renewcommand{\emailAuthor} { andresprofemates@outlook.es }
\renewcommand{\autorSing}{ Profesor: Andrés } 

\renewcommand{\thepartno}{\arabic{partno}}

\usepackage{xfrac}

%%%%%%%%%%%%%%%%%%%%%%%%%%%
% Exam configuration
%\pointsdroppedatright   %% No mostrar la puntuación
\pointsinrightmargin{} % Para poner las puntuaciones a la derecha. Se puede cambiar. Si se comenta, sale a la izquierda.
\extrawidth{-1.5cm} %Un poquito más de margen por si ponemos textos largos.
\marginpointname{ \emph{\points}}

%% Si se comenta no aparecerán los espacios de la solución.
%\nocancelspace

%% Esto es de la clase exam. Si dejamos sin comentar \printanswers, se mostraran las soluciones. 
%% Si la comentamos y dejamos sin comentar \noprintanswers, pues no se muestran las soluciones.
% \printanswers
%\noprintanswers

%%%%%%%%%%%%%%%%%%%%%%%%%%%

\begin{document}

% \StudentData{}
% \GradeTableHeader{}

\justifying{}

    % https://www.brain-training.org/problemas-de-logica/

    \begin{questions}
        \question 

        \question EL PASTOR Un pastor tiene que pasar un lobo, una cabra y una lechuga a la otra orilla de un río, dispone de una barca en la que solo caben el y una de las otras tres cosas.
        Si el lobo se queda solo con la cabra se la come, si la cabra se queda sola con la lechuga se la come.
        
        \question EL CONDENADO A MUERTE. En los tiempos de la antigüedad la gracia o el castigo se dejaban frecuentemente al azar.
        Así, éste es el caso de un reo al que un sultán decidió que se salvase o muriese sacando al azar una papeleta de entre dos posibles: una con la sentencia «muerte», la otra con la palabra «vida», indicando gracia.     
        Lo malo es que el Gran Visir, que deseaba que el acusado muriese, hizo que en las dos papeletas se escribiese la palabra «muerte».      
        ¿Cómo se las arregló el reo, enterado de la trama del Gran Visir, para estar seguro de salvarse?
        Al reo no le estaba permitido hablar y descubrir así el enredo del Visir.

        \question LAS DEPORTISTAS. Ana, Beatriz y Carmen. Una es tenista, otra gimnasta y otra nadadora.
        La gimnasta, la más baja de las tres, es soltera. Ana, que es suegra de Beatriz, es más alta que la tenista.
        ¿Qué deporte practica cada una?

        \question SILOGISMOS. Ejemplo que está en todos los manuales de lógica elemental. El silogismo: «Los hombres son mortales, Sócrates es hombre.
        Luego, Sócrates es mortal».

        Los gatos son seres vivos, Sócrates está vivo, luego Sócrates es un gato.

        \question EL TORNEO DE AJEDREZ. En un torneo de ajedrez participaron 30 concursantes que fueron divididos, de acuerdo con su categoría, en dos grupos.
        En cada grupo los participantes jugaron una partida contra todos los demás.
        En total se jugaron 87 partidas más en el segundo grupo que en el primero.
        El ganador del primer grupo no perdió ninguna partida y totalizó 7’5 puntos.
        ¿En cuántas partidas hizo tablas el ganador?

        \question LAS TRES CARTAS. Tres naipes, sacados de una baraja francesa, yacen boca arriba en una fila horizontal.
        A la derecha de un Rey hay una o dos Damas. A la izquierda de una Dama hay una o dos Damas.
        A la izquierda de un corazón hay una o dos picas.
        A la derecha de una pica hay una o dos picas. Dígase de qué tres cartas se trata.

    \end{questions}

\end{document}
