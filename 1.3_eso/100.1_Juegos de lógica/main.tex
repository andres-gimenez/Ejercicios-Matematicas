%%%%%%%%%%%%%%%%%%%%%%%%%%%
\newcommand{\documentName} { Lógica }
\newcommand{\documentContent} { \phantom{ } } 
\newcommand{\waterMark} {  } 
%%%%%%%%%%%%%%%%%%%%%%%%%%%

% Configuración del documento.
\newcommand{\schoolSubject} { Matemáticas 3º ESO - Recuperación}
\newcommand{\school} { IES La Serna }
\newcommand{\academicPeriod} { Curso 2020/2021 }


\newcommand{\autor} { Andrés Giménez Muñoz }
\newcommand{\emailAuthor} { agimenezmunoz@ieslaserna.com }
\newcommand{\autorSing}{ Profesores: Andrés } 
\renewcommand{\schoolSubject} { Examen Matemáticas 2º ESO  }
\renewcommand{\school} { IES José de Churriguera  }
\renewcommand{\academicPeriod} { Curso 2022/2023 }

\renewcommand{\autor} { Andrés Giménez Muñoz }
\renewcommand{\emailAuthor} { andresprofemates@outlook.es }
\renewcommand{\autorSing}{ Profesor: Andrés } 

\renewcommand{\thepartno}{\arabic{partno}}

\usepackage{xfrac}

%%%%%%%%%%%%%%%%%%%%%%%%%%%
% Exam configuration
%\pointsdroppedatright   %% No mostrar la puntuación
\pointsinrightmargin{} % Para poner las puntuaciones a la derecha. Se puede cambiar. Si se comenta, sale a la izquierda.
\extrawidth{-1.5cm} %Un poquito más de margen por si ponemos textos largos.
\marginpointname{ \emph{\points}}

%% Si se comenta no aparecerán los espacios de la solución.
%\nocancelspace

%% Esto es de la clase exam. Si dejamos sin comentar \printanswers, se mostraran las soluciones. 
%% Si la comentamos y dejamos sin comentar \noprintanswers, pues no se muestran las soluciones.
% \printanswers
%\noprintanswers

%%%%%%%%%%%%%%%%%%%%%%%%%%%

\begin{document}

% \StudentData{}
% \GradeTableHeader{}

\justifying{}

    % https://www.brain-training.org/problemas-de-logica/

    \begin{questions}
        \question \textbf{La isla de los ojos azules}
        En una isla hay 100 habitantes. Todos ellos tienen o bien ojos azules o bien ojos marrones. Todos ven el color de los otros, pero no el color propio. No pueden hablar del tema y no hay espejos. Eso sí: una ley establece que si alguien descubre que tiene los ojos azules, ha de abandonar la isla a las 8 de la mañana siguientes. Todos los isleños tienen la misma capacidad para razonar y todos son capaces de usar una lógica impecable.
        Un día, una persona llega de visita a la isla y, mientras los mira a todos, dice, sin señalar a nadie en concreto: “¡Qué bueno es ver al menos una persona con ojos azules después de tanto tiempo de estar en alta mar!”
        ¿Qué consecuencias trajo este comentario a los habitantes de la isla?

        \begin{solution}
            Todos los que tengan ojos azules abandonarán la isla.
            Si sólo hubiera una persona con ojos azules, lo sabría ya que vería que los 99 restantes los tienen marrones, así que se marcharía.
            Si hubiera dos, el primero (A) podría pensar que se refiere al segundo (B) y que sólo hay uno, pero el segundo pensaría lo mismo del primero. Cuando uno ve que el otro no deja la isla el primer día, sólo le queda deducir que él también tiene los ojos azules, por lo que ambos se tendrán que marchar al segundo día.
            Lo mismo ocurre si hubiera tres, ya que A vería que B y C no dejan la isla y que, por tanto, él también tiene los ojos azules, así que se tendrían que ir los tres el tercer día, al ver A (por ejemplo) que ni B ni C se han ido el segundo día.
            Y así hasta que se vayan todos los habitantes con ojos azules, sean cuantos sean.
        \end{solution}

        \question \textbf{El camino del monje}
            Un monje parte al amanecer de su monasterio hasta la cima de una montaña, donde llega tras un camino de varias horas. Se queda a descansar y a dormir, y sale por la mañana de la montaña a la misma hora para regresar a su monasterio.
            Es posible que no tardara lo mismo en ir que en volver y da igual que su velocidad no fuera constante o cuándo y cuántas veces se parara a descansar: el monje pasó por algún punto del camino exactamente a la misma hora, pero con un día de diferencia. ¿Por qué?
        \begin{solution}
            Imaginemos que se trata de dos monjes que salen a la misma hora de puntos opuestos: si siguen el mismo camino, en algún momento se tendrán que cruzar. Ahora parece obvio, ¿verdad?        
        \end{solution}

        \question \textbf{La puerta infernal}
            Uno de mentirosos. Estás encerrado en una habitación en la que hay dos puertas vigiladas por dos centinelas. Una lleva a la libertad, pero la otra a la muerte segura. Puedes elegir una puerta y antes puedes hacer una pregunta a uno de los centinelas. Hay un problema: uno de ellos siempre dice la verdad, pero el otro siempre miente.
            \\
            ¿Qué pregunta harías para salvar tu vida?
            \begin{solution}
                ¿Qué diría el otro centinela si le pregunto cuál es la puerta segura?.
                \\
                Si mi centinela miente y el otro dice la verdad, mi centinela me dirá cuál es la puerta que lleva a la muerte. Si mi centinela dice la verdad y el otro miente, también me dirá cuál es esa puerta, ya que es la que el otro me diría. Sólo hay que escoger la opuesta a la que me contesten.
            \end{solution}

        \question \textbf{Los sombreros}
            En una mesa hay tres sombreros negros y dos blancos. Tres personas se ponen un sombrero al azar sin mirar el color y se colocan en fila india. No sé, es una fiesta un poco rara.
            El tercero ve el color de los dos que tiene delante y se le pregunta si sabría decir cuál es el color de su sombrero. Contesta que no.
            El segundo sólo puede ver el sombrero del primero. Se le hace la misma pregunta y contesta que no.
            El primero no ve ningún sombrero, pero sabe perfectamente de qué color es el suyo.
            ¿Qué lógica siguió?
            \begin{solution}
                Si el último no sabe de qué color es su sombrero, eso significa que los otros dos no son blancos, porque si no, sabría que el suyo es negro. Así que o bien hay uno blanco o los dos son negros.
                El segundo ha deducido esto mismo al oír lo que dice el primero, así que si no sabe de qué color es su sombrero es porque el primero es negro. Si el del primero fuera blanco, sabría que el suyo es negro porque los dos no pueden ser blancos.
                Por tanto, el primero sabe que su sombrero es negro.
            \end{solution}

        \question \textbf{Manzanas traigo}
            Tienes una frutería y te han repartido tres cajas: una tiene sólo manzanas; otra, sólo naranjas; la tercera, manzanas y naranjas. Cada caja tiene una etiqueta: “manzanas”, “naranjas” y “manzanas y naranjas”. Ninguna de las cajas tiene la etiqueta que le corresponde. ¿Cómo puedes saber la fruta que contiene cada una de las cajas sacando una sola pieza de una sola de ellas?
            \begin{solution}
                Has de coger una pieza de la caja que dice “manzanas y naranjas”. Como todas están mal etiquetadas, incluida esta, no necesitas saber más.
                Si es una manzana, esta es la caja de las manzanas. Las naranjas están en la etiquetada como “manzanas” y la caja que queda, la de “naranjas”, contiene naranjas y manzanas.
                Si es una naranja, tienes la caja de las naranjas. La etiquetada como “naranjas” contiene manzanas y la que tiene la etiqueta “manzanas” guarda naranjas y manzanas.
            \end{solution}

        \question \textbf{Hyperborea}
        Hyperborea existió hace mucho tiempo, en el tiempo cuando había monstruos, cuando se creía que el mundo era plano, y se creía que el sol se levantaba por el mar del este y se iba por el mar del oeste. Hyperborea estaba situada al norte del monte Olimpo, casa de los dioses. Los Hyperboreanos estaban muy favorecidos por los dioses, especialmente de Apolo. Ellos vivían felizmente en un pais donde el sol brillaba con abundancia, y siempre era primavera. \\
        Hyperborea estaba dividida en tres regiones: Los que vivían al sur (Sororeanos), que siempre decían la verdad; los que vivían al norte (Nororeanos), que siempre mentían; y los que vivían en medio (Midoreanos), que alternaban entre mentir y decir la verdad en un orden desconocido. \\
        
        Entonces Apolo decidió visitar a sus ciudadanos favoritos (los Hyperboreanos) a escondidas. Se acercó a tres habitantes y le dijeron: \\
        
        A: Soy Sororeano \\
        B: Soy Nororeano \\
        C: Los dos mienten, Soy Midoreano \\
        
        Asumiendo que cada persona es de una región diferente, ¿quién es Sororeano, quién Nororeano y quién Midoreano?
        \begin{solution}
            B no puede ser Nororeano como dijo, porque eso supondría que dice la verdad, y los Nororeanos siempre mienten. B no puede ser Sororeano, porque entonces estaría mintiendo, y los sororeanos nunca mienten. Entonces B es Midoreano; A es Sororeano, y C, Sería Nororeano ya que sus dos afirmaciones son falsas.        
        \end{solution}



        
        \question \textbf{EL Pastor} Un pastor tiene que pasar un lobo, una cabra y una lechuga a la otra orilla de un río, dispone de una barca en la que solo caben el y una de las otras tres cosas.
        Si el lobo se queda solo con la cabra se la come, si la cabra se queda sola con la lechuga se la come.
        
        \question \textbf{El condenado a muerte}. En los tiempos de la antigüedad la gracia o el castigo se dejaban frecuentemente al azar.
        Así, éste es el caso de un reo al que un sultán decidió que se salvase o muriese sacando al azar una papeleta de entre dos posibles: una con la sentencia «muerte», la otra con la palabra «vida», indicando gracia.     
        Lo malo es que el Gran Visir, que deseaba que el acusado muriese, hizo que en las dos papeletas se escribiese la palabra «muerte».      
        ¿Cómo se las arregló el reo, enterado de la trama del Gran Visir, para estar seguro de salvarse?
        Al reo no le estaba permitido hablar y descubrir así el enredo del Visir.

        \question \textbf{Las deportistas}. Ana, Beatriz y Carmen. Una es tenista, otra gimnasta y otra nadadora.
        La gimnasta, la más baja de las tres, es soltera. Ana, que es suegra de Beatriz, es más alta que la tenista.
        ¿Qué deporte practica cada una?

        \question \textbf{Silogismos}. Ejemplo que está en todos los manuales de lógica elemental. El silogismo: «Los hombres son mortales, Sócrates es hombre.
        Luego, Sócrates es mortal».

        Los gatos son seres vivos, Sócrates está vivo, luego Sócrates es un gato.

        \question \textbf{El torneo de ajedrez}. En un torneo de ajedrez participaron 30 concursantes que fueron divididos, de acuerdo con su categoría, en dos grupos.
        En cada grupo los participantes jugaron una partida contra todos los demás.
        En total se jugaron 87 partidas más en el segundo grupo que en el primero.
        El ganador del primer grupo no perdió ninguna partida y totalizó 7’5 puntos.
        ¿En cuántas partidas hizo tablas el ganador?

        \question \textbf{Las tres cartas}. Tres naipes, sacados de una baraja francesa, yacen boca arriba en una fila horizontal.
        A la derecha de un Rey hay una o dos Damas. A la izquierda de una Dama hay una o dos Damas.
        A la izquierda de un corazón hay una o dos picas.
        A la derecha de una pica hay una o dos picas. Dígase de qué tres cartas se trata.

    \end{questions}

\end{document}
