%%%%%%%%%%%%%%%%%%%%%%%%%%%
\newcommand{\documentName} { Fracciones }
\newcommand{\documentContent} { \phantom{ } } 
\newcommand{\waterMark} {  } 
%%%%%%%%%%%%%%%%%%%%%%%%%%%

% Configuración del documento.
\newcommand{\schoolSubject} { Matemáticas 3º ESO - Recuperación}
\newcommand{\school} { IES La Serna }
\newcommand{\academicPeriod} { Curso 2020/2021 }


\newcommand{\autor} { Andrés Giménez Muñoz }
\newcommand{\emailAuthor} { agimenezmunoz@ieslaserna.com }
\newcommand{\autorSing}{ Profesores: Andrés } 
\renewcommand{\schoolSubject} { Examen Matemáticas 2º ESO  }
\renewcommand{\school} { IES José de Churriguera  }
\renewcommand{\academicPeriod} { Curso 2022/2023 }

\renewcommand{\autor} { Andrés Giménez Muñoz }
\renewcommand{\emailAuthor} { andresprofemates@outlook.es }
\renewcommand{\autorSing}{ Profesor: Andrés } 

\renewcommand{\thepartno}{\arabic{partno}}

\usepackage{xfrac}

%%%%%%%%%%%%%%%%%%%%%%%%%%%
% Exam configuration
%\pointsdroppedatright   %% No mostrar la puntuación
\pointsinrightmargin{} % Para poner las puntuaciones a la derecha. Se puede cambiar. Si se comenta, sale a la izquierda.
\extrawidth{-1.5cm} %Un poquito más de margen por si ponemos textos largos.
\marginpointname{ \emph{\points}}

%% Si se comenta no aparecerán los espacios de la solución.
%\nocancelspace

%% Esto es de la clase exam. Si dejamos sin comentar \printanswers, se mostraran las soluciones. 
%% Si la comentamos y dejamos sin comentar \noprintanswers, pues no se muestran las soluciones.
% \printanswers
%\noprintanswers

%%%%%%%%%%%%%%%%%%%%%%%%%%%

\begin{document}

% \StudentData{}
% \GradeTableHeader{}

\justifying{}

    \begin{questions}
        \question Realizar las siguientes operaciones con fracciones simplificando en todo momento los pasos intermedios y el resultado.
        
        \begin{parts}
            \begin{multicols}{2}
                \part
                    $\frac{5}{6} \cdot \frac{2}{3} $
                \part
                    $\frac{2}{15} : \frac{2}{3} $
                    
                \part
                $\frac{3}{4} - \frac{1}{3} - \frac{2}{12} + \frac{5}{6}$
                \begin{solution}
                    $m.c.m(3, 4,6,12) = 2^2 \cdot 3$ \\ \\
                    $\frac{3}{4} - \frac{1}{3} - \frac{2}{12} + \frac{5}{6} = \frac{9}{12} - \frac{4}{12} - \frac{2}{12} + \frac{10}{12}$ 
                \end{solution}
                
                \part $-\frac{2}{5}+\frac{1}{3}\cdot\frac{4}{5} - \frac{1}{3} \cdot \frac{6}{5}$           

                \part $\frac{1}{4} + \frac{1}{3} \cdot \frac{6}{5}$
                
                \part $\left(\frac{1}{4} + \frac{1}{3}\right) \cdot \frac{6}{5}$
                
                \part $1-\frac{2}{3}\cdot\frac{1}{5}$
                
                \part $\left(1-\frac{2}{3}\right) \cdot \frac{1}{5}$
                
                \part $-\frac{2}{3}+\frac{4}{3}\cdot\frac{1}{2}$

                \part
                $\left( 4 + \frac{3}{4}\right) - \left(3 + \frac{2}{3} \right)$
                \begin{solution}
                    $m.c.m(3, 4) = 12$ \\ \\
                    $\left( 4 + \frac{3}{4}\right) - \left(3 + \frac{2}{3} \right) = \left( \frac{48}{12} + \frac{9}{12} \right) - \left( \frac{36}{12} + \frac{8}{12} \right) = \frac{57}{12} + \frac{44}{12} = \frac{13}{12}$        
                \end{solution}

                \part
                    $\left( \frac{7}{5} - \frac{1}{2} \right) : \left( 1 - \frac{3}{10} \right) $
                \part
                    $\frac{5}{8} \cdot \left[ \frac{17}{4} - 3 \cdot \left( 2 -  \frac{2}{3} \right) \right] $
                
                \part $\left(-1+\frac{1}{2}-\frac{1}{3}\right) \cdot \frac{6}{5}$

                \part $\left(-\frac{2}{5}+\frac{1}{3}\right)\cdot \frac{4}{5} - \frac{1}{3} \cdot \frac{6}{5}$
                
                \part $\frac{1}{2}- \frac{1}{3} \cdot \frac{4}{3} - \frac{1}{12} + \frac{5}{4} \cdot \frac{8}{3}$
                
                \part $\left(\frac{1}{2} + \frac{1}{3}\right) \cdot \frac{4}{3} - \frac{1}{12} + \frac{5}{4} \cdot \frac{8}{3}$
                
                \part $\left(1 - \frac{1}{2} + \frac{1}{3}\right) \cdot \frac{2}{5}$

                \part $1- \frac{1}{2} + \frac{1}{3} \cdot \frac{2}{5}$

                \part $- \frac{1}{2} \cdot \frac{4}{7} - \frac{2}{14} + \frac{1}{2} \cdot \frac{5}{7}$

                \part $-\frac{1}{2} \cdot \left( \frac{4}{7} - \frac{2}{14} \right) + \frac{1}{2} \cdot \frac{5}{7}$

                \part $\frac{17}{9} - \frac{15}{5} + \frac{4}{3} : \left(\frac{1}{5} + \frac{2}{3} - \frac{1}{15}\right) + \frac{14}{3} : \frac{16}{8}$

                \part $\frac{1}{3} + \frac{4}{3} : \frac{5}{6} \cdot \left(\frac{1}{2} - \frac{3}{2} \cdot \frac{10}{9} + 4\right)$

                \part $\frac{4}{5} - \frac{7}{3} \cdot \frac{3}{7} + \frac{1}{5} \left(2 + \frac{1}{2}\right) - \frac{7}{3} + 4: \frac{6}{5}$

                \part $\frac{2}{3} + \frac{5}{4} \left(\frac{3}{5} + \frac{4}{10}\right) - \frac{5}{4} + \left(\frac{3}{5} : 4\right) + \frac{12}{5}$

                \part $2 + \frac{1}{5} : \left(2 + \frac{7}{3} - \frac{2}{4} + \frac{5}{3} \right)$

                \part $\left(\frac{2}{7} - \frac{4}{5} + \frac{2}{8}\right) \cdot \frac{3}{2} - \frac{7}{5} : \frac{4}{7}$

                \part $\frac{2}{3} + \left[1 - \left(\frac{3}{4} - \frac{1}{6}\right)\right]$

            \end{multicols}
        \end{parts}

        \newpage 

        \question Realizar las siguientes operaciones con fracciones simplificando en todo momento los pasos intermedios y el resultado.
        \begin{parts}
            \part $\frac{17}{9} - \frac{15}{5} + \frac{4}{3} : \left(\frac{1}{5} + \frac{2}{3} - \frac{1}{15}\right) + \frac{14}{3} : \frac{16}{8}$

            \part $\frac{2}{3}- \left[\frac{3}{2} - \frac{1}{5} - \left(\frac{2}{5} - \frac{1}{3}\right) + \left(- \frac{6}{5} - \frac{1}{2}\right) \right] - \frac{3}{4} + \left(\frac{1}{2} - \frac{1}{3} \right)$

            \part $2+\left(\frac{5}{2} - 3\right) - \left[\frac{7}{10} - \left( \frac{2}{5} + \frac{1}{4} \right)\right] $

            \part $2 - \left[\frac{4}{3} - \left(\frac{1}{2} + \frac{2}{5}\right) - \frac{1}{3}\right] - \left(\frac{4}{3} + 2 \right) - \frac{1}{2}$

            \part $\left(\frac{4}{3} - \frac{-1}{9}\right) + \left[2 - \left(- \frac{5}{4} + \frac{2}{3} \right)\right] - \frac{7}{2}$

            \part $\left[\left(\frac{4}{6} + \frac{\sfrac{1}{7}}{2} \right) : \left(\frac{4}{3} - \frac{5}{12} \right)\right] \cdot \left(\frac{1}{6} + \frac{1}{15}\right)$

            \part $\left[-\frac{3}{8} + \left(4 - \frac{1}{2} \right)\right] - \left[\left(2 - \frac{5}{4}\right) + \left(\frac{7}{2} - \frac{1}{8}\right) \right]$

            \part $\left(\frac{1}{3} - \frac{4}{5}\right) \cdot \left[\left(\frac{1}{3} - 1\right) \cdot 3 - \frac{1 + \sfrac{2}{5}}{3}\right]$

            \part $\frac{4}{5} : \left[\frac{12}{16} \left(\frac{1}{6} + \frac{2}{3}\right) - \frac{3}{8} \right] - 3 \left[\frac{1}{6} : \left(1 - \frac {2}{5}\right) \right]$

            \part $\frac{3}{2} - \frac{1}{2}\cdot \frac{4}{3} : \left(\frac{4}{3} - \frac{2}{3} \cdot \frac{15}{8} + 1\right)$
            
            \part $\left[\frac{\frac{5}{3}}{3 - \frac{1}{2}} \cdot \left(\frac{6}{4} - \frac{3}{2} \cdot \frac{1}{3}\right)\right] \cdot \left( \frac{1 + \frac{1}{2} - \frac{1}{3}}{\frac{1}{4} - \frac{2}{3} \cdot \frac{5}{2}} + 1\right)$ 
        \end{parts}
        
    \end{questions}

\end{document}
