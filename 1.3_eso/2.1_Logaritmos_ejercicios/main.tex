 %%%%%%%%%%%%%%%%%%%%%%%%%%%
\newcommand{\documentName} { Logaritmos }
\newcommand{\documentContent} { Propiedades y ejercicios } 
\newcommand{\waterMark} {  } 
%%%%%%%%%%%%%%%%%%%%%%%%%%%

% Configuración del documento.
\newcommand{\schoolSubject} { Matemáticas 3º ESO - Recuperación}
\newcommand{\school} { IES La Serna }
\newcommand{\academicPeriod} { Curso 2020/2021 }


\newcommand{\autor} { Andrés Giménez Muñoz }
\newcommand{\emailAuthor} { agimenezmunoz@ieslaserna.com }
\newcommand{\autorSing}{ Profesores: Andrés } 
\renewcommand{\schoolSubject} { Examen Matemáticas 2º ESO  }
\renewcommand{\school} { IES José de Churriguera  }
\renewcommand{\academicPeriod} { Curso 2022/2023 }

\renewcommand{\autor} { Andrés Giménez Muñoz }
\renewcommand{\emailAuthor} { andresprofemates@outlook.es }
\renewcommand{\autorSing}{ Profesor: Andrés } 

% \renewcommand{\thepartno}{\arabic{partno}}

\usepackage{yhmath}

\usepackage{amsthm}
\theoremstyle{definition}
\newtheorem*{theorem}{Theorem}
\newtheorem{definition}{Definición}
\newtheorem{property}{Propiedad}

%%%%%%%%%%%%%%%%%%%%%%%%%%%
% Exam configuration
%\pointsdroppedatright   %% No mostrar la puntuación
\pointsinrightmargin{} % Para poner las puntuaciones a la derecha. Se puede cambiar. Si se comenta, sale a la izquierda.
\extrawidth{-1.5cm} %Un poquito más de margen por si ponemos textos largos.
\marginpointname{ \emph{\points}}

%% Si se comenta no aparecerán los espacios de la solución.
%\nocancelspace

%% Esto es de la clase exam. Si dejamos sin comentar \printanswers, se mostraran las soluciones. 
%% Si la comentamos y dejamos sin comentar \noprintanswers, pues no se muestran las soluciones.
% \printanswers
%\noprintanswers

%%%%%%%%%%%%%%%%%%%%%%%%%%%

\begin{document}

% \StudentData{}
% \GradeTableHeader{}

\justifying{}

\begin{center}
    \fbox{\parbox{6.5in}{
        \vspace{3mm}
            {\Large
                \centerline{\textbf{Definión de logaritmo}}
            }
        \vspace{5mm}

    Si \textbf{a} es un número positivo y distinto de 1, el \textbf{logaritmo en base a de un número positivo N}
    es el exponente al que hay que elvar la base a para obtener N. 
    
    \center
    $\log_a N =x \Leftrightarrow a^x=N$

    \vspace{5mm}
    }}
\end{center}

{\Large
    % \centerline{
        \textbf{Propiedades de los logaritmos:}
    % }
}

\vspace{5mm}

\begin{enumerate}

\item El logaritmo de la base es simpre igual a 1:

\begin{equation*}
    \log_a a = 1
\end{equation*}

\item El logaritmo de cualquier base de 1 es igual a 0:

\begin{equation*}
    \log_a 1 = 0
\end{equation*}

\item El logaritmo del producto es igual a la suma de los logaritmos:

\begin{equation*}
    \log_a (N \cdot M) =\log_a (N)+\log_a (M)
\end{equation*}

\item El logaritmo del cociente es igual a la diferencia de los logaritmos:

\begin{equation*}
    \log_a (\dfrac {N}{M}) = \log_a (N)-\log_a (M)
\end{equation*}

\item El logaritmo de una potencia es igual al producto del exponente por el logaritmo de la base:

\begin{equation*}
    \log_a (N^r) =r\log_a (N)
\end{equation*}

\item La relación entre los logaritmos de un mismo número A en distintas bases viene dada por:

\begin{equation*}
    \log_a (N) =\dfrac {\log_b (N)}{\log_b (a)} =\dfrac {\ln (N)}{\ln (a)}
\end{equation*}

\item Si no se especifica la base de un logaritmo, se sobreentiende que es base 10.

\begin{equation*}
    \log_{10} (N) =\log (N)
\end{equation*}

\item El logaritmo Neperiano, tiene como base el número $e = 2,718281828459045235360...$ y se indica como $ln$.

\begin{equation*}
    \log_e (N)=\ln (N)
\end{equation*}

\end{enumerate}

\newpage
\textbf{Ejercicios}

\begin{questions}
    \question
    Calcula los siguientes logaritmos
    \begin{multicols}{3}
        \begin{parts}
            \part $\log_2 8$
            \part $\log_2 32$
            \part $\log_2 \frac{1}{16}$
            \part $\log_3 81$
            \part $\log_3 729$
            \part $\log_3 \frac{1}{81}$
            \part $\log_3 \sqrt[3]{3}$
            \part $\log_3 \left(\frac{\sqrt[4]{3}}{9}\right)$
            \part $\log_{81} 3$
            \part $\log_{81} \left(\frac{\sqrt{3}}{3}\right)$

            \part $\log 100000000$
            \part $\log \frac{1}{10000}$
            \part $\log \sqrt{8}$
            \part $\log_{\frac{1}{2}} 8$
            \part $\log_3 \sqrt[3]{243}$
        \end{parts}
    \end{multicols}

    \question
    Halla el valor de $x$ en cada caso:
    \begin{multicols}{3}
        \begin{parts}
            \part $\log_x 16 = -4$
            \part $\log_x \frac{1}{16} = -8$
            \part $\log_{\frac{1}{7}} x = -3$
            \part $\log_{11} 1331 = x$
            \part $\log_x 125 = 3$
            \part $\log_x 24 = 4$
        \end{parts}
    \end{multicols}

    \question
    Conociendo el valor de $\log 2 \approx 0,3010$, calcula los siguientes logaritmos:
    \begin{multicols}{3}
        \begin{parts}
            \part $\log 0,0002$
            \part $\log \sqrt[5]{16}$
            \part $\log \frac{\sqrt[7]{2^5}}{2}$
            \part $\log 0,0125$
            \part $\log \frac{0,64^3 \cdot \sqrt[3]{0,32}}{80 \cdot \sqrt{6,25}}$
        \end{parts}
    \end{multicols}

    \question
    Conociendo el valor de $\log 2 \approx 0,3010$, y el de $\log 3 \approx 0,4771$, calcula:
    \begin{multicols}{3}
        \begin{parts}
            \part $\log 12$
            \part $\log \sqrt[5]{4,8}$
            \part $\log \sqrt[3]{0,6}$
            \part $\log 3,\wideparen{3}$
            \part $\log 40,5$
            \part $\log \frac{\left(0,027\right)^3 \cdot \sqrt[4]{540}}{96 \cdot \sqrt[5]{51,84}}$
        \end{parts}
    \end{multicols}

    \question 
    Si $\log 8 \approx 0,9031$, halla:
    \begin{multicols}{3}
        \begin{parts}
            \part $\log 800$
            \part $\log 2$
            \part $\log 0,64$
            \part $\log 40$
            \part $\log 5$
            \part $\log \sqrt[5]{8}$
        \end{parts}
    \end{multicols}



    %% 4º ESO Editorial SM.
    \question La relación entre el número de decibelios, n, y la intensidad del sonido, I, en voltios por centímetro cuadrado $(V/cm^2)$ viene dada por:
    \begin{equation*}
    n = 10 \cdot log \left(\frac{I}{10^{-16}}\right)
    \end{equation*}

    \begin{parts}
        \part Determina la intensidad de un sonido si el nivel de decibelios es 125.
        \part Si la intensidad del sonido se hace 10 veces mayor, ¿cómo aumenta el número de decibelios?
        \part Si el número de decibelios aumenta el doble, ¿qué relación hay entre las intensidades?
    \end{parts}

    % En el ciclismo de persecución en pista, uno de los corredores da 
    % una vuelta al circuito cada 54 segundos, y el otro cada 72 segundos. Parten 
    % juntos de la línea de salida. ¿Cuánto tiempo tardarán en volverse a encontrar 
    % por primera vez en la línea de salida? ¿Cuántas vueltas habrá dado cada 
    % ciclista en ese tiempo?

    % \begin{solution}
    %     Para calcular el tiempo que tardan en coincidir habrá que calcular el mínimo común múltiplo: 

    %     \begin{align*}
    %         54 &= 2 \cdot 3^3 \\
    %         72 &= 2^3 \cdot 3^2
    %     \end{align*}

    %     \begin{equation*}
    %         mcm(54,72) = 2^3 \cdot 3^3 = 8 \cdot 27 = 216
    %     \end{equation*}

    %     Volverán a coincidir en \textbf{216 segundos}, que son 3 minutos y 36 segundos. \\

    %     \begin{itemize}
    %         \item Primer ciclista: $ 216 : 54 = \textbf{4 vueltas} $
    %         \item Segundo ciclista: $ 216 : 72 = \textbf{3 vuletas} $
    %     \end{itemize}

    % \end{solution}
    
    % \question Un grupo de 60 niños, acompañados de 36 padres, acude a un 
    % campamento en la montaña. Para dormir ocupan cabañas todas iguales, 
    % pero sin mezclarse los padres con los niños. Cuantas menos cabañas ocupen, 
    % menos pagan. ¿Cuántas personas dormirán en cada cabaña? 

    % \begin{solution}
    %     Para calcular el mínimo número de personas que pueden entrar en cada cabaña, sin mezclarse, se busca el máximo común divisor: 

    %     \begin{align*}
    %         60 &= 2^2 \cdot 3 \cdot 5 \\
    %         36 &= 2 \cdot 3^2
    %     \end{align*}

    %     \begin{equation*}
    %         MCD(36,60) = 2^2 \cdot 3 = 4 \cdot 3 = 12
    %     \end{equation*}

    %     Dormirán \textbf{12 personas} en cada cabaña.\\

    % \end{solution}

    % \question Deseamos partir 2 cuerdas de 20 y 30 metros en trozos iguales lo 
    % más grandes posible y sin desperdiciar ningún cabo. ¿Cuánto medirá cada 
    % trozo? ¿Cuántos trozos habrá? 

    % \begin{solution}
    %     Para calcular la longitud de cada trozo habrá que calcular el máximo común divisor: 

    %     \begin{align*}
    %         20 &= 2^2 \cdot 5 \\
    %         30 &= 2 \cdot 3 \cdot 5
    %     \end{align*}

    %     \begin{equation*}
    %         MCD(20,30) = 2 \cdot 5 = 10
    %     \end{equation*}

    %     Cada toroz medrirá \textbf{10 metros}. \\

    %     \begin{itemize}
    %         \item De la primera cuerda se obtendrán: $ 20 : 10 = 2 trozos $
    %         \item De la segunda cuerda se obtendrán: $ 30 : 10 = 3 trozos $
    %     \end{itemize}

    %     Por tanto se obtendrán en total \textbf{5 trozos}.

    % \end{solution}

    % \question Un panadero necesita envases para colocar 250 magdalenas y 75 
    % mantecados en cajas, lo más grandes que sean posible, pero sin mezclar 
    % ambos productos en la misma caja. ¿Cuántas cajas harán falta, y cuántos 
    % bollos irán en cada caja? 

    % \begin{solution}
    %     Para calcular los bollos que irán en cada caja hay que calcular el máximo común divisor: 

    %     \begin{align*}
    %         250 &= 2 \cdot 5^3 \\
    %         75 &= 3 \cdot 5^2
    %     \end{align*}

    %     \begin{equation*}
    %         MCD(250,75) = 5^2 = 25
    %     \end{equation*}

    %     Irán \textbf{25 bollos} en cada caja. \\

    %     \begin{itemize}
    %         \item Cajas de magdalenas: $ 250 : 25 = 10 cajas $
    %         \item Cajas de mantecados: $ 75 : 25 = 3 cajas $
    %     \end{itemize}

    %     Por tanto habrá un total de \textbf{13 cajas}.

    % \end{solution}

    % \question El mayor de los 3 hijos de una familia visita a sus padres cada 15 
    % días, el mediano cada 10, y la menor cada 12. La cena de Nochebuena se 
    % reúne toda la familia. ¿Cuándo volverán a encontrarse los tres juntos?

    % \begin{solution}
    %     Para calcular cuándo coinciden los tres juntos de nuevo habrá que calcular el mínimo común múltiplo:

    %     \begin{align*}
    %         15 &= 3 \cdot 5 \\
    %         10 &= 2 \cdot 5 \\
    %         12 &= 2^2 \cdot 3  
    %     \end{align*}

    %     \begin{equation*}
    %         mcm(10,12,15) = 2^2 \cdot 3 \cdot 5 = 4 \cdot 3 \cdot 5 = 60
    %     \end{equation*}

    %     Volverán a encontrarse en \textbf{60 días}. Es decir, Nochebuena, como es el 24 de diciembre, más 60 días, el 22 de febrero.

    % \end{solution}

    % \question Alan y Pedro comen en la misma taquería, pero Alan asiste cada 20 días y Pedro cada 38. ¿Cuánto volverán a encontrarse?

    % \question David tiene 24 dulces para repartir y Fernando tiene 18. 
    % Si desean regalar los dulces a sus respectivos familiares de modo que todos tengan la misma cantidad y que sea la mayor posible, 
    % ¿cuántos dulces repartirán a cada persona? ¿A cuántos familiares regalará dulces cada uno de ellos?

    % \question Andrés tiene una cuerda de 120 metros y otra de 96 metros.
    % Desea cortarlas de modo que todos los trozos sean iguales pero lo más largos posibles.
    % ¿Cuántos trozos de cuerda obtendrá?

    % \question En un vecindario, un camión de helados pasa cada 8 días y un food truck pasa cada dos semanas.
    % Se sabe que 15 días atrás ambos vehículos pasaron en el mismo días. \\

    % Raúl cree que dentro de un mes los vehículos volverán a encontrarse y Oscar cree esto ocurrirá dentro de dos semanas.
    % ¿Quién está en lo cierto?

    % \question En una banda compuesta por un baterista, un guitarrista, un bajista y un saxofonista, el baterista toca en lapsos de 8 tiempos, 
    % el guitarrista en 12 tiempos, el bajista en 6 tiempos y el saxofonista en 16 tiempos.
    % Si todos empiezas al mismo tiempo, ¿en cuántos tiempos sus periodos volverán a iniciar al mismo tiempo?

\end{questions}

\end{document}