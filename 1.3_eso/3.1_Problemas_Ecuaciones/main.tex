 %%%%%%%%%%%%%%%%%%%%%%%%%%%
\newcommand{\documentName} { Ecuaciones }
\newcommand{\documentContent} { Problemas con ecuaciones de 1º grado } 
\newcommand{\waterMark} {  } 
%%%%%%%%%%%%%%%%%%%%%%%%%%%

% Configuración del documento.
\newcommand{\schoolSubject} { Matemáticas 3º ESO - Recuperación}
\newcommand{\school} { IES La Serna }
\newcommand{\academicPeriod} { Curso 2020/2021 }


\newcommand{\autor} { Andrés Giménez Muñoz }
\newcommand{\emailAuthor} { agimenezmunoz@ieslaserna.com }
\newcommand{\autorSing}{ Profesores: Andrés } 
\renewcommand{\schoolSubject} { Examen Matemáticas 2º ESO  }
\renewcommand{\school} { IES José de Churriguera  }
\renewcommand{\academicPeriod} { Curso 2022/2023 }

\renewcommand{\autor} { Andrés Giménez Muñoz }
\renewcommand{\emailAuthor} { andresprofemates@outlook.es }
\renewcommand{\autorSing}{ Profesor: Andrés } 

% \renewcommand{\thepartno}{\arabic{partno}}

\usepackage{yhmath}

\usepackage{amsthm}
\theoremstyle{definition}
\newtheorem*{theorem}{Theorem}
\newtheorem{definition}{Definición}
\newtheorem{property}{Propiedad}

%%%%%%%%%%%%%%%%%%%%%%%%%%%
% Exam configuration
%\pointsdroppedatright   %% No mostrar la puntuación
\pointsinrightmargin{} % Para poner las puntuaciones a la derecha. Se puede cambiar. Si se comenta, sale a la izquierda.
\extrawidth{-1.5cm} %Un poquito más de margen por si ponemos textos largos.
\marginpointname{ \emph{\points}}

%% Si se comenta no aparecerán los espacios de la solución.
%\nocancelspace

%% Esto es de la clase exam. Si dejamos sin comentar \printanswers, se mostraran las soluciones. 
%% Si la comentamos y dejamos sin comentar \noprintanswers, pues no se muestran las soluciones.
% \printanswers
%\noprintanswers

%%%%%%%%%%%%%%%%%%%%%%%%%%%

\begin{document}

% \StudentData{}
% \GradeTableHeader{}

\justifying{}

\begin{questions}
    \question
    ¿Cuántos centímetros hay que sumar a la base y a la altura de un rectángulo de $7 cm$ de base y $3 cm$ de altura 
    para que el área del nuevo rectángulo sea de $96 cm^2$?

    \question
    Marta tiene que leer un libro para la clase de Lengua. 
    La primera semana leyó las $\sfrac{2}{7}$ partes del libro, la segunda semana las $\sfrac{2}{5}$ partes de lo que le quedaba y la tercera semana las 60 páginas restantes.
    ¿Cuántas páginas tiene el libro? ¿Cuántas páginas leyó las dos primeras semanas?

    \question 
    Tres hermanos se reparten 1300\euro{}. El mayor recibe doble que el mediano y este el cuádruple que el pequeño. ¿Cuánto recibe cada uno?

    % Calcula los siguientes logaritmos
    % \begin{multicols}{3}
    %     \begin{parts}
    %         \part $\log_2 8$
    %         \part $\log_2 32$
    %         \part $\log_2 \frac{1}{16}$
    %         \part $\log_3 81$
    %         \part $\log_3 729$
    %         \part $\log_3 \frac{1}{81}$
    %         \part $\log_4 (8) + \log_4(2)$ 
    %         \part $\log_4 4^4$
    %         \part $\log_5 (125) + \log_5(5)$    
    %         \part $\log_5 \left(100\right) + \log_5 \left(\frac{1}{4}\right)$ 
    %         \part $\log_6 36^2$
    %         \part $\log_{17} {17} $
    %         \part $\log_8 1$
    %         \part $\log_5 10$
    %         \part $\log 100000000$
    %         \part $\log \frac{1}{10000}$
    %         \part $\log 0,001$
    %     \end{parts}
    % \end{multicols}

    % \question
    % Conociendo el valor de $\log 2 \approx 0,3010$, calcula los siguientes logaritmos:
    % \begin{multicols}{3}
    %     \begin{parts}
    %         \part $\log 0,0002$
    %         \part $\log 16$
    %         \part $\log 32^4$
    %         \part $\log 4^5$
    %         \part $\log 0,0125$
    %         \part $\log \frac{0,64^3 \cdot 0,32}{80 \cdot 6,25}$
    %     \end{parts}
    % \end{multicols}

    % \question
    % Conociendo el valor de $\log 2 \approx 0,3010$, y el de $\log 3 \approx 0,4771$, calcula:
    % \begin{multicols}{3}
    %     \begin{parts}
    %         \part $\log 12$
    %         \part $\log (4,8)^2$
    %         \part $\log (0,6)^3$
    %         \part $\log 3,\wideparen{3}$
    %         \part $\log 40,5$
    %         \part $\log 5000$
    %         \part $\log \frac{\left(0,027\right)^3 \cdot 540}{96 \cdot 51,84}$
    %     \end{parts}
    % \end{multicols}

    % \question 
    % Si $\log 8 \approx 0,9031$, halla:
    % \begin{multicols}{3}
    %     \begin{parts}
    %         \part $\log 800$
    %         \part $\log 2$
    %         \part $\log 0,64$
    %         \part $\log 40$
    %         \part $\log 5$
    %         \part $\log 64^4$
    %     \end{parts}
    % \end{multicols}

    % \question 
    % Resuelve las siguientes ecuaciones
    %     \begin{parts}
    %         \part $12^{x-2} = 4^x$
    %         \part $2^{2x+2} = 9 \cdot 2^x - 2$
    %         \part $\left(8^{x+2}\right)\left(4^{x-6}\right) = 16$
    %     \end{parts}

\end{questions}

\end{document}