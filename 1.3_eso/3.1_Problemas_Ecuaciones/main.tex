 %%%%%%%%%%%%%%%%%%%%%%%%%%%
\newcommand{\documentName} { Ecuaciones }
\newcommand{\documentContent} { Problemas con ecuaciones } 
\newcommand{\waterMark} {  } 
%%%%%%%%%%%%%%%%%%%%%%%%%%%

% Configuración del documento.
\newcommand{\schoolSubject} { Matemáticas 3º ESO - Recuperación}
\newcommand{\school} { IES La Serna }
\newcommand{\academicPeriod} { Curso 2020/2021 }


\newcommand{\autor} { Andrés Giménez Muñoz }
\newcommand{\emailAuthor} { agimenezmunoz@ieslaserna.com }
\newcommand{\autorSing}{ Profesores: Andrés } 
\renewcommand{\schoolSubject} { Examen Matemáticas 2º ESO  }
\renewcommand{\school} { IES José de Churriguera  }
\renewcommand{\academicPeriod} { Curso 2022/2023 }

\renewcommand{\autor} { Andrés Giménez Muñoz }
\renewcommand{\emailAuthor} { andresprofemates@outlook.es }
\renewcommand{\autorSing}{ Profesor: Andrés } 

% \renewcommand{\thepartno}{\arabic{partno}}

\usepackage{yhmath}

\usepackage{amsthm}
\theoremstyle{definition}
\newtheorem*{theorem}{Theorem}
\newtheorem{definition}{Definición}
\newtheorem{property}{Propiedad}

%%%%%%%%%%%%%%%%%%%%%%%%%%%
% Exam configuration
%\pointsdroppedatright   %% No mostrar la puntuación
\pointsinrightmargin{} % Para poner las puntuaciones a la derecha. Se puede cambiar. Si se comenta, sale a la izquierda.
\extrawidth{-1.5cm} %Un poquito más de margen por si ponemos textos largos.
\marginpointname{ \emph{\points}}

%% Si se comenta no aparecerán los espacios de la solución.
%\nocancelspace

%% Esto es de la clase exam. Si dejamos sin comentar \printanswers, se mostraran las soluciones. 
%% Si la comentamos y dejamos sin comentar \noprintanswers, pues no se muestran las soluciones.
% \printanswers
%\noprintanswers

%%%%%%%%%%%%%%%%%%%%%%%%%%%

\begin{document}

% \StudentData{}
% \GradeTableHeader{}

\justifying{}

\begin{questions}
    \question
    ¿Cuántos centímetros hay que sumar a la base y a la altura de un rectángulo de $7 cm$ de base y $3 cm$ de altura 
    para que el área del nuevo rectángulo sea de $96 cm^2$?

    \question 
    Un padre tiene 47 años y su hijo 11. ¿Cuántos años han de transcurrir para que la edad del padre sea triple que la del hijo?

    \question
    En un rectángulo la base mide 18 cm más que la altura y el perímetro mide 76 cm. ¿Cuáles son las dimensiones del rectángulo?

    \question
    Calcula tres números consecutivos cuya suma sea 51.

    \question
    Halla los números que sumados con su anterior y con su siguiente sea 114.

    \question
    Calcula el número que se triplica al sumarle 26.

    \question
    La tercera parte de un número es 45 unidades menor que su doble. ¿Cuál es el número?

    \question
    ¿Qué edades tiene Rosa sabiendo que dentro de 56 años tendrá el quíntuplo de su edad actual?
    
    \question
    Tres hermanos se reparten 1.300\euro{}. El mayor recibe doble que el mediano y este el cuádruple que el pequeño. ¿Cuánto recibe cada uno?

    \question
    Si a la edad de Rodrigo se le suma su mitad se obtiene la edad de Andrea. ¿Cuál es la edad de Rodrigo si Andrea tiene 24 años?

    \question
    Un padre tiene 47 años y su hijo 11. ¿Cuántos años han de transcurrir para que la edad del padre sea triple que la del hijo?
    
    \question
    Dos ciclistas avanzan uno hacia el otro por una misma carretera. Sus velocidades son de 20km/h y de 15 km/h. Si les separan 78 km. ¿Cuánto tardarán en encontrarse?
    
    \question
    Un camión sale de una ciudad a una velocidad de 60Km/h. Dos horas más tarde sale en su persecución un coche a 100 km/h ¿cuánto tardarán en encontrase?
    
    \question
    En un rectángulo la base mide 18 cm más que la altura y el perímetro mide 76 cm. ¿Cuáles son las dimensiones del rectángulo?
    
    \question
    En un control de Biología había que contestar 20 preguntas. Por cada pregunta bien contestada dan tres puntos y por cada fallo restan dos. ¿Cuántas preguntas acertó Elena sabiendo que ha obtenido 30 puntos y que contestó todas?
    
    \question
    Cada vez que un jugador gana una partida recibe 7\euro{} y cada vez que pierde paga 3\euro{}. Al cabo de 15 partidas ha ganado 55\euro{}. ¿Cuántas partidas ha ganado y cuántas ha perdido?
    
    \question
    La mitad de un número multiplicada por su quinta parte es igual a 160. ¿Cuál es ese número?
    
    \question
    En un garaje hay 110 vehículos entre coches y motos y sus ruedas suman 360. ¿Cuántas motos y coches hay?
    
    \question
    Un granjero lleva al mercado una cesta de huevos, de tan mala suerte que tropieza y se le rompen 2/5 partes de la mercancía. Entonces vuelve al gallinero y recoge 21 huevos más, con lo que ahora tiene 1/8 más de la cantidad inicial. ¿Cuántos huevos tenía al principio? 22
    
    \question
    De un barril lleno de agua se saca la mitad de contenido y después un tercio del resto, quedando en él 200 litros. Calcula la capacidad del barril.
    
    \question
    Un reloj marca las 4 de la tarde. ¿A qué hora se superpondrán las manecillas?
    
    \question
    Se han consumido las 7/8 partes de un bidón de gasolina. Añadiendo 38 litros se llena hasta las 3/5 partes. Calcula la capacidad del bidón.

    \question
    Un padre tiene 35 años y su hijo 5. ¿Al cabo de cuántos años la edad del padre será tres veces mayor que la del hijo?
    
    \question
    Si al doble de un número le sumas su mitad resulta 90. ¿Cuál es el número?
    
    \question
    La base de un rectángulo es doble que su altura. ¿Cuáles son sus dimensiones si el perímetro mide 30 cm?
    
    \question
    En una granja hay doble número de gatos que de perros y triple número de gallinas que de perros y gatos juntos. ¿Cuántos gatos, perros y gallinas hay si en total son 96 animales?
    
    \question
    Una granja tiene cerdos y pavos, en total hay 35 cabezas y 116 patas. ¿Cuántos cerdos y pavos hay?
    
    \question
    Luis hizo un viaje en el coche, en el cuál consumió 20 litros de gasolina. El trayecto lo hizo en 2 etapas, en la primera consumió 2/3 de la gasolina que tenía el depósito y en la segunda etapa la mitad de lo que le quedaba. ¿Cuántos litros tenía? ¿Cuántos litros consumió en cada etapa?
    
    \question
    En una librería Ana compra un libro con la tercera parte de su dinero y un comic con las dos terceras partes de lo que le quedaba. Al salir de la librería tenía 12e. ¿Cuánto dinero tenía Ana?
    
    \question
    Las tres cuartas partes de la edad del padre de Juan excede en 15 años a la edad de este. Hace cuatro años la edad del padre era el doble que la edad del hijo. Hallar las edades de ambos.
    
    \question
    Halla el valor de los tres ángulos de un triángulo sabiendo que B mide 40º más que C y que A mide 40º más que B.
    
    \question
    Una madre tiene 60 años y su hijo la mitad. ¿Cuántos años hace que la madre tenía tres veces la edad del hijo?
    
    \question
    Ana tiene 7 años más que su hermano Juan. Dentro de dos años la edad de Ana será el doble de la de Juan. ¿Qué edad tiene cada uno en la actualidad?
    
    \question
    Un padre tiene 34 años y su hijo 12. ¿Al cabo de cuántos años la edad del padre será el doble que la del hijo?
    
    \question
    La edad de una madre y un hijo suman 40 años y dentro de 14 años la edad de la madre será el triple de la del hijo. Calcula la edad actual de cada uno.
    
    \question
    Un padre tiene 37 años y las edades de sus tres hijos suman 25 años. ¿Dentro de cuántos años las edades de los hijos sumarán como la edad del padre?
    
    \question
    Preguntado el padre por la edad de su hijo contesta: “si el doble de los años que tiene se le quitan el triple de los que tenía hace 6 años se tendrá su edad actual”. Halla la edad del hijo en el momento actual.
    
    \question
    Una madre es 21 años mayor que su hijo y en 6 años el niño será 5 veces menor que ella. ¿Qué edad tiene el hijo?
    
    \question
    Se distribuyen 400 bolsas en tres urnas sabiendo que la primera tiene 80 menos que la segunda y esta tiene 60 menos que la tercera, averigua cuántas bolsas tiene cada una.
    
    \question
    Reparten 390\euro{} entre dos personas de tal modo que la parte de la primera sea igual al doble de la parte de la segunda menos 60.
    
    \question
    Un granjero tiene 12 caballos de 9 y 11 años. La suma de sus edades es de 122 años. ¿Cuántos caballos había de cada edad?
    
    \question
    En una empresa trabajan 160 personas y todas ellas deben someterse a un reconocimiento médico en el plazo de tres días. El primer día lo hace la tercera parte de los que lo hacen durante los otros dos días. El segundo día y el tercero lo hacen el mismo número de personas. Calcule el número de trabajadores que acuden al reconocimiento cada día.
    
    \question
    Trabajando juntos, 2 obreros tardan en hacer un trabajo 17 horas. ¿Cuánto tardarán en hacerlo por separado si uno es el doble de rápido que el otro?

    \question
    Marta tiene que leer un libro para la clase de Lengua. 
    La primera semana leyó las $\sfrac{2}{7}$ partes del libro, la segunda semana las $\sfrac{2}{5}$ partes de lo que le quedaba y la tercera semana las 60 páginas restantes.
    ¿Cuántas páginas tiene el libro? ¿Cuántas páginas leyó las dos primeras semanas?

    \question
    Roberto tiene 18\euro{} en monedas de 20 céntimos y 50 céntimos. 
    ¿Cuántas monedas tiene si hay el doble de monedas de 20 céntimos que de 50 céntimos?

    \question
    Hace 4 años la edad de Manuel era la mitad de la edad que tendrá dentro de 7 años.
    ¿Qué edad tiene Manuel hoy?

    

    \question
    Una habitación de $13,5m^2$ de superficie mide un metro y medio más de largo que de ancho.
    ¿Cuáles son las dimensiones exactas de la habitación?

    \question
    Un rectángulo de $20cm$ de perímetro tiene una superficie de $21cm^2$. 
    ¿Cuánto miden los lados?

    \question
    Cristina ha diseñado una pancarta para celebrar el 40º aniversario del instituto.
    Calcula el perímetro y el área de la pancarta, sabiendo que la diagonal mide $17dm$ y el ancho $7dm$ más que el alto.

    \question
    Mercedes quiere cercar una finca. Sabe que la razón entre sus lados es de 2 a 1 y su superficie mide $450m^2$.
    ¿Cuántos metros de valla necesita comprar?

    % Ecuaciones de segundo grado

    \question 
    Queremos poner una cerca a una finca rectangular de $7.000 m^2$
    y sabemos que la finca mide $30m$ más de largo que de ancho 
    ¿Cuánto nos costara la cerca si la valla metálica tiene un precio de 5\euro{}/m.

    \question  % Ecuaciones de tercer grado (solución, x=3, (3m x 6m x 1,5m))
    Queremos poner una piscina en el jardín y tenemos permiso del ayuntamiento para que la piscina no exceda de $23 m^3$.
    Queremos que la piscina sea rectangular, que el largo tenga 3 m más que el ancho y la profundidad sea la mitad del ancho.
    Calcula las dimensiones de la piscina.

    \question 
    Marta, Mercedes y María están organizando sus colecciones de dados.
    Mercedes tiene 3 dados más que María y Marta tiene el doble de dados que Mercedes.
    Si el producto de las tres cantidades es igual a 100.
    ¿Cuántos dados tiene cada una?

    \question
    El producto de cuatro números consecutivos es 360. 
    ¿Puedes determinar de forma exacta los cuatro números?

    % Sistemas de ecuaciones
    \question 
    Seis amigos han alquilado un apartamento en la playa para las vacaciones.
    Cuándo llegan las vacaciones uno de ellos no puede ir y los demás deberán pagar 20\euro{} más cada uno.
    ¿Cuánto les ha costado el apartamento?

    \question
    Antonio se ha comprado dos pantalones y tres camisas en las rebajas. Los pantalones tenían un $30\%$ de descuento y las camisas un $20\%$.
    El precio original de un pantalón era el doble que el de una camisa, pero con el descuento solo ha pagado 104\euro{}
    ¿Cuánto costaba cada artículo antes de las rebajas?

\end{questions}

\end{document}