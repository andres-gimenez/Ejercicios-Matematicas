 %%%%%%%%%%%%%%%%%%%%%%%%%%%
\newcommand{\documentName} { Logaritmos }
\newcommand{\documentContent} { Propiedades y ejercicios } 
\newcommand{\waterMark} {  } 
%%%%%%%%%%%%%%%%%%%%%%%%%%%

% Configuración del documento.
\newcommand{\schoolSubject} { Matemáticas 3º ESO - Recuperación}
\newcommand{\school} { IES La Serna }
\newcommand{\academicPeriod} { Curso 2020/2021 }


\newcommand{\autor} { Andrés Giménez Muñoz }
\newcommand{\emailAuthor} { agimenezmunoz@ieslaserna.com }
\newcommand{\autorSing}{ Profesores: Andrés } 
\renewcommand{\schoolSubject} { Examen Matemáticas 2º ESO  }
\renewcommand{\school} { IES José de Churriguera  }
\renewcommand{\academicPeriod} { Curso 2022/2023 }

\renewcommand{\autor} { Andrés Giménez Muñoz }
\renewcommand{\emailAuthor} { andresprofemates@outlook.es }
\renewcommand{\autorSing}{ Profesor: Andrés } 

% \renewcommand{\thepartno}{\arabic{partno}}

\usepackage{yhmath}

\usepackage{amsthm}
\theoremstyle{definition}
\newtheorem*{theorem}{Theorem}
\newtheorem{definition}{Definición}
\newtheorem{property}{Propiedad}

%%%%%%%%%%%%%%%%%%%%%%%%%%%
% Exam configuration
%\pointsdroppedatright   %% No mostrar la puntuación
\pointsinrightmargin{} % Para poner las puntuaciones a la derecha. Se puede cambiar. Si se comenta, sale a la izquierda.
\extrawidth{-1.5cm} %Un poquito más de margen por si ponemos textos largos.
\marginpointname{ \emph{\points}}

%% Si se comenta no aparecerán los espacios de la solución.
%\nocancelspace

%% Esto es de la clase exam. Si dejamos sin comentar \printanswers, se mostraran las soluciones. 
%% Si la comentamos y dejamos sin comentar \noprintanswers, pues no se muestran las soluciones.
% \printanswers
%\noprintanswers

%%%%%%%%%%%%%%%%%%%%%%%%%%%

\begin{document}

% \StudentData{}
% \GradeTableHeader{}

\justifying{}

\begin{center}
    \fbox{\parbox{6.5in}{
        \vspace{3mm}
            {\Large
                \centerline{\textbf{Definión de logaritmo}}
            }
        \vspace{5mm}

    Si \textbf{a} es un número positivo y distinto de 1, el \textbf{logaritmo en base a de un número positivo N}
    es el exponente al que hay que elvar la base a para obtener N. 
    
    \center
    $\log_a N =x \Leftrightarrow a^x=N$

    \vspace{5mm}
    }}
\end{center}

{\Large
    % \centerline{
        \textbf{Propiedades de los logaritmos:}
    % }
}

\vspace{5mm}

\begin{enumerate}

\item El logaritmo de la base es simpre igual a 1:

\begin{equation*}
    \log_a a = 1
\end{equation*}

\item El logaritmo de cualquier base de 1 es igual a 0:

\begin{equation*}
    \log_a 1 = 0
\end{equation*}

\item El logaritmo del producto es igual a la suma de los logaritmos:

\begin{equation*}
    \log_a (N \cdot M) =\log_a (N)+\log_a (M)
\end{equation*}

\item El logaritmo del cociente es igual a la diferencia de los logaritmos:

\begin{equation*}
    \log_a (\dfrac {N}{M}) = \log_a (N)-\log_a (M)
\end{equation*}

\item El logaritmo de una potencia es igual al producto del exponente por el logaritmo de la base:

\begin{equation*}
    \log_a (N^r) =r\log_a (N)
\end{equation*}

\item La relación entre los logaritmos de un mismo número A en distintas bases viene dada por:

\begin{equation*}
    \log_a (N) =\dfrac {\log_b (N)}{\log_b (a)} =\dfrac {\ln (N)}{\ln (a)}
\end{equation*}

\item Si no se especifica la base de un logaritmo, se sobreentiende que es base 10.

\begin{equation*}
    \log_{10} (N) =\log (N)
\end{equation*}

\item El logaritmo Neperiano, tiene como base el número $e = 2,718281828459045235360...$ y se indica como $ln$.

\begin{equation*}
    \log_e (N)=\ln (N)
\end{equation*}

\end{enumerate}

\newpage
\textbf{Ejercicios}

\begin{questions}
    \question
    Calcula los siguientes logaritmos
    \begin{multicols}{3}
        \begin{parts}
            \part $\log_2 8$
            \part $\log_2 32$
            \part $\log_2 \frac{1}{16}$
            \part $\log_3 81$
            \part $\log_3 729$
            \part $\log_3 \frac{1}{81}$
            \part $\log_3 \sqrt[3]{3}$
            \part $\log_3 \left(\frac{\sqrt[4]{3}}{9}\right)$
            \part $\log_{81} 3$
            \part $\log_{81} \left(\frac{\sqrt{3}}{3}\right)$

            \part $\log 100000000$
            \part $\log \frac{1}{10000}$
            \part $\log \sqrt{8}$
            \part $\log_{\frac{1}{2}} 8$
            \part $\log_3 \sqrt[3]{243}$
        \end{parts}
    \end{multicols}

    \question
    Halla el valor de $x$ en cada caso:
    \begin{multicols}{3}
        \begin{parts}
            \part $\log_x 16 = -4$
            \part $\log_x \frac{1}{16} = -8$
            \part $\log_{\frac{1}{7}} x = -3$
            \part $\log_{11} 1331 = x$
            \part $\log_x 125 = 3$
            \part $\log_x 24 = 4$
        \end{parts}
    \end{multicols}

    \question
    Conociendo el valor de $\log 2 \approx 0,3010$, calcula los siguientes logaritmos:
    \begin{multicols}{3}
        \begin{parts}
            \part $\log 0,0002$
            \part $\log \sqrt[5]{16}$
            \part $\log \frac{\sqrt[7]{2^5}}{2}$
            \part $\log 0,0125$
            \part $\log \frac{0,64^3 \cdot \sqrt[3]{0,32}}{80 \cdot \sqrt{6,25}}$
        \end{parts}
    \end{multicols}

    \question
    Conociendo el valor de $\log 2 \approx 0,3010$, y el de $\log 3 \approx 0,4771$, calcula:
    \begin{multicols}{3}
        \begin{parts}
            \part $\log 12$
            \part $\log \sqrt[5]{4,8}$
            \part $\log \sqrt[3]{0,6}$
            \part $\log 3,\wideparen{3}$
            \part $\log 40,5$
            \part $\log \frac{\left(0,027\right)^3 \cdot \sqrt[4]{540}}{96 \cdot \sqrt[5]{51,84}}$
        \end{parts}
    \end{multicols}

    \question 
    Si $\log 8 \approx 0,9031$, halla:
    \begin{multicols}{3}
        \begin{parts}
            \part $\log 800$
            \part $\log 2$
            \part $\log 0,64$
            \part $\log 40$
            \part $\log 5$
            \part $\log \sqrt[5]{8}$
        \end{parts}
    \end{multicols}



    %% 4º ESO Editorial SM.
    \question La relación entre el número de decibelios, n, y la intensidad del sonido, I, en voltios por centímetro cuadrado $(V/cm^2)$ viene dada por:
    \begin{equation*}
    n = 10 \cdot log \left(\frac{I}{10^{-16}}\right)
    \end{equation*}

    \begin{parts}
        \part Determina la intensidad de un sonido si el nivel de decibelios es 125.
        \part Si la intensidad del sonido se hace 10 veces mayor, ¿cómo aumenta el número de decibelios?
        \part Si el número de decibelios aumenta el doble, ¿qué relación hay entre las intensidades?
    \end{parts}

\end{questions}

\end{document}