\documentclass[addpoints,spanish, 12pt,a4paper,cancelspace]{./include/gexercises}

 %%%%%%%%%%%%%%%%%%%%%%%%%%%
\renewcommand{\documentName} { Proporcionalidad }
\renewcommand{\documentContent} { Proporcionalidad compuesta } 
\renewcommand{\waterMark} {  } 

% Configuración del documento.
\renewcommand{\schoolSubject} { Examen Matemáticas 2º ESO  }
\renewcommand{\school} { IES José de Churriguera  }
\renewcommand{\academicPeriod} { Curso 2022/2023 }

\renewcommand{\autor} { Andrés Giménez Muñoz }
\renewcommand{\emailAuthor} { andresprofemates@outlook.es }
\renewcommand{\autorSing}{ Profesor: Andrés } 
%%%%%%%%%%%%%%%%%%%%%%%%%%%

% \renewcommand{\thepartno}{\arabic{partno}}

%%%%%%%%%%%%%%%%%%%%%%%%%%%
% Exam configuration
%\pointsdroppedatright   %% No mostrar la puntuación
\pointsinrightmargin{} % Para poner las puntuaciones a la derecha. Se puede cambiar. Si se comenta, sale a la izquierda.
\extrawidth{-1.5cm} %Un poquito más de margen por si ponemos textos largos.
\marginpointname{ \emph{\points}}

%% Si se comenta no aparecerán los espacios de la solución.
%\nocancelspace

%% Esto es de la clase exam. Si dejamos sin comentar \printanswers, se mostraran las soluciones. 
%% Si la comentamos y dejamos sin comentar \noprintanswers, pues no se muestran las soluciones.
% \printanswers
%\noprintanswers

%%%%%%%%%%%%%%%%%%%%%%%%%%%

\begin{document}

% \StudentData{}
% \GradeTableHeader{}

\justifying{}

\begin{questions}
    \question
		Una receta para cocinar trufas de chocolate, nos dice que para 30 unidades se requieren 200 gramos de chocolate negro de buena calidad, 120 gramos de nata líquida para montar,
		media vaina de vainilla y 30 gramos de mantequilla. ¿Qué cantidad de ingredientes necesitaremos para cocinar 1400 unidades?

      \question
      Para transportar 640 kilogramos de mercancía a 87 kilómetro de distancia,
      se han gastado 2.700\euro. 
      ¿Cuánto costará transportar 1.218 kilogramos de la misma mercancía a 320 kilómetros?

      \question
      Un hombre que camina durante 7 días a razón de 8 horas diarias, ha recorrido 225 kilómetros. 
      ¿Cuántos habrá recorrido otro que camina 12 días durante 7 horas diarias?      

      \question 
      Nueve grifos abiertos durante 40 horas han consumido 200 litros de agua
      ¿Cuántos litros consumen 15 grifos durante 9 horas?

      \question Cuatro empleados de una tienda de moda tardan 8 días en coser 6 vestidos.
      ¿Cuánto tiempo tardarían en coser 24 vestidos si se duplica la plantilla?

      \question Cuatro agricultores recolectan 10.000 Kg de cerezas en 9 días. 
      ¿Cuántos Kilos recolectarán seis agricultores en 15 días?

      \question Cinco trabajadores tardan 16 días en construir una pequeña caseta de aperos trabajando 6 horas diarias.
      ¿Cuántos trabajadores serán necesarios para construir dicha casita en 10 días si trabajan 8 horas diarias?

      \question En 8 días, 6 máquinas cavan una zanja de 2.100 metros de largo. 
      ¿Cuántas máquinas serán necesarias para cavar 525 m trabajando durante 3 días?

      \question Para pavimentar 2 km de carretera 50 trabajadores han empleado 20 días trabajando 8 horas diarias. 
      ¿Cuántos días tardarán 100 trabajadores trabajando 10 horas al día en construir 6 km más de carretera?

      \question Una estufa de 4 quemadores ha consumido 50 € de gas al estar encendidos 2 de ellos durante 3 horas. 
      ¿Cuál es el precio del gas consumido si se encienden los 4 quemadores durante el mismo tiempo?

      \question 4 coches llevan a 16 personas en un recorrido de 120 km en 90 minutos.
      ¿Cuántos coches se necesitan para transportar a 58 personas en el mismo recorrido y en el mismo tiempo?

      \question 6 elefantes consumen 345 kilos de heno en una semana, ¿Cuál es el consumo de 8 elefantes en 10 días?

      \question Para pavimentar 180 metros de pista, 18 obreros tardan 21 días. 
      ¿Cuántos días se necesitarán para pavimentar 120 metros de la misma pista con 4 obreros menos?

      \question Si 16 obreros, trabajando 9 horas diarias en 12 días, hacen 60 sillas. 
      ¿Cuántos días necesitarán 40 obreros trabajando una hora diaria menos para hacer un ciento de las mismas sillas?

      \question Doce obreros trabajando 15 días de 8 horas diarias pueden construir 160 metros de un muro. 
      ¿Cuántos días se demorarán 10 obreros trabajando 10 horas diarias para construir 200 metros del mismo muro?

      \question Si 180 hombres en 6 días; trabajando 10 horas cada día pueden hacer una zanja de 200 metros de largo, 3 metros de ancho y 2 metros de profundidad. 
      ¿En cuántos días de 8 horas, harían 100 hombres una zanja de 400 metros de largo, 4 metros de ancho y 3 metros de profundidad?

      \question Si un grifo, dando por minuto 100 litros de agua, llena en 8 horas un pozo; cinco grifos, dando cada uno 40 litros por minuto. 
      ¿En cuántas horas llenará un pozo 6 veces el anterior?

      \question 15 obreros han hecho la mitad de un trabajo en 20 días. En ese momento abandonan el trabajo 5 obreros. 
      ¿Cuántos días tardarán en terminar el trabajo los obreros que quedan?

      \question Un móvil aumenta su velocidad en $\sfrac{1}{3}$. 
      ¿Cuántas horas diarias debe estar en movimiento para recorrer en 4 días, la distancia cubierta en 6 días a razón de 8 horas diarias?

      \question Si 7 psicólogos pueden tomar 1800 test en 4 horas 30 minutos.
      ¿Cuántos psicólogos lograrán tomar 3200 test en 3 horas 40 minutos, cuya dificultad sea de 0,375 mayor respecto a los primeros?

      \question 10 sastres trabajando 8 horas diarias durante 10 días, confeccionan 800 trajes. 
      ¿Cuántos sastres de igual rendimiento lograrán confeccionar 600 trajes trabajando 2 horas diarias durante 12 días?

      \question Una familia de 10 personas cuenta con 15.120\euro{} para vivir 8 meses en una ciudad. 
      A los 3 meses, mueren 4 de sus integrantes y el costo de vida aumenta en $\sfrac{1}{5}$. 
      ¿Cuánto dinero sobrará después de los 8 meses?

      \question
      Si una mujer tiene un niño en 9 meses. ¿Cuánto tardarán 9 mujeres en tener un niño?
      \\
      \small Nota: Obviamente este ejercicio no hay que resolverlo. Este es un ejemplo que se pone en las escuelas de negocio para hacer reflexionar que hay problemas que no se resuelven con una regla de tres.  

\end{questions}

\end{document}