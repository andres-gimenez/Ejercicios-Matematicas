\documentclass[addpoints,spanish, 12pt,a4paper,cancelspace]{./include/gexam}

%%%%%%%%%%%%%%%%%%%%%%%%%%%
\newcommand{\documentName} { Porcentajes }
\newcommand{\documentContent} { Problemas con porcentajes } 
\newcommand{\waterMark} {  } 
%%%%%%%%%%%%%%%%%%%%%%%%%%%

% Configuración del documento.
\renewcommand{\schoolSubject} { Examen Matemáticas 2º ESO  }
\renewcommand{\school} { IES José de Churriguera  }
\renewcommand{\academicPeriod} { Curso 2022/2023 }

\renewcommand{\autor} { Andrés Giménez Muñoz }
\renewcommand{\emailAuthor} { andresprofemates@outlook.es }
\renewcommand{\autorSing}{ Profesor: Andrés } 

% \renewcommand{\thepartno}{\arabic{partno}}

\usepackage{amsthm}
\theoremstyle{definition}
\newtheorem*{theorem}{Theorem}
\newtheorem{definition}{Definición}
\newtheorem{property}{Propiedad}

%%%%%%%%%%%%%%%%%%%%%%%%%%%
% Exam configuration
%\pointsdroppedatright   %% No mostrar la puntuación
\pointsinrightmargin{} % Para poner las puntuaciones a la derecha. Se puede cambiar. Si se comenta, sale a la izquierda.
\extrawidth{-1.5cm} %Un poquito más de margen por si ponemos textos largos.
\marginpointname{ \emph{\points}}

%% Si se comenta no aparecerán los espacios de la solución.
%\nocancelspace

%% Esto es de la clase exam. Si dejamos sin comentar \printanswers, se mostraran las soluciones. 
%% Si la comentamos y dejamos sin comentar \noprintanswers, pues no se muestran las soluciones.
% \printanswers
%\noprintanswers

%%%%%%%%%%%%%%%%%%%%%%%%%%%

\begin{document}

% \StudentData{}
% \GradeTableHeader{}

\justifying{}

\begin{questions}
    \question Tres jinetes disputan una carrera invirtiendo para ello $\frac{7}{5}$ de hora, $\frac{20}{12}$ horas y $\frac{16}{9}$ horas, respectivamente.
    ¿Cuál de ellos es más veloz?

    \question La familia García ha invertido la cuarta parte de su presupuesto para vacaciones en los billetes de avión;
    la tercera parte, en el hotel; y el resto, que son 600\euro{}, en gastos varios. ¿A cuánto asciende el presupuesto?

    \question Un contribuyente paga al principio del año la mitad de sus impuestos; al cabo de seis meses, la tercera parte de ellos, y al final del año el resto.
    ¿Qué parte de los impuestos paga al final del año?
    Suponiendo que tiene que pagar 1.440\euro{}, ¿qué cantidad ha pagado en cada uno de los tres plazos?

\end{questions}

\end{document}