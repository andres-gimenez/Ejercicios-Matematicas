 %%%%%%%%%%%%%%%%%%%%%%%%%%%
\newcommand{\documentName} { Ecuaciones }
\newcommand{\documentContent} { \phantom{Ecuaciones} } 
\newcommand{\waterMark} {  } 
%%%%%%%%%%%%%%%%%%%%%%%%%%%

% Configuración del documento.
\newcommand{\schoolSubject} { Matemáticas 3º ESO - Recuperación}
\newcommand{\school} { IES La Serna }
\newcommand{\academicPeriod} { Curso 2020/2021 }


\newcommand{\autor} { Andrés Giménez Muñoz }
\newcommand{\emailAuthor} { agimenezmunoz@ieslaserna.com }
\newcommand{\autorSing}{ Profesores: Andrés } 
\renewcommand{\schoolSubject} { Examen Matemáticas 2º ESO  }
\renewcommand{\school} { IES José de Churriguera  }
\renewcommand{\academicPeriod} { Curso 2022/2023 }

\renewcommand{\autor} { Andrés Giménez Muñoz }
\renewcommand{\emailAuthor} { andresprofemates@outlook.es }
\renewcommand{\autorSing}{ Profesor: Andrés } 

% \renewcommand{\thepartno}{\arabic{partno}}

\usepackage{yhmath}

\usepackage{amsthm}
\theoremstyle{definition}
\newtheorem*{theorem}{Theorem}
\newtheorem{definition}{Definición}
\newtheorem{property}{Propiedad}

%%%%%%%%%%%%%%%%%%%%%%%%%%%
% Exam configuration
%\pointsdroppedatright   %% No mostrar la puntuación
\pointsinrightmargin{} % Para poner las puntuaciones a la derecha. Se puede cambiar. Si se comenta, sale a la izquierda.
\extrawidth{-1.5cm} %Un poquito más de margen por si ponemos textos largos.
\marginpointname{ \emph{\points}}

%% Si se comenta no aparecerán los espacios de la solución.
%\nocancelspace

%% Esto es de la clase exam. Si dejamos sin comentar \printanswers, se mostraran las soluciones. 
%% Si la comentamos y dejamos sin comentar \noprintanswers, pues no se muestran las soluciones.
% \printanswers
%\noprintanswers

%%%%%%%%%%%%%%%%%%%%%%%%%%%

\begin{document}

% \StudentData{}
% \GradeTableHeader{}

\justifying{}

\begin{questions}
    \question Calcula el valor de las siguientes expresiones:
    \begin{parts}
        \part El doble de cuatro.
        \part El triple de cinco.
        \part El doble de la mitad de tres.
        \part La mitad de diez menos dos.
        \part La mitad de diez, menos dos.
        \part El triple de cinco más cuatro.
        \part El triple de cinco, más cuatro.
    \end{parts}

    \question Escribe las siguientes expresiones en notación algebraica:
    \begin{parts}
        \part El doble de un número.
        \part La mitad de un número.
        \part El doble de un número más su mitad.
        \part La mitad de un número menos dos.
        \part La mitad de un número, menos dos.
        \part El triple de la mitad de un número.
        \part La mitad del triple de un número.
        \part El siguiente número uno dado.
        \part El triple de un número al sumarle siete.

        \part Un número par.
        \part Un número impar.

    \end{parts}

    \question Calcula las siguientes operaciones convidadas
    \begin{parts}
        \begin{multicols}{2}
            \part $12- 2 \cdot 5$
            \part $2 + 6 \cdot \left(13 - 2 \cdot 5 \right)$
            \part $5 \cdot 3 - 2 \cdot 6$
            \part $\left(14 - 9\right) \cdot 3 - \left(22 - 5 \cdot 4\right) \cdot 6$
            \part $3 \cdot 2^3 - 7 + 1$
            \part $3 \cdot 2^3 - \left(7 + 1\right)$
            \part $3 \cdot \left(2^3 - 7\right) + 1$
            \part $3 \cdot \left(2^3 - 7 + 1\right)$
            \part $14-2\cdot \left(5^2 - 3 \cdot 6\right)$
            \part $35 - 2 \cdot 4^2 - \left(2^3 - 10:2\right)$
            \part $\left(6^2 : 4 + 2\right)-\left(6^2 - 5^2\right)$
            \part $\left[\left(1 - 4\right)-\left(5-3\right)-\left(-6\right)\right]\cdot\left[-3+\left(-7\right)\right]$
            \part $-4\left(3-8\right)-\left[4\cdot\left(-5\right)\right]\cdot\left[\left(-3\right)\cdot11\right]$
            \part $15 + 2 \cdot \left[8 - 3\cdot5\right]$
            \part $\left(-3\right) \cdot \left(+5\right) - 3 \cdot \left[11+3\cdot\left(5-11\right)\right]$
            \part $28:\left(-7\right)-\left(-6\right)\cdot \left[23-5\cdot\left(9-4\right)\right]$
        \end{multicols}
    \end{parts}


    \question Resuelve las siguientes ecuaciones:
    \begin{parts}
        \begin{multicols}{2}
            \part $x-3=5$
            \part $x-6=3$
            \part $2x+3=13$
            \part $x-6=3x$
            \part $x+6=9$
            \part $x+9=3$

            \part $x-3=3-x$
            \part $5x-9=4x-20$
            \part $2x-7=x+5$
            \part $\frac{x}{5}=-30$
            \part $2x+6=3x-1$


            \part $3x + 1 = 3 - \left(2- 2x\right)$
            \part $\frac{3x}{2} + \frac{2x}{3} = \frac{1 + 3x}{2} $

            \part $2 \left(2+x\right)-\left(6-7x\right) = 13x - \left(1 + 4x \right)$
            
            \part $3 \left(x+1\right)-2x=x-\left(2+3\left(3-x\right)\right)$

            \part $1-2\left(1+3x-2 \left(x +2\right)+3x\right) = -1$
            
            
        \end{multicols}
    \end{parts}

    \question Resuelve las siguientes ecuaciones
    \begin{parts}
        \part $5 \left(x+1\right)-\left(1-x\right)= 2 \left(x-1\right)-4 \left(1-x\right)$
        \part $2- \left(3 - 2 \left(x+1\right)\right)=3x + 2 \left(x - \left(3+2x\right)\right)$
        \part $x + \frac{1}{2} \left(x - 3  \frac{1}{2} \left(4 -3x\right)\right) = \frac{2}{3} \left(1 - \frac{5x}{2}\right)$ 
    \end{parts}

\end{questions}

\end{document}