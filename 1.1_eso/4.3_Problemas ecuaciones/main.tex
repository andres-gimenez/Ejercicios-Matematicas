\documentclass[addpoints,spanish, 12pt,a4paper,cancelspace]{./include/gexercises}

%%%%%%%%%%%%%%%%%%%%%%%%%%%
\renewcommand{\documentName} { Expresiones algebraicas}
\renewcommand{\documentContent} { Problemas con ecuaciones de 1º grado } 
\renewcommand{\waterMark} {  } 

% Configuración del documento.
\renewcommand{\schoolSubject} { Examen Matemáticas 2º ESO  }
\renewcommand{\school} { IES José de Churriguera  }
\renewcommand{\academicPeriod} { Curso 2022/2023 }

\renewcommand{\autor} { Andrés Giménez Muñoz }
\renewcommand{\emailAuthor} { andresprofemates@outlook.es }
\renewcommand{\autorSing}{ Profesor: Andrés } 
%%%%%%%%%%%%%%%%%%%%%%%%%%%

\renewcommand{\thepartno}{\arabic{partno}}

\usepackage{yhmath}

\usepackage{amsthm}
\theoremstyle{definition}
\newtheorem*{theorem}{Theorem}
\newtheorem{definition}{Definición}
\newtheorem{property}{Propiedad}

%%%%%%%%%%%%%%%%%%%%%%%%%%%
% Exam configuration
%\pointsdroppedatright   %% No mostrar la puntuación
\pointsinrightmargin{} % Para poner las puntuaciones a la derecha. Se puede cambiar. Si se comenta, sale a la izquierda.
\extrawidth{-1.5cm} %Un poquito más de margen por si ponemos textos largos.
\marginpointname{ \emph{\points}}

%% Si se comenta no aparecerán los espacios de la solución.
%\nocancelspace

%% Esto es de la clase exam. Si dejamos sin comentar \printanswers, se mostraran las soluciones. 
%% Si la comentamos y dejamos sin comentar \noprintanswers, pues no se muestran las soluciones.
\printanswers
%\noprintanswers

%%%%%%%%%%%%%%%%%%%%%%%%%%%

\begin{document}

% \StudentData{}
% \GradeTableHeader{}

\justifying{}

\begin{questions}
    \question Hallar tres números consecutivos cuya suma sea 219.

    \question Héctor guarda 25 euros en su hucha, que supone sumar una cuarta parte del dinero que ya había. ¿Cuánto dinero hay en la hucha?

    \question El padre de Ana tiene 5 años menos que su madre y la mitad de la edad de la madre es 23. ¿Qué edad tiene el padre de Ana?

    \question Dado un número, la suma de su mitad, su doble y su triple es 55. ¿Qué número es?

    \question Vicente se gasta 20 euros en un pantalón y una camisa. No sabe el precio de cada prenda, pero sí sabe que la camisa vale dos quintas partes de lo que vale el pantalón. ¿Cuánto vale el pantalón?

    \question Carmen tiene 16 años y sus dos hermanos pequeños tienen 2 y 3 años. ¿Cuántos años han de pasar para que el doble de la suma de las edades de los hermanos de Carmen sea la misma que la que tiene ella?

    \question Tenemos tres peceras y 56 peces. Los tamaños de las peceras son pequeño, mediano y grande, 
    siendo la pequeña la mitad de la mediana y la grande el doble. Como no tenemos ninguna preferencia en cuanto al reparto de los peces, 
    decidimos que en cada una de ellas haya una cantidad de peces proporcional al tamaño de cada pecera. 
    ¿Cuántos peces pondremos en cada pecera?

    \question Queremos repartir 510 caramelos entre un grupo de 3 niños, 
    de tal forma que dos de ellos tengan la mitad de los caramelos pero que uno de estos dos tenga la mitad de caramelos que el otro. 
    ¿Cuántos caramelos tendrá cada niño?

    \question La tercera parte de las cucharas de la casa estaban en el lavaplatos y las restantes en el cajón. Pero la mitad de las cucharas del cajón, 15, se llevan a la mesa. 
    ¿Cuántas cucharas hay en el lavaplatos?

    \question Una tienda vende en dos días la tercera parte de sus productos. 
    Al día siguiente recibe del almacén la mitad de la cantidad de los productos vendidos, que son 15 unidades. 
    ¿Cuántas unidades vendió en los dos primeros días? ¿Cuántas unidades hay en la tienda después de abastecerla?

    \question Juan tiene 400 euros y Rosa tiene 350. Ambos se compran el mismo libro. 
    Después de la compra, a Rosa le quedan cinco sextas partes del dinero que le queda a Juan.

    \question Ester tiene el triple de dinero que Ana y la mitad que Héctor. Héctor les da a Ana y a Ester 25 euros a cada una. 
    Ahora Ester tiene la misma cantidad que Héctor. 
    ¿Cuánto dinero tenía cada uno al principio? ¿Y después?

    \question En una casa, el depósito de agua se encuentra al 2/7 de su capacidad.
     Se duchan tres personas: el primero en ducharse consume una quinta parte de la cantidad del depósito; 
     el segundo, una tercera parte de la cantidad que queda; 
     y el tercero, tres cuartas partes de la cantidad del primero.
    \\
    ¿Cuál es la capacidad del depósito y la cantidad de agua que consumen los dos primeros si sabemos que el tercero consume 10 litros al ducharse?
    Calcular el precio del libro.

\end{questions}

\end{document}