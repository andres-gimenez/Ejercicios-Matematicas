\documentclass[addpoints,spanish, 12pt,a4paper,cancelspace]{./include/gexercises}

%%%%%%%%%%%%%%%%%%%%%%%%%%%
\renewcommand{\documentName} { Operaciones combinadas }
\renewcommand{\documentContent} { Operaciones combinadas con números enteros } 
\renewcommand{\waterMark} {  } 

% Configuración del documento.
\renewcommand{\schoolSubject} { Examen Matemáticas 2º ESO  }
\renewcommand{\school} { IES José de Churriguera  }
\renewcommand{\academicPeriod} { Curso 2022/2023 }

\renewcommand{\autor} { Andrés Giménez Muñoz }
\renewcommand{\emailAuthor} { andresprofemates@outlook.es }
\renewcommand{\autorSing}{ Profesor: Andrés } 
%%%%%%%%%%%%%%%%%%%%%%%%%%%

\renewcommand{\thepartno}{\arabic{partno}}

%%%%%%%%%%%%%%%%%%%%%%%%%%%
% Exam configuration
%\pointsdroppedatright   %% No mostrar la puntuación
\pointsinrightmargin{} % Para poner las puntuaciones a la derecha. Se puede cambiar. Si se comenta, sale a la izquierda.
\extrawidth{-1.5cm} %Un poquito más de margen por si ponemos textos largos.
\marginpointname{ \emph{\points}}

%% Si se comenta no aparecerán los espacios de la solución.
%\nocancelspace

%% Esto es de la clase exam. Si dejamos sin comentar \printanswers, se mostraran las soluciones. 
%% Si la comentamos y dejamos sin comentar \noprintanswers, pues no se muestran las soluciones.
% \printanswers
%\noprintanswers

%%%%%%%%%%%%%%%%%%%%%%%%%%%

\begin{document}

% \StudentData{}
% \GradeTableHeader{}

\justifying{}

\begin{questions}
    \question Realiza las siguientes operaciones combinadas:
    \begin{parts}
        \part $\frac{75}{30} - \frac{10}{6}$
        \part $\frac{1}{2}\cdot \frac{6}{5}+\frac{7}{5}:\frac{4}{3}$
        \part $\left(\frac{3}{5}- \frac{1}{4}\right):\frac{7}{12} + \left(\frac{1}{6}-\frac{1}{4}\right):\frac{5}{6}$
        \part $2-\left(\frac{1}{6}+\frac{1}{2}\right) - 3:\left(1 +\frac{1}{2}\right)$
    \end{parts}
    
    \question Realiza las siguientes operaciones combinadas con potencias y raíces:
    \begin{parts}
        \part $\left(\frac{4}{5}\right)^3$
    \end{parts}

\end{questions}

\end{document}