 %%%%%%%%%%%%%%%%%%%%%%%%%%%
 \newcommand{\documentName} { 1ª evaluación }
 \newcommand{\documentContent} {Expresiones combinadas y divisibilidad} 
 \newcommand{\waterMark} { Modelo B } 
 %%%%%%%%%%%%%%%%%%%%%%%%%%%
 
 % Configuración del documento.
 \newcommand{\schoolSubject} { Matemáticas 3º ESO - Recuperación}
\newcommand{\school} { IES La Serna }
\newcommand{\academicPeriod} { Curso 2020/2021 }


\newcommand{\autor} { Andrés Giménez Muñoz }
\newcommand{\emailAuthor} { agimenezmunoz@ieslaserna.com }
\newcommand{\autorSing}{ Profesores: Andrés } 
 \renewcommand{\schoolSubject} { Examen Matemáticas 2º ESO  }
\renewcommand{\school} { IES José de Churriguera  }
\renewcommand{\academicPeriod} { Curso 2022/2023 }

\renewcommand{\autor} { Andrés Giménez Muñoz }
\renewcommand{\emailAuthor} { andresprofemates@outlook.es }
\renewcommand{\autorSing}{ Profesor: Andrés } 
 
 % \renewcommand{\thepartno}{\arabic{partno}}
 \usepackage{xfrac}
 
 %%%%%%%%%%%%%%%%%%%%%%%%%%%
 % Exam configuration
 %\pointsdroppedatright   %% No mostrar la puntuación
 \pointsinrightmargin{} % Para poner las puntuaciones a la derecha. Se puede cambiar. Si se comenta, sale a la izquierda.
 \extrawidth{-1.5cm} %Un poquito más de margen por si ponemos textos largos.
 \marginpointname{ \emph{\points}}
 
 %% Si se comenta no aparecerán los espacios de la solución.
 %\nocancelspace
 
 %% Esto es de la clase exam. Si dejamos sin comentar \printanswers, se mostraran las soluciones. 
 %% Si la comentamos y dejamos sin comentar \noprintanswers, pues no se muestran las soluciones.
 % \printanswers
 %\noprintanswers
 
 %%%%%%%%%%%%%%%%%%%%%%%%%%%
 
 \begin{document}
 
 \StudentData{}
 \GradeTableHeader{}
 
 \justifying{}
 
 \begin{questions}
    \question[2]
    Resuelve las siguientes operaciones combinadas
    \begin{parts}
      \part
      $3+5 \cdot \left(-2\right) - 3 +12:\left(-2\right)$
      \vspace{\stretch{5}}
      \part
      $\left( - 6 -2 \right) \cdot \left[-4 + 2 : \left(-5 + 6\right) \right] $
      \vspace{\stretch{5}}
    \end{parts}

    \question[2] Completa
    \begin{parts}
      \part Todos los divisores de 15 = $\left\{ \phantom{--------------------} \right\}$
      \vspace{\stretch{1}}
      \part Todos los divisores de 28 = $\left\{ \phantom{--------------------} \right\}$
      \vspace{\stretch{1}}
      \part 5 primeros múltiplos de 2 = $\left\{ \phantom{--------------------} \right\}$
      \vspace{\stretch{1}}
      \part 5 primeros múltiplos de 5 = $\left\{ \phantom{--------------------} \right\}$
      \vspace{\stretch{1}}
    \end{parts}

    \newpage{}

    \question[2] Calcula el máximo común divisor y el mínimo como un múltiplo de las siguientes cantidades:
    \begin{parts}
      \part 3, 5 y 15
      \vspace{\stretch{1}}
      \part 5 y 50 
      \vspace{\stretch{1}}
      \part 10, 40 y 80
      \vspace{\stretch{1}}
      \part 14, 21 y 35
      \vspace{\stretch{1}}
    \end{parts}

    \question[2]
    Mari Carmen y Juan Carlos van a visitar a su abuela cada cierto tiempo.
    Mari Carmen va cada 16 días y Juan Carlos cada 40.
    ¿Cuándo volverán a encontrarse si hoy se han visto allí?

    \vspace{\stretch{5}}

    \newpage{}

    \question[2] Tenemos 40 lápices de colores y 16 rotuladores.
     Queremos repartirlos en los estuches de manera que contengan el mismo número de lápices o de rotuladores y, 
     además, el mayor número posible de ellos.
     \\ \\
     Calcula el número de lápices que repartiremos en cada estuche.
     \vspace{\stretch{1}}
     
 \end{questions}
 
 \end{document}