\documentclass[addpoints,spanish, 12pt,a4paper,cancelspace]{./include/gexam}

 %%%%%%%%%%%%%%%%%%%%%%%%%%%
 \renewcommand{\documentName} { 1ª evaluación }
 \renewcommand{\documentContent} {Expresiones combinadas y divisibilidad} 
 \renewcommand{\waterMark} {  } 
 % Configuración del documento.
 \renewcommand{\schoolSubject} { Examen Matemáticas 2º ESO  }
\renewcommand{\school} { IES José de Churriguera  }
\renewcommand{\academicPeriod} { Curso 2022/2023 }

\renewcommand{\autor} { Andrés Giménez Muñoz }
\renewcommand{\emailAuthor} { andresprofemates@outlook.es }
\renewcommand{\autorSing}{ Profesor: Andrés } 
 %%%%%%%%%%%%%%%%%%%%%%%%%%%
  
 % \renewcommand{\thepartno}{\arabic{partno}}
  
 %%%%%%%%%%%%%%%%%%%%%%%%%%%
 % Exam configuration
 %\pointsdroppedatright   %% No mostrar la puntuación
 \pointsinrightmargin{} % Para poner las puntuaciones a la derecha. Se puede cambiar. Si se comenta, sale a la izquierda.
 \extrawidth{-1.5cm} %Un poquito más de margen por si ponemos textos largos.
 \marginpointname{ \emph{\points}}
 
 %% Si se comenta no aparecerán los espacios de la solución.
 %\nocancelspace
 
 %% Esto es de la clase exam. Si dejamos sin comentar \printanswers, se mostraran las soluciones. 
 %% Si la comentamos y dejamos sin comentar \noprintanswers, pues no se muestran las soluciones.
 % \printanswers
 %\noprintanswers
 
 %%%%%%%%%%%%%%%%%%%%%%%%%%%
 
 \begin{document}
 
 \StudentData{}
 \GradeTableHeader{}
 
 \justifying{}
 
 \begin{questions}
    \question[2]
    Resuelve las siguientes operaciones combinadas
    \begin{parts}
      \part
      $3+5 \cdot \left(-2\right) - 4 +12:\left(-6\right)$
      \vspace{\stretch{5}}
      \part
      $\left( -6 -3 \right) \cdot \left[-4 + 2 : \left(-8 + 7\right) \right] $
      \vspace{\stretch{5}}
    \end{parts}

    \question[2] Completa
    \begin{parts}
      \part Todos los divisores de 12 = $\left\{ \phantom{--------------------} \right\}$
      \vspace{\stretch{1}}
      \part Todos los divisores de 30 = $\left\{ \phantom{--------------------} \right\}$
      \vspace{\stretch{1}}
      \part 5 primeros múltiplos de 3 = $\left\{ \phantom{--------------------} \right\}$
      \vspace{\stretch{1}}
      \part 5 primeros múltiplos de 6 = $\left\{ \phantom{--------------------} \right\}$
      \vspace{\stretch{1}}
    \end{parts}

    \newpage{}

    \question[2] Calcula el máximo común divisor y el mínimo como un múltiplo de las siguientes cantidades:
    \begin{parts}
      \part 3, 5 y 7
      \vspace{\stretch{1}}
      \part 10 y 100 
      \vspace{\stretch{1}}
      \part 20, 30 y 90
      \vspace{\stretch{1}}
      \part 33, 77 y 231
      \vspace{\stretch{1}}
    \end{parts}

    \question[2]
    A Eva le encantan las manualidades y esta tarde ha decidido hacer pulseras con su madre. 
    Quiere adornarlas con perlas. Si tiene 24 perlas blancas y 36 azules, 
    y quiere hacer el máximo número de pulseras posibles de manera que haya la misma cantidad de cada color de perla en cada una de ellas 
    ¿cuántas pulseras podrá hacer como máximo?

    \vspace{\stretch{5}}

    \newpage{}

    \question[2] En la estación de trenes de cercanías de Leganes, 
    salen un tren cada 20 minutos en dirección Humanes y un tren cada 24 minutos en dirección Madrid.
    \begin{parts}
      \part ¿Cada cuánto tiempo ha de estar más atento el jefe de estación, porque coinciden la salida de los dos trenes a la vez?
      \vspace{\stretch{1}}

      \part Le han pedido al jefe de estación que permita cruzar las vías del tren a peatones a una hora determinada. 
      Si los dos primeros trenes por cada sentido salen a las 8 en punto de la mañana, 
      ¿a qué hora debería permitir el paso? ¿a las 12:00 o a las 12:30?.\\
      \footnotesize{(*)Justifica tu respuesta en función de múltiplos o divisores.}
      \vspace{\stretch{1}}
    \end{parts}
 \end{questions}
 
 \end{document}