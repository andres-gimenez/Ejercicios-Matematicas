 %%%%%%%%%%%%%%%%%%%%%%%%%%%
 \newcommand{\documentName} { 1ª evaluación }
 \newcommand{\documentContent} { Números decimales} 
 \newcommand{\waterMark} {  } 
 %%%%%%%%%%%%%%%%%%%%%%%%%%%
 
 % Configuración del documento.
 \newcommand{\schoolSubject} { Matemáticas 3º ESO - Recuperación}
\newcommand{\school} { IES La Serna }
\newcommand{\academicPeriod} { Curso 2020/2021 }


\newcommand{\autor} { Andrés Giménez Muñoz }
\newcommand{\emailAuthor} { agimenezmunoz@ieslaserna.com }
\newcommand{\autorSing}{ Profesores: Andrés } 
 \renewcommand{\schoolSubject} { Examen Matemáticas 2º ESO  }
\renewcommand{\school} { IES José de Churriguera  }
\renewcommand{\academicPeriod} { Curso 2022/2023 }

\renewcommand{\autor} { Andrés Giménez Muñoz }
\renewcommand{\emailAuthor} { andresprofemates@outlook.es }
\renewcommand{\autorSing}{ Profesor: Andrés } 
 
 % \renewcommand{\thepartno}{\arabic{partno}}
 \usepackage{xfrac}
 \usepackage{yhmath}
 
 %%%%%%%%%%%%%%%%%%%%%%%%%%%
 % Exam configuration
 %\pointsdroppedatright   %% No mostrar la puntuación
 \pointsinrightmargin{} % Para poner las puntuaciones a la derecha. Se puede cambiar. Si se comenta, sale a la izquierda.
 \extrawidth{-1.5cm} %Un poquito más de margen por si ponemos textos largos.
 \marginpointname{ \emph{\points}}
 
 %% Si se comenta no aparecerán los espacios de la solución.
 %\nocancelspace
 
 %% Esto es de la clase exam. Si dejamos sin comentar \printanswers, se mostraran las soluciones. 
 %% Si la comentamos y dejamos sin comentar \noprintanswers, pues no se muestran las soluciones.
 % \printanswers
 %\noprintanswers
 
 %%%%%%%%%%%%%%%%%%%%%%%%%%%
 
 \begin{document}
 
 \StudentData{}
 \GradeTableHeader{}
 
 \justifying{}
 
 \begin{questions}
    \question[1] Indica si las siguientes medidas son exactas o aproximadas
    \begin{parts}
        \part La superficie de la Península Ibérica es de 583.254 $km^2$.
        \vspace{\stretch{1}}
        \part En el instituto hay unos 1.000 alumnos.
        \vspace{\stretch{1}}
        \part La manga de la camisa, desde la axila al puño, mide 55,3 cm.
        \vspace{\stretch{1}}
        \part En la manifestación hubo 30.000 asistentes.
        \vspace{\stretch{1}}
    \end{parts}

    \question[1] Redondea las siguientes cantidades a la centésima:
    \begin{parts}
        \part $4,23423$
        \vspace{\stretch{1}}
        \part $5,549$
        \vspace{\stretch{1}}
        \part $934,39422$
        \vspace{\stretch{1}}
        \part $9902,85832$
        \vspace{\stretch{1}}
    \end{parts}

    \newpage

    \question[1] Trunca las siguientes cantidades a la milésima:
    \begin{parts}
        \part $4,23423$
        \vspace{\stretch{1}}
        \part $5,54998$
        \vspace{\stretch{1}}
        \part $934,39482$
        \vspace{\stretch{1}}
        \part $9902,85832$
        \vspace{\stretch{1}}
    \end{parts}

    % \question[1] Expresa en forma de fracción las siguientes cantidades:
    % \begin{parts}
    %     \part $4,23$
    %     \vspace{\stretch{1}}
    %     \part $5,01$
    %     \vspace{\stretch{1}}
    %     \part $0,\wideparen{3}$
    %     \vspace{\stretch{1}}
    % \end{parts}

    \question[2] Expresa estos números en notación científica:
    \begin{parts}
        \part $23400000000$
        \vspace{\stretch{1}}
        \part $0,0000325$
        \vspace{\stretch{1}}
        \part Masa de la tierra $5.974.000.000.000.000.000.000.000$ kg.
        \vspace{\stretch{1}}
        \part Masa del protón $0,000 000 000 000 000 000 000 000 000 000 91$ kg.
        \vspace{\stretch{1}}
    \end{parts}

    \newpage{}

    \question[1] Realiza las siguientes operaciones y exprésalas en notación científica:
    \begin{parts}
        \part $3,23 \cdot 10^5 + 2,8 \cdot 10^4$
        \vspace{\stretch{2}}

        \part $3,23 \cdot 10^{-4} - 2,8 \cdot 10^{-3}$
        \vspace{\stretch{2}}

        \part $2,18 \cdot 10^4 \cdot 4,7 \cdot 10^3$
        \vspace{\stretch{2}}
    \end{parts}

    \newpage

    \question[2] En una carnicería compran una res entera que pesa $450 kg$, 
    si de esta solo se pueden vender $300 kg$, 
    calcula la merma que se ha producido en valor absoluto y relativo.
    \vspace{\stretch{1}}
    
    \question[2] Un fabricante de discos de frenos especifica que sus productos tienen $90 mm$ de espesor, 
    pero cuando nosotros hacemos la medición obtenemos la medida de $88,4 mm$.
    Calcula el error absoluto y relativo que comete el fabricante al fabricar la pieza. 
    \vspace{\stretch{1}}

 \end{questions}
 
 \end{document}