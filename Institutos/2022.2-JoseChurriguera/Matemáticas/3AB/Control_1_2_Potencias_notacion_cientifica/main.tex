 %%%%%%%%%%%%%%%%%%%%%%%%%%%
 \newcommand{\documentName} { 1ª evaluación }
 \newcommand{\documentContent} { Potencias, raíces y notación científica} 
 \newcommand{\waterMark} {  } 
 %%%%%%%%%%%%%%%%%%%%%%%%%%%
 
 % Configuración del documento.
 \newcommand{\schoolSubject} { Matemáticas 3º ESO - Recuperación}
\newcommand{\school} { IES La Serna }
\newcommand{\academicPeriod} { Curso 2020/2021 }


\newcommand{\autor} { Andrés Giménez Muñoz }
\newcommand{\emailAuthor} { agimenezmunoz@ieslaserna.com }
\newcommand{\autorSing}{ Profesores: Andrés } 
 \renewcommand{\schoolSubject} { Examen Matemáticas 2º ESO  }
\renewcommand{\school} { IES José de Churriguera  }
\renewcommand{\academicPeriod} { Curso 2022/2023 }

\renewcommand{\autor} { Andrés Giménez Muñoz }
\renewcommand{\emailAuthor} { andresprofemates@outlook.es }
\renewcommand{\autorSing}{ Profesor: Andrés } 
 
 % \renewcommand{\thepartno}{\arabic{partno}}
 
 %%%%%%%%%%%%%%%%%%%%%%%%%%%
 % Exam configuration
 %\pointsdroppedatright   %% No mostrar la puntuación
 \pointsinrightmargin{} % Para poner las puntuaciones a la derecha. Se puede cambiar. Si se comenta, sale a la izquierda.
 \extrawidth{-1.5cm} %Un poquito más de margen por si ponemos textos largos.
 \marginpointname{ \emph{\points}}
 
 %% Si se comenta no aparecerán los espacios de la solución.
 %\nocancelspace
 
 %% Esto es de la clase exam. Si dejamos sin comentar \printanswers, se mostraran las soluciones. 
 %% Si la comentamos y dejamos sin comentar \noprintanswers, pues no se muestran las soluciones.
 % \printanswers
 %\noprintanswers
 
 %%%%%%%%%%%%%%%%%%%%%%%%%%%
 
 \begin{document}
 
 \StudentData{}
 \GradeTableHeader{}
 
 \justifying{}
 
 \begin{questions}
     \question[2]Escribe como una sola potencia (sin resolver la potencia)
     \begin{parts}
         \part $\left(-3\right)^3$
         \vspace{\stretch{1}}
         \part $-3^3$
         \vspace{\stretch{1}}
         \part $\left(\frac{1}{2}\right)^{-3}$
         \vspace{\stretch{1}}
         \part $\left(\frac{4}{3}\right)^{0}$
         \vspace{\stretch{1}}
         \part $\left(a^{-3} \cdot a^2\right) ^{-4} : a^{-6}$
         \vspace{\stretch{1}}
     \end{parts}
 
     \question[2]Calcula el valor de las siguientes raíces y su resto
     \begin{parts}
           \part $\sqrt[4]{81}$
           \vspace{\stretch{1}}
           \part $\sqrt{\frac{16}{25}}$
           \vspace{\stretch{1}}
           \part $\sqrt[3]{\frac{1}{8}}$
           \vspace{\stretch{1}}
           \part $\sqrt{-1}$
           \vspace{\stretch{1}}
           \part $\sqrt[3]{-1}$
           \vspace{\stretch{1}}
     \end{parts}

     \newpage{}

     \question[2]
     Escribe en cada caso el valor de $n$:
     \begin{parts}
        \part $256=2^n$
        \vspace{\stretch{1}}
        \part $\frac{1}{27} = 3^n$
        \vspace{\stretch{1}}
        \part $-125=-5^n$
        \vspace{\stretch{1}}
    \end{parts}

     \question[2]
     ¿Cuál es el número de monedas que hay en el lado de un cuadrado que se puede formar con 131 monedas?
     ¿Cuántas monedas sobran?
     \vspace{\stretch{2}}
    
     
     \question[2]
     Efectúa las siguientes operaciones y expresa el resultado en notación científica:
     \begin{parts}
        \part $5,3 \cdot 10^{11} - 1,2 \cdot 10^{12} + 7,2 \cdot 10^{10}$
        \vspace{\stretch{1}}
        \part $\left(2,25 \cdot 10^{22} \right) \cdot \left(4 \cdot 10^{-15} \right) : \left(3 \cdot 10^{-3} \right) $
        \vspace{\stretch{1}}
     \end{parts}

 \end{questions}
 
 \end{document}