 %%%%%%%%%%%%%%%%%%%%%%%%%%%
 \newcommand{\documentName} { 1ª evaluación }
 \newcommand{\documentContent} { Potencias, radicales, notación científica y logaritmos} 
 \newcommand{\waterMark} { Modelo B  } 
 %%%%%%%%%%%%%%%%%%%%%%%%%%%
 
 % Configuración del documento.
 \newcommand{\schoolSubject} { Matemáticas 3º ESO - Recuperación}
\newcommand{\school} { IES La Serna }
\newcommand{\academicPeriod} { Curso 2020/2021 }


\newcommand{\autor} { Andrés Giménez Muñoz }
\newcommand{\emailAuthor} { agimenezmunoz@ieslaserna.com }
\newcommand{\autorSing}{ Profesores: Andrés } 
 \renewcommand{\schoolSubject} { Examen Matemáticas 2º ESO  }
\renewcommand{\school} { IES José de Churriguera  }
\renewcommand{\academicPeriod} { Curso 2022/2023 }

\renewcommand{\autor} { Andrés Giménez Muñoz }
\renewcommand{\emailAuthor} { andresprofemates@outlook.es }
\renewcommand{\autorSing}{ Profesor: Andrés } 
 
 % \renewcommand{\thepartno}{\arabic{partno}}
 \usepackage{xfrac}
 
 %%%%%%%%%%%%%%%%%%%%%%%%%%%
 % Exam configuration
 %\pointsdroppedatright   %% No mostrar la puntuación
 \pointsinrightmargin{} % Para poner las puntuaciones a la derecha. Se puede cambiar. Si se comenta, sale a la izquierda.
 \extrawidth{-1.5cm} %Un poquito más de margen por si ponemos textos largos.
 \marginpointname{ \emph{\points}}
 
 %% Si se comenta no aparecerán los espacios de la solución.
 %\nocancelspace
 
 %% Esto es de la clase exam. Si dejamos sin comentar \printanswers, se mostraran las soluciones. 
 %% Si la comentamos y dejamos sin comentar \noprintanswers, pues no se muestran las soluciones.
 % \printanswers
 %\noprintanswers
 
 %%%%%%%%%%%%%%%%%%%%%%%%%%%
 
 \begin{document}
 
 \StudentData{}
 \GradeTableHeader{}
 
 \justifying{}
 
 \begin{questions}
     \question[3]Simplifica las siguientes expresiones:
     \begin{parts}
         \part $\left(-5\right)^3$
         \vspace{\stretch{1}}
         \part $\left(\frac{1}{3}\right)^{-3}$
         \vspace{\stretch{1}}
         \part $\left(\frac{2}{7}\right)^{0}$
         \vspace{\stretch{1}}
         \part $\left(a^{-2} \cdot a^3\right) ^3 : a^{-8}$
         \vspace{\stretch{1}}
         \part $\left(-2\right)^2 \cdot \left(2^2\right)^3 : \left(-3\right)^7$
         \vspace{\stretch{1}}
         \part $\left(\frac{1}{3^2}\right)^{-2} \cdot 3^3$
         \vspace{\stretch{1}}
         \part $\left(-5\right)^{-1} \cdot \left[\left(-5\right)^2 \right]^3$
         \vspace{\stretch{1}}

     \end{parts}
 
     \newpage{}

     \question[1] Simplifica:
     \\
     \\
     $\frac{20 \cdot 16 \cdot 5^2 \cdot 14^2}{10 \cdot 7 \cdot 350}$ % = 8
     \vspace{\stretch{2}}

     \question[2]Calcula el valor de las siguientes raíces o simplifica lo que se pueda:
     \begin{parts}
           \part $\sqrt[5]{125}$
           \vspace{\stretch{1}}
           \part $\sqrt{\frac{9}{49}}$
           \vspace{\stretch{1}}
           \part $\sqrt[3]{\frac{1}{125}}$
           \vspace{\stretch{1}}
           \part $\frac{1}{5}\sqrt{3} + 3\sqrt{3} - \frac{3}{2} \sqrt{3}$
           \vspace{\stretch{1}}
           \part $\frac{\sqrt{27}}{\sqrt{3}}$
           \vspace{\stretch{1}}

     \end{parts}

     \newpage{}

     \question[2]
     Calcula el valor de las siguientes expresiones:
     \begin{parts}
        \part $\log_{7} 343$
        \vspace{\stretch{1}}
        \part $\log 100000$
        \vspace{\stretch{1}}       
        \part $\log 0,0001$
        \vspace{\stretch{1}}
        \part $\log_2 64$
        \vspace{\stretch{1}}
        \part $\log_4 (8) + \log_9 (2)$
        \vspace{\stretch{1}}
        \part $\log_7 (7^8)$
        \vspace{\stretch{1}}

    \end{parts}   

    \newpage{}

    \question[2]
    Efectúa las siguientes operaciones y expresa el resultado en notación científica:
    \begin{parts}
        \part $5,3 \cdot 10^{11} - 1,3 \cdot 10^{12} + 7,2 \cdot 10^{10}$
        \vspace{\stretch{1}}
        \part $\left(2,25 \cdot 10^{22} \right) \cdot \left(4 \cdot 10^{-15} \right) : \left(5 \cdot 10^{-3} \right) $
        \vspace{\stretch{1}}
        \part $\frac{1,2 \cdot 10^5 - 7,7 \cdot 10^4}{2,8 \cdot 10^{-5}}$
        \vspace{\stretch{1}}
    \end{parts}

 \end{questions}
 
 \end{document}