 %%%%%%%%%%%%%%%%%%%%%%%%%%%
 \newcommand{\documentName} { 1ª evaluación }
 \newcommand{\documentContent} { Examen de evaluación } 
 \newcommand{\waterMark} {  } 
 %%%%%%%%%%%%%%%%%%%%%%%%%%%
 
 % Configuración del documento.
 \newcommand{\schoolSubject} { Matemáticas 3º ESO - Recuperación}
\newcommand{\school} { IES La Serna }
\newcommand{\academicPeriod} { Curso 2020/2021 }


\newcommand{\autor} { Andrés Giménez Muñoz }
\newcommand{\emailAuthor} { agimenezmunoz@ieslaserna.com }
\newcommand{\autorSing}{ Profesores: Andrés } 
 \renewcommand{\schoolSubject} { Examen Matemáticas 2º ESO  }
\renewcommand{\school} { IES José de Churriguera  }
\renewcommand{\academicPeriod} { Curso 2022/2023 }

\renewcommand{\autor} { Andrés Giménez Muñoz }
\renewcommand{\emailAuthor} { andresprofemates@outlook.es }
\renewcommand{\autorSing}{ Profesor: Andrés } 
 
 % \renewcommand{\thepartno}{\arabic{partno}}
 \usepackage{xfrac}
 \usepackage{yhmath}
 
 %%%%%%%%%%%%%%%%%%%%%%%%%%%
 % Exam configuration
 %\pointsdroppedatright   %% No mostrar la puntuación
 \pointsinrightmargin{} % Para poner las puntuaciones a la derecha. Se puede cambiar. Si se comenta, sale a la izquierda.
 \extrawidth{-1.5cm} %Un poquito más de margen por si ponemos textos largos.
 \marginpointname{ \emph{\points}}
 
 %% Si se comenta no aparecerán los espacios de la solución.
 %\nocancelspace
 
 %% Esto es de la clase exam. Si dejamos sin comentar \printanswers, se mostraran las soluciones. 
 %% Si la comentamos y dejamos sin comentar \noprintanswers, pues no se muestran las soluciones.
 % \printanswers
 %\noprintanswers
 
 %%%%%%%%%%%%%%%%%%%%%%%%%%%
 
 \begin{document}
 
 \StudentData{}
 \GradeTableHeader{}
 
 \justifying{}
 
 \begin{questions}
    \question[2] Calcula:
    \begin{parts}
        \part $\left[\left(1 - 7\right) - \left( 8 - 3 \right) - 2 \right] \cdot \left(15 -11\right)$
        \vspace{\stretch{1}}
        \part $\left(\frac {13}{15} - \frac{7}{25} \right) \cdot \left(\frac {9}{22} + \frac{-13}{33} \right)$
        \vspace{\stretch{1}}
    \end{parts}

    \question[1] Expresa en forma de fracción las siguientes cantidades:
    \begin{parts}
        \part $4,23$
        \vspace{\stretch{1}}
        \part $5,01$
        \vspace{\stretch{1}}
        \part $0,\wideparen{3}$
        \vspace{\stretch{1}}
        \part $1,5\wideparen{8}$
        \vspace{\stretch{1}}
    \end{parts}

    %  \question[3]Simplifica las siguientes expresiones:
    %  \begin{parts}
    %      \part $\left(-3\right)^3$
    %      \vspace{\stretch{1}}
    %      \part $\left(\frac{1}{2}\right)^{-3}$
    %      \vspace{\stretch{1}}
    %      \part $\left(\frac{7}{3}\right)^{0}$
    %      \vspace{\stretch{1}}
    %      \part $\left(a^{-2} \cdot a^3\right) ^{-3} : a^{-7}$
    %      \vspace{\stretch{1}}
    %      \part $\left(-2\right)^4 \cdot \left(2^2\right)^2 : \left(-2\right)^7$
    %      \vspace{\stretch{1}}
    %      \part $\left(\frac{1}{3^2}\right)^{-2} \cdot 3^3$
    %      \vspace{\stretch{1}}
    %      \part $\left(-3\right)^{-1} \cdot \left[\left(-3\right)^2 \right]^3$
    %      \vspace{\stretch{1}}

    %  \end{parts}
 
    %  \newpage{}

    %  \question[1] Simplifica:
    %  \\
    %  \\
    %  $\frac{10 \cdot 8 \cdot 5^2 \cdot 14^2}{40 \cdot 7 \cdot 175}$ % = 8
    %  \vspace{\stretch{2}}

    %  \question[2]Calcula el valor de las siguientes raíces o simplifica lo que se pueda:
    %  \begin{parts}
    %        \part $\sqrt[4]{81}$
    %        \vspace{\stretch{1}}
    %        \part $\sqrt{\frac{16}{25}}$
    %        \vspace{\stretch{1}}
    %        \part $\sqrt[3]{\frac{1}{8}}$
    %        \vspace{\stretch{1}}
    %        \part $\frac{1}{2}\sqrt{2} + 4\sqrt{2} - \frac{3}{4} \sqrt{2}$
    %        \vspace{\stretch{1}}
    %        \part $\frac{\sqrt{8}}{\sqrt{2}}$
    %        \vspace{\stretch{1}}

    %  \end{parts}

    %  \newpage{}

    %  \question[2]
    %  Calcula el valor de las siguientes expresiones:
    %  \begin{parts}
    %     \part $\log_{11} 121$
    %     \vspace{\stretch{1}}
    %     \part $\log 10000$
    %     \vspace{\stretch{1}}       
    %     \part $\log 0,001$
    %     \vspace{\stretch{1}}
    %     \part $\log_2 32$
    %     \vspace{\stretch{1}}
    %     \part $\log_9 (3) + \log_9 (3)$
    %     \vspace{\stretch{1}}
    %     \part $\log_5 (5^3)$
    %     \vspace{\stretch{1}}

    % \end{parts}   

    % \newpage{}

    % \question[2]
    % Efectúa las siguientes operaciones y expresa el resultado en notación científica:
    % \begin{parts}
    %     \part $5,3 \cdot 10^{11} - 1,2 \cdot 10^{12} + 7,2 \cdot 10^{10}$
    %     \vspace{\stretch{1}}
    %     \part $\left(2,25 \cdot 10^{22} \right) \cdot \left(4 \cdot 10^{-15} \right) : \left(3 \cdot 10^{-3} \right) $
    %     \vspace{\stretch{1}}
    %     \part $\frac{1,2 \cdot 10^5 + 7,7 \cdot 10^3}{2,8 \cdot 10^{-5}}$
    %     \vspace{\stretch{1}}
    % \end{parts}

 \end{questions}
 
 \end{document}