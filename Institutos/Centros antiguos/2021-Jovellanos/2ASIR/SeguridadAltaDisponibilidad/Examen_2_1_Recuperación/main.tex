%%%%%%%%%%%%%%%%%%%%%%%%%%%
\newcommand{\documentName} { Examen 2ª evaluación }
\newcommand{\documentContent} { Examen recuperación } 
\newcommand{\waterMark} {  } 
%%%%%%%%%%%%%%%%%%%%%%%%%%%

% Configuración del documento.
\newcommand{\schoolSubject} { Matemáticas 3º ESO - Recuperación}
\newcommand{\school} { IES La Serna }
\newcommand{\academicPeriod} { Curso 2020/2021 }


\newcommand{\autor} { Andrés Giménez Muñoz }
\newcommand{\emailAuthor} { agimenezmunoz@ieslaserna.com }
\newcommand{\autorSing}{ Profesores: Andrés } 
\renewcommand{\schoolSubject} { Examen Matemáticas 2º ESO  }
\renewcommand{\school} { IES José de Churriguera  }
\renewcommand{\academicPeriod} { Curso 2022/2023 }

\renewcommand{\autor} { Andrés Giménez Muñoz }
\renewcommand{\emailAuthor} { andresprofemates@outlook.es }
\renewcommand{\autorSing}{ Profesor: Andrés } 

%%%%%%%%%%%%%%%%%%%%%%%%%%%
% Exam configuration
%\pointsdroppedatright   %% No mostrar la puntuación
\pointsinrightmargin % Para poner las puntuaciones a la derecha. Se puede cambiar. Si se comenta, sale a la izquierda.
\extrawidth{-1.5cm} %Un poquito más de margen por si ponemos textos largos.
\marginpointname{ \emph{\points}}

%% Si se comenta no aparecerán los espacios de la solución.
%\nocancelspace

%% Esto es de la clase exam. Si dejamos sin comentar \printanswers, se mostraran las soluciones. 
%% Si la comentamos y dejamos sin comentar \noprintanswers, pues no se muestran las soluciones.
\printanswers
%\noprintanswers

%%%%%%%%%%%%%%%%%%%%%%%%%%%

\begin{document}

\StudentData{}
\GradeTableHeader{}

\justifying{}

\begin{questions}
    \setcounter{question}{0}

    \question[7]
    Indica la respuesta más adecuada.
    \begin{center}
        \fbox{\fbox{\parbox{6.5in}{
                    La forma de valor el resultado las preguntas tipo test será:

                    \begin{equation*} \label{eqn}
                        Nota = \left( correctas - \frac{incorrecta}{2} \right) \cdot \left( \frac{nota\:máxima}{preguntas}  \right)
                    \end{equation*}

                    \vspace{0.2cm}
                    \par
                    Cuando contestes a cada pregunta del cuestionario ten presente:
                    \begin{itemize}
                        \item Lee atentamente las preguntas y las contestaciones, tienes tiempo de sobra para resolver el examen.
                        \item Los fallos restan puntos, si no estás seguro de la respuesta no la contestes.
                        \item Aunque en algunas preguntas pueda parecer que existe más de una respuesta correcta, tienes que dar una respuesta profesional que indique que dominas los conceptos teóricos.
                    \end{itemize}
                }}}
    \end{center}

    \begin{parts}
        \PartNoBreak{
            El realizar gran cantidad de peticiones a un servicio, para que no pueda atender a sus usuarios legítimos se considera \dots
            \newline
            \begin{oneparcheckboxes}
                \choice{}
                \dots{} ataque por colapso del servicio.
                \newline{}
                \choice{}
                \dots ataque por redundancia de peticiones.
                \newline{}
                \CorrectChoice{}
                \dots{} un ataque DDoS o denegación de servicio distribuido.
                \newline{}
            \end{oneparcheckboxes}
        }

        \PartNoBreak{
        El Sniffing es \dots
        \newline
            \begin{oneparcheckboxes}
                \choice{}
                \dots el rastrear todos los posibles fallos de seguridad de una red informática, el término viene del inglés \emph{sniff} que significa husmear.
                \newline
                \choice{}
                \dots escuchar las conversaciones de las posibles víctimas de un ataque, intentando conocer sus hábitos para deducir las contraseñas de acceso a los servicios que usa.
                \newline
                \CorrectChoice{}
                \dots una técnica consistente en monitorizar el tráfico de red.
                \newline
            \end{oneparcheckboxes}
        }

        \PartNoBreak{
            El sppofing es \dots
            \newline
            \begin{oneparcheckboxes}
                \choice{}
                \dots suplantación de la identidad falsificando la documentación de una persona.
                \newline
                \choice{}
                \dots suplantar un servicio de Internet, para que los usuarios introduzcan la contraseña en el sistema del suplantador.
                \newline
                \CorrectChoice{}
                \dots una técnica de suplantación de identidad falsificando la IP, MAC, web o mail.
                \newline
            \end{oneparcheckboxes}
        }

        \PartNoBreak{
            El modificar el fichero \emph{host} fraudulentamente \dots
            \newline
            \begin{oneparcheckboxes}
                \choice
                \dots no tiene efectos, ya que el fichero \emph{host} solo afecta a direcciones locales.
                \newline
                \choice
                \dots el fichero \emph{host} no se puede modificar.
                \newline
                \CorrectChoice
                \dots es una técnica de ataque denominada \emph{pharming}.
                \newline
            \end{oneparcheckboxes}
        }

        \PartNoBreak{
            Los firewall o cortafuegos se encargan de \dots
            \newline
            \begin{oneparcheckboxes}
                \choice
                \dots redireccionar el tráfico de red, cuando detecta una amenaza.
                \newline
                \CorrectChoice
                \dots controlar el acceso a una red, un host o un proceso.
                \newline
                \choice
                \dots coordinar las medidas de seguridad antincendios.
                \newline
            \end{oneparcheckboxes}
        }

        \PartNoBreak{
            El introducir las contraseñas de acceso a servicios, en ficheros de configuración, \dots
            \newline
            \begin{oneparcheckboxes}
                \CorrectChoice
                \dots es un fallo de seguridad.
                \newline
                \choice
                \dots es un fallo de seguridad, solo si las claves están accesibles a personal ajeno a la organización.
                \newline
                \choice
                \dots es una práctica imprescindible, ya que los servidores las necesitan para acceder a dichos servicios. Para que esta información no sea accesible, se puede proteger los ficheros de seguridad ante amenazas.
                \newline
            \end{oneparcheckboxes}
        }

        \PartNoBreak{
            Los firewall o cortafuegos puede filtrar la información por  \dots
            \newline
            \begin{oneparcheckboxes}
                \choice
                \dots IP ó puerto de origen y destino, la MAC no pueden filtrarla por pertenecer a la capa de enlace.
                \newline
                \choice
                \dots el puerto de destino, ya que, solo está destinado a controlar que recursos están disponibles para los usuarios.
                \newline
                \CorrectChoice
                \dots la MAC, IP ó puerto de origen y de destino.
                \newline
            \end{oneparcheckboxes}
        }

        \PartNoBreak{
            Una DMZ \dots
            \newline
            \begin{oneparcheckboxes}
                \choice
                \dots un término que se usan solo en redes de uso militar.
                \newline
                \CorrectChoice
                \dots consiste en una zona de una red en la que se aplica menores medidas de seguridad.
                \newline
                \choice
                \dots una zona de la red con acceso libre a cualquier usuario ajeno a la corporación.
                \newline
            \end{oneparcheckboxes}
        }

        \PartNoBreak{
            Containerd se puede definir como \dots
            \newline
            \begin{oneparcheckboxes}
                \CorrectChoice{}
                \dots el motor de ejecución de contenedores. Utilizado en diversas herramientas de contenedores como docker ó kubernetes.
                \newline
                \choice{}
                \dots un sistema de virtualización de contenedores en remoto.
                \newline
                \choice{}
                \dots un comando que muestra todos contenedores en ejecución.
                \newline
            \end{oneparcheckboxes}
        }

        \PartNoBreak{
            Los contenedores los podemos ver como \dots
            \newline
            \begin{oneparcheckboxes}
                \choice{}
                \dots una máquina virtual de bajo consumo.
                \newline
                \choice{}
                \dots un instalador de aplicaciones Web.
                \newline
                \CorrectChoice{}
                \dots un mecanismo para empaquetar y distribuir una aplicación y todas sus dependencias.
                \newline
            \end{oneparcheckboxes}
        }
        
        \PartNoBreak{
            La relación entre docker e imágenes es \dots
            \newline
            \begin{oneparcheckboxes}
                \CorrectChoice{}
                \dots podemos tener varios docker de una imagen.
                \newline
                \choice{}
                \dots podemos tener varias imágenes de un cocker.
                \newline
                \choice{}
                \dots de uno a uno, cada docker tiene su imagen.
                \newline{}
            \end{oneparcheckboxes}
        }

        \PartNoBreak{
            Un ataque de denegación de servicio o DoS consiste en\dots
            \newline
            \begin{oneparcheckboxes}
                \choice{}
                \dots causar un error de un servicio denegando le el acceso a los recursos que necesita dicho servicio.
                \newline{}
                \CorrectChoice{}
                \dots{} causar que un servicio deje de estar accesible a los usuarios legítimos.
                \newline{}
                \choice{}
                \dots cortando el acceso de un sistema informático a los recursos necesarios para su correcto funcionamiento.
                \newline{}
            \end{oneparcheckboxes}
        }

        \PartNoBreak{
            Un Proxy es \dots
            \newline
            \begin{oneparcheckboxes}
                \choice
                \dots un tipo especial de cortafuegos.
                \newline
                \choice
                \dots un elemento de red que nos permite conectarnos a servidores remotos.
                \newline
                \CorrectChoice
                \dots una aplicación o sistema que hace de intermediario ente el host cliente y el host servidor.
                \newline
            \end{oneparcheckboxes}
        }

        \PartNoBreak{
            Los sistemas de redundancia en la distribución de procesos \dots
            \newline
            \begin{oneparcheckboxes}
                \choice
                \dots se denominan multiproceso.
                \newline
                \choice
                \dots es un sistema consistente en disponer de varias máquinas en reserva para utilizarlas en el supuesto de que falle una de ellas.
                \newline
                \CorrectChoice
                \dots se denomina clustering.
                \newline
            \end{oneparcheckboxes}
        }

        \PartNoBreak{
            Un sistema de RAID de discos en el que se duplica toda la información en dos discos se denomina \dots
            \newline
            \begin{oneparcheckboxes}
                \choice
                \dots RAID 0.
                \newline
                \CorrectChoice
                \dots RAID 1.
                \newline
                \choice
                \dots RAID 2.
                \newline
            \end{oneparcheckboxes}
        }

        \PartNoBreak{
            La jerarquía de las normas jurídicas en España sigue el siguiente orden\dots
            \newline
            \begin{oneparcheckboxes}
                \CorrectChoice{}
                \dots Constitución, tratados internacionales, leyes orgánicas leyes ordinarias, decretos legislativos.
                \newline
                \choice{}
                \dots Constitución, leyes orgánicas, tratados internacionales, leyes ordinarias, decretos legislativos.
                \newline
                \choice{}
                \dots Constitución, tratados internacionales, leyes ordinarias, leyes orgánicas, decretos legislativos.
                \newline
            \end{oneparcheckboxes}
        }

        \PartNoBreak{
            La LOPD o Ley Orgánica de Protección de Datos protege los datos referentes\dots
            \newline
            \begin{oneparcheckboxes}
                \choice
                \dots a las personas y asociaciones.
                \newline
                \choice
                \dots a las personas y asociaciones empresariales.
                \newline
                \CorrectChoice
                \dots únicamente a las personas físicas.
                \newline
            \end{oneparcheckboxes}
        }

        \PartNoBreak{
            La LOPD o Ley Orgánica de Protección de Datos establece derechos fundamentales, \dots
            \newline
            \begin{oneparcheckboxes}
                \choice
                \dots a los cuales, solo se puede renunciar por consentimiento explícito.
                \newline
                \choice
                \dots a los cuales, solo se puede renunciar parcialmente mediante un consentimiento por escrito.
                \newline
                \CorrectChoice
                \dots a los cuales no se puede renunciar.
                \newline
            \end{oneparcheckboxes}
        }

        \PartNoBreak{
            En la LOPD o Ley Orgánica de Protección de Datos cuando habla de un fichero se refiere \dots
            \newline
            \begin{oneparcheckboxes}
                \choice
                \dots tanto a un fichero de ordenador como un documento físico guardado en un archivador.
                \newline
                \choice
                \dots a ficheros informático, pudiendo se almacenar en un ordenador personal, en un servidor o en la nube.
                \newline
                \CorrectChoice
                \dots a cualquier información concerniente a personas físicas, identificadas o identificables.
                \newline
            \end{oneparcheckboxes}
        }

        \PartNoBreak{
            La LOPD o Ley Orgánica de Protección de Datos establece que si se produce un robo de información de un fichero se debe comunicar a la Agencia de Protección de Datos \dots
            \newline
            \begin{oneparcheckboxes}
                \CorrectChoice
                \dots siempre.
                \newline
                \choice
                \dots únicamente cuando el propietario del fichero es responsable de la filtración.
                \newline
                \choice
                \dots únicamente cuando el propietario del fichero no es responsable de la filtración.
                \newline
            \end{oneparcheckboxes}
        }

        \PartNoBreak{
            Según la LOPD o Ley Orgánica de Protección de Datos, los datos referentes a información financiero son datos de seguridad de nivel\dots
            \newline
            \begin{oneparcheckboxes}
                \choice
                \dots básico.
                \newline
                \CorrectChoice
                \dots medio.
                \newline
                \choice
                \dots alto.
                \newline
            \end{oneparcheckboxes}
        }

        \PartNoBreak{
            Según la LOPD o Ley Orgánica de Protección de Datos, los datos referentes a información médica son datos de seguridad de nivel\dots
            \newline
            \begin{oneparcheckboxes}
                \choice
                \dots básico.
                \newline
                \choice
                \dots medio.
                \newline
                \CorrectChoice
                \dots alto.
                \newline
            \end{oneparcheckboxes}
        }

        \PartNoBreak{
        Mara modificar una imagen de un docker se utiliza el comando \dots
        \newline
            \begin{oneparcheckboxes}
                \CorrectChoice{}
                \dots{} la imágenes de docker no se pueden modificar. 
                \newline
                \choice{}
                \dots docker image push
                \newline
                \choice{}
                \dots{} docker image update.
                \newline
            \end{oneparcheckboxes}
        }
    
        \PartNoBreak{
        DevOps se refiere a \dots
        \newline
            \begin{oneparcheckboxes}
                \choice
                \dots una metodología que define como implantar docker en una corporación.
                \newline
                \choice
                \dots un software orientado al control de cambios en la configuración de servidores.
                \newline
                \CorrectChoice
                \dots un conjunto de prácticas que agrupa a desarrolladores y operaciones TI utilizado para controlar el ciclo de vida del desarrollo de software. 
                \newline
            \end{oneparcheckboxes}
        }

        \PartNoBreak{
            Según la LOPD o Ley Orgánica de Protección de Datos, el registro de incidencias es requerido para archivos con nivel de seguridad básico.
            Si somos responsable de un archivo en el que se guarda información de ideología y creencias religiosas las cuales requieren seguridad de nivel alto \dots
            \newline
            \begin{oneparcheckboxes}
                \CorrectChoice
                \dots también debemos mantener un registro de incidencias, ya que los requisitos de nivel de seguridad básico, también han de ser cumplidos en los niveles de seguridad medio y alto.
                \newline
                \choice
                \dots no en necesario mantener un registro de incidencias, ya que el nivel de seguridad alto dispone de otros mecanismos que lo suplementan.
                \newline
                \choice
                \dots no podemos guardar un registro de incidencia porque eso vulnera los derechos de la persona a la que se refieren los datos.
                \newline
            \end{oneparcheckboxes}
        }
    \end{parts}

    \newpage
    \question[3]
    En los Anexo I y II, puedes encontrar los ficheros \emph{docker-compose.yaml} y \emph{WorkpressDockerfile}, los cuales se utilizan para 
    ejecutar ciertos servicios sobre docker-compose. Explica lo que hacen, los servicios que utiliza, las conexiones entre los servicios,
    las redes virtuales y volúmenes que genera y los puertos que utiliza. 
    Realiza un esquema en el que se pueda apreciar todas las conexiones entre los servicios. 
    \\ \\
    \footnotesize{(*) Recuerda que debes demostrar tus conocimientos y no limitarte a enumerar lo que hace cada línea.}

    \newpage
    \newpage 
    \begin{center}
        Anexo I:  docker-compose.yaml
    \end{center}
    \begin{figure}[htbp!]
        \centering
            \includegraphics[width=.9\textwidth]{./docker-compose.1.png} 
     \end{figure}

     \newpage 
     \begin{figure}[htbp!]
        \centering
            \includegraphics[width=.9\textwidth]{./docker-compose.2.png} 
     \end{figure}

     \begin{center}
        Anexo II: WorkpressDockerfile
     \end{center}
     \begin{figure}[htbp!]
        \centering
            \includegraphics[width=.9\textwidth]{./WorkpressDockerfile.png} 
     \end{figure}

\end{questions}

\end{document}