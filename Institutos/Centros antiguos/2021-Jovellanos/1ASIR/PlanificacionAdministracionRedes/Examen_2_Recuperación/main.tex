%%%%%%%%%%%%%%%%%%%%%%%%%%%
\newcommand{\documentName} { Examen 2ª evaluación }
\newcommand{\documentContent} { Examen recuperación } 
\newcommand{\waterMark} { } 
%%%%%%%%%%%%%%%%%%%%%%%%%%%

% Configuración del documento.
\newcommand{\schoolSubject} { Matemáticas 3º ESO - Recuperación}
\newcommand{\school} { IES La Serna }
\newcommand{\academicPeriod} { Curso 2020/2021 }


\newcommand{\autor} { Andrés Giménez Muñoz }
\newcommand{\emailAuthor} { agimenezmunoz@ieslaserna.com }
\newcommand{\autorSing}{ Profesores: Andrés } 
\renewcommand{\schoolSubject} { Examen Matemáticas 2º ESO  }
\renewcommand{\school} { IES José de Churriguera  }
\renewcommand{\academicPeriod} { Curso 2022/2023 }

\renewcommand{\autor} { Andrés Giménez Muñoz }
\renewcommand{\emailAuthor} { andresprofemates@outlook.es }
\renewcommand{\autorSing}{ Profesor: Andrés } 

% \renewcommand{\thepartno}{\roman{partno}}
\renewcommand{\thepartno}{\arabic{partno}}
% \renewcommand{\thepartno}{\thesubpart}

%%%%%%%%%%%%%%%%%%%%%%%%%%%
% Exam configuration
%\pointsdroppedatright   %% No mostrar la puntuación
\pointsinrightmargin % Para poner las puntuaciones a la derecha. Se puede cambiar. Si se comenta, sale a la izquierda.
\extrawidth{-1.5cm} %Un poquito más de margen por si ponemos textos largos.
\marginpointname{ \emph{\points}}

%% Si se comenta no aparecerán los espacios de la solución.
%\nocancelspace

%% Esto es de la clase exam. Si dejamos sin comentar \printanswers, se mostraran las soluciones. 
%% Si la comentamos y dejamos sin comentar \noprintanswers, pues no se muestran las soluciones.
% \printanswers
%\noprintanswers

%%%%%%%%%%%%%%%%%%%%%%%%%%%

\begin{document}

\StudentData
\GradeTableHeader

\justifying

\begin{questions}
    \setcounter{question}{0}

    \question[6]
    Indica las afirmaciones más apropiadas
    \ifprintanswers
    \else
    \begin{center}
        \fbox{\fbox{\parbox{6.5in}{
                    La forma de valor el resultado las preguntas tipo test será:

                    \begin{equation*} \label{eqn}
                        \text{Nota} = \left( \text{correctas} - \frac{\text{incorrecta}}{2} \right) \cdot \left( \frac{\text{nota máxima}}{\text{número de preguntas}}  \right)
                    \end{equation*}

                    \vspace{0.2cm}
                    \par
                    Cuando contestes a cada pregunta del cuestionario ten presente:
                    \begin{itemize}
                        \item Lee atentamente las preguntas y las contestaciones, tienes tiempo de sobra para resolver el examen.
                        \item Los fallos restan puntos, si no estás seguro de la respuesta no la contestes.
                        \item Aunque en algunas preguntas pueda parecer que existe más de una respuesta correcta, tienes que dar una respuesta profesional que indique que dominas los conceptos teóricos.
                    \end{itemize}
                }}}
    \end{center}
    \fi

    \begin{parts}
        \PartNoBreak{
            El tamaño de una red lógica con protocolo IP está determinada por \dots
            \newline
            \begin{oneparcheckboxes}
                \choice 
                \dots el número de dispositivos de interconexión de redes.
                \newline
                \CorrectChoice 
                \dots el valor de la máscara.
                \newline
                \choice 
                \dots los tres primeros valores de la dirección IP.
                \newline
            \end{oneparcheckboxes}
        }

        \PartNoBreak{
            Si tenemos una red Ethernet para la red local y salimos a internet mediante un router IP,  \dots
            \newline
            \begin{oneparcheckboxes}
                \choice 
                \dots solo se utiliza las direcciones IP.
                \newline
                \choice 
                \dots se utilizan las direcciones MAC para conectarse a direcciones remotas.
                \newline
                \CorrectChoice 
                \dots se utilizan las direcciones MAC, para los enlaces locales.
                \newline
            \end{oneparcheckboxes}
        }

        \PartNoBreak{
            En el protocolo IP, los valores indispensables de configuración que ha de tener un Host para tener acceso a una red local son,  \dots
            \newline
            \begin{oneparcheckboxes}
                \CorrectChoice 
                \dots dirección IP y máscara.
                \newline
                \choice 
                \dots dirección IP, máscara y puerta de enlace.
                \newline
                \choice 
                \dots dirección IP, máscara, puerta de enlace y dirección de servidor DNS.
                \newline
            \end{oneparcheckboxes}
        }

        % \PartNoBreak{
        %     En el protocolo IP, los valores indispensables de configuración que ha de tener un Host para tener acceso a Internet son,  \dots
        %     \newline
        %     \begin{oneparcheckboxes}
        %         \CorrectChoice 
        %         \dots dirección IP, máscara y puerta de enlace.
        %         \newline
        %         \choice 
        %         \dots dirección IP y máscara.
        %         \newline
        %         \choice 
        %         \dots dirección IP, máscara, puerta de enlace y dirección de servidor DNS.
        %         \newline
        %     \end{oneparcheckboxes}
        % }

        \PartNoBreak{
            En la configuración de un HOST con protocolo IP,  \dots
            \newline
            \begin{oneparcheckboxes}
                \choice 
                \dots la dirección IP, la dirección del DNS y la puerta de enlace han de pertenecer a la misma red, determinada por la máscara.
                \newline
                \CorrectChoice 
                \dots la dirección IP y la puerta de enlace han de pertenecer a la misma red, determinada por la máscara.
                \newline
                \choice 
                \dots dirección IP y la puerta de enlace pueden pertenecer a redes distintas, determinada por la máscara. 
                \newline
            \end{oneparcheckboxes}
        }

        \PartNoBreak{
           Un router es \dots
            \newline
            \begin{oneparcheckboxes}
                \CorrectChoice 
                \dots un dispositivo de conexión de redes destinado a conectar múltiples redes.
                \newline
                \choice 
                \dots un dispositivo de conexión de redes destinado a conectar una oficina o domicilio a Internet.
                \newline
                \choice 
                \dots un dispositivo de conexión de redes que busca el camino más corto al destino. 
                \newline
            \end{oneparcheckboxes}
        }

        % \PartNoBreak{
        % El termino LAN se refiere a  \dots
        %     \newline
        %     \begin{oneparcheckboxes}
        %         \CorrectChoice 
        %         \dots una red de área local, generalmente utilizada en entornos domésticos o empresariales.
        %         \newline
        %         \choice 
        %         \dots una red con IP privadas conectadas a internet mediante un router.
        %         \newline
        %         \choice 
        %         \dots una red que solo utiliza dispositivos de conexión de redes de tipo Hub o Switch.
        %         \newline
        %     \end{oneparcheckboxes}
        % }

        \PartNoBreak{
            Las interfaces de un router\dots
            \newline
            \begin{oneparcheckboxes}
                \CorrectChoice 
                \dots pueden utilizar diferentes protocolos de la capa de enlace.
                \newline
                \choice 
                \dots deben utilizar todas el mismo protocolo de la capa de enlace.
                \newline
                \choice 
                \dots pueden utilizar fibra óptica o Ethernet.
                \newline
            \end{oneparcheckboxes}
        }

        \PartNoBreak{
            Las IP privadas son\dots
            \newline
            \begin{oneparcheckboxes}
                \choice 
                \dots las que se utilizan en una red de área local.
                \newline
                \CorrectChoice 
                \dots aquellas que están traducidas mediante el protocolo NAT.
                \newline
                \choice 
                \dots aquellas que tienen en propiedad una persona u organización.
                \newline
            \end{oneparcheckboxes}
        }

        % \PartNoBreak{
        %     En las tablas de enrutamiento de varios router interconectados, \dots
        %     \newline
        %     \begin{oneparcheckboxes}
        %         \CorrectChoice 
        %         \dots pueden tener información diferente.
        %         \newline
        %         \choice 
        %         \dots todas han de tener la misma información.
        %         \newline
        %         \choice 
        %         \dots pueden tener información complementaria.
        %         \newline
        %     \end{oneparcheckboxes}
        % }

        % \PartNoBreak{
        %     Un Host,\dots
        %     \newline
        %     \begin{oneparcheckboxes}
        %         \CorrectChoice 
        %         \dots puede tener varias direcciones IP conectadas y se denomina Multihomed.
        %         \newline
        %         \choice 
        %         \dots solo puede tener una dirección IP asignada.
        %         \newline
        %         \choice 
        %         \dots solo puede tener una dirección MAC asignada.
        %         \newline
        %     \end{oneparcheckboxes}
        % }

        \PartNoBreak{
            La tabla de rutas, \dots
            \newline
            \begin{oneparcheckboxes}
                \choice 
                \dots es característico de la interfaz de red.
                \newline
                \CorrectChoice 
                \dots es común para todas las interfaces de un Host.
                \newline
                \choice 
                \dots es solo característico de los routers.
                \newline
            \end{oneparcheckboxes}
        }

        \PartNoBreak{
            En una tabla de rutas, \dots
            \newline
            \begin{oneparcheckboxes}
                \choice 
                \dots cada ruta solo pueden especificar un destino.
                \newline
                \choice 
                \dots cada ruta especifica una ruta única para la red de destino.
                \newline
                \CorrectChoice 
                \dots pueden aparecer diferentes rutas con el mismo destino.
                \newline
            \end{oneparcheckboxes}
        }

        % \PartNoBreak{
        %     Las rutas por defecto, \dots
        %     \newline
        %     \begin{oneparcheckboxes}
        %         \CorrectChoice 
        %         \dots se indican con la dirección de red 0.0.0.0/0.
        %         \newline
        %         \choice 
        %         \dots se indican con la dirección de red 255.255.255.255/32.
        %         \newline
        %         \choice 
        %         \dots se configuran de forma automática al configurar las interfaces de un router.
        %         \newline
        %     \end{oneparcheckboxes}
        % }

        \PartNoBreak{
            Se denomina ruta host, \dots
            \newline
            \begin{oneparcheckboxes}
                \CorrectChoice 
                \dots aquella que se dirige a un solo host y se indica con la dirección de red 255.255.255.255/32.
                \newline
                \choice 
                \dots aquella que se dirige a un solo host y se indica con la dirección de red 0.0.0.0/0.
                \newline
                \choice 
                \dots las rutas definidas en un host, las cuales se pueden listar con el comando tracert.
                \newline
            \end{oneparcheckboxes}
        }

        % \PartNoBreak{
        %     Un protocolo de encaminamiento, \dots
        %     \newline
        %     \begin{oneparcheckboxes}
        %         \CorrectChoice 
        %         \dots se encarga de definir las tablas de enrutamiento.
        %         \newline
        %         \choice 
        %         \dots se encarga de redirigir los paquetes por el camino más corto.
        %         \newline
        %         \choice 
        %         \dots se encarga de enviar paquetes de forma dinámica.
        %         \newline
        %     \end{oneparcheckboxes}
        % }

        % \PartNoBreak{
        %     En un protocolo de encaminamiento, las métricas\dots
        %     \newline
        %     \begin{oneparcheckboxes}
        %         \CorrectChoice 
        %         \dots indican de una forma cuantitativa el coste de enviar paquetes por diferentes caminos.
        %         \newline
        %         \choice 
        %         \dots miden el tiempo que tardan los paquetes en llegar por los diferentes caminos.
        %         \newline
        %         \choice 
        %         \dots no aplican, porque el protocolo se encarga de buscar la mejor ruta.
        %         \newline
        %     \end{oneparcheckboxes}
        % }

        \newpage{}
        \PartNoBreak{
            Ejemplos de protocolos de encaminamiento son\dots
            \newline
            \begin{oneparcheckboxes}              
                \choice 
                \dots IPv4, IPv6 y IPX.
                \newline
                \choice 
                \dots TCP y UPD.
                \newline
                \CorrectChoice 
                \dots RIP, IGRP y OSPF.
                \newline
            \end{oneparcheckboxes}
        }

        \PartNoBreak{
            Un dominio de enrutamiento es\dots
            \newline
            \begin{oneparcheckboxes}
                \choice 
                \dots un registro de DNS destinado a nombres de routes.
                \newline
                \CorrectChoice 
                \dots un conjunto de routers que se encuentran bajo una administración común.
                \newline
                \choice 
                \dots un protocolo para enrutar direcciones DNS.
                \newline
            \end{oneparcheckboxes}
        }

        \PartNoBreak{
            Si hablamos de encaminamiento estático, \dots
            \newline
            \begin{oneparcheckboxes}
                \CorrectChoice 
                \dots nos definimos a definir las rutas de enrutamiento de manera manual.
                \newline
                \choice 
                \dots una serie de protocolos de encaminamiento que mantiene las rutas de enrutamiento de forma estáticas.
                \newline
                \choice 
                \dots mantener la configuración de los encaminadores de forma estática.
                \newline
            \end{oneparcheckboxes}
        }

        % \PartNoBreak{
        %     La capa de trasporte \dots
        %     \newline
        %     \begin{oneparcheckboxes}
        %         \CorrectChoice 
        %         \dots nos permite enviar múltiples conexiones de trasporte por una sola conexión de red.
        %         \newline
        %         \choice 
        %         \dots nos permite enviar paquetes mediante en una conexión de red.
        %         \newline
        %         \choice 
        %         \dots se refiere a la forma de enviar los paquetes por una red.
        %         \newline
        %     \end{oneparcheckboxes}
        % }

        % \PartNoBreak{
        %     La capa de trasporte \dots
        %     \newline
        %     \begin{oneparcheckboxes}
        %         \CorrectChoice 
        %         \dots nos permite indicar con que proceso de aplicación nos queremos conectar.
        %         \newline
        %         \choice 
        %         \dots nos permite indicar con que Host nos queremos comunicar al final del transporte.
        %         \newline
        %         \choice 
        %         \dots nos permite indicar la ruta por la que se transportan los datos.
        %         \newline
        %     \end{oneparcheckboxes}
        % }

        \PartNoBreak{
            Ejemplos de protocolos de la capa de transporte son \dots
            \newline
            \begin{oneparcheckboxes}
                \choice 
                \dots IPv4, IPv6, IPX.
                \newline
                \CorrectChoice 
                \dots TCP y UDP.
                \newline
                \choice 
                \dots ftp, telnet, http.
                \newline
            \end{oneparcheckboxes}
        }

        \PartNoBreak{
            El puerto 23 corresponde al protocolo \dots
            \newline
            \begin{oneparcheckboxes}
                \CorrectChoice 
                \dots telnet.
                \newline
                \choice 
                \dots ssh.
                \newline
                \choice 
                \dots http.
                \newline
            \end{oneparcheckboxes}
        }

        \PartNoBreak{
            El puerto 20 corresponde al protocolo \dots
            \newline
            \begin{oneparcheckboxes}
                \CorrectChoice 
                \dots ftp.
                \newline
                \choice 
                \dots ssh.
                \newline
                \choice 
                \dots telnet.
                \newline
            \end{oneparcheckboxes}
        }

        % \PartNoBreak{
        %     Se denomina puerto bien conocido \dots
        %     \newline
        %     \begin{oneparcheckboxes}
        %         \CorrectChoice 
        %         \dots a los puertos por defecto para ciertos protocolos, que ha definido la ICANN.
        %         \newline
        %         \choice 
        %         \dots a los puertos que tiene abiertos un host.
        %         \newline
        %         \choice 
        %         \dots a los puertos que están habilitados en el router.
        %         \newline
        %     \end{oneparcheckboxes}
        % }

        % \PartNoBreak{
        %     Se denomina puerto registrado \dots
        %     \newline
        %     \begin{oneparcheckboxes}
        %         \CorrectChoice 
        %         \dots a los puertos utilizados por ciertos programas, registrados por terceros, ajenos a la ICANN.
        %         \newline
        %         \choice 
        %         \dots a los puertos registrados por terceros a través de la ICANN.
        %         \newline
        %         \choice 
        %         \dots a los puertos que están abiertos en un host.
        %         \newline
        %     \end{oneparcheckboxes}
        % }

        % \PartNoBreak{
        %     Se denomina puerto dinámicos o privados \dots
        %     \newline
        %     \begin{oneparcheckboxes}
        %         \CorrectChoice 
        %         \dots a aquellos que abre una aplicación de forma temporal para establecer una conexión.
        %         \newline
        %         \choice 
        %         \dots a aquellos que asigna un router NAS junto a la IP privada.
        %         \newline
        %         \choice 
        %         \dots a los puertos que están abiertos en un host.
        %         \newline
        %     \end{oneparcheckboxes}
        % }

        \PartNoBreak{
            Para el envío de correo electrónico se utiliza el protocolo\dots
            \newline
            \begin{oneparcheckboxes}
                \choice 
                \dots ftp.
                \newline
                \choice 
                \dots http.
                \newline
                \CorrectChoice 
                \dots smtp.
                \newline                
            \end{oneparcheckboxes}
        }

        \PartNoBreak{
            El DNS se puede definir como \dots
            \newline
            \begin{oneparcheckboxes}
                \choice
                \dots un sistema de nombres para que funcionen las páginas web.
                \newline
                \CorrectChoice
                \dots una base de datos distribuida y jerárquica que almacena información asociada a nombres de dominio en redes como Internet.
                \newline
                \choice
                \dots un traductor para convertir las direcciones de Internet a direcciones más legibles de manera única.
                \newline
            \end{oneparcheckboxes}
        }

        \PartNoBreak{
            En un servidor DNS un registro A \dots
            \newline
            \begin{oneparcheckboxes}
                \choice
                \dots indica la dirección que ha de tomar la consulta hacia otras zonas del árbol del DNS.
                \newline
                \choice
                \dots asocia un dominio con un servidor de Internet.
                \newline
                \CorrectChoice
                \dots asocia un dominio con una dirección IP.
                \newline
            \end{oneparcheckboxes}
        }

        % \PartNoBreak{
        %    El protocolo BOOTP \dots
        %    \newline
        %    \begin{oneparcheckboxes}
        %        \CorrectChoice 
        %        \dots nos permite realizar instalaciones remotas del sistema operativo.
        %        \newline
        %        \choice 
        %        \dots es un protocolo de enrutamiento dinámico para grandes redes.
        %        \newline
        %        \choice 
        %        \dots es un protocolo de enrutamiento estático para grandes redes. 
        %        \newline
        %    \end{oneparcheckboxes}
        % }
       
        \PartNoBreak{ 
            ¿En el modelo de capas en la que se descompone el protocolo TCP/IP el protocolo DHCP se encuentra en la capa?
            \newline
            \begin{oneparcheckboxes}
                \choice
                Capa de transporte.
                \newline
                \choice
                Capa de red.
                \newline
                \CorrectChoice
                Capa de aplicación.
                \newline
            \end{oneparcheckboxes}
        }

        \PartNoBreak{ 
            Si nos encontramos un host con la dirección IP 169.254.2.6/16 que podemos interpretar que \dots
            \newline
            \begin{oneparcheckboxes}
                \choice
                \dots que se trata de una IP pública.
                \newline
                \CorrectChoice
                \dots que no se le ha asignado una IP fija al host y que no hay ningún servidor DHCP en la red.
                \newline
                \choice
                \dots que se trata de una IP privada.
                \newline
            \end{oneparcheckboxes}
        }

        \PartNoBreak{ 
            Una dirección IP asignada por el protocolo DHCP se denomina \dots
            \newline
            \begin{oneparcheckboxes}
                \CorrectChoice
                \dots IP dinámica.
                \newline
                \choice
                \dots IP estática.
                \newline
                \choice
                \dots IP enrutable.
                \newline
            \end{oneparcheckboxes}
        }

        \PartNoBreak{ 
            El protocolo DHCP \dots
            \newline
            \begin{oneparcheckboxes}
                \choice
                \dots solo asigna direcciones variables, a partir de un rango dado.
                \newline
                \CorrectChoice
                \dots puede asignar direcciones fijas, en función de la MAC.
                \newline
                \choice
                \dots solo asigna direcciones IP y la máscara.
                \newline
            \end{oneparcheckboxes}
        }

        \PartNoBreak{ 
            El protocolo NAT sustituye direcciones privadas por públicas de manera que\dots
            \newline
            \begin{oneparcheckboxes}
                \CorrectChoice
                \dots por la red pública no aparecen direcciones privadas.
                \newline
                \choice
                \dots por la red privada no aparecen direcciones públicas.
                \newline
                \choice
                \dots no se mezclan direcciones públicas con privadas.
                \newline
            \end{oneparcheckboxes}
        }

        \PartNoBreak{ 
            Cuando un paquete con IP pública llega a un router NAT \dots
            \newline
            \begin{oneparcheckboxes}
                \choice
                \dots el paquete se enruta en función de las tablas de enrutamiento.
                \newline
                \choice
                \dots el paquete siempre es rechazado.
                \newline
                \CorrectChoice
                \dots el paquete se enruta en función de las tablas NAT.
                \newline
            \end{oneparcheckboxes}
        }

        \PartNoBreak{ 
            El despliegue del protocolo IPv6 se está centrando en \dots
            \newline
            \begin{oneparcheckboxes}
                \choice
                \dots aumentar al máximo posible el número de dispositivos que se pueden conectar a Internet.
                \newline
                \CorrectChoice
                \dots realizar subneting para reducir las tablas de enrutamiento y optimizar el funcionamiento de los routers.
                \newline
                \choice
                \dots protocolos de enrutamiento dinámico para conseguir que las rutas en Internet estén lo más optimizadas posibles.
                \newline
            \end{oneparcheckboxes}
        }
    \end{parts}

    \newpage

    \question En un instituto se quiere ampliar el sistema informático y te han pedido que reconfigures los dispositivos de red para dar soporte a los nuevos equipos.
    El instituto consta de dos secciones, que por su distancia no es posible conectarlos a una misma red local y se ha optado por contratar dos conexiones a Internet con el ISP. 
    En la sección principal, hay una secretaria en la que disponen de un servidor y una fotocopiadora en la que se quiere tener capacidad para 20 ordenadores, 
    20 aulas en la que se quieren poner entre 20 y 40 PC’s por aula y en la sección delegada 4 aulas con el mismo número de PC’s por aula. 
    Los routers que nos ha proporcionado el ISP ya están preconfigurados con las IP’s públicas 83.26.36.45 (R1) y 72.26.89.80 (R2) y en ambos IP privada 192.168.0.1/24. 
    Nos han pedido expresamente que se pueda acceder desde todos los PC’s a la impresora y al servidor ya que allí tienen alojado el aula virtual y necesitan que todos
    los profesores y alumnos tengan acceso a ello. También disponen de un servidor DNS en la nube con IP 201.65.23.9 al que podemos agregar los registros necesarios.
 
    \begin{center}
        \centering
        \includegraphics[width=1\textwidth]{red02}
    \end{center}

    \begin{parts}
        \part[2]
            Configura los protocolos NAT y DHCP para los router R1 y R2 para la interface privada de estos, así como los registros del servidor DNS.

        \newpage
        \phantom{new page}
        
        \newpage
        \phantom{new page}

        \part[2]
            Explica la finalidad y los efectos de la configuración del apartado anterior, haciendo referencia a la configuración de los HOST de la red y conexión entre ellos.
    \end{parts}

    \newpage
    \phantom{new page}

\end{questions}

\end{document}