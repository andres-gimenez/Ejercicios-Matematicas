\documentclass[addpoints,spanish, 12pt,a4paper,cancelspace]{./include/gexam}
 
 %%%%%%%%%%%%%%%%%%%%%%%%%%%
 \renewcommand{\documentName} { 3ª evaluación }
 \renewcommand{\documentContent} { Recuperación } 
 \renewcommand{\waterMark} { } 

 % Configuración del documento.
 \renewcommand{\schoolSubject} { Examen Matemáticas 2º ESO  }
\renewcommand{\school} { IES José de Churriguera  }
\renewcommand{\academicPeriod} { Curso 2022/2023 }

\renewcommand{\autor} { Andrés Giménez Muñoz }
\renewcommand{\emailAuthor} { andresprofemates@outlook.es }
\renewcommand{\autorSing}{ Profesor: Andrés } 
 %%%%%%%%%%%%%%%%%%%%%%%%%%%
  
 % \renewcommand{\thepartno}{\arabic{partno}}
 \usepackage{nicefrac, xfrac}
 \usepackage{amsmath}

 %%%%%%%%%%%%%%%%%%%%%%%%%%%
 % Exam configuration
 %\pointsdroppedatright   %% No mostrar la puntuación
 \pointsinrightmargin{} % Para poner las puntuaciones a la derecha. Se puede cambiar. Si se comenta, sale a la izquierda.
 \extrawidth{-1.5cm} %Un poquito más de margen por si ponemos textos largos.
 \marginpointname{ \emph{\points}}
 
 %% Si se comenta no aparecerán los espacios de la solución.
 %\nocancelspace
 
 %% Esto es de la clase exam. Si dejamos sin comentar \printanswers, se mostraran las soluciones. 
 %% Si la comentamos y dejamos sin comentar \noprintanswers, pues no se muestran las soluciones.
 % \printanswers
 %\noprintanswers
 
 %%%%%%%%%%%%%%%%%%%%%%%%%%%
 
 \begin{document}
 
 \StudentData{}
 \GradeTableHeader{}
 
 \justifying{}

 \begin{center}
    \fbox{\fbox{\parbox{6.5in}{             
                \begin{itemize}
                    \item Copiar, hablar, levantarse de la silla o molestar al resto de la clase pueden ser motivos de la retirada del examen que se valorará con un cero.
                    \item Deben aparecer todas las operaciones, no vale solo con indicar el resultado.
                    \item Se valorará que se indiquen las cuentas en línea o columna, realizando las operaciones en el margen.
                    \item No se puede utilizar la calculadora.
                \end{itemize}
            }}}
\end{center}

 \Large
 
 \checkboxchar{$\Box$}
 \checkedchar{$\blacksquare$}

 \begin{questions}
    \question Se han tomado las siguientes medidas de temperatura a lo largo de 10 días.
        \begin{center}
        \begin{tabular}{ |l|l|l|l|l| }
            \hline
            19ºC & 20ºC & 20ºC & 19ºC & 19ºC \\ \hline
            19ºC & 18ºC & 17ºC & 16ºC & 16ºC \\ \hline
        \end{tabular}
        \end{center}

        \begin {parts} 
            \part[1]
            Completa la siguiente tabla de frecuencia:
            \vspace{5mm}
            \begin{center}
                % \begin{table}[]
                    \begin{tabular}{|r|l|l|}
                    \hline
                    \multicolumn{1}{|l|}{Temperaturas} & Frecuencia absoluta & Frecuencia relativa \\ \hline
                    16º                                &                     &                     \\ \hline
                    17º                                &                     &                     \\ \hline
                    18º                                &                     &                     \\ \hline
                    19º                                &                     &                     \\ \hline
                    20º                                &                     &                     \\ \hline
                    \end{tabular}
                % \end{table}
            \end{center}
            \vspace{5mm}

            \part[1] Calcula la media de las temperaturas:
            \vspace{\stretch{3}}

            \part[1] Calcula la moda de las temperaturas:
            \vspace{\stretch{2}}
        \end{parts}
    \newpage{}
    
    \question[1] Manuel tiene una cesta con 5 manzanas rojas y 3 amarilla. 
              Si saca una manzana al azar ¿Cuál es la probabilidad de sacar una manzana roja?
    \vspace{\stretch{1}}

    \question[2] Completa la tabla de los múltiplos y submúltiplos de las unidades de masa:
    \\
    \begin{center}
    % \begin{table}[]
        \begin{tabular}{|l|l|l|l|}
        \hline
        \multirow{4}{*}{Múltiplos}             & Miriagramo & \multicolumn{1}{r|}{Mag} & \multicolumn{1}{r|}{10.000 g} \\ \cline{2-4} 
                                               &            &     &          \\ \cline{2-4} 
                                               &            &     &          \\ \cline{2-4} 
                                               &            &     &          \\ \hline
        \multicolumn{1}{|r|}{Unidad principal} & Gramo      & \multicolumn{1}{r|}{g} &  \multicolumn{1}{r|}{1 g} \\ \hline
        \multirow{3}{*}{Submúltiplos}          &            &     &          \\ \cline{2-4} 
                                               &            &     &          \\ \cline{2-4} 
                                               &            &     &          \\ \hline
        \end{tabular}
        % \end{table}
    \end{center}

    \question[1] Realiza las siguientes conversiones de unidades:
    \begin {parts} 
        \part $2.000 \text{ m}$ = \fillin[2] $\text{ Km}$
        \part $2.000 \text{ cm}$ = \fillin[2]  $\text{ m}$
        \part $0,1 \text{ km}$ = \fillin[100]  $\text{ m}$
        \part $300 \text{ Hm}$ = \fillin[0,3]  $\text{ Km}$
        \part $300 \text{ ml}$ = \fillin[0,3] $\text{ l}$
    \end{parts}

    \newpage 
    \question[3] Indica el nombre y calcula el área y el perímetro del siguiente polígono.

    \begin {parts} 
        \part   
        \begin{minipage}{\linewidth}
            % \centering
            \includegraphics[width=7cm]{Romboide01}
        \end{minipage}
        \vspace{\stretch{1}}
        \part   
        \begin{minipage}{\linewidth}
            % \centering
            \includegraphics[width=3cm]{Rombo01}
        \end{minipage}
        \vspace{\stretch{1}}
        \part   
        \begin{minipage}{\linewidth}
            % \centering
            \includegraphics[width=5cm]{Rectangulo01}
        \end{minipage}
        \vspace{\stretch{1}}
    \end{parts}

    \newpage
    
\end{questions}
 
\end{document}