\documentclass[addpoints,spanish, 12pt,a4paper,cancelspace]{./include/gexam}

 %%%%%%%%%%%%%%%%%%%%%%%%%%%
 \renewcommand{\documentName} { 2ª evaluación }
 \renewcommand{\documentContent} { Fracciones II } 
 \renewcommand{\waterMark} { } 

 % Configuración del documento.
 \renewcommand{\schoolSubject} { Examen Matemáticas 2º ESO  }
\renewcommand{\school} { IES José de Churriguera  }
\renewcommand{\academicPeriod} { Curso 2022/2023 }

\renewcommand{\autor} { Andrés Giménez Muñoz }
\renewcommand{\emailAuthor} { andresprofemates@outlook.es }
\renewcommand{\autorSing}{ Profesor: Andrés }
 %%%%%%%%%%%%%%%%%%%%%%%%%%%
 
 % \renewcommand{\thepartno}{\arabic{partno}}
 \usepackage{nicefrac, xfrac}
 
 %%%%%%%%%%%%%%%%%%%%%%%%%%%
 % Exam configuration
 %\pointsdroppedatright   %% No mostrar la puntuación
 \pointsinrightmargin{} % Para poner las puntuaciones a la derecha. Se puede cambiar. Si se comenta, sale a la izquierda.
 \extrawidth{-1.5cm} %Un poquito más de margen por si ponemos textos largos.
 \marginpointname{ \emph{\points}}
 
 %% Si se comenta no aparecerán los espacios de la solución.
 %\nocancelspace
 
 %% Esto es de la clase exam. Si dejamos sin comentar \printanswers, se mostraran las soluciones. 
 %% Si la comentamos y dejamos sin comentar \noprintanswers, pues no se muestran las soluciones.
 % \printanswers
 %\noprintanswers
 
 %%%%%%%%%%%%%%%%%%%%%%%%%%%
 
 \begin{document}
 
 \StudentData{}
 \GradeTableHeader{}
 
 \justifying{}
 
 \begin{questions}
      
    \question[5] Realiza las siguientes operaciones con fracciones:
        \begin {parts} 
            \part $\frac{4}{5} + \frac{4}{15}$
            \vspace{\stretch{1}}

            \part $5 - \frac{1}{3}$
            \vspace{\stretch{1}}

            \part $\frac{5}{3} - \frac{2}{5} + \frac{4}{15}$
            \vspace{\stretch{1}}

            \part $\frac{3}{7} + \frac{3}{14} - \frac{7}{4}$
            \vspace{\stretch{1}}

            \part $\frac{1}{6} + \left(\frac{5}{3} - \frac{1}{2}\right)$
            \vspace{\stretch{1}}
        \end{parts}
   
        \newpage

        \question[5] Resuelve las siguientes operaciones con fracciones:
        \begin {parts} 
            \part $\frac{4}{5} \cdot \frac{15}{12}$
            \vspace{\stretch{1}}

            \part $\frac{3}{4} \cdot \frac{7}{5}$
            \vspace{\stretch{1}}

            \part $\frac{7}{3} : \frac{5}{3}$
            \vspace{\stretch{1}}

            \part $\frac{1}{6} + \left(\frac{5}{3} \cdot \frac{1}{2}\right)$
            \vspace{\stretch{1}}

            \part $\frac{1}{2} + \left(\frac{5}{3} : \frac{2}{3}\right)$
            \vspace{\stretch{1}}
        \end{parts}

    \end{questions}
 
\end{document}