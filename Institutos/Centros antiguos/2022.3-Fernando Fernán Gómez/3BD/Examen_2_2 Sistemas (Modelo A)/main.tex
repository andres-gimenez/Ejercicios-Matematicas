\documentclass[addpoints,spanish, 12pt,a4paper,cancelspace]{./include/gexam}

%%%%%%%%%%%%%%%%%%%%%%%%%%%
\renewcommand{\documentName} { Examen 2ª evaluación }
\renewcommand{\documentContent} { Sistemas de ecuaciones } 
\renewcommand{\waterMark} { Modelo A} 

% Configuración del documento.
\renewcommand{\schoolSubject} { Examen Matemáticas 2º ESO  }
\renewcommand{\school} { IES José de Churriguera  }
\renewcommand{\academicPeriod} { Curso 2022/2023 }

\renewcommand{\autor} { Andrés Giménez Muñoz }
\renewcommand{\emailAuthor} { andresprofemates@outlook.es }
\renewcommand{\autorSing}{ Profesor: Andrés } 
%%%%%%%%%%%%%%%%%%%%%%%%%%%

%%%%%%%%%%%%%%%%%%%%%%%%%%%
% Exam configuration
%\pointsdroppedatright   %% No mostrar la puntuación
\pointsinrightmargin % Para poner las puntuaciones a la derecha. Se puede cambiar. Si se comenta, sale a la izquierda.
\extrawidth{-1.5cm} %Un poquito más de margen por si ponemos textos largos.
\marginpointname{ \emph{\points}}

%% Si se comenta no aparecerán los espacios de la solución.
%\nocancelspace

%% Esto es de la clase exam. Si dejamos sin comentar \printanswers, se mostraran las soluciones. 
%% Si la comentamos y dejamos sin comentar \noprintanswers, pues no se muestran las soluciones.
%\printanswers
%\noprintanswers

%%%%%%%%%%%%%%%%%%%%%%%%%%%

\begin{document}

\StudentData
\GradeTableHeader

\justifying

\begin{questions}
    \setcounter{question}{0}

    \question[2]
    Resuelve el siguiente sistema de ecuaciones por el método gráfico.
    \begin{flushleft}
        $\begin{cases}
                \nonumber
                2x + 3y  = 14 \\
                \nonumber
                -x + 2y = 0
            \end{cases}$
    \end{flushleft}

    \begin{figure}[h]
        \begin{tikzpicture}[scale=1]
            \tkzInit[xmax=7,ymax=7,xmin=-7,ymin=-7]
            \tkzGrid[color=black!50]
            \tkzAxeXY
        \end{tikzpicture}
    \end{figure}

    \newpage
    \question[4]
    Resuelve los siguientes sistemas de ecuaciones:
    \begin{parts}
        \part
        Por el método de sustitución:
        \begin{flushleft}
            $\begin{cases}
                    \nonumber
                    2x - 3y  = -1 \\
                    \nonumber
                    3x +2y = 5
                \end{cases}$
        \end{flushleft}
        \vspace{\stretch{1}}

        \part
        Por el método de igualación.
        \begin{flushleft}
            $\begin{cases}
                    \nonumber
                    -3x - 4y = 5 \\
                    \nonumber
                    -2x + 3y = 9
                \end{cases}$
        \end{flushleft}
        \vspace{\stretch{1}}

        \part
        Por el método de reducción.
        \begin{flushleft}
            $\begin{cases}
                    \nonumber
                    x + 2y = 5 \\
                    \nonumber
                    x + 3y = 6
                \end{cases}$
        \end{flushleft}
        \vspace{\stretch{1}}
    \end{parts}

    \newpage

    \question[2]
    Dos trenes circulan por una misma vía. Sus velocidades son de 200km/h y de 150 km/h. 
    Si les separan 780 km. ¿Cuánto tardarán en encontrarse?
    \vspace{\stretch{1}}

    \question[2]
    En una granja entre corderos y pollos hay 120 animales. 
    Si las patas de todos ellos suman 450. ¿Cuántos corderos y cuántos pollos hay?
    \vspace{\stretch{1}}

\end{questions}
\end{document}