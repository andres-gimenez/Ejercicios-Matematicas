\documentclass[addpoints,spanish, 12pt,a4paper,cancelspace]{./include/gexam}

%%%%%%%%%%%%%%%%%%%%%%%%%%%
\renewcommand{\documentName} { Recuperación final junio }
\renewcommand{\documentContent} { 2º Evaluación } 
\renewcommand{\waterMark} { \phantom{Modelo A} } 

% Configuración del documento.
\renewcommand{\schoolSubject} { Examen Matemáticas 2º ESO  }
\renewcommand{\school} { IES José de Churriguera  }
\renewcommand{\academicPeriod} { Curso 2022/2023 }

\renewcommand{\autor} { Andrés Giménez Muñoz }
\renewcommand{\emailAuthor} { andresprofemates@outlook.es }
\renewcommand{\autorSing}{ Profesor: Andrés } 
%%%%%%%%%%%%%%%%%%%%%%%%%%%

%%%%%%%%%%%%%%%%%%%%%%%%%%%
% Exam configuration
%\pointsdroppedatright   %% No mostrar la puntuación
\pointsinrightmargin % Para poner las puntuaciones a la derecha. Se puede cambiar. Si se comenta, sale a la izquierda.
\extrawidth{-1.5cm} %Un poquito más de margen por si ponemos textos largos.
\marginpointname{ \emph{\points}}

%% Si se comenta no aparecerán los espacios de la solución.
%\nocancelspace

%% Esto es de la clase exam. Si dejamos sin comentar \printanswers, se mostraran las soluciones. 
%% Si la comentamos y dejamos sin comentar \noprintanswers, pues no se muestran las soluciones.
%\printanswers
%\noprintanswers

%%%%%%%%%%%%%%%%%%%%%%%%%%%

% \usepackage{tabularx}
% \usepackage{array}

\usepackage{caption}
\usepackage{subcaption}

\begin{document}

\StudentData
\GradeTableHeader

\justifying

\begin{center}
    \fbox{\fbox{\parbox{6.5in}{             
                \begin{itemize}
                    \item Copiar, hablar, levantarse de la silla o molestar al resto de la clase pueden ser motivos de la retirada del examen que se valorará con un cero.
                    \item Deben aparecer todas las operaciones, no vale solo con indicar el resultado.
                    \item Se podrán quitar hasta cinco décimas por falta de claridad o rigor en el desarrollo de las respuestas o por una mala presentación.
                    % \item Se valorará que se indiquen las cuentas en línea, realizando las operaciones en el margen.
                    \item Está permitido utilizar la calculadora.
                \end{itemize}
            }}}
\end{center}

\begin{questions}
    \setcounter{question}{0}

    \question[3]
    Resuelve los siguientes sistemas de ecuaciones, indicando el nombre del método que has utilizado:
    \begin{parts}
        \part
        % Por el método de sustitución:
        \begin{flushleft}
            $\begin{cases}
                    \nonumber
                    2x - y  = 1 \\
                    \nonumber
                    -4x + y = 25
                \end{cases}$
        \end{flushleft}
        \vspace{\stretch{1}}

        % \part
        % Por el método de igualación.
        % \begin{flushleft}
        %     $\begin{cases}
        %             \nonumber
        %             -3x - 4y = 5 \\
        %             \nonumber
        %             -2x + 3y = 9
        %         \end{cases}$
        % \end{flushleft}
        % \vspace{\stretch{1}}

        \part
        % Por el método de reducción.
        \begin{flushleft}
            $\begin{cases}
                    \nonumber
                    2x + 3y = 1 \\
                    \nonumber
                    x + y = -2
                \end{cases}$
        \end{flushleft}
        \vspace{\stretch{1}}
    \end{parts}

    \newpage

    \question[2]
    Resuelve el siguiente sistema de ecuaciones por el método gráfico.
    \begin{flushleft}
        $\begin{cases}
                \nonumber
                2x + 3y  = 14 \\
                \nonumber
                -x + 2y = 0
            \end{cases}$
    \end{flushleft}

    \begin{figure}[h]
        \centering
        \begin{subfigure}[b]{0.4\textwidth}
            \begin{tabular}{|w{c}{1cm}|w{c}{1cm}|}
                \multicolumn{2}{c}{$ 2x + 3y  = 14 $}   \\ \hline
                x & y \\ \hline
                &   \\
                &   \\
                &   \\
                &   \\
                &   \\
                &   \\
                &   \\ \hline
            \end{tabular}        \end{subfigure}
        \begin{subfigure}[b]{0.4\textwidth}
            \begin{tabular}{|w{c}{1cm}|w{c}{1cm}|}
                \multicolumn{2}{c}{$ -x + 2y = 0 $}   \\ \hline
                x & y \\ \hline
                &   \\
                &   \\
                &   \\
                &   \\
                &   \\
                &   \\
                &   \\ \hline
            \end{tabular}
        \end{subfigure}
    \end{figure}

    \vspace{\stretch{1}}

    \begin{figure}[h]
        \centering
        \begin{tikzpicture}[scale=0.7]
            \tkzInit[xmax=10,ymax=10,xmin=-10, ymin=-10]
            \tkzGrid[color=black!50]
            \tkzAxeXY
        \end{tikzpicture}
    \end{figure}

    \newpage
    \question[2]
    Resuelve las siguientes ecuaciones:
    \begin{parts}
        % \part
        % $-(-2x-1)-3(x+5)=x+12$
        % \vspace{\stretch{1}}

        \part
        $\frac{x-3}{2}-\frac{3x-5}{2} = -1$
        \vspace{\stretch{1}}

        \part
        $x-13=4\left[3x-4\left(x-2\right)\right]$
        \vspace{\stretch{1}}
    \end{parts}

    \newpage

    \question[3]
    Resuelve las siguientes ecuaciones de segundo y tercer grado
    \begin{parts}
        \part 
        $(x-2) (x+2) (x-3) = 0$
        \vspace{\stretch{1}}

        \part 
        $3x^2+3x=2x^2$
        \vspace{\stretch{1}}

        \part 
        $x^2+x-6=0$
        \vspace{\stretch{1}}

    \end{parts}


    % \newpage
    % \question[2]
    % Dada la función de la gráfica, identifica el dominio, la imagen, continuidad y puntos de discontinuidad, intervalos de crecimiento y decrecimiento, máximos y mínimos relativos y absolutos.

    % \begin{tikzpicture}[scale=1]
    %     \begin{axis}[
    %             %title={$f(x)=x^2-10$},
    %             axis lines=middle,
    %             axis line style={<->},
    %             legend style={inner ysep=7pt},
    %             %x label style={at={(axis description cs:0.5,-0.06)},anchor=north},
    %             %y label style={at={(axis description cs:-0.06,.5)},rotate=90,anchor=south},
    %             %xlabel={Nº de bombillas},ylabel={Vida en horas},
    %             no markers,
    %             black!50,
    %             grid,
    %             width=15cm,
    %             xmax=20,ymax=20,xmin=-20,ymin=-20,
    %             %axis lines=middle
    %             % restrict y to domain=-20:20
    %         ]
    %         \addplot[domain=-15:-5,black] {-x-10};
    %         \addplot[domain=-5:0,black] {x*x+6*x};
    %         \addplot[domain=0:6,black] {x};
    %         \addplot[domain=6:15,black] {-5};
    %         \draw[dashed] (axis cs:6,6) -- (axis cs:6,-5);
    %         \addplot[holdotBlack] coordinates{(-15,5)(6,-5)};
    %         \addplot[soldotBlack] coordinates{(6,6)(15,-5)};
    %     \end{axis}
    % \end{tikzpicture}
    % \\
    % \renewcommand{\arraystretch}{1.8}
    % \begin{table}[h]
    %     \begin{tabular}{|l|p{10cm}|}   
    %         \hline
    %         Dominio:                     &  \\ \hline
    %         Imagen o recorrido:          &  \\ \hline
    %         Continuidad:                 &  \\ \hline
    %         Puntos de discontinuidad:    &  \\ \hline
    %         Intervalos de crecimiento:   &  \\ \hline
    %         Intervalos de decrecimiento: &  \\ \hline
    %         Máximos relativos:           &  \\ \hline
    %         Mínimos relativos:           &  \\ \hline
    %         Máximo absoluto:             &  \\ \hline
    %         Mínimo absoluto:             &  \\ \hline
    %     \end{tabular}
    % \end{table}

\end{questions}
\end{document}