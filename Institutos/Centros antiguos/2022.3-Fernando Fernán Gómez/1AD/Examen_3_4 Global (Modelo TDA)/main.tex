\documentclass[addpoints,spanish, 12pt,a4paper,cancelspace]{./include/gexam}

 %%%%%%%%%%%%%%%%%%%%%%%%%%%
 \renewcommand{\documentName} { Examen global 3ª evaluación }
 \renewcommand{\documentContent} { Regla de tres, lenguaje algebraico y ecuaciones de 1º grado } 
 \renewcommand{\waterMark} { Modelo TDA } 

 % Configuración del documento.
 \renewcommand{\schoolSubject} { Examen Matemáticas 2º ESO  }
\renewcommand{\school} { IES José de Churriguera  }
\renewcommand{\academicPeriod} { Curso 2022/2023 }

\renewcommand{\autor} { Andrés Giménez Muñoz }
\renewcommand{\emailAuthor} { andresprofemates@outlook.es }
\renewcommand{\autorSing}{ Profesor: Andrés } 
 %%%%%%%%%%%%%%%%%%%%%%%%%%%
 
% \renewcommand{\thepartno}{\arabic{partno}}
%  \renewcommand{\thepartno}{\thecurrentpartno.\arabic{partno}}

% \renewcommand{\partlabel}{(\thequestion.\arabic{partno})}
% \renewcommand{\subpartlabel}{(\thepart.\arabic{subpartno})}

\renewcommand\subpartlabel{(\thesubpart)}
\renewcommand\subpartshook{\renewcommand\makelabel[1]{##1\hfil }} 
 
 %%%%%%%%%%%%%%%%%%%%%%%%%%%
 % Exam configuration
 %\pointsdroppedatright   %% No mostrar la puntuación
 \pointsinrightmargin{} % Para poner las puntuaciones a la derecha. Se puede cambiar. Si se comenta, sale a la izquierda.
 \extrawidth{-1.5cm} %Un poquito más de margen por si ponemos textos largos.
 \marginpointname{ \emph{\points}}
 
 %% Si se comenta no aparecerán los espacios de la solución.
 %\nocancelspace
 
 %% Puntuación a la izquierda.
%  \nopointsinrightmargin 

 %% Esto es de la clase exam. Si dejamos sin comentar \printanswers, se mostraran las soluciones. 
 %% Si la comentamos y dejamos sin comentar \noprintanswers, pues no se muestran las soluciones.
 % \printanswers
 %\noprintanswers
 
 %%%%%%%%%%%%%%%%%%%%%%%%%%%
 
 \begin{document}
 
 \StudentData{}
 \GradeTableHeader{}
 
 \justifying{}

 \Large

\begin{center}
    \fbox{\fbox{\parbox{6.5in}{             
                \begin{itemize}
                    \item Copiar, hablar, levantarse de la silla o molestar al resto de la clase pueden ser motivos de la retirada del examen que se valorará con un cero.
                    \item Deben aparecer todas las operaciones, no vale solo con indicar el resultado.
                    \item Se podrán quitar hasta cinco décimas por falta de claridad o rigor en el desarrollo de las respuestas o por una mala presentación.
                    \item Se valorará que se indiquen las cuentas en línea, realizando las operaciones en el margen.
                    \item No se puede utilizar la calculadora.
                \end{itemize}
            }}}
\end{center}
 
 \begin{questions}
    
    \question[3] Realiza las siguientes operaciones:
    \begin {parts}
        \part $6x^2+3x^2$
        \vspace{\stretch{1}}
        \part $x-8x$
        \vspace{\stretch{1}}
        \part $x^2-4x-3x^2$
        \vspace{\stretch{1}}
        \part $2x \cdot \left(-4x^3\right)$
        \vspace{\stretch{1}}
        \part $3x^2 \cdot 7x^4$
        \vspace{\stretch{1}}

        \newpage{}

        \part $2x^3-x^2-2 + 7x^3-2x^2+3x-1$
        \vspace{\stretch{1}}

        \part $2x^3-x^2-2- 7x^3+2x^2-3x+1$
        \vspace{\stretch{1}}
    \end {parts}

    \newpage

    \question[1] Expresa en lenguaje algebraico:
    \begin {parts}
        \part El cuadrado de un número menos su doble.
        \vspace{\stretch{1}}

        \part Un número disminuido en tres unidades.
        \vspace{\stretch{1}}

        \part El cubo de un número.
        \vspace{\stretch{1}}

        \part La quinta parte de un número.
        \vspace{\stretch{1}}
    \end {parts}
    
    \question[2] Halla el valor numérico de las siguientes expresiones algebraicas con los valores dados:
    \begin {parts}
        \part $4x^3+5x^2$ para $x=-2$
        \vspace{\stretch{2}}

        \part $2x^2+x-15$ para $x=3$
        \vspace{\stretch{2}}
    \end {parts}

    \newpage{}
    
    \question[2] Resuelve las siguientes ecuaciones
    \begin {parts}
        \part $3x - 7 = 8 - 2x$
        \vspace{\stretch{1}}

        \part $-x-x+8=7x-6-4$
        \vspace{\stretch{1}}

        \newpage{}
    
        \part $x-3\left(x-2\right) = 6x-2$
        \vspace{\stretch{1}}

        \part $3x-9=2x+2$
        \vspace{\stretch{1}}
    \end {parts}

    \newpage{}

    \question[2] Irene ha recibido 25 \euro{} por un trabajo de reparto de publicidad durante 5 horas. 
    ¿Cuánto recibirá Eduardo, que ha trabajado 4 horas? 
    \vspace{\stretch{1}}
    
\end{questions}
 
\end{document}