\documentclass[addpoints,spanish, 12pt,a4paper,cancelspace]{./include/gexam}

 %%%%%%%%%%%%%%%%%%%%%%%%%%%
 \renewcommand{\documentName} { 3ª evaluación }
 \renewcommand{\documentContent} { Lenguaje algebraico y ecuaciones de 1º grado } 
 \renewcommand{\waterMark} { Modelo B } 

 % Configuración del documento.
 \renewcommand{\schoolSubject} { Examen Matemáticas 2º ESO  }
\renewcommand{\school} { IES José de Churriguera  }
\renewcommand{\academicPeriod} { Curso 2022/2023 }

\renewcommand{\autor} { Andrés Giménez Muñoz }
\renewcommand{\emailAuthor} { andresprofemates@outlook.es }
\renewcommand{\autorSing}{ Profesor: Andrés } 
 %%%%%%%%%%%%%%%%%%%%%%%%%%%
 
% \renewcommand{\thepartno}{\arabic{partno}}
%  \renewcommand{\thepartno}{\thecurrentpartno.\arabic{partno}}

% \renewcommand{\partlabel}{(\thequestion.\arabic{partno})}
% \renewcommand{\subpartlabel}{(\thepart.\arabic{subpartno})}

\renewcommand\subpartlabel{(\thesubpart)}
\renewcommand\subpartshook{\renewcommand\makelabel[1]{##1\hfil }} 
 
 %%%%%%%%%%%%%%%%%%%%%%%%%%%
 % Exam configuration
 %\pointsdroppedatright   %% No mostrar la puntuación
 \pointsinrightmargin{} % Para poner las puntuaciones a la derecha. Se puede cambiar. Si se comenta, sale a la izquierda.
 \extrawidth{-1.5cm} %Un poquito más de margen por si ponemos textos largos.
 \marginpointname{ \emph{\points}}
 
 %% Si se comenta no aparecerán los espacios de la solución.
 %\nocancelspace
 
 %% Puntuación a la izquierda.
%  \nopointsinrightmargin 

 %% Esto es de la clase exam. Si dejamos sin comentar \printanswers, se mostraran las soluciones. 
 %% Si la comentamos y dejamos sin comentar \noprintanswers, pues no se muestran las soluciones.
 % \printanswers
 %\noprintanswers
 
 %%%%%%%%%%%%%%%%%%%%%%%%%%%
 
 \begin{document}
 
 \StudentData{}
 \GradeTableHeader{}
 
 \justifying{}

 \justifying

% \begin{center}
%     \fbox{\fbox{\parbox{6.5in}{             
%                 \begin{itemize}
%                     \item Deben aparecer todas las operaciones, no vale solo con indicar el resultado.
%                     \item Se podrán quitar hasta cinco décimas por falta de claridad o rigor en el desarrollo de las respuestas o por una mala presentación.
%                     \item Se valorará que se indiquen las cuentas en línea, realizando las operaciones en el margen.
%                     \item No se puede utilizar la calculadora.
%                 \end{itemize}
%             }}}
% \end{center}
 
 \begin{questions}
    
    \question[1] Expresa las siguientes frases en lenguaje algebraico:
    \begin {parts}
        \part El cuadrado de un número menos su cuarta parte.
        \vspace{10mm}
        \part La mitad de un número menos cuatro.
        \vspace{10mm}
    \end{parts} 

    \question[2] Reduce las siguientes expresiones algebraicas todo lo que se pueda:
    \begin {parts}
        \part $\frac{2}{3}x^2+\frac{1}{5}x^2 = $
        \vspace{\stretch{1}}
        \part $2x^2+3x-5x^2+5x-4 = $
        \vspace{\stretch{1}}
        \part $4x \cdot 3x^2 = $
        \vspace{\stretch{1}}
        \part $8x \cdot \left(-3x^2\right) = $
        \vspace{\stretch{1}}
        \part $3 \cdot 7x^3 =$
        \vspace{\stretch{1}}
    \end {parts}

    \newpage{}

    \question[2] Elimina los paréntesis y simplifica las siguientes expresión algebraica:
    \begin {parts}
        \part $6x^2-\left(4x^2+3x^2\right) = $
        \vspace{\stretch{1}}
        \part $-\left(-2x+3\right)-\left(5x+2\right) = $
        \vspace{\stretch{1}}
    \end {parts}

    \question[2] Resuelve las siguientes ecuaciones
    \begin {parts}
        \part $3x+11 = 2$
        \vspace{\stretch{1}}
        \part $-2x + 16 = 18$
        \vspace{\stretch{1}}
        \part $x + 4 = 2$
        \vspace{\stretch{1}}
        \part $3+5x-3x = -5+4x$
        \vspace{\stretch{1}}
    \end {parts}

    \newpage

    \question[1] Calcula el valor de la expresión $-2x^2+3x-2$ para los siguientes valores de $x$:
    \begin {parts}
        \part $x=3$
        \vspace{\stretch{1}}
        \part $x=-3$
        \vspace{\stretch{1}}
    \end{parts}

    \question[2] Resuelve las siguientes ecuaciones complejas
    \begin {parts}
        \part $4x + 2(x-2)= 8$
        \vspace{\stretch{2}}
        \part $3x-(5x+2)=2x+6$
        \vspace{\stretch{2}}
    \end{parts}

\end{questions}
 
\end{document}