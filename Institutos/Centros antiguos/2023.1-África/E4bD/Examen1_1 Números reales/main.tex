\documentclass[addpoints,spanish, 12pt,a4paper,cancelspace]{./include/gexam}

%%%%%%%%%%%%%%%%%%%%%%%%%%%
\renewcommand{\documentName} { 1ª evaluación 2º examen }
\renewcommand{\documentContent} { Expresiones algebraicas y ecuaciones } 
\renewcommand{\waterMark} { Modelo A } 

% Configuración del documento.
\renewcommand{\schoolSubject} { Examen Matemáticas 2º ESO  }
\renewcommand{\school} { IES José de Churriguera  }
\renewcommand{\academicPeriod} { Curso 2022/2023 }

\renewcommand{\autor} { Andrés Giménez Muñoz }
\renewcommand{\emailAuthor} { andresprofemates@outlook.es }
\renewcommand{\autorSing}{ Profesor: Andrés } 
%%%%%%%%%%%%%%%%%%%%%%%%%%%

%%%%%%%%%%%%%%%%%%%%%%%%%%%
% Exam configuration
%\pointsdroppedatright   %% No mostrar la puntuación
\pointsinrightmargin % Para poner las puntuaciones a la derecha. Se puede cambiar. Si se comenta, sale a la izquierda.
\extrawidth{-1.5cm} %Un poquito más de margen por si ponemos textos largos.
\marginpointname{ \emph{\points}}

%% Si se comenta no aparecerán los espacios de la solución.
%\nocancelspace

%% Esto es de la clase exam. Si dejamos sin comentar \printanswers, se mostraran las soluciones. 
%% Si la comentamos y dejamos sin comentar \noprintanswers, pues no se muestran las soluciones.
%\printanswers
%\noprintanswers

%%%%%%%%%%%%%%%%%%%%%%%%%%%

\begin{document}

\StudentData{}
\GradeTableHeader{}

\justifying{}

\begin{center}
    \fbox{\fbox{\parbox{6.5in}{             
                \begin{itemize}
                    \item Copiar, hablar, levantarse de la silla o molestar al resto de la clase pueden ser motivos de la retirada del examen que se valorará con un cero.
                    \item Deben aparecer todas las operaciones, no vale solo con indicar el resultado.
                    % \item Se podrán quitar hasta cinco décimas por falta de claridad o rigor en el desarrollo de las respuestas o por una mala presentación.
                    % \item Se valorará que se indiquen las cuentas en línea, realizando las operaciones en el margen.
                    \item Está permitido utilizar la calculadora.
                \end{itemize}
            }}}
\end{center}

\begin{questions}
    \setcounter{question}{0}

    \question[1] Expresa las siguientes operaciones con conjuntos como un solo intervalo:
        \begin{parts}
            \part $\left[-5, 6\right) \cup \left(4, 8 \right)$
            \vspace{\stretch{1}}

            \part $\left(-\infty, 4\right] \cap \left(3, 9\right)$
            \vspace{\stretch{1}}

        \end{parts}
    
    \question[2] Realiza las siguientes operaciones con notación científica:
    \begin{parts}
        \part $\left(7,28 \cdot 10^{83}\right) \cdot \left(6,2 \cdot 10^{-57}\right)$
        \vspace{\stretch{1}}

        \part $2,25 \cdot 10^{34} + 9,35 \cdot 10^{35}$
        \vspace{\stretch{1}}

        \part $2,25 \cdot 10^{-15} - 2,25 \cdot 10^{-16}$
        \vspace{\stretch{1}}
    \end{parts}
    \newpage 

    \question[2] Simplifica las siguientes expresiones con radicales, extrayendo del radical todo lo que se pueda:
    \begin{parts}
        \part $\sqrt[3]{5^9}$
        \vspace{\stretch{1}}

        \part $\sqrt[5]{\frac{2^{23} \cdot 3^{10}}{5^7}}$
        \vspace{\stretch{1}}

        % Han tenido problema en sacar que 27436 es divisible entre 19. 
        \part $\sqrt{\frac{9375}{27436}}$
        \vspace{\stretch{1}}

        \part $\sqrt[9]{\frac{a^{30} \cdot b^{24}}{c^9 \cdot d^{78}}}$
        \vspace{\stretch{1}}
    \end{parts}
    
    \question[2] Racionaliza las siguientes fracciones con radicales, simplificando todo lo que se pueda:
    \begin{parts}
        \part $-\frac{5}{\sqrt{5}}$
        \vspace{\stretch{1}}

        \part $\frac{6 - \sqrt{3}}{\sqrt{3}}$
        \vspace{\stretch{1}}

        \part $\frac{9}{6-\sqrt{15}}$
        \vspace{\stretch{1}}

        \part $\frac{2+\sqrt{9}}{\sqrt{25}}$
        \vspace{\stretch{1}}
    \end{parts}
    \newpage 

    % Les han resultado muy fácil y tiran mucho de calculadora.
    \question[3] Resuelve los siguientes logaritmos:
    \begin{parts}
        \part $\log_2{32}$
        \vspace{\stretch{1}}

        \part $\log_5{\left(0,04\right)}$
        \vspace{\stretch{1}}

        \part $\log_{\frac{1}{2}} {\left(128\right)}$
        \vspace{\stretch{1}}

        \part $\log \left(10.000\right)$
        \vspace{\stretch{1}}

        \part $\log_2\left(\frac{1}{256}\right)$
        \vspace{\stretch{1}}

        \part $\log_3{\left(729\right)}$
        \vspace{\stretch{1}}
    \end{parts}

\end{questions}
\end{document}