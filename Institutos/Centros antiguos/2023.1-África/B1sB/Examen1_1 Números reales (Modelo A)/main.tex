\documentclass[addpoints,spanish, 12pt,a4paper,cancelspace]{./include/gexam}

%%%%%%%%%%%%%%%%%%%%%%%%%%%
\renewcommand{\documentName} { 1ª evaluación }
\renewcommand{\documentContent} { Números reales } 
\renewcommand{\waterMark} { Modelo A }

% Configuración del documento.
\renewcommand{\schoolSubject} { Examen Matemáticas 2º ESO  }
\renewcommand{\school} { IES José de Churriguera  }
\renewcommand{\academicPeriod} { Curso 2022/2023 }

\renewcommand{\autor} { Andrés Giménez Muñoz }
\renewcommand{\emailAuthor} { andresprofemates@outlook.es }
\renewcommand{\autorSing}{ Profesor: Andrés } 
%%%%%%%%%%%%%%%%%%%%%%%%%%%

%%%%%%%%%%%%%%%%%%%%%%%%%%%
% Exam configuration
%\pointsdroppedatright   %% No mostrar la puntuación
\pointsinrightmargin % Para poner las puntuaciones a la derecha. Se puede cambiar. Si se comenta, sale a la izquierda.
\extrawidth{-1.5cm} %Un poquito más de margen por si ponemos textos largos.
\marginpointname{ \emph{\points}}

%% Si se comenta no aparecerán los espacios de la solución.
%\nocancelspace

%% Esto es de la clase exam. Si dejamos sin comentar \printanswers, se mostraran las soluciones. 
%% Si la comentamos y dejamos sin comentar \noprintanswers, pues no se muestran las soluciones.
%\printanswers
%\noprintanswers

%%%%%%%%%%%%%%%%%%%%%%%%%%%

\begin{document}

\StudentData{}
\GradeTableHeader{}

\justifying{}

\begin{center}
    \fbox{\fbox{\parbox{6.5in}{             
                \begin{itemize}
                    \item Copiar, hablar, levantarse de la silla o molestar a al resto de la clase pueden ser motivos de la retirada del examen que se valorará con un cero.
                    \item Deben aparecer todas las operaciones, no vale solo con indicar el resultado.
                    % \item Se podrán quitar hasta cinco décimas por falta de claridad o rigor en el desarrollo de las respuestas o por una mala presentación.
                    % \item Se valorará que se indiquen las cuentas en línea, realizando las operaciones en el margen.
                    \item Está permitido utilizar la calculadora.
                \end{itemize}
            }}}
\end{center}

\begin{questions}
    \setcounter{question}{0}

    \question[2] Racionaliza las siguientes expresiones:
        \begin{parts}
            \part $\frac{2\sqrt{2}}{\sqrt{3}-\sqrt{2}}$
            \vspace{\stretch{1}}

            \part $\frac{\sqrt{3}}{\sqrt{2}}$
            \vspace{\stretch{1}}

            \part $\frac{2+\sqrt{9}}{\sqrt{25}}$
            \vspace{\stretch{1}}

            \part $\frac{\sqrt{216.000}}{\sqrt{38.808}}$
            \vspace{\stretch{1}}
        \end{parts}
    \newpage 

    \question[2] Calcula, desarrollando el proceso de cálculo y sin usar la calculadora, el valor de los siguientes logaritmos:
        \begin{parts}
            \part $\log_2 4$ 
            \vspace{\stretch{1}}

            % \part $\log_2{1024}$
            % \vspace{\stretch{1}}

            \part $\log_4{1024}$
            \vspace{\stretch{1}}

            % \part $\log_2 8$
            % \part $\log_2 32$
            % \part $\log_2 \frac{1}{16}$
            \part $\log_3 81$
            \vspace{\stretch{1}}
            % \part $\log_3 729$
            % \part $\log_3 \frac{1}{81}$
            % \part $\log_4 (8) + \log_4(2)$ 
            \part $\log_4 4^5$
            \vspace{\stretch{1}}
            % \part $\log_5 (125) + \log_5(5)$    
            % \part $\log_5 \left(100\right) + \log_5 \left(\frac{1}{4}\right)$ 
            % \part $\log_6 36^2$
            % \part $\log_{17} {17} $
            % \part $\log_8 1$
            % \part $\log_5 10$
            \part $\log 100000000$
            \vspace{\stretch{1}}
            % \part $\log \frac{1}{10000}$
            \part $\log 0,001$
            \vspace{\stretch{1}}
        \end{parts}
    \newpage 

    \question[3] Halla el valor de x en estas expresiones aplicando las propiedades de los logaritmos:
        \begin{parts}
            \part $\ln{x} = \ln{8} + \ln{2}$
            \vspace{\stretch{1}}

            \part $\ln{x} = \ln{5} + \ln{2} - \ln{10}$
            \vspace{\stretch{1}}

            % \part $\log{x} = \log{36} - \log{6}$
            % \vspace{\stretch{1}}

            \part $\log{x} = 4\log{2} - \frac{1}{2} \log{25}$
            \vspace{\stretch{1}}

            % \part $\ln{x} = 3 \ln{4}$
            % \vspace{\stretch{1}}

            % \part $\log{x} = 3 \log{2} - \frac{1}{4} \log{16}$
            % \vspace{\stretch{1}}
        \end{parts}
    \newpage 

    \question[3] Resuelve las siguientes ecuaciones
        \begin{parts}
            \part $12^{x-3} = 3^x$
            \vspace{\stretch{1}}

            % \part $3^{2x+2} = 9 \cdot 3^x - 2$
            % \vspace{\stretch{1}}

            \part $\log \left(3x+13\right) = 2$
            \vspace{\stretch{1}}

            % \part $12^{x-2} = 4^x$
            % \part $2^{2x+2} = 9 \cdot 2^x - 2$
            % \part $\left(8^{x+2}\right)\left(4^{x-6}\right) = 16$

            % \part $\log{\left(2x + 6\right)} = 2$
            % \part $\log{\left(x + 3\right)} - \log{\left(2x -2\right)} = 1 - \log{5}$

            % \part $\frac{3}{2} + \log_x{2} = 2$
            % \part $\frac{3}{2} \cdot 10^{-2} = e^x$

            % \part $2^{x+1} + 4 = 80$
            \part $2 \cdot 3^x = 3^{2x} - 3$
            \vspace{\stretch{1}}
            % \part $3^{2x-3} = 8^{x+1}$
            % \part $3^{x+2} + 9^{x+1} = 810$
            % \part $2^{x-3} = -3$
            % \part $5^{x-1} = 2 + \frac{2}{5^{x-2}}$
            % \part $2 e^{x-4} = 3$
            % \part $100 \cdot 10^x = \sqrt[x]{1000^5}$

            % \part $\log{\left(\log{x}\right)} = 1 $
        \end{parts}

\end{questions}
\end{document}