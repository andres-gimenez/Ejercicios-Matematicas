%%%%%%%%%%%%%%%%%%%%%%%%%%%
\newcommand{\numeroHoja} { Hoja 2.5 }
\newcommand{\nombreHoja} { Identidades notables }
%%%%%%%%%%%%%%%%%%%%%%%%%%%

% Configuración del documento.
\renewcommand{\schoolSubject} { Examen Matemáticas 2º ESO  }
\renewcommand{\school} { IES José de Churriguera  }
\renewcommand{\academicPeriod} { Curso 2022/2023 }

\renewcommand{\autor} { Andrés Giménez Muñoz }
\renewcommand{\emailAuthor} { andresprofemates@outlook.es }
\renewcommand{\autorSing}{ Profesor: Andrés } 
\newcommand{\schoolSubject} { Matemáticas 3º ESO - Recuperación}
\newcommand{\school} { IES La Serna }
\newcommand{\academicPeriod} { Curso 2020/2021 }


\newcommand{\autor} { Andrés Giménez Muñoz }
\newcommand{\emailAuthor} { agimenezmunoz@ieslaserna.com }
\newcommand{\autorSing}{ Profesores: Andrés } 

%%%%%%%%%%%%%%%%%%%%%%%%%%%
% Exam configuration
\pointsdroppedatright   %% No mostrar la puntuación

%% Si se comenta no aparecerán los espacios de la solución.
%\nocancelspace

%% Esto es de la clase exam. Si dejamos sin comentar \printanswers, se mostraran las soluciones. 
%% Si la comentamos y dejamos sin comentar \noprintanswers, pues no se muestran las soluciones.
\printanswers
%\noprintanswers

%%%%%%%%%%%%%%%%%%%%%%%%%%%

\begin{document}

    %%\StudentData

    \begin{questions}
        \question
        Demuestra que el cuadrado de una suma es:
            \begin{equation*}
                (a+b)^2 = a^2 + b^2 + 2ab
            \end{equation*}
            \begin{solution}
                \begin{multline*}
                    (a+b)^2 = (a+b)(a+b) = a^2 +ab + ba + b^2 =  \cdots \\
                        \cdots = a^2 + b^2 + ab + ba = a^2 + b^2 + ab + ab = a^2 + b^2 + 2ab
                \end{multline*}
            \end{solution}
        \question
        Demuestra que el cuadrado de una resta es:
            \begin{equation*}
                (a-b)^2 = a^2 + b^2 - 2ab
            \end{equation*}
            \begin{solution}
                \begin{multline*}
                    (a-b)^2 = (a-b)(a-b) = a^2 - ab - ba + b^2 =  \cdots \\
                        \cdots = a^2 + b^2 - ab - ba = a^2 + b^2 - ab - ab = a^2 + b^2 - 2ab
                \end{multline*}
            \end{solution}
        \question
        Demuestra que suma por diferencia es la diferencia de cuadrados:
            \begin{equation*}
                (a+b)(a-b) = a^2-b^2 
            \end{equation*}
            \begin{solution}
                \begin{equation*}
                    (a+b)(a-b) = a^2 + ab - ba - b^2 = a^2 + ab - ab - b^2 = a^2 - b^2                    
                \end{equation*}
            \end{solution}
        \question
        Desarrolla las siguientes expresiones algebraicas utilizando las igualdades notables:
        \begin{parts}
            \part
                $(x^3 + 2x)^2$
                \begin{solution}
                    {\footnotesize (*) Usamos $(a+b)^2 = a^2 + b^2 + 2ab$}
                    \begin{equation*}
                        (x^3 + 2x)^2 = \left(x^3\right)^2 + (2x)^2 + 2(x^3)(2x) = x^6 + 4x^2 + 4x^4 = \boxed{x^6 + 4x^4 + 4x^2}
                    \end{equation*}
                \end{solution}
            \part
                $(2x^3-5)^2$
                \begin{solution}
                    {\footnotesize (*) Usamos $(a-b)^2 = a^2 + b^2 - 2ab$}
                    \begin{equation*}
                        (2x^3-5)^2 = \left(2x^3\right)^2 + 5^2 - 2(2x^3)5 = 4x^6 + 25 - 20x^3 = \boxed{4x^6 - 20x^3 + 25 }
                    \end{equation*}
                \end{solution}
            \part
                $(5x-6)(5x+6)$
                \begin{solution}
                    {\footnotesize (*) Usamos $(a+b)(a-b) = a^2-b^2$ }
                    \begin{equation*}
                        (5x-6)(5x+6) = \left(5x\right)^2 - \left(6\right)^2 = 5^2 \cdot x^2 - 36 = \boxed{25x^2 - 36 }
                    \end{equation*}
                \end{solution}
        \end{parts}

    \end{questions}
\end{document}