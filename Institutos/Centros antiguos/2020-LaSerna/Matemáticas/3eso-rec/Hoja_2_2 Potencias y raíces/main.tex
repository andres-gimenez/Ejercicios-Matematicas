%%%%%%%%%%%%%%%%%%%%%%%%%%%

\newcommand{\numeroHoja} { Hoja 2.2 }
\newcommand{\nombreHoja} { Potencias y raíces }
%%%%%%%%%%%%%%%%%%%%%%%%%%%

% Configuración del documento.
\renewcommand{\schoolSubject} { Examen Matemáticas 2º ESO  }
\renewcommand{\school} { IES José de Churriguera  }
\renewcommand{\academicPeriod} { Curso 2022/2023 }

\renewcommand{\autor} { Andrés Giménez Muñoz }
\renewcommand{\emailAuthor} { andresprofemates@outlook.es }
\renewcommand{\autorSing}{ Profesor: Andrés } 
\newcommand{\schoolSubject} { Matemáticas 3º ESO - Recuperación}
\newcommand{\school} { IES La Serna }
\newcommand{\academicPeriod} { Curso 2020/2021 }


\newcommand{\autor} { Andrés Giménez Muñoz }
\newcommand{\emailAuthor} { agimenezmunoz@ieslaserna.com }
\newcommand{\autorSing}{ Profesores: Andrés } 

%%%%%%%%%%%%%%%%%%%%%%%%%%%
% Exam configuration
\pointsdroppedatright   %% No mostrar la puntuación

%% Si se comenta no aparecerán los espacios de la solución.
%\nocancelspace

%% Esto es de la clase exam. Si dejamos sin comentar \printanswers, se mostraran las soluciones. 
%% Si la comentamos y dejamos sin comentar \noprintanswers, pues no se muestran las soluciones.
\printanswers
%\noprintanswers

%%%%%%%%%%%%%%%%%%%%%%%%%%%

\begin{document}

    %%\StudentData

    \begin{center}
		\fbox{\fbox{\parbox{5.5in}{\centering
		Estos ejercicios están destinados a ejercitar la destreza en la aritmética. No uséis la calculadora para realizarlos. En ninguno de ellos debéis obtener decimales.}}}
	\end{center}

    \begin{questions}
        \question
        Expresa los productos en forma de potencia.
        \begin{parts}
            \part
            $6 \cdot 6 \cdot 6 \cdot 6 \cdot 6 \cdot 6 \cdot 6$
                \begin{solution}
                    $6 \cdot 6 \cdot 6 \cdot 6 \cdot 6 \cdot 6 \cdot 6 = 6^7$
                \end{solution}
            \part
            $3 \cdot 3 \cdot 3 \cdot 3 \cdot 3 \cdot 3 \cdot 3 \cdot 3 \cdot 3 \cdot 3 \cdot 3 = 3^{11}$
                \begin{solution}
                    $3 \cdot 3 \cdot 3 \cdot 3 \cdot 3 \cdot 3 \cdot 3 \cdot 3 \cdot 3 \cdot 3 \cdot 3 = 3$
                \end{solution}
            \part
            $(-1) \cdot (-1) \cdot (-1) \cdot (-1) \cdot (-1)$
            \begin{solution}
                $(-1) \cdot (-1) \cdot (-1) \cdot (-1) \cdot (-1) = (-1)^5$
            \end{solution}
            \part
            $5 \cdot 5 \cdot 5 \cdot 5 \cdot 5$
            \begin{solution}
                $5 \cdot 5 \cdot 5 \cdot 5 \cdot 5 = 5^5$
            \end{solution}
        \end{parts}

        \question
        Calcula las siguiente potencias.
        \begin{parts}
            \part
                $5^2$
                    \begin{solution}
                        $5^2 = 25$
                    \end{solution}
            \part
                $2^8$
                    \begin{solution}
                        $2^8 = 256$
                    \end{solution}
            \part
                $3^5$
                    \begin{solution}
                        $3^5 = 243$
                    \end{solution}
            \part
                $(-3)^4$
                    \begin{solution}
                        $(-3)^4 = 81$
                    \end{solution}
            \part
                $(-2)^5$
                    \begin{solution}
                        $(-2)^5 = -32$
                    \end{solution}
            \part
                $(-11)^2$
                    \begin{solution}
                        $(-11)^2 = 121$
                    \end{solution}
            \part
                $(-1)^{34}$
                    \begin{solution}
                        $(-1)^{34} = 1$
                    \end{solution}
            \part
                $(-3)^5$
                    \begin{solution}
                        $(-3)^5 = -243$
                    \end{solution}
            \part
                $7^3$
                    \begin{solution}
                        $(-3)^5 = 343$
                    \end{solution}
        \end{parts}

        \question
        Expresa como una única potencia.
        \begin{parts}
            \part
                $2^6 \cdot 5^6 \cdot 7^6$
                    \begin{solution}
                        $2^6 \cdot 5^6 \cdot 7^6 = {\left(2 \cdot 5 \cdot 7 \right)}^6 = \boxed{70^6}$
                    \end{solution}
            \part
                $21^8 : 3^8$
                    \begin{solution}
                        $21^8 : 3^8 = \boxed{{21 : 3}^8} = 7^8$
                    \end{solution}
            \part
                $12^5 : 4^5 : 3^5$
                    \begin{solution}
                        $12^5 : 4^5 : 3^5 = \boxed{{\left( 12 : 4 : 3 \right)}^5} = {\left(\frac{12}{4 \cdot 3}\right)}^5 = {\left(\frac{12}{12}\right)}^5 = {1}^5 = 1 $
                    \end{solution}
        \end{parts}

        \question
        Calcula
        \begin{parts}
            \part 
                $\frac{16 \cdot 81 \cdot 25}{12 \cdot 300}$
                \begin{solution}
                    \begin{multline*}
                        \frac{16 \cdot 81 \cdot 25}{12 \cdot 300} = \frac{2^4 \cdot 3^4 \cdot 5^2}{\left(2^2 \cdot 3\right) \cdot 3 \cdot 10^2}
                         = \frac{2^4 \cdot 3^4 \cdot 5^2}{2^2 \cdot 3 \cdot 3\cdot \left(2 \cdot 5\right)^2} = \dots \\
                         \dots = \frac{2^4 \cdot 3^4 \cdot 5^2}{2^2 \cdot 3 \cdot 3 \cdot 2^2 \cdot 5^2} = 
                         \frac{2^4}{2^2 \cdot 2^2} \cdot \frac{3^4}{3 \cdot 3} \cdot \frac{5^2}{5^2} = \dots \\ 
                         \dots = 2^{(4-2-2)} \cdot 3^{(4-1-1)} \cdot 5^{(2-2)}
                         = 2^0 \cdot 3^2 \cdot 5^0 = \boxed{3^2} =  9
                    \end{multline*}
                \end{solution}
            \part 
                $\frac{16^3 \cdot 12^5}{8^5 \cdot 9^2}$
                \begin{solution}
                    \begin{multline*}
                        \frac{16^3 \cdot 12^5}{8^5 \cdot 9^2} = \frac{{\left(2^4\right)}^3 \cdot \left(2^2 \cdot 3 \right)^5}{{\left(2^3\right)}^5 \cdot {\left(3^2\right)}^2} = 
                        \frac{{\left(2^4\right)}^3 \cdot \left(2^2 \right)^5 \cdot 3^5}{{\left(2^3\right)}^5 \cdot {\left(3^2\right)}^2} = 
                        \frac{2^{12} \cdot 2^{10} \cdot 3^5}{2^{15} \cdot 3^{4}} = \dots \\
                        \dots = \frac{2^{12} \cdot 2^{10} }{2^{15}} \cdot \frac{ 3^5}{3^{4}} = 2^{(12+10-15)} \cdot 3^{(5-4)} = \boxed{2^7 \cdot 3} = 384
                    \end{multline*}
                \end{solution}
            \part 
                $\frac{9^4 \cdot 27^3}{81^4}$
                \begin{solution}
                    $\frac{9^4 \cdot 27^3}{81^4} = \frac{(3^2)^4 \cdot (3^3)^3}{(3^4)^4} = \frac{3^8 \cdot 3^9}{3^{16}} = 3^{(8+9-16)} = \boxed{3}$
                \end{solution}
        \end{parts}

        \question
        Simplifica
        \begin{parts}
            \part
                $\frac{2^7 \cdot 3^6 \cdot (2+3)^8}{(32 - 2)^6}$
                \begin{solution}
                    $\\
                    \frac{2^7 \cdot 3^6 \cdot (2+3)^8}{(32 - 2)^6} = \frac{2^7 \cdot 3^6 \cdot 5^8}{{30}^6} = \frac{2^7 \cdot 3^6 \cdot 5^8}{2 \cdot 3 \cdot 5} = 
                    \boxed{2^6 \cdot 3^5 \cdot 5^7}
                    \\$
                \end{solution}
            \part
                $\frac{(4^2 + 3^2) \cdot 2^6}{(5-1)^3 \cdot (4+1)}$
                \begin{solution}
                    \begin{multline*}
                        \frac{(4^2 + 3^2) \cdot 2^6}{(5-1)^3 \cdot (4+1)} = \frac{(16 + 9) \cdot 2^6}{4^3 \cdot 5} = \frac{25 \cdot 2^6}{4^3 \cdot 5} = \dots \\
                        \dots = \frac{5^2 \cdot 2^6}{(2^2)^3 \cdot 5}
                        = \frac{5^2 \cdot 2^6}{2^6 \cdot 5} = \frac{2^6}{2^6} \cdot \frac{5^2}{5} = \boxed{5}
                    \end{multline*}
                \end{solution}
        \end{parts}

        %\newpage

        \question
        Coloca en orden decreciente los números: \\
        $3^{0}$ ; $3^{-1}$ ; ${3}^{3}$ ; ${3}^{-2}$ ; ${3}^{2}$
        \begin{solution}
            Para ordenar potencias con la misma base, se ordenan los exponentes. \\ \\
            ${3}^{-2} < 3^{-1} < 3^{0} <  {3}^{3}  < {3}^{2}$
            \\
        \end{solution}

        \question
        Simplifica los siguientes radicales, extrayendo fuera lo que se pueda.
        \begin{parts}
            \part
                $\sqrt{12}$
                \begin{solution}
                    $\sqrt{12} = \sqrt{2^2 \cdot 3} = \sqrt{2^2} \cdot \sqrt{3} = \boxed{2 \sqrt{3}}$
                \end{solution}
            \part
                $\sqrt{75}$
                \begin{solution}
                    $\sqrt{75} = \sqrt{3 \cdot 5^2} = \sqrt{3} \cdot \sqrt{5^2} = \boxed{5 \sqrt{3}}$
                \end{solution}
            \part
                $\sqrt{114}$
                \begin{solution}
                    $\sqrt{114} = \boxed{\sqrt{2 \cdot 3 \cdot 19}}$
                \end{solution}
            \part
                $\sqrt{841}$
                \begin{solution}
                    $\sqrt{841} = \sqrt{29^2} = \boxed{29}$
                \end{solution}
            \part
                $\sqrt[3]{-8}$
                \begin{solution}
                    $\sqrt[3]{-8} = \sqrt[3]{-(2^3)} = \boxed{-2}$
                \end{solution}
            \part
                $\sqrt{128}$
                \begin{solution}
                    $\sqrt{2^7} = \sqrt{2^6 \cdot 2} = \sqrt{2^6} \cdot \sqrt{2} = \boxed{2^3 \sqrt{2}} $
                \end{solution}
            \part
                $\sqrt[5]{-1024}$
                \begin{solution}
                    $\sqrt[5]{-1024} = \sqrt[5]{-2^5} = \boxed{-5}$
                \end{solution}
        \end{parts}

    \end{questions}
\end{document}