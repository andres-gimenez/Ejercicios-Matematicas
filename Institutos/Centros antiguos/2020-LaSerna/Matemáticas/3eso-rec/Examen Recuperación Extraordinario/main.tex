%%%%%%%%%%%%%%%%%%%%%%%%%%%
\newcommand{\documentName} { Examen Recuperación }
\newcommand{\documentContent} { Convocatoria extraordinaria } 
\newcommand{\waterMark} { } 
%%%%%%%%%%%%%%%%%%%%%%%%%%%

% Configuración del documento.
\newcommand{\schoolSubject} { Matemáticas 3º ESO - Recuperación}
\newcommand{\school} { IES La Serna }
\newcommand{\academicPeriod} { Curso 2020/2021 }


\newcommand{\autor} { Andrés Giménez Muñoz }
\newcommand{\emailAuthor} { agimenezmunoz@ieslaserna.com }
\newcommand{\autorSing}{ Profesores: Andrés } 
\renewcommand{\schoolSubject} { Examen Matemáticas 2º ESO  }
\renewcommand{\school} { IES José de Churriguera  }
\renewcommand{\academicPeriod} { Curso 2022/2023 }

\renewcommand{\autor} { Andrés Giménez Muñoz }
\renewcommand{\emailAuthor} { andresprofemates@outlook.es }
\renewcommand{\autorSing}{ Profesor: Andrés } 

%%%%%%%%%%%%%%%%%%%%%%%%%%%
% Exam configuration
%\pointsdroppedatright   %% No mostrar la puntuación
\pointsinrightmargin % Para poner las puntuaciones a la derecha. Se puede cambiar. Si se comenta, sale a la izquierda.
\extrawidth{-1.5cm} %Un poquito más de margen por si ponemos textos largos.
\marginpointname{ \emph{\points}}


%% Si se comenta no aparecerán los espacios de la solución.
%\nocancelspace

%% Esto es de la clase exam. Si dejamos sin comentar \printanswers, se mostraran las soluciones. 
%% Si la comentamos y dejamos sin comentar \noprintanswers, pues no se muestran las soluciones.
%\printanswers
%\noprintanswers

%%%%%%%%%%%%%%%%%%%%%%%%%%%

\begin{document}

\StudentData
\GradeTableHeader

\justifying

\begin{questions}
	\setcounter{question}{0}

    \question[1\half]
    Realiza las siguientes operaciones y expresa el resultado en forma de fracción irreducible.
    \begin{parts}
        \part
			$\frac{3}{5} \cdot \frac{5}{4} \cdot \frac{4}{12} \cdot \frac{12}{3} \cdot \frac{3}{2} \cdot \frac{2}{7} : \frac{3}{7} $
			\vspace{\stretch{3}}

        \part
        	$\frac{1}{25} + \frac{3}{5} - \frac{6}{9} : \frac{14}{12}$
        	\vspace{\stretch{3}}

		\part
			$\frac{9^4 \cdot 27^3}{81^4}$
			\vspace{\stretch{3}}
    \end{parts}

	\question[1\half]
	Si $P(x)=x^4 - 3x^3 + 2 x^2 - 2x - 6$, evalua.
	\begin{parts}
		\part
		$P(-2)$ 
		\vspace{\stretch{2}}

		\part
		$P(0)$ 
		\vspace{\stretch{2}}

		\part
		$P(2)$ 
		\vspace{\stretch{2}}

	\end{parts}

	\newpage

	\question[1\half]
	Si $P(x)=x^3 - x^2 -2x +2$, $Q(x)=2x^3-3x^2+x-1$, $R(x)=2x - 3$ calcula.
	\begin{parts}
		\part
		$P(x) + Q(X)$
		\vspace{\stretch{3}}
		\part
		$P(x) \cdot R(X)$
		\vspace{\stretch{3}}
	\end{parts}

	\question[1\half]
	Resuelve las siguientes ecuaciones.
	\begin{parts}
		\part
		$5x+4=49$
		\vspace{\stretch{2}}

		\part
		$\frac{5x+7}{4}-\frac{2x+1}{3}=2$
		\vspace{\stretch{2}}
	\end{parts}

	\newpage

	\question[2]
	Resuelve las siguientes ecuaciones.
	\begin{parts}
		\part
		$3x^2-3x-18=0$
		\vspace{\stretch{1}}

		\part
		%  Copiada del libro.
		$4x^2-100=0$
		\vspace{\stretch{1}}

	\end{parts}

	\newpage

	\question[2]
	Resuelve el siguientes sistema de ecuaciones por el metodo que prefieras.
	\begin{flushleft}
		$\begin{cases}
				\nonumber
				3x - y  = 7 \\
				\nonumber
				2x + 5y = 16
			\end{cases}$
	\end{flushleft}
	\vspace{\stretch{1}}
\end{questions}

\end{document}