%%%%%%%%%%%%%%%%%%%%%%%%%%%
\newcommand{\numeroHoja} { 3º Evaluación }
\newcommand{\nombreHoja} { Ecuaciones de 1º y 2º grado } 
%%%%%%%%%%%%%%%%%%%%%%%%%%%

% Configuración del documento.
\renewcommand{\schoolSubject} { Examen Matemáticas 2º ESO  }
\renewcommand{\school} { IES José de Churriguera  }
\renewcommand{\academicPeriod} { Curso 2022/2023 }

\renewcommand{\autor} { Andrés Giménez Muñoz }
\renewcommand{\emailAuthor} { andresprofemates@outlook.es }
\renewcommand{\autorSing}{ Profesor: Andrés } 
\newcommand{\schoolSubject} { Matemáticas 3º ESO - Recuperación}
\newcommand{\school} { IES La Serna }
\newcommand{\academicPeriod} { Curso 2020/2021 }


\newcommand{\autor} { Andrés Giménez Muñoz }
\newcommand{\emailAuthor} { agimenezmunoz@ieslaserna.com }
\newcommand{\autorSing}{ Profesores: Andrés } 

%%%%%%%%%%%%%%%%%%%%%%%%%%%
% Exam configuration
%\pointsdroppedatright   %% No mostrar la puntuación
\pointsinrightmargin % Para poner las puntuaciones a la derecha. Se puede cambiar. Si se comenta, sale a la izquierda.
\extrawidth{-1.5cm} %Un poquito más de margen por si ponemos textos largos.
\marginpointname{ \emph{\points}}

%% Si se comenta no aparecerán los espacios de la solución.
%\nocancelspace

%% Esto es de la clase exam. Si dejamos sin comentar \printanswers, se mostraran las soluciones. 
%% Si la comentamos y dejamos sin comentar \noprintanswers, pues no se muestran las soluciones.
%\printanswers
%\noprintanswers

%%%%%%%%%%%%%%%%%%%%%%%%%%%

% \usepackage{tikz}
% \usetikzlibrary{arrows}

\begin{document}

	\StudentData
	\GradeTableHeader

    \justifying

	\begin{questions}
		\setcounter{question}{0}

        \question[1]
        Comprueba si la solución de cada ecuación es correcta.
        \begin{parts}
            \part
            $\frac{x-3}{2}-\frac{5x-10}{7} = \frac{-1}{2}$, $\boxed{x=2}$
            \vspace{\stretch{1}}
            
            \part
            $x^2-x-2 = 0$, $\boxed{x=-1}$ y $\boxed{x=2}$
            \vspace{\stretch{1}}
		\end{parts}

		\question[2]
        Resuelve las siguientes ecuaciones de 1º grado
        \begin{parts}
            \part
            $2x-3 = 8x+21$
            \vspace{\stretch{1}}
            
            \part
            $16x-9-41x=13-15x+34$
            \vspace{\stretch{1}}

            % \part
            % $2(x-1)+3 = 9$
            % \vspace{\stretch{1}}

            \part
            $6(2x-3) = 10(2x-5)$
            \vspace{\stretch{1}}
		\end{parts}

        \newpage
		\question[2]
        Resuelve las siguientes ecuaciones de 2º grado
        \begin{parts}
            \part
            $x^2-5x+4=0$
            \vspace{\stretch{1}}
            
            \part
            $-x^2-7x-10=0$
            \vspace{\stretch{1}}

            % \part
            % $5x^2-7x-6=0$
            % \vspace{\stretch{1}}
		\end{parts}

        \question[2]
        Resuelve las ecuaciones de 2º grados incompletas, sin utilizar la formula
        \begin{parts}
            % \part
            % $x^2-7x=0$
            % \vspace{\stretch{1}}

            \part
            $8x-4x^2=0$
            \vspace{\stretch{1}}
            
            % \part
            % $5x^2-80=0$
            % \vspace{\stretch{1}}

            \part
            $5+2x^2=3x^2-11$
            \vspace{\stretch{1}}
		\end{parts}
        
        \newpage
        \question[3]
        Resuelve las siguientes ecuaciones
        \begin{parts}
            \part
            $\frac{(2x-8)}{3}-\frac{x+2}{3}=2$
            \vspace{\stretch{3}}

            \part
            $\frac{(2x-8)}{5}-\frac{3(x+2)}{6}=2$
            \vspace{\stretch{3}}
		\end{parts}
	\end{questions}
\end{document}