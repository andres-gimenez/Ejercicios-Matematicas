%%%%%%%%%%%%%%%%%%%%%%%%%%%
\newcommand{\documentName} { Examen Recuperación - 2ª Evaluación }
\newcommand{\documentContent} { Convocatoria ordinaria } 
\newcommand{\waterMark} { 2ª Evaluación } 
%%%%%%%%%%%%%%%%%%%%%%%%%%%

% Configuración del documento.
\newcommand{\schoolSubject} { Matemáticas 3º ESO - Recuperación}
\newcommand{\school} { IES La Serna }
\newcommand{\academicPeriod} { Curso 2020/2021 }


\newcommand{\autor} { Andrés Giménez Muñoz }
\newcommand{\emailAuthor} { agimenezmunoz@ieslaserna.com }
\newcommand{\autorSing}{ Profesores: Andrés } 
\renewcommand{\schoolSubject} { Examen Matemáticas 2º ESO  }
\renewcommand{\school} { IES José de Churriguera  }
\renewcommand{\academicPeriod} { Curso 2022/2023 }

\renewcommand{\autor} { Andrés Giménez Muñoz }
\renewcommand{\emailAuthor} { andresprofemates@outlook.es }
\renewcommand{\autorSing}{ Profesor: Andrés } 

%%%%%%%%%%%%%%%%%%%%%%%%%%%
% Exam configuration
%\pointsdroppedatright   %% No mostrar la puntuación
\pointsinrightmargin % Para poner las puntuaciones a la derecha. Se puede cambiar. Si se comenta, sale a la izquierda.
\extrawidth{-1.5cm} %Un poquito más de margen por si ponemos textos largos.
\marginpointname{ \emph{\points}}

%% Si se comenta no aparecerán los espacios de la solución.
%\nocancelspace

%% Esto es de la clase exam. Si dejamos sin comentar \printanswers, se mostraran las soluciones. 
%% Si la comentamos y dejamos sin comentar \noprintanswers, pues no se muestran las soluciones.
%\printanswers
%\noprintanswers

%%%%%%%%%%%%%%%%%%%%%%%%%%%

\begin{document}

\StudentData
% \GradeTableHeader

\setcounter{question}{0}

\center{2ª Evaluación}
\begin{center}
    \partialgradetable{Evaluacion2}[h][questions]
\end{center}

\justifying

\begin{questions}
%%%%%%%%%%%%%%%%%%%%%%%%%%%%%%%%%%%%%%%%%%  2ª Evaluación  %%%%%%%%%%%%%%%%%%%%%%%%%%%%%%%%%%%%%%%%%%%%%%%%%%%%%%
\begingradingrange{Evaluacion2}

% \question[1]
% Expresa los productos en forma de potencia.
% \begin{parts}
%     \part
%     $2 \cdot 2 \cdot 2 \cdot 2 $
%     \vspace{\stretch{1}}
%     \part
%     $5 \cdot 5 \cdot 5 \cdot 5 \cdot 5 \cdot 5 $
%     \vspace{\stretch{1}}
%     \part
%     $(-3) \cdot (-3) \cdot (-3) $
%     \vspace{\stretch{1}}
%     \part
%     $7 \cdot 7 \cdot 7 \cdot 7 \cdot 7$
%     \vspace{\stretch{1}}
% \end{parts}

\question[1]
Expresa como una única potencia.
\begin{parts}
    \part
    $3^{15} \cdot 4^{15} \cdot 2^{15}$
    \vspace{\stretch{1}}
    \part
    $840^8 : 7^8$
    \vspace{\stretch{1}}
    \part
    $12^5 : 4^5 : 3^5$
    \vspace{\stretch{1}}
\end{parts}

\question[2]
    Calcula el IVA de $1.342,32\euro{}$ al tipo general del $21\%$, tipo reducido al $10\%$ y superreducido al $4\%$.
    \vspace{\stretch{3}}
\newpage

\question[2]
Calcula
\begin{parts}
    \part
    $\frac{16 \cdot 81 \cdot 25}{12 \cdot 300}$
    \vspace{\stretch{1}}
    \part
    $\frac{16^3 \cdot 12^5}{8^5 \cdot 9^2}$
    \vspace{\stretch{1}}

    % \part
    % $\frac{9^4 \cdot 27^3}{81^4}$
    % \vspace{\stretch{1}}
\end{parts}

% \question[1]
% Calcula las siguientes raices
% \begin{parts}
%     \part
%     $\sqrt{16}$
%     \vspace{\stretch{1}}
%     \part
%     $\sqrt{49}$
%     \vspace{\stretch{1}}
%     \part
%     $\sqrt{1024}$
%     \vspace{\stretch{1}}
% \end{parts}

% \question[1]
%         Selecciona la expresión algebraica que corresponda a las siguientes frases
%         \begin{parts}
%             \part
%             Un número par: \\ \\
%             \begin{oneparcheckboxes}
%                 \CorrectChoice $2n$
%                 \choice $3n$
%                 \choice $n^2$
%                 \choice $n^2 - 2$
%             \end{oneparcheckboxes}
%             \\
%             \part
%             Un número impar: \\ \\
%             \begin{oneparcheckboxes}
%                 \choice $2n$
%                 \choice $3n$
%                 \choice $(n-1)^2$
%                 \CorrectChoice $2n - 1 $
%             \end{oneparcheckboxes}
%             \\
%             \part
%             El tripe de un número, menos cinco: \\ \\
%             \begin{oneparcheckboxes}
%                 \choice $3n$
%                 \choice $5n$
%                 \choice $3(n-5)$
%                 \CorrectChoice $3n-5$
%             \end{oneparcheckboxes}
%             \\
%             \part
%             Raiz cuadrada del doble de un número: \\ \\
%             \begin{oneparcheckboxes}
%                 \choice $\sqrt{n + 2}$
%                 \choice $2 \sqrt{n}$
%                 \CorrectChoice $\sqrt{2n}$
%                 \choice $\sqrt{n^2}$
%             \end{oneparcheckboxes}
%             \\
%         \end{parts}

        \question[2]
        Calcula las siguientes operaciones con monomios
        \begin{parts}
            \part
                $2x^7+4x^7$
                \vspace{\stretch{1}}
            \part
                $2x^2 + \frac{3}{2}x^2$
                \vspace{\stretch{1}}
            % \part
            %     $x + 5x - 3x$
            %     \vspace{\stretch{1}}
            % \part
            %     $3x^3 \cdot 2x^7$
            %     \vspace{\stretch{1}}
            % \part
            %     $7x^2 \cdot \frac{1}{7}x^3$
            %     \vspace{\stretch{1}}
        \end{parts}

        \newpage

        \question[2]
        Si $P(x)=2 x^2 - x - 5$, evalúa.
        \begin{parts}
            \part
                $P\left(-\frac{1}{2}\right)$
                \vspace{\stretch{1}}
            \part
                $P\left(-1\right)$
                \vspace{\stretch{1}}
            \part
                $P\left(0\right)$
                \vspace{\stretch{1}}
            % \part
            %     $P\left(\frac{1}{3}\right)$
            %     \vspace{\stretch{1}}
        \end{parts}

        \question[2]
        Si $P(x)=x^4 + 3x^3 - 2x^2 +5x -1$ y $Q(x)=2x^4-3x^2 + 3x-2$, calcula $P(x) + Q(X)$.
        \vspace{\stretch{4}}

        \question[2]
        Si $P(x)=x^2 +2x +3$, $Q(x)=3x-2$, calcula $P(x) \cdot Q(x)$.
        \vspace{\stretch{4}}

\endgradingrange{Evaluacion2}

\end{questions}
\end{document}