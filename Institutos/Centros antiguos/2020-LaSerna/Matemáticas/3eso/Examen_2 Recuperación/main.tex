\documentclass[addpoints,spanish, 12pt,a4paper,cancelspace]{./include/gexam}

% Configuración del documento.
\renewcommand{\schoolSubject} { Examen Matemáticas 2º ESO  }
\renewcommand{\school} { IES José de Churriguera  }
\renewcommand{\academicPeriod} { Curso 2022/2023 }

\renewcommand{\autor} { Andrés Giménez Muñoz }
\renewcommand{\emailAuthor} { andresprofemates@outlook.es }
\renewcommand{\autorSing}{ Profesor: Andrés } 

%%%%%%%%%%%%%%%%%%%%%%%%%%%
\renewcommand{\numeroHoja} { 2º Evaluación }
\renewcommand{\nombreHoja} { Recuperación } 
\renewcommand{\waterMark} { Modelo: B } 
%%%%%%%%%%%%%%%%%%%%%%%%%%%

%%%%%%%%%%%%%%%%%%%%%%%%%%%
% Exam configuration
%\pointsdroppedatright   %% No mostrar la puntuación
\pointsinrightmargin % Para poner las puntuaciones a la derecha. Se puede cambiar. Si se comenta, sale a la izquierda.
\extrawidth{-1.5cm} %Un poquito más de margen por si ponemos textos largos.
\marginpointname{ \emph{\points}}

%% Si se comenta no aparecerán los espacios de la solución.
%\nocancelspace

%% Esto es de la clase exam. Si dejamos sin comentar \printanswers, se mostraran las soluciones. 
%% Si la comentamos y dejamos sin comentar \noprintanswers, pues no se muestran las soluciones.
%\printanswers
%\noprintanswers

%%%%%%%%%%%%%%%%%%%%%%%%%%%

% \usepackage{tikz}
% \usetikzlibrary{arrows}

\begin{document}

	\StudentData
	\GradeTableHeader

    \justifying

	\begin{questions}
		\setcounter{question}{0}

        \question[2]
        Simplifica desarrollando. \\   
        \tiny{*{No abuses de la calculadora, no pongas la solución directamente, no utilices decimales.}}
        \normalsize
        
        \begin{parts}
            \part
				$\frac{(4^2 + 3^2) \cdot 2^6}{(5-1)^3 \cdot (4+1)} = $
				\vspace{\stretch{5}}
            \part
				$\left(\frac{3}{4} \right)^4 \cdot \left(\frac{9}{25}\right)^{-2} = $
				\vspace{\stretch{5}}
			\part
				$\left(\frac{5}{2} + \frac{3}{4} \right)^2 - \left(\frac{1}{4}\right)^2 = $
				\vspace{\stretch{5}}
        \end{parts}

        \newpage
        \question[2]
		Simplifica los siguientes radicales, extrayendo fuera lo que se pueda. \\
        \tiny{*{No abuses de la calculadora, no pongas la solución directamente, no utilices decimales.}}
        \normalsize
		\begin{parts}
			\begin{multicols}{2}
			\part
				$\sqrt{a^4} = $ \\
            \part
				$\sqrt[3]{a^{12}} = $  \\
            \part
				$\sqrt[3]{64} = $ \\
	        \part
				$\sqrt{128} = $ \\
	        \part
				$\sqrt{-1} = $ \\
	        \part
				$\sqrt{1} = $ \\
			\part
				$1^{23} = $ \\
            \part
				$\sqrt[5]{0} = $ \\
            \part
				$\sqrt[5]{-1024} = $ \\
            \part
				$\sqrt[3]{\frac{1}{216}} = $ \\
            \part
				$\sqrt[n]{b^{2n}} = $ \\
			\part
				$\sqrt{4a^2} = $ \\
			\end{multicols}
		\end{parts}

        \question[2]
        Si $P(x)=x^4 + 2x^3 - 1x^2 + x +2$, evalúa.
        \begin{parts}
            \part
                $P\left(-1\right)$
                \vspace{\stretch{5}}
            \part
                $P\left(0\right)$
                \vspace{\stretch{5}}
            \part
                $P\left(\frac{1}{2}\right)$
                \vspace{\stretch{5}}                
        \end{parts}

        \newpage
        \question[2]
        Sea $P(x)=x^5+2x^3-3x+2$, $Q(x)=3x^4-2x^2-3x+1$, $R(x)=x^2 -3x + 1$ calcula.
        \begin{parts}
            \part
                $P(x) - Q(x)$
                \vspace{\stretch{3}}
            % \part
            %     $4 Q(x)$
            %     \vspace{\stretch{2}}

            % \part
            %     $P(x) + 4 Q(x)$
            %     \vspace{\stretch{2}}

            \part
                $Q(x) \cdot R(x)$
                \vspace{\stretch{3}}
            % \part
            %     $P(x) \cdot 2 Q(x)$
            %     \vspace{\stretch{4}}
        \end{parts}

        \newpage

        \question[2]
            Realiza la siguiente división por el método de Ruffini \\
            $\left(\polynomial[reciprocal]{4,-8,-9,7}\right) : \left( \polynomial[reciprocal]{1, -3} \right)$
            
            \ifprintanswers
                \begin{solution}
                    $$\polyhornerscheme[x=3,resultbottomrule=true,resultleftrule=true]{4x^3-8x^2-9x+7}$$

                    $$\boxed{\polynomial[reciprocal]{4,-8,-9,7} = \left( \polynomial[reciprocal]{1,-3} \right) \left( \polynomial[reciprocal]{4, 4, 3} \right) + 16}$$
                \end{solution}
            \else
                \vspace{\stretch{3}}
            \fi

	\end{questions}
\end{document}