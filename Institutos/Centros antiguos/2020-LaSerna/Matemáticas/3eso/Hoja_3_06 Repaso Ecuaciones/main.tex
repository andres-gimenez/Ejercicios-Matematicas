%%%%%%%%%%%%%%%%%%%%%%%%%%%
\newcommand{\numeroHoja} { Hoja 3.06 }
\newcommand{\nombreHoja} { Repaso ecuaciones de 1º y 2º grado }
%%%%%%%%%%%%%%%%%%%%%%%%%%%

% Configuración del documento.
\renewcommand{\schoolSubject} { Examen Matemáticas 2º ESO  }
\renewcommand{\school} { IES José de Churriguera  }
\renewcommand{\academicPeriod} { Curso 2022/2023 }

\renewcommand{\autor} { Andrés Giménez Muñoz }
\renewcommand{\emailAuthor} { andresprofemates@outlook.es }
\renewcommand{\autorSing}{ Profesor: Andrés } 
\newcommand{\schoolSubject} { Matemáticas 3º ESO - Recuperación}
\newcommand{\school} { IES La Serna }
\newcommand{\academicPeriod} { Curso 2020/2021 }


\newcommand{\autor} { Andrés Giménez Muñoz }
\newcommand{\emailAuthor} { agimenezmunoz@ieslaserna.com }
\newcommand{\autorSing}{ Profesores: Andrés } 

%%%%%%%%%%%%%%%%%%%%%%%%%%%
% Exam configuration
\pointsdroppedatright   %% No mostrar la puntuación

%% Si se comenta no aparecerán los espacios de la solución.
%\nocancelspace

%% Esto es de la clase exam. Si dejamos sin comentar \printanswers, se mostraran las soluciones. 
%% Si la comentamos y dejamos sin comentar \noprintanswers, pues no se muestran las soluciones.
\printanswers
%\noprintanswers

%%%%%%%%%%%%%%%%%%%%%%%%%%%

\begin{document}

    %%\StudentData

    \begin{questions}
        \question
        Resuelve las siguientes ecuaciones de 1º grado.
        \begin{parts}
            \part
            $3x+\frac{1}{2}x+6=2x$
            \part
            $\frac{3}{2}+8=\frac{3}{5}x-1$
            \part
            $\frac{4}{3}(x+1)=2x-1$
            \part
            $\frac{x}{2}+\frac{2x}{3}-\frac{5x}{6}=5x-14$
            \part
            $\frac{x-2}{4}-\frac{2x+6}{3}=0$
            \part
            $\frac{x+1}{5}+\frac{x-2}{6}=1$
            \part
            $\frac{-1}{2}\left(1-\frac{3x}{2}\right)+\frac{6x}{2}=\frac{-3}{2}\left(\frac{6+x}{2}\right)$
        \end{parts}

        \question
        Resuelve las siguientes ecuaciones de 2º grado.
        \begin{parts}
            \part
            $x^2-5x+6=0$
            \part
            $x^2+x-6=0$
            \part
            $2x^2-7x+3=0$
        \end{parts}

        \question
        Resuelve sin utilizar la formula de ecuaciones de 2º grado.
        \begin{parts}
            \part
            $(x+3)(x-5)=0$
            \begin{sample} Si tenemos la ecuación factorizada la pondemos resolver facilmente. \\
                $\begin{array}{ccc} 
                    & & x+3 = 0 \Rightarrow \boxed{x= -3}\\ 
                    & \nearrow &\\ 
                    (x+3)(x-5)=0
                    & &\\ 
                    & \searrow &\\
                    & & x-5 = 0 \Rightarrow \boxed{x= 5}\\ 
                \end{array}$
             \end{sample}
            \part
            $(x+3)(x-5)=0$
            \part
            $(x+4)^2=0$
            \part
            $(x-5)^5-9=0$
            \part
            $x^2=10x$
            \part
            $2x^2-18=0$
            \part
            $16x+4x^2=0$
        \end{parts}
    \end{questions}
\end{document}