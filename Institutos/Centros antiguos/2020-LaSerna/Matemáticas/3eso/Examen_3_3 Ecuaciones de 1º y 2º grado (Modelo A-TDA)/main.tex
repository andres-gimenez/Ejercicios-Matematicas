% Configuración del documento.
\renewcommand{\schoolSubject} { Examen Matemáticas 2º ESO  }
\renewcommand{\school} { IES José de Churriguera  }
\renewcommand{\academicPeriod} { Curso 2022/2023 }

\renewcommand{\autor} { Andrés Giménez Muñoz }
\renewcommand{\emailAuthor} { andresprofemates@outlook.es }
\renewcommand{\autorSing}{ Profesor: Andrés } 
\newcommand{\schoolSubject} { Matemáticas 3º ESO - Recuperación}
\newcommand{\school} { IES La Serna }
\newcommand{\academicPeriod} { Curso 2020/2021 }


\newcommand{\autor} { Andrés Giménez Muñoz }
\newcommand{\emailAuthor} { agimenezmunoz@ieslaserna.com }
\newcommand{\autorSing}{ Profesores: Andrés }

%%%%%%%%%%%%%%%%%%%%%%%%%%%
\newcommand{\numeroHoja} { 3º Evaluación }
\newcommand{\nombreHoja} { Ecuaciones } 
\renewcommand{\waterMark} { Modelo: A - TDA } 
%%%%%%%%%%%%%%%%%%%%%%%%%%%

%%%%%%%%%%%%%%%%%%%%%%%%%%%
% Exam configuration
%\pointsdroppedatright   %% No mostrar la puntuación
\pointsinrightmargin % Para poner las puntuaciones a la derecha. Se puede cambiar. Si se comenta, sale a la izquierda.
\extrawidth{-1.5cm} %Un poquito más de margen por si ponemos textos largos.
\marginpointname{ \emph{\points}}

%% Si se comenta no aparecerán los espacios de la solución.
%\nocancelspace

%% Esto es de la clase exam. Si dejamos sin comentar \printanswers, se mostraran las soluciones. 
%% Si la comentamos y dejamos sin comentar \noprintanswers, pues no se muestran las soluciones.
%\printanswers
%\noprintanswers

%%%%%%%%%%%%%%%%%%%%%%%%%%%

% \usepackage{tikz}
% \usetikzlibrary{arrows}

\begin{document}

	\StudentData
	\GradeTableHeader

    \justifying

	\begin{questions}
		\setcounter{question}{0}

        \question[1]
        Comprueba, sin resolver, si la solución de cada ecuación es correcta.
        \begin{parts}
            \part
            $2x^2-3x-2 = 0$, $\boxed{x=-\frac{1}{2}}$ y $\boxed{x=2}$
            \vspace{\stretch{1}}

            \part
            $x^2+\frac{5}{2}x+1 = 0$, $\boxed{x=-\frac{1}{2}}$ y $\boxed{x=-2}$
            \vspace{\stretch{1}}
		\end{parts}

        \newpage
        \question[2]
		Resuelve las siguientes ecuaciones.
        \begin{parts}
            \part
            $3x+\frac{1}{2}x +6 = \frac{2x}{5}-1$
            \vspace{\stretch{2}}

            \part
            $\frac{x-5}{2}-\frac{8-3x}{2}=2-\frac{9x}{2}$
            \vspace{\stretch{2}}
        \end{parts}

        \newpage
        \question[2]
        Resuelve las siguientes ecuaciones de 2º grado.
        \begin{parts}
            \part
            $2x^2+4x=30$
            \vspace{\stretch{1}}

            \part
            % Solución x=8;x=2
            $(x-5)^2-9 =0$
            \vspace{\stretch{1}}
	    \end{parts}

        \newpage
        \question[3]
		Resuelve las siguientes ecuaciones utilizando la regla de Ruffini.
        \begin{parts}
            \part
            % Solucion x = -1, 2, -2 
            $x^3 + x^2 - 4 x - 4=0$
            \vspace{\stretch{1}}

            % \part
            % % Solución: x = -1, -1, -1 
            % $ x^3 + 2 x^2 - x - 2=0$
            % \vspace{\stretch{1}}

            % \part
            % % Solución: x = 2, 2, 3 
            % $ x^3 - 7 x^2 + 16 x - 12=0$
            % % \vspace{\stretch{1}}

            \part
            % Solución: x = -1, -1, 2 
            $ x^3 - 3 x - 2 = 0$
            \vspace{\stretch{1}}
	    \end{parts}

        \newpage
        \question[2]
		Resuelve las siguientes ecuaciones.
        \begin{parts}
            \part
            $x^4-25x^2+144=0$
            \vspace{\stretch{1}}

            \part
            $(x+3)(x-2)(x-52)(x^2-10x+9)=0$
            \vspace{\stretch{1}}
	    \end{parts}
	\end{questions}
\end{document}