%%%%%%%%%%%%%%%%%%%%%%%%%%%
\newcommand{\numeroHoja} { Hoja 2.5 }
\newcommand{\nombreHoja} { Potencias y Raíces }
%%%%%%%%%%%%%%%%%%%%%%%%%%%

% Configuración del documento.
\renewcommand{\schoolSubject} { Examen Matemáticas 2º ESO  }
\renewcommand{\school} { IES José de Churriguera  }
\renewcommand{\academicPeriod} { Curso 2022/2023 }

\renewcommand{\autor} { Andrés Giménez Muñoz }
\renewcommand{\emailAuthor} { andresprofemates@outlook.es }
\renewcommand{\autorSing}{ Profesor: Andrés } 
\newcommand{\schoolSubject} { Matemáticas 3º ESO - Recuperación}
\newcommand{\school} { IES La Serna }
\newcommand{\academicPeriod} { Curso 2020/2021 }


\newcommand{\autor} { Andrés Giménez Muñoz }
\newcommand{\emailAuthor} { agimenezmunoz@ieslaserna.com }
\newcommand{\autorSing}{ Profesores: Andrés } 

%%%%%%%%%%%%%%%%%%%%%%%%%%%
% Exam configuration
\pointsdroppedatright   %% No mostrar la puntuación

%% Si se comenta no aparecerán los espacios de la solución.
%\nocancelspace

%% Esto es de la clase exam. Si dejamos sin comentar \printanswers, se mostraran las soluciones. 
%% Si la comentamos y dejamos sin comentar \noprintanswers, pues no se muestran las soluciones.
%\printanswers
%\noprintanswers

%%%%%%%%%%%%%%%%%%%%%%%%%%%

\begin{document}

    %%\StudentData

    \begin{center}
		\fbox{\fbox{\parbox{5.5in}{\centering
		Estos ejercicios están destinados a ejercitar la destreza en la aritmética. No uséis la calculadora para realizarlos. En ninguno de ellos debéis obtener decimales.}}}
	\end{center}

    \begin{questions}
        \question
        Calcular
        \begin{parts}
            \part
                $2^{-3} - {4}^{-2}$
            \part
                $(-2)^{-3} + (-3)^{-2}$
            \part
                ${\left(\frac{1}{2} \right)}^2 + {\left(\frac{1}{2}\right)}^{-2}$
            \part
                $(3-1)^2 - (3-1)^{-2}$
        \end{parts}

       \question
       Calcula los siguientes radicales descomponiendo cada radicando en factores primos.
       \begin{parts}
           \part
           $\sqrt{441}$
           \part
           $\sqrt[3]{1728}$
           \part
           $\sqrt[2]{64}$
           \part
           $\sqrt[4]{625}$
           \part
           $\sqrt{2025}$
       \end{parts}

       \question
       Calcula el valor de $x$ en cada una de las igualdades.
       \begin{parts}
           \part
            ${\left(\frac{1}{4} \right)}^{x} = 16$
           \part
           $10^x \cdot 100 = 0,001$
           \part
           $5^{-7} \cdot {\left(\frac{1}{25} \right)}^{x} = 5$
           \part
           $2 \cdot 4^3 \cdot 32 = 2^{x}$
       \end{parts}

       \question
       Extrae factores de cada raíz.
       \begin{parts}
          \part 
          $\sqrt[3]{\frac{24 x^9}{27 y^4}}$
          \part 
          $\sqrt[3]{a^{11} \cdot b^{16} \cdot c^{21} }$
       \end{parts}

       \question 
       Extrae factores, obtén radicales semejantes y reduce.
       \begin{parts}
            \part 
            $7\sqrt[3]{81} + 5\sqrt[3]{24} + 2\sqrt[3]{375}$
            \part 
            $2\sqrt{25}-5\sqrt{54}-12\sqrt{600}$
            \part 
            $4\sqrt{27}-7\sqrt{12}-2\sqrt{75}$
        \end{parts}

        \question
        Opera y simplifca al máximo
        \begin{parts}
            \part 
            $\frac{\sqrt[3]{16} \cdot \sqrt{2\sqrt{2}}}{\left(\sqrt[12]{8} \right)^5}$
            \part
            $\sqrt{\frac{63}{4}} - \frac{5}{2} \sqrt{\frac{28}{25}} + \frac{1}{3} \sqrt{112}$
        \end{parts}
    \end{questions}
\end{document}