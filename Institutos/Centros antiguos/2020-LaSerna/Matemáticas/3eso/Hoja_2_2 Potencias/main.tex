%%%%%%%%%%%%%%%%%%%%%%%%%%%

\newcommand{\numeroHoja} { Hoja 2.2 }
\newcommand{\nombreHoja} { Potencias }
%%%%%%%%%%%%%%%%%%%%%%%%%%%

% Configuración del documento.
\renewcommand{\schoolSubject} { Examen Matemáticas 2º ESO  }
\renewcommand{\school} { IES José de Churriguera  }
\renewcommand{\academicPeriod} { Curso 2022/2023 }

\renewcommand{\autor} { Andrés Giménez Muñoz }
\renewcommand{\emailAuthor} { andresprofemates@outlook.es }
\renewcommand{\autorSing}{ Profesor: Andrés } 
\newcommand{\schoolSubject} { Matemáticas 3º ESO - Recuperación}
\newcommand{\school} { IES La Serna }
\newcommand{\academicPeriod} { Curso 2020/2021 }


\newcommand{\autor} { Andrés Giménez Muñoz }
\newcommand{\emailAuthor} { agimenezmunoz@ieslaserna.com }
\newcommand{\autorSing}{ Profesores: Andrés } 

%%%%%%%%%%%%%%%%%%%%%%%%%%%
% Exam configuration
\pointsdroppedatright   %% No mostrar la puntuación

%% Si se comenta no aparecerán los espacios de la solución.
%\nocancelspace

%% Esto es de la clase exam. Si dejamos sin comentar \printanswers, se mostraran las soluciones. 
%% Si la comentamos y dejamos sin comentar \noprintanswers, pues no se muestran las soluciones.
%\printanswers
%\noprintanswers

%%%%%%%%%%%%%%%%%%%%%%%%%%%

\begin{document}

    %%\StudentData

    \begin{center}
		\fbox{\fbox{\parbox{5.5in}{\centering
		Estos ejercicios estan destinados a ejercitar la destreza en la aritmética. No uséis la calculadora para realizarlos. En ninguno de ellos debéis obtener decimales.}}}
	\end{center}

    \begin{questions}
        \question
        Expresa como una única potencia.
        \begin{parts}
            \part
                $2^6 \cdot 5^6 \cdot 7^6$
            \part
                $21^8 : 3^8$
            \part
                $12^5 : 4^5 : 3^5$
        \end{parts}

        \question
        Calcula
        \begin{parts}
            \part 
                $\frac{16 \cdot 81 \cdot 25}{12 \cdot 300}$
            \part 
                $\frac{16^3 \cdot 12^5}{8^5 \cdot 9^2}$
            \part 
                $\frac{9^4 \cdot 27^3}{81^4}$
            \part 
                $\frac{30^{20} \cdot 49^5}{(210^4)^2 \cdot 100^5 \cdot 9^6}$
        \end{parts}

        \question
        Simplifica
        \begin{parts}
            \part
                $\frac{2^7 \cdot 3^6 \cdot (2+3)^8}{(32 - 2)^6}$
            \part
                $\frac{(4^2 + 3^2) \cdot 2^6}{(5-1)^3 \cdot (4+1)}$
            \part
                $\frac{(6^2)^4 \cdot 3^3}{2^6 \cdot 9^4}$
            \part
                $(6^2)^4 \cdot 9^3 \cdot (2^6 : 2^4)$
            \part
                $\left(\frac{3}{4} \right)^4 \cdot \left(\frac{9}{25}\right)^{-2}$
        \end{parts}

        % \question
        % Simplifica los siguientes radicales, extrayendo fuera lo que se pueda.
        % \begin{parts}
        %     \part
        %         $\sqrt{12}$
        %     \part
        %         $\sqrt{75}$
        %     \part
        %         $\sqrt{114}$
        %     \part
        %         $\sqrt{841}$
        %     \part
        %         $\sqrt[3]{-8}$
        %     \part
        %         $\sqrt{128}$
        %     \part
        %         $\sqrt[5]{-1024}$
        %     \part
        %         $\sqrt[4]{\frac{1}{625}}$
        %     \part
        %         $\sqrt[3]{-\frac{125}{512}}$
        % \end{parts}

        % \question
        % Simplifica los siguientes radicales, extrayendo fuera lo que se pueda: \\
        % \begin{parts}
        %     \part
        %         $\sqrt{12}$
        %     \part
        %         $\sqrt{75}$
        %     \part
        %         $\sqrt[3]{-8}$
        %     \part
        %         $\sqrt{128}$
        %     \part
        %         $\sqrt[5]{-1024}$
        %     \part
        %         $\sqrt[4]{\frac{1}{625}}$
        %     \part
        %         $\sqrt[3]{-\frac{125}{512}}$
        % \end{parts}

        % \question
        % Simplifica los siguientes radicales, extrayendo fuera lo que se pueda: \\
        % \begin{parts}
        %     \part
        %         $\sqrt[n]{\left(\frac{1}{3}\right)^{n+1}}$
        %     \part
        %         $\sqrt{324 a^5 b^2 c}$
        %     \part
        %         $\sqrt[3]{81 x^3 y ^6}$
        %     \part
        %         $\sqrt[3]{-16 a^6 c^9}$
        %     \part
        %         $\sqrt[3]{-64 a^6 b^3}$
        %     \part
        %         $\sqrt[4]{32 a^4 b^8 c}$
        %     \part
        %         $\sqrt{\frac{125 x^3}{96 y^3}}$
        %     \part
        %         $\sqrt[3]{-\frac{16 a^3 b^4}{27}}$
        %     \part
        %         $\sqrt{\frac{(a+b)^3}{a-b}}$
        % \end{parts}
        
        % \question
        % Ejecuta y simplifica: \\
        % \begin{parts}
        %     \part
        %         $\sqrt{\sqrt[3]{b^2}}$
        %     \part
        %         $\sqrt[4]{\sqrt[3]{a^{18}}}$
        %     \part
        %         $\sqrt[3]{\sqrt{\sqrt[4]{a}}}$
        %     \part
        %         $\sqrt{9 \sqrt{3}}$
        % \end{parts}

        \question
        Coloca en orden decreciente los números: \\
        ${({-5})}^{-1}$ ; ${({-5})}^{0}$ ; ${({-5})}^{2}$ ; ${({-5})}^{3}$ ; ${({-5})}^{-2}$

    \end{questions}
\end{document}