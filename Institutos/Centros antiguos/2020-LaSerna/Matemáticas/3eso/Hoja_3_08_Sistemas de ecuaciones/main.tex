%%%%%%%%%%%%%%%%%%%%%%%%%%%
\newcommand{\documentName} { Hoja 3.08 }
\newcommand{\documentContent} { Sistema de ecuaciones lineales } 
\newcommand{\waterMark} { } 
%%%%%%%%%%%%%%%%%%%%%%%%%%%

% Configuración del documento.
\newcommand{\schoolSubject} { Matemáticas 3º ESO - Recuperación}
\newcommand{\school} { IES La Serna }
\newcommand{\academicPeriod} { Curso 2020/2021 }


\newcommand{\autor} { Andrés Giménez Muñoz }
\newcommand{\emailAuthor} { agimenezmunoz@ieslaserna.com }
\newcommand{\autorSing}{ Profesores: Andrés } 
\renewcommand{\schoolSubject} { Examen Matemáticas 2º ESO  }
\renewcommand{\school} { IES José de Churriguera  }
\renewcommand{\academicPeriod} { Curso 2022/2023 }

\renewcommand{\autor} { Andrés Giménez Muñoz }
\renewcommand{\emailAuthor} { andresprofemates@outlook.es }
\renewcommand{\autorSing}{ Profesor: Andrés } 

%%%%%%%%%%%%%%%%%%%%%%%%%%%
% Exam configuration
\pointsdroppedatright   %% No mostrar la puntuación
%\pointsinrightmargin % Para poner las puntuaciones a la derecha. Se puede cambiar. Si se comenta, sale a la izquierda.
%\extrawidth{-1.5cm} %Un poquito más de margen por si ponemos textos largos.
%\marginpointname{ \emph{\points}}

%% Si se comenta no aparecerán los espacios de la solución.
%\nocancelspace

%% Esto es de la clase exam. Si dejamos sin comentar \printanswers, se mostraran las soluciones. 
%% Si la comentamos y dejamos sin comentar \noprintanswers, pues no se muestran las soluciones.
%\printanswers
%\noprintanswers

%%%%%%%%%%%%%%%%%%%%%%%%%%%

% \usepackage{tikz}
% \usetikzlibrary{arrows}

\begin{document}
    \begin{questions}
        \question
        Resuelve los siguiens sistemas de ecuaciones por el método gráfico.
        \begin{parts}
            \part
            \begin{flushleft}
                $\begin{cases}
                    \nonumber
                    2x + y = 10 \\
                    \nonumber
                    -3x - 2y= -16 
                \end{cases}$
            \end{flushleft}

            \begin{solution}

                \begin{multicols}{2}
                    \center
                    $2x + y = 10$ 
                    \vspace{ 1mm } \\
                    \begin{tabular}{c|c} 
                        \textbf{x} & \textbf{y} \\
                        \hline
                        0 & 10 \\
                        5 & 0 \\
                    \end{tabular}

                    \center
                    $-3x - 2y= -16 $ 
                    \vspace{ 1mm } \\
                    \begin{tabular}{c|c} 
                        \textbf{x} & \textbf{y} \\
                        \hline
                        0 & 8 \\
                        6 & -1 \\
                    \end{tabular}
                \end{multicols}

                \begin{tikzpicture}[scale=0.7]
                    \tkzInit[xmax=8,ymax=8,xmin=-8,ymin=-8]
                    % \tkzGrid
                    \tkzAxeXY
                    \draw[thick] (1,8) -- (9,-8) node[anchor=south west] {$2x + y = 10$};
                    \draw[thick] (0,8) -- (8,-4) node[anchor=south west] {$-3x - 2y= -16$};  
                    \filldraw[black] (4,2) circle (2pt) node[anchor=north east] {(4,2)};
                    %\tkzText[above](0,6.75){Desired Output}
                \end{tikzpicture}

            \end{solution}
           
            %\newpage

            \part
            \begin{flushleft}
                $\begin{cases}
                    \nonumber
                    -3x+2y=17 \\
                    \nonumber
                    4x+5y=8
                \end{cases}$
            \end{flushleft}

            \begin{solution}
                \begin{multicols}{2}
                    \center
                    $-3x+2y=17$ 
                    \vspace{ 1mm } \\
                    \begin{tabular}{c|c} 
                        \textbf{x} & \textbf{y} \\
                        \hline
                        -1 & 7 \\
                        -3 & 4 \\
                    \end{tabular}

                    \center
                    $4x+5y=8$ 
                    \vspace{ 1mm } \\
                    \begin{tabular}{c|c} 
                        \textbf{x} & \textbf{y} \\
                        \hline
                         2 & 0 \\
                        -3 & 4 \\
                    \end{tabular}
                \end{multicols}

                \begin{tikzpicture}[scale=0.7]
                    \tkzInit[xmax=8,ymax=8,xmin=-8,ymin=-8]
                    % \tkzGrid
                    \tkzAxeXY
                    \draw[thick] (-8,-3.5) -- (-0.3,8) node[anchor=south west] {$-3x+2y=17$};
                    \draw[thick] (8,-4.8) -- (-8,8) node[anchor=south west] {$4x+5y=8$};  
                    \filldraw[black] (-3,4) circle (2pt) node[anchor=west] {(-3,4)};
                    %\tkzText[above](0,6.75){Desired Output}
                \end{tikzpicture}
            \end{solution}
            % \begin{figure}[h]
            %     \begin{tikzpicture}
            %         \tkzInit[xmax=6,ymax=6,xmin=-6,ymin=-6]
            %         \tkzGrid
            %         \tkzAxeXY
            %     \end{tikzpicture}
            % \end{figure}

            % \begin{tikzpicture}
            %     \draw[gray, thick] (-1,2) -- (2,-4);
            %     \draw[gray, thick] (-1,-1) -- (2,2);
            %     \filldraw[black] (0,0) circle (2pt) node[anchor=west] {Intersection point};    
            % \end{tikzpicture}
        \end{parts}

        %\newpage
        \question
        Resuelve las siguientes ecuaciones por el metodo de reducción.

        \begin{parts}
        \part
        \begin{flushleft}
            $\begin{cases}
                \nonumber
                2x + y = 10 \\
                \nonumber
                -3x - 2y= -16 
            \end{cases}$
        \end{flushleft}

        % \begin{solution}
        %     \begin{minipage}[t]{0.95\linewidth}
        %         \centering
        %         \includegraphics[width=\textwidth]{sol21}
        %     \end{minipage}
        % \end{solution}

        \part
        %Incompatible. (No copiar aquí)
        \begin{flushleft}
            $\begin{cases}
                \nonumber
                x - 4y  = 3 \\
                \nonumber
                2x - 6y = 15
            \end{cases}$
        \end{flushleft}

        % \begin{solution}
        %     \begin{minipage}[t]{0.95\linewidth}
        %         \centering
        %         \includegraphics[width=\textwidth]{sol22}
        %     \end{minipage}
        % \end{solution}

       \end{parts}

        \question
        Resuelve las siguientes ecuaciones por el metodo de sustitución.
        \begin{parts} 
            \part
            \begin{flushleft}
                $\begin{cases}
                    \nonumber
                    2x + y = 10 \\
                    \nonumber
                    -3x - 2y= -16 
                \end{cases}$
            \end{flushleft}

            % \begin{solution}
            %     \begin{minipage}[t]{0.95\linewidth}
            %         \centering
            %         \includegraphics[width=\textwidth]{sol31}
            %     \end{minipage}
            % \end{solution}

            
            %     \part
            %         $\begin{alignedat}{3}
            %             2x & +{} & 5 &= & & 5x-9 \\
            %             -2x & & & = {}& - & 2x\\
            %             \midrule
            %             & &5 & = & & 3x-9
            %         \end{alignedat}$

            \part
            \begin{flushleft}
                $\begin{cases}
                    \nonumber
                    x=\frac{y}{4} \\
                    \nonumber
                    x-7=\frac{y}{3}
                \end{cases}$
            \end{flushleft}

            % \begin{solution}
            %     \begin{minipage}[t]{0.95\linewidth}
            %         \centering
            %         \includegraphics[width=\textwidth]{sol32}
            %     \end{minipage}
            % \end{solution}
        \end{parts}  

        \question
        Resuelve las siguientes ecuaciones por el metodo de igualación.
        \begin{parts}
            \part
            \begin{flushleft}
                $\begin{cases}
                    \nonumber
                    2x + y = 10 \\
                    \nonumber
                    -3x - 2y= -16  
                \end{cases}$
            \end{flushleft}

            % \begin{solution}
            %     \begin{minipage}[t]{0.95\linewidth}
            %         \centering
            %         \includegraphics[width=\textwidth]{sol41}
            %     \end{minipage}
            % \end{solution}

            \part
            \begin{flushleft}
                $\begin{cases}
                    \nonumber
                   -3(x-5y)-7y+6 = 4 \\
                    \nonumber
                    7(x-2)-3y=5
                \end{cases}$
            \end{flushleft}

            % \begin{solution}
            %     \begin{minipage}[t]{0.95\linewidth}
            %         \centering
            %         \includegraphics[width=\textwidth]{sol42}
            %     \end{minipage}
            % \end{solution}

        \end{parts}

        \question
        Indica si tinene solución y en caso de tenerla que tipo de solución tienen los siguientes sistemas de ecuaciones.
        \begin{parts} 
            \part
                %Incompatible.
                \begin{flushleft}
                    $\begin{cases}
                        \nonumber
                        x - 2y  = 3 \\
                        \nonumber
                        2x - 4y = 15
                    \end{cases}$
                \end{flushleft}

                % \begin{solution}
                %     \begin{minipage}[t]{0.95\linewidth}
                %         \centering
                %         \includegraphics[width=\textwidth]{sol51}
                %     \end{minipage}

                %     %\newpage
                %     \begin{multicols}{2}
                %         \center
                %         $2x + y = 2$ 
                %         \vspace{ 1mm } \\
                %         \begin{tabular}{c|c} 
                %             \textbf{x} & \textbf{y} \\
                %             \hline
                %             3 & 0 \\
                %             5 & 1 \\
                %         \end{tabular}
                %         \vspace{ 1mm }
                %         \center
                %         $ x - y = 4$ 
                %         \vspace{ 1mm } \\
                %         \begin{tabular}{c|c} 
                %             \textbf{x} & \textbf{y} \\
                %             \hline
                %             0 & $-\frac{15}{4}$ \\
                %             4 & $-\frac{9}{4}$\\
                %         \end{tabular}
                %     \end{multicols}
    
                %     \begin{tikzpicture}[scale=0.7]
                %         \tkzInit[xmax=6,ymax=6,xmin=-6,ymin=-6]
                %         % \tkzGrid
                %         \tkzAxeXY
                %         \draw[thick] (-6,-4.5) -- (6,1.5) node[anchor=south west] {$x - 2y = 3$};
                %         \draw[thick] (-4.5,-6) -- (6,-0.75) node[anchor=north west] {$2x - 4y = 15$};  
                %     \end{tikzpicture}
                % \end{solution}

            \part
                % Compatible indeterminado.
                \begin{flushleft}
                    $\begin{cases}
                        \nonumber
                        x - 2y  = 3 \\
                        \nonumber
                        3x - 6y = 9
                    \end{cases}$
                \end{flushleft}

                % \begin{solution}
                %     \begin{minipage}[t]{0.95\linewidth}
                %         \centering
                %         \includegraphics[width=\textwidth]{sol52}
                %     \end{minipage}
                % \end{solution}
        \end{parts}
    \end{questions}    

\end{document}