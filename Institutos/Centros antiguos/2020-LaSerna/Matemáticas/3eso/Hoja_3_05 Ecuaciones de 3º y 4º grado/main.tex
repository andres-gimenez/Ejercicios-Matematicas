%%%%%%%%%%%%%%%%%%%%%%%%%%%
\newcommand{\numeroHoja} { Hoja 3.05 }
\newcommand{\nombreHoja} { Ecuaciones de 3º y 4º grado }
%%%%%%%%%%%%%%%%%%%%%%%%%%%

% Configuración del documento.
\renewcommand{\schoolSubject} { Examen Matemáticas 2º ESO  }
\renewcommand{\school} { IES José de Churriguera  }
\renewcommand{\academicPeriod} { Curso 2022/2023 }

\renewcommand{\autor} { Andrés Giménez Muñoz }
\renewcommand{\emailAuthor} { andresprofemates@outlook.es }
\renewcommand{\autorSing}{ Profesor: Andrés } 
\newcommand{\schoolSubject} { Matemáticas 3º ESO - Recuperación}
\newcommand{\school} { IES La Serna }
\newcommand{\academicPeriod} { Curso 2020/2021 }


\newcommand{\autor} { Andrés Giménez Muñoz }
\newcommand{\emailAuthor} { agimenezmunoz@ieslaserna.com }
\newcommand{\autorSing}{ Profesores: Andrés } 

%%%%%%%%%%%%%%%%%%%%%%%%%%%
% Exam configuration
\pointsdroppedatright   %% No mostrar la puntuación

%% Si se comenta no aparecerán los espacios de la solución.
%\nocancelspace

%% Esto es de la clase exam. Si dejamos sin comentar \printanswers, se mostraran las soluciones. 
%% Si la comentamos y dejamos sin comentar \noprintanswers, pues no se muestran las soluciones.
\printanswers
%\noprintanswers

%%%%%%%%%%%%%%%%%%%%%%%%%%%

\begin{document}

    %%\StudentData

    %\begin{center}
        \begin{wrapfigure}{l}{0.12\textwidth}
		    \includegraphics[scale=2]{Diophantos}
        \end{wrapfigure}
		Diofanto fué un matemático de la escuela de Alejandría, durante el siglo III, el cual empezó a estudiar las ecuaciones con soluciones en los números enteros.
        Por este motivo, las ecuaciones con soluciones en los números enteros $(\mathbb Z)$ se las llama ecuaciones diofánticas. \\

        Un problema abierto durante siglos fué encontrar una solución general para todas las ecuaciones diofánticas, 
        o lo que es lo mismo, un método para encontrar todas las soluciones enteras para cualquier ecuación.
        En 1970 se demostró que dicho método general no existe,
        por ello, hay que abordar este tipo de problemas por separado para distintos tipos de ecuación diofántica.

	%\end{center}

    \begin{questions}
        \question
        Utilizando los problemas, ya resueltos de la hoja 3.2, resuelve las siguientes ecuaciones.
        \begin{parts}
            \part
                $x^2-x-12 = 0$
            \part
                $x^3+x^2-17x+15 = 0$
            \part
                $x^3+7x^2+2x-40 = 0$
            \part
                $x^3+6x^2+x+6 = 0$
        \end{parts}

        \question
        Calcula todas las soluciones de las ecuaciones \\
        \scriptsize{*{Si no se puede encontrar todas las soluciones por Ruffini, se puede usar la fórmula de la ecuación de 2º grado.}}
        \normalsize

        \begin{parts}
            \part
                $4x^3-8x^2-x+2 = 0$
            \part
                $6x^4-17x^3+17x^2-7x+1 = 0$
        \end{parts}

        \question
        Resuelve las siguientes ecuaciones bicuadradas \\
        \scriptsize{*{Las ecuaciones de 4º grados, de la forma $ax^4+bx^2+c = 0$ se les llama ecuaciones bicuadradas y se pueden resolver haciendo el cambio de variable $t=x^2$.}}
        \normalsize
        \begin{parts}
            \part
                $-2x^4+58x^2-200=0$
               \begin{sample} Realizamos el cambio de variable $t=x^2$. \\ \\
                    Resolvemos la ecuación $-2t^2+58t-200=0$.

                    \begin{multline*}
                        t = \frac{{ - 58 \pm \sqrt {58^2 - 4\cdot (-2) \cdot (-200)} }}{2\cdot (-2)} = 
                        \frac{ - 58 \pm \sqrt {3364 - 1600}}{-4} = \dots \\
                        \dots = \frac{ - 58 \pm \sqrt {1764}}{-4} = \frac{ - 58 \pm 42}{-4}
                        % 2 P(x) \cdot 4 Q(x) = 2\left(2x-3x^2\right) \cdot 4 \left(2x^2 - 3\right) = \left(4x-6x^2\right) \left(8x^2-12\right) = \dots \\
                        % 32x^3 -48x - 48x^4 + 72x^2 = \boxed{-48x^4+32x^3+72x^2-48x}
                        %\boxed{3} \\
                    \end{multline*}
                    \begin{align*}
                        t & = \frac{-58 + 42}{-4} = \frac{-16}{-4} = 4 \\
                        t & = \frac{-58 - 42}{-4} = \frac{-100}{-4} = 25
                    \end{align*}

                    Deshacemos el cambio de variable:
                    \begin{equation*}
                    t=x^2 \Rightarrow x = \pm \sqrt{t}
                    \end{equation*}
                    \begin{align*}
                        x = \pm \sqrt{4} = \pm 2 \\
                        x = \pm \sqrt{5} = \pm 5 \\
                    \end{align*}
                    Tenemos cuatro soluciones de la ecuación.
                    \begin{align*}
                        \boxed{x = \phantom{-} 2} \\
                        \boxed{x = -2} \\
                        \boxed{x = \phantom{-} 5} \\
                        \boxed{x = -5} \\
                    \end{align*}
                \end{sample}

            \part
                $x^4-13x^2+36=0$
            \part
                $x^4-6x^2+8=0$
        \end{parts}
    \end{questions}
\end{document}