%%%%%%%%%%%%%%%%%%%%%%%%%%%
\newcommand{\numeroHoja} { Unidad 3 }
\newcommand{\nombreHoja} { Proporcionalidad } 
%%%%%%%%%%%%%%%%%%%%%%%%%%%

% Configuración del documento.
\renewcommand{\schoolSubject} { Examen Matemáticas 2º ESO  }
\renewcommand{\school} { IES José de Churriguera  }
\renewcommand{\academicPeriod} { Curso 2022/2023 }

\renewcommand{\autor} { Andrés Giménez Muñoz }
\renewcommand{\emailAuthor} { andresprofemates@outlook.es }
\renewcommand{\autorSing}{ Profesor: Andrés } 
\newcommand{\schoolSubject} { Matemáticas 3º ESO - Recuperación}
\newcommand{\school} { IES La Serna }
\newcommand{\academicPeriod} { Curso 2020/2021 }


\newcommand{\autor} { Andrés Giménez Muñoz }
\newcommand{\emailAuthor} { agimenezmunoz@ieslaserna.com }
\newcommand{\autorSing}{ Profesores: Andrés } 

%%%%%%%%%%%%%%%%%%%%%%%%%%%
% Exam configuration
%\pointsdroppedatright   %% No mostrar la puntuación
\pointsinrightmargin % Para poner las puntuaciones a la derecha. Se puede cambiar. Si se comenta, sale a la izquierda.
\extrawidth{-1.5cm} %Un poquito más de margen por si ponemos textos largos.
\marginpointname{ \emph{\points}}

%% Si se comenta no aparecerán los espacios de la solución.
%\nocancelspace

%% Esto es de la clase exam. Si dejamos sin comentar \printanswers, se mostraran las soluciones. 
%% Si la comentamos y dejamos sin comentar \noprintanswers, pues no se muestran las soluciones.
%\printanswers
%\noprintanswers

%%%%%%%%%%%%%%%%%%%%%%%%%%%

\begin{document}

	\StudentData
	\GradeTableHeader
	
	\justifying

	\begin{questions}
		\setcounter{question}{0}
		\question[2]
		Juan reparte 240\euro{} entre sus tres sobrinos, de 15, 16 y 17 años, 
		de forma directamente proporcional a su edad. ¿Cuánto dinero recibe cada uno?
		\vspace{\stretch{1}}

		\question[2]
		Calcula el valor de la incógnita en cada una de las relaciones de proporcionalidad:
		\begin{parts}
			\setcounter{partno}{0}
			\part $\frac{3}{x} = \frac{1}{3}$
			\part $\frac{2}{x} = \frac{7}{5}$
			\part $\frac{x}{8} = \frac{2}{3}$
			\part $\frac{6}{x} = 2$
			\part $\frac{4}{3} = \frac{x}{6}$

			\part $\frac{1}{2} = \frac{4}{x}$
			\part $\frac{2x}{3} = 4$
			\part $\frac{5}{2x} = \frac{1}{10}$
			% \part Divide $v^6+1$ by $v-1$.
			% \begin{solution}
			% 	Solution
			% \end{solution}
		\end{parts}

		\question[2]
		Tres personas se han repartido una cantidad de dinero directamente proporcional a los números 6, 3 y 2. 
		Si la que menos recibe ha recibido 900\euro{}. ¿qué cantidad total se repartió?
		\vspace{\stretch{1}}

		\question[2]
		Al final de la liga, el Real Madrid reparte una prima de 662.000\euro{} entre Marcelo, Carvajal y Modric, 
		los tres jugadores que menos faltas han retenido en la temporada. 
		El reparto se hará de forma inversamente proporcional a las faltas cometidas, que fueron 2, 3 y 5, respectivamente. 
		¿Cuánto recibirá cada uno?
		\vspace{\stretch{1}}

		\question[2]
		Juan, Olivia y Rafa tienen un bar y se reparten las ganancias del mes de forma inversamente proporcional al número de días que han descansado este mes.
		Las ganancias del mes fueron de 4.230\euro{} y los socios descansaron 8, 6 y 10 días respectivamente. ¿Cuánto dinero le corresponde a cada uno?
		\vspace{\stretch{1}}

	\end{questions}
\end{document}