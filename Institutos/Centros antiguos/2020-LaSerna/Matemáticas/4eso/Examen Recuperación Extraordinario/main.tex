%%%%%%%%%%%%%%%%%%%%%%%%%%%
\newcommand{\documentName} { Examen recuperación }
\newcommand{\documentContent} { Convocatoria extraordinaria } 
\newcommand{\waterMark} { } 
%%%%%%%%%%%%%%%%%%%%%%%%%%%

% Configuración del documento.
\newcommand{\schoolSubject} { Matemáticas 3º ESO - Recuperación}
\newcommand{\school} { IES La Serna }
\newcommand{\academicPeriod} { Curso 2020/2021 }


\newcommand{\autor} { Andrés Giménez Muñoz }
\newcommand{\emailAuthor} { agimenezmunoz@ieslaserna.com }
\newcommand{\autorSing}{ Profesores: Andrés } 
\renewcommand{\schoolSubject} { Examen Matemáticas 2º ESO  }
\renewcommand{\school} { IES José de Churriguera  }
\renewcommand{\academicPeriod} { Curso 2022/2023 }

\renewcommand{\autor} { Andrés Giménez Muñoz }
\renewcommand{\emailAuthor} { andresprofemates@outlook.es }
\renewcommand{\autorSing}{ Profesor: Andrés } 

%%%%%%%%%%%%%%%%%%%%%%%%%%%
% Exam configuration
%\pointsdroppedatright   %% No mostrar la puntuación
\pointsinrightmargin % Para poner las puntuaciones a la derecha. Se puede cambiar. Si se comenta, sale a la izquierda.
\extrawidth{-1.5cm} %Un poquito más de margen por si ponemos textos largos.
\marginpointname{ \emph{\points}}

%% Si se comenta no aparecerán los espacios de la solución.
%\nocancelspace

%% Esto es de la clase exam. Si dejamos sin comentar \printanswers, se mostraran las soluciones. 
%% Si la comentamos y dejamos sin comentar \noprintanswers, pues no se muestran las soluciones.
%\printanswers
%\noprintanswers

%%%%%%%%%%%%%%%%%%%%%%%%%%%

% \usepackage{tikz}
% \usetikzlibrary{arrows}

\begin{document}

	\StudentData
	\GradeTableHeader

    \justifying

	\begin{questions}
		\setcounter{question}{0}

        \question[1]
        Calcula y simplifica las siguientes expresiones.
        \begin{parts}
            \part
            $\left( \frac{1}{9} \right)^{-1}$
            \vspace{\stretch{1}}
        
            \part
            $\left( - \frac{2}{4} \right)^{-4}$
            \vspace{\stretch{1}}
        
            % \part
            % $\left( - \frac{5}{8} \right)^{-3}$
            % \vspace{\stretch{1}}
            
            % \part
            % $\left( \frac{3}{5} \right)^{4}$
            % \vspace{\stretch{1}}

            % \part
            % $\frac{3}{5} \cdot \frac{5}{4} \cdot \frac{4}{12} \cdot \frac{12}{3} \cdot \frac{3}{2} \cdot \frac{2}{7} : \frac{3}{7} $
            % \vspace{\stretch{1}}
            % \part
            % 	$\frac{1}{25} + \frac{3}{5} - \frac{6}{9} : \frac{14}{12}$
            % 	\vspace{\stretch{1}}
            % \part
            % 	$\left( \frac{3}{5} \cdot \frac{1}{2} - \frac{5}{6} \right) : \frac{1}{3}$
            % 	\vspace{\stretch{1}}
        
            % \part
            % $\frac{3}{5} \cdot \left( \frac{1}{2} + \frac{5}{15} \right) : \frac{1}{5}$
            % \vspace{\stretch{1}}
        
            \part
            $2 + \frac{3}{3} - \left( \frac{5}{6} + \frac{7}{12} \right) - \frac{1}{3}$
            \vspace{\stretch{3}}

        \end{parts}

        \question[2]
        A tres amigos les han tocado el segundo premio de la lotería de navidad valorado en $125.000\euro{}$
        por un décimo que costó $20\euro{}$ al que aportaron 8\euro{}, 7\euro{} y 5\euro{} respectivamente,
        ¿cómo han de repartir el premio?
        \vspace{\stretch{6}}

        \newpage

        % \question[2]
        % Cuatro empleados de una tienda de moda tardan 8 días en coser 6 vestidos.
        % ¿Cuánto tiempo tardarían en coser 24 vestidos si se duplica la plantilla?
        % \vspace{\stretch{1}}

        \question[1]
        ¿Qué interés producen ¡7.000\euro{} al 2 \% anual durante 3 años?
        \vspace{\stretch{1}}

        \question[2]
        Ayer compré para mis alumnos de la clase de Matemáticas 7 cuadernos y 12 rotuladores y hoy he comprado 4 cuadernos y 6 rotuladores. 
        Si la factura de ayer fue de 51 €, y la de hoy, de 27 €, ¿cuánto vale cada cuaderno y cada rotulador?
        \vspace{\stretch{2}}

        \newpage
		\question[2]
        Resuelve el siguiente sistema de ecuaciones por el método gráfico.
            \begin{flushleft}
                $\begin{cases}
                    \nonumber
                    x - y  = 1 \\
                    \nonumber
                    x + y = 3
                \end{cases}$
            \end{flushleft}

            \begin{figure}[h]
                \begin{tikzpicture}[scale=0.9]
                    \tkzInit[xmax=8,ymax=8,xmin=-8,ymin=-8]
                    \tkzGrid
                    \tkzAxeXY
                \end{tikzpicture}
            \end{figure}

            \newpage
            % \question[3]
            % Resuelve los siguientes sistemas de ecuaciones
            % \begin{parts}
            %     \part
            %     \begin{flushleft}
            %         $\begin{cases}
            %             \nonumber
            %             x + y  = 9 \\
            %             \nonumber
            %             2x - y = -3 
            %         \end{cases}$
            %     \end{flushleft}
            %     \vspace{\stretch{1}}

            %     \part
            %     \begin{flushleft}
            %         $\begin{cases}
            %             \nonumber
            %             x - 5y = 4 \\
            %             \nonumber
            %             3x - y = 2 
            %         \end{cases}$
            %     \end{flushleft}
            %     \vspace{\stretch{1}}
            % \end{parts}
        % \newpage
        \newpage
        \question[2]
            Durante el mes de junio las temperaturas máximas en una ciudad han sido: \\
                
             8, 27, 28, 29, 25, 24, 23, 25, 27, 30, 29, 28, 29, 30, 26, 28, 31, 30, 30, 27, 29, 28, 30, 29, 30, 31, 32, 29, 30, 31

             \vspace{1cm}
             \begin{parts}
             \part
             Calcula la tabla de frecuencias relativas y acumuladas.
             \vspace{\stretch{1}}
         
             \part
             %Media = 2,75
             Calcula la media, la mediana y moda.
             \begin{equation*}
                 \bar{x}=\frac{\sum{x_i f_i}}{N} = \sum{x_i h_i}
             \end{equation*}
             \begin{equation*}
                 N = \sum{f_i}
             \end{equation*}
             \vspace{\stretch{1}}
            \end{parts}

	\end{questions}
\end{document}