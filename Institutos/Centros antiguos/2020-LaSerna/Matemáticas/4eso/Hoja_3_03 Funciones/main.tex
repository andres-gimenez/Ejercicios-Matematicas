%%%%%%%%%%%%%%%%%%%%%%%%%%%
\newcommand{\numeroHoja} { Hoja 3.3 }
\newcommand{\nombreHoja} { Funciones }
%%%%%%%%%%%%%%%%%%%%%%%%%%%

% Configuración del documento.
\renewcommand{\schoolSubject} { Examen Matemáticas 2º ESO  }
\renewcommand{\school} { IES José de Churriguera  }
\renewcommand{\academicPeriod} { Curso 2022/2023 }

\renewcommand{\autor} { Andrés Giménez Muñoz }
\renewcommand{\emailAuthor} { andresprofemates@outlook.es }
\renewcommand{\autorSing}{ Profesor: Andrés } 
\newcommand{\schoolSubject} { Matemáticas 3º ESO - Recuperación}
\newcommand{\school} { IES La Serna }
\newcommand{\academicPeriod} { Curso 2020/2021 }


\newcommand{\autor} { Andrés Giménez Muñoz }
\newcommand{\emailAuthor} { agimenezmunoz@ieslaserna.com }
\newcommand{\autorSing}{ Profesores: Andrés } 

%%%%%%%%%%%%%%%%%%%%%%%%%%%
% Exam configuration
\pointsdroppedatright   %% No mostrar la puntuación

%% Si se comenta no aparecerán los espacios de la solución.
%\nocancelspace

%% Esto es de la clase exam. Si dejamos sin comentar \printanswers, se mostraran las soluciones. 
%% Si la comentamos y dejamos sin comentar \noprintanswers, pues no se muestran las soluciones.
\printanswers
%\noprintanswers

%%%%%%%%%%%%%%%%%%%%%%%%%%%

\usepackage{tkz-euclide}

\begin{document}
    \begin{questions}
        \question
        Representa las siguientes funciones en un sistema de coordenadas cartesianas. \\
        \scriptsize{*{Con la calculadora, prueba con distintos valores de $x$, y representa los valores en un eje de coordenadas.}}
        \normalsize

        \begin{parts}
            \part
            $f(x)=x^2$
            \part
            $f(x)=x^3$
            \part
            $f(x)=\frac{1}{x}$
            \part
            $f(x)=\sqrt{x}$
        \end{parts}

        \question
        Compara la grafica de las siguientes funciones, representando cada pareja en un eje de coordenadas.
        \begin{parts}
            \part
            $f(x)=x$, $f(x)=\sqrt{x}$
            \part
            $f(x)=x^2$, $f(x)=\sqrt{x}$
            \part
            $f(x)=x^2$, $f(x)=x^3$

        \end{parts}

        \question
        %Copiada del libro
        Para un grupo de 50 o más personas una empresa de trasportes ofrece, para una excursión, un precio por persona, en euros, según la fórmula: \\
        $P(n)=40-0,5(n-50), n> 5$, donde $n$ es el número de excursionistas.

        \begin{parts}
            \part
            Escribe cuál será el ingreso, G(n), para la empresa en función del número de excursionistas.

            \part
            Completa la tabla
                \begin{table}[h]
                    \centering
                    \begin{tabular}{|
                    >{\columncolor[HTML]{CBCEFB}}c |c|c|c|c|c|c|c|}
                    \hline
                    n & 60 & 70 & 90 & 110 & 120 & 140 & 160 \\ \hline
                    G &    &    &    &     &     &     &     \\ \hline
                    \end{tabular}
                \end{table}
            \part
            A la vista de los resultados, ¿cuál parece ser el número de personas más conveniente para la empresa?

            % \part
            % Determina de manera exacta y razonada cuál es ese número de personas y cuánto ingresaría la empresa.
        \end{parts}

        \question
        %Copiada del libro
        El coste de producir $n$ palas de pádel viene dado por la expresión: \\ $p(n)=40+16\sqrt{n-1}$, con $n\leq50$. \\
        Si la empresa pretende ganar un $50\%$ en la venta de cada pala, determina:
        \begin{parts}
            \part
                El precio $U(n)$ de cada una de las plas al producir $n$.
            \part
                ¿A qué precio deberá vender cada pala, si produce 17?
            \part
                ¿Cuánto dinero ganará si produce 30 palas, pero solo logran vender 26?
            \part
                La ganancia, G(n), al producir y vender $n$ palas.
            % \part
            %     Analiza si $P(N)$, $U(n)$ y $G(n)$ son crecientes o decrecientes.
        \end{parts}

    \end{questions}    

\end{document}