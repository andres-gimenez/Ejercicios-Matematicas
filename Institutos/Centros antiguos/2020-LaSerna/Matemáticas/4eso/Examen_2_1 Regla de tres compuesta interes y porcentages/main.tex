%%%%%%%%%%%%%%%%%%%%%%%%%%%
\newcommand{\numeroHoja} { 2º Evaluación }
\newcommand{\nombreHoja} { Regla de tres compuesto, interes y porcentajes }
%%%%%%%%%%%%%%%%%%%%%%%%%%%

% Configuración del documento.
\renewcommand{\schoolSubject} { Examen Matemáticas 2º ESO  }
\renewcommand{\school} { IES José de Churriguera  }
\renewcommand{\academicPeriod} { Curso 2022/2023 }

\renewcommand{\autor} { Andrés Giménez Muñoz }
\renewcommand{\emailAuthor} { andresprofemates@outlook.es }
\renewcommand{\autorSing}{ Profesor: Andrés } 
\newcommand{\schoolSubject} { Matemáticas 3º ESO - Recuperación}
\newcommand{\school} { IES La Serna }
\newcommand{\academicPeriod} { Curso 2020/2021 }


\newcommand{\autor} { Andrés Giménez Muñoz }
\newcommand{\emailAuthor} { agimenezmunoz@ieslaserna.com }
\newcommand{\autorSing}{ Profesores: Andrés } 

%%%%%%%%%%%%%%%%%%%%%%%%%%%
% Exam configuration
%\pointsdroppedatright   %% No mostrar la puntuación
\pointsinrightmargin % Para poner las puntuaciones a la derecha. Se puede cambiar. Si se comenta, sale a la izquierda.

%% Si se comenta no aparecerán los espacios de la solución.
%\nocancelspace

%% Esto es de la clase exam. Si dejamos sin comentar \printanswers, se mostraran las soluciones. 
%% Si la comentamos y dejamos sin comentar \noprintanswers, pues no se muestran las soluciones.
%\printanswers
%\noprintanswers

%%%%%%%%%%%%%%%%%%%%%%%%%%%

\begin{document}

	\StudentData
	\GradeTableHeader

	\begin{questions}

		\question[2]
		Una receta para cocinar Mousse de yogur con frutos rojos, nos dice que para 7 raciones se requieren 250 gramos de yogur griego, 
		150 gramos de nata líquida para montar, 2 claras de huevo, 50 gramos de azúcar glasé y 50 gramos de fresa.
		¿Qué cantidad de ingredientes necesitaremos para cocinar 150 raciones?
		\vspace{\stretch{1}}

		\question[2]
		Nueve grifos abiertos durante 40 horas han consumido 200 litros de agua ¿Cuántos litros consumen 15 grifos durante 9 horas?
		\vspace{\stretch{1}}

		\newpage

		\question[2]
		Cuatro empleados de una tienda de moda tardan 8 días en coser 6 vestidos.
		¿Cuánto tiempo tardarían en coser 24 vestidos si se duplica la plantilla?
		\vspace{\stretch{1}}

		\question[2]
		Calcula el 7\% de 42,18\euro{}, 1.200\euro{} y 354,07\euro{}.
		\vspace{\stretch{1}}

		\question[2]
		¿Qué interés producen 4.000\euro{} al 7 \% anual durante 3 años?
		\vspace{\stretch{1}}

	\end{questions}
\end{document}