%%%%%%%%%%%%%%%%%%%%%%%%%%%
\newcommand{\numeroHoja} { Hoja 2.3 }
\newcommand{\nombreHoja} { Interés simple }
%%%%%%%%%%%%%%%%%%%%%%%%%%%

% Configuración del documento.
\renewcommand{\schoolSubject} { Examen Matemáticas 2º ESO  }
\renewcommand{\school} { IES José de Churriguera  }
\renewcommand{\academicPeriod} { Curso 2022/2023 }

\renewcommand{\autor} { Andrés Giménez Muñoz }
\renewcommand{\emailAuthor} { andresprofemates@outlook.es }
\renewcommand{\autorSing}{ Profesor: Andrés } 
\newcommand{\schoolSubject} { Matemáticas 3º ESO - Recuperación}
\newcommand{\school} { IES La Serna }
\newcommand{\academicPeriod} { Curso 2020/2021 }


\newcommand{\autor} { Andrés Giménez Muñoz }
\newcommand{\emailAuthor} { agimenezmunoz@ieslaserna.com }
\newcommand{\autorSing}{ Profesores: Andrés } 

%%%%%%%%%%%%%%%%%%%%%%%%%%%
% Exam configuration
\pointsdroppedatright   %% No mostrar la puntuación

%% Si se comenta no aparecerán los espacios de la solución.
%\nocancelspace

%% Esto es de la clase exam. Si dejamos sin comentar \printanswers, se mostraran las soluciones. 
%% Si la comentamos y dejamos sin comentar \noprintanswers, pues no se muestran las soluciones.
%\printanswers
%\noprintanswers

%%%%%%%%%%%%%%%%%%%%%%%%%%%

\begin{document}

    %%\StudentData

    \begin{questions}

        \setcounter{question}{0}
		% Aritmética Razonada
		% Cálculo mental rápido
		\question
		Calcula el 5\% de 800\euro{}, 1.500\euro{} y 250\euro{}.

		\question
		Calcula el 3,5\% de 200\euro{}, 5.000\euro{} y 6.600\euro{}.

		%Interes simple
		\question 
		¿Qué interés producen 4.000\euro{} al 6\% anual durante:
		\begin{parts}
            \part
                3 meses.
            \part
				60 días.
			\part
				40 días.
			\part
				2 años.
		\end{parts}

		\question
        Me quiero comprar unos pantalones que valen 60\euro{}. ¿Cuánto me ahorro si me hacen unas rebajas del $35\%$? 
		
		\question
		¿Durante cuánto tiempo hay que colocar 900\euro{} al 5\% para tener 180\euro{}?

		\question
		¿Qué capital hay que colocar al 6\% para que en 4 años produzcan 120\euro{}?

		\question
		Un restaurante compra el solomillo de ternera a $31,90 \euro{}/kg$, las cuales al hacerlos raciones tiene una merma del 10\%. 
		¿A cuánto le cuesta al restaurante la ración de 200 gramos de solomillo de ternera?

		\question
		Un pastelero necesita comprar almendras para su pastelería. Pregunta a sus proveedores a cuánto le venden la almendra y recibe las siguientes ofertas.
		Paquetes de 5Kg de almendra a 59\euro{}, paquete de 10kg a 123\euro{} con un 3\% de descuento y paquete de 20Kg a 245\euro{} con un 4\% de descuento.

		\begin{parts}
			\part
				¿Donde le interesa comprar?
				\begin{solution}
					Calculamos el precio real de cada oferta por kilogramo. \\ \\
					1º oferta: $\frac{59 {\euro{}}}{5Kg} = 11,8 {\euro{}}/Kg$ \\ \\
					2º oferta: $\frac{121 {\euro{}}}{10 Kg} \cdot (1 - 3\%) = \frac{121}{10} \cdot (1 - \frac{3}{100}) = \frac{121}{10} \cdot 0,97 = 11,737 {\euro{}}/Kg$ \\ \\
					3º oferta: $\frac{245 {\euro{}}}{20 Kg} \cdot (1 - 4\%) = \frac{245}{20} \cdot (1 - \frac{4}{100}) = \frac{245}{20} \cdot 0.96 = 11,76 {\euro{}}/Kg$ \\ \\
					\\
					La segunda oferta es la más barata.
				\end{solution}
			\part
				¿Si consume 100 Kg de almendra al mes, cuánto se ahorra al año entre la más cara y la más barata?
				\begin{solution}
					La diferencia entre la más cara y la más barata son de 
					\\ \\ 
					$11,8 {\euro}/kg - 11,737 {\euro}/Kg = 0,063 {\euro}/Kg$
					\\ \\
					En un año:
					\\ \\
					$0,063 \euro/Kg \cdot 100Kg \cdot 12 meses = 75,6 \euro$

				\end{solution}
		\end{parts}
		\question
		¿Si a una cantidad fija le hacemos un descuento primero del 5\% y luego del 7\%, es lo mismo que hacer primero el descuento del 7\% y luego del 5\%?
		¿Serias capaz de dar una justificación matemática?

		\question
		Un padre quiere repartir 52.338 \euro{} entre sus 4 hijos, de manera que a la edad de 18 años tengan cantidades iguales. 
		Los hijos tienen 9, 11, 13 y 15 años, y el dinero que reciben se impone en un banco al 5\% de interés simple.
		¿Qué cantidad le ha de dar a cada hijo?
    \end{questions}
\end{document}