%%%%%%%%%%%%%%%%%%%%%%%%%%%
\newcommand{\numeroHoja} { Hoja 2.6 }
\newcommand{\nombreHoja} { Expresiones algebraicas y figuras }
%%%%%%%%%%%%%%%%%%%%%%%%%%%

% Configuración del documento.
\renewcommand{\schoolSubject} { Examen Matemáticas 2º ESO  }
\renewcommand{\school} { IES José de Churriguera  }
\renewcommand{\academicPeriod} { Curso 2022/2023 }

\renewcommand{\autor} { Andrés Giménez Muñoz }
\renewcommand{\emailAuthor} { andresprofemates@outlook.es }
\renewcommand{\autorSing}{ Profesor: Andrés } 
\newcommand{\schoolSubject} { Matemáticas 3º ESO - Recuperación}
\newcommand{\school} { IES La Serna }
\newcommand{\academicPeriod} { Curso 2020/2021 }


\newcommand{\autor} { Andrés Giménez Muñoz }
\newcommand{\emailAuthor} { agimenezmunoz@ieslaserna.com }
\newcommand{\autorSing}{ Profesores: Andrés } 

%%%%%%%%%%%%%%%%%%%%%%%%%%%
% Exam configuration
\pointsdroppedatright   %% No mostrar la puntuación

%% Si se comenta no aparecerán los espacios de la solución.
%\nocancelspace

%% Esto es de la clase exam. Si dejamos sin comentar \printanswers, se mostraran las soluciones. 
%% Si la comentamos y dejamos sin comentar \noprintanswers, pues no se muestran las soluciones.
\printanswers
%\noprintanswers

%%%%%%%%%%%%%%%%%%%%%%%%%%%

\begin{document}

    %%\StudentData

    \begin{questions}
        % \question 
        % Un campo de fútbol mide 30 metros más de largo que de ancho y su área es de $7.000 m^2$, halla sus dimensiones.

        \question
            Expresa algebraicamente el área y el perímetro de la siguiente figura.

            \begin{minipage}[t]{\linewidth}
                \centering
                \includegraphics[width=7cm]{figura1}
                %\captionof{figure}{Figure of Q.\ref{label:a}}
                \label{label:q1}
            \end{minipage}

            \begin{solution}
                \begin{minipage}[t]{0.95\linewidth}
                    \centering
                    \includegraphics[width=\textwidth]{sol11}
                \end{minipage}
            \end{solution}

            \begin{parts}
                \part
                Calcula el aria y el perímetro para x = 5 e y = 2.
                \begin{solution}
                    \begin{minipage}[t]{\linewidth}
                        \centering
                        \includegraphics[width=\textwidth]{sol12}
                    \end{minipage}
                \end{solution}
                \part
                Calcula el aria y el perímetro para x = 10 e y = 5.
                \begin{solution}
                    \begin{minipage}[t]{\linewidth}
                        \centering
                        \includegraphics[width=\textwidth]{sol13}
                    \end{minipage}
                \end{solution}
            \end{parts}

            \newpage
            \question
            Expresa algebraicamente el área de y el perímetro de la siguiente figura y calcula los para x = 5 e x = 10. 

            \begin{minipage}[t]{\linewidth}
                \centering
                \includegraphics[width=7cm]{figura2}
                %\captionof{figure}{Figure of Q.\ref{label:a}}
                \label{label:q1}
            \end{minipage}
            \begin{solution}
                \begin{minipage}[t]{0.85\linewidth}
                    \centering
                    \includegraphics[width=\textwidth]{sol21}
                \end{minipage}
            \end{solution}

            \begin{solution}
                \begin{minipage}[t]{0.95\linewidth}
                    \centering
                    \includegraphics[width=\textwidth]{sol22}
                \end{minipage}
            \end{solution}

            %¿Tiene sentido calcular el área y el perímetro para x = 1? ¿Por qué?
           
        % \question
        %     Invéntate una figura geométrica y indica la expresión algebraica que representa su área y su perímetro.
    \end{questions}
\end{document}