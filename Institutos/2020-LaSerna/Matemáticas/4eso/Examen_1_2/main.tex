%%%%%%%%%%%%%%%%%%%%%%%%%%%
\newcommand{\numeroHoja} { Unidad 3 }
\newcommand{\nombreHoja} { Potencias y raíces }
%%%%%%%%%%%%%%%%%%%%%%%%%%%

% Configuración del documento.
\renewcommand{\schoolSubject} { Examen Matemáticas 2º ESO  }
\renewcommand{\school} { IES José de Churriguera  }
\renewcommand{\academicPeriod} { Curso 2022/2023 }

\renewcommand{\autor} { Andrés Giménez Muñoz }
\renewcommand{\emailAuthor} { andresprofemates@outlook.es }
\renewcommand{\autorSing}{ Profesor: Andrés } 
\newcommand{\schoolSubject} { Matemáticas 3º ESO - Recuperación}
\newcommand{\school} { IES La Serna }
\newcommand{\academicPeriod} { Curso 2020/2021 }


\newcommand{\autor} { Andrés Giménez Muñoz }
\newcommand{\emailAuthor} { agimenezmunoz@ieslaserna.com }
\newcommand{\autorSing}{ Profesores: Andrés } 

%%%%%%%%%%%%%%%%%%%%%%%%%%%
% Exam configuration
%\pointsdroppedatright   %% No mostrar la puntuación
\pointsinrightmargin % Para poner las puntuaciones a la derecha. Se puede cambiar. Si se comenta, sale a la izquierda.

%% Si se comenta no aparecerán los espacios de la solución.
%\nocancelspace

%% Esto es de la clase exam. Si dejamos sin comentar \printanswers, se mostraran las soluciones. 
%% Si la comentamos y dejamos sin comentar \noprintanswers, pues no se muestran las soluciones.
%\printanswers
%\noprintanswers

%%%%%%%%%%%%%%%%%%%%%%%%%%%

\begin{document}

	\StudentData
	\GradeTableHeader

	\begin{questions}
		\setcounter{question}{0}
		\question[2]
		Calcula el valor de las siguientes potencias de fracciones.
		\begin{parts}
            \part
            $\left( \frac{1}{9} \right)^{-1}$
			\part
            $\left( - \frac{3}{4} \right)^{-4}$
			\part
            $\left( - \frac{5}{8} \right)^{-3}$
			\part
            $\left( \frac{3}{5} \right)^{4}$
        \end{parts}

		\question[2]
		Completa la siguiente tabla. \\
		
		\begin{table}[h]
			\centering
			%\caption{}
			%\label{tab:my-table}
			\begin{tabular}{|c|c|c|}
			\hline
			\rowcolor[gray]{.9}
			\textbf{Escritura decimal} & \textbf{Escritura P10} & \textbf{Notación científica} \\ \hline
			$25.000.000$  &                     &                       \\ \hline
			$0,0000043$   &                     &                       \\ \hline
			              & $29 \cdot 10^{-3}$  &                       \\ \hline
			              & $438 \cdot 10^{5}$  &                       \\ \hline
						  &                     & $3,48 \cdot 10^{-4}$  \\ \hline 
						  &                     & $1,3 \cdot 10^{5}$    \\ \hline						  
			\end{tabular}
		\end{table}
		
		\question[2]
		Reduce a índice común y ordena de mayor a menor.
		\begin{parts}
			\part
			$\sqrt[4]{4}, \sqrt[8]{8}, \sqrt[6]{6}$
			\part
			$\sqrt[5]{5}, \sqrt{2}, \sqrt[10]{20}$
		\end{parts}

		\question[2]
		Racionaliza y simplifica.
		\begin{parts}
			\part
			$\frac{3}{\sqrt{6}}$
			\part
			$\frac{10}{\sqrt[3]{5}}$
			\part
			$\frac{\sqrt{3}}{\sqrt[3]{12}}$
			\part
			$\frac{\sqrt{2}}{\sqrt[7]{4^3}}$
		\end{parts}

		\newpage

		\question[2]
		Halla el valor de $x$ en cada uno de los casos, recordando $log_b a = x \Rightarrow b^x = a$
		\begin{parts}
			\part
			$log_2 {\sqrt{2}} = x$
			\part
			$log_3 {729} = x$
			\part
			$log_x {16} = 2$
			\part
			$log_5 x = \frac{1}{2}$
			\part
			$log_x 9 = 4$
			\part
			$log_x 125 = 3$
			\part
			$log_2 \sqrt{8} = x$
			\part
			$log {100.000} = x$
		\end{parts}
	\end{questions}
\end{document}