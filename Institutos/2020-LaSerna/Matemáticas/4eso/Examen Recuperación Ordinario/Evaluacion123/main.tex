%%%%%%%%%%%%%%%%%%%%%%%%%%%
\newcommand{\documentName} { Examen Recuperación }
\newcommand{\documentContent} { Convocatoria ordinaria } 
\newcommand{\waterMark} { } 
%%%%%%%%%%%%%%%%%%%%%%%%%%%

% Configuración del documento.
\newcommand{\schoolSubject} { Matemáticas 3º ESO - Recuperación}
\newcommand{\school} { IES La Serna }
\newcommand{\academicPeriod} { Curso 2020/2021 }


\newcommand{\autor} { Andrés Giménez Muñoz }
\newcommand{\emailAuthor} { agimenezmunoz@ieslaserna.com }
\newcommand{\autorSing}{ Profesores: Andrés } 
\renewcommand{\schoolSubject} { Examen Matemáticas 2º ESO  }
\renewcommand{\school} { IES José de Churriguera  }
\renewcommand{\academicPeriod} { Curso 2022/2023 }

\renewcommand{\autor} { Andrés Giménez Muñoz }
\renewcommand{\emailAuthor} { andresprofemates@outlook.es }
\renewcommand{\autorSing}{ Profesor: Andrés } 

%%%%%%%%%%%%%%%%%%%%%%%%%%%
% Exam configuration
%\pointsdroppedatright   %% No mostrar la puntuación
\pointsinrightmargin % Para poner las puntuaciones a la derecha. Se puede cambiar. Si se comenta, sale a la izquierda.
\extrawidth{-1.5cm} %Un poquito más de margen por si ponemos textos largos.
\marginpointname{ \emph{\points}}


%% Si se comenta no aparecerán los espacios de la solución.
%\nocancelspace

%% Esto es de la clase exam. Si dejamos sin comentar \printanswers, se mostraran las soluciones. 
%% Si la comentamos y dejamos sin comentar \noprintanswers, pues no se muestran las soluciones.
%\printanswers
%\noprintanswers

%%%%%%%%%%%%%%%%%%%%%%%%%%%

% \usepackage{tikz}
% \usetikzlibrary{arrows}

% \useslantedhalf

% \usepackage{titling}
% \usepackage{float}

\begin{document}

\StudentData
% \GradeTableHeader

\setcounter{question}{0}

\vspace{0.1cm}
\center{1ª Evaluación}
\begin{center}
    \partialgradetable{Evaluacion1}[h][questions]
\end{center}

\center{2ª Evaluación}
\begin{center}
    \partialgradetable{Evaluacion2}[h][questions]
\end{center}

\center{3ª Evaluación}
\begin{center}
    \partialgradetable{Evaluacion3}[h][questions]
\end{center}

\justifying

\begin{questions}
\setcounter{question}{0}

%%%%%%%%%%%%%%%%%%%%%%%%%%%%%%%%%%%%%%%%%%  1ª Evaluación  %%%%%%%%%%%%%%%%%%%%%%%%%%%%%%%%%%%%%%%%%%%%%%%%%%%%%%

\section*{1ª Evaluación}

\begingradingrange{Evaluacion1}

\question[2\half]
Calcula el valor de las siguientes potencias de fracciones y expresa el resultado en forma de fracción irreducible.
\begin{parts}
    \part
    $\left( \frac{1}{9} \right)^{-1}$
    \vspace{\stretch{1}}

    % \part
    % $\left( - \frac{3}{4} \right)^{-4}$
    % \vspace{\stretch{1}}

    % \part
    % $\left( - \frac{5}{8} \right)^{-3}$
    % \vspace{\stretch{1}}
    
    \part
    $\left( \frac{3}{5} \right)^{4}$
    \vspace{\stretch{1}}
\end{parts}

\newpage

\question[2\half]
Realiza las siguientes operaciones y expresa el resultado en forma de fracción irreducible.
\begin{parts}
    \part
    $\frac{3}{5} \cdot \frac{5}{4} \cdot \frac{4}{12} \cdot \frac{12}{3} \cdot \frac{3}{2} \cdot \frac{2}{7} : \frac{3}{7} $
    \vspace{\stretch{1}}
    % \part
    % 	$\frac{1}{25} + \frac{3}{5} - \frac{6}{9} : \frac{14}{12}$
    % 	\vspace{\stretch{1}}
    % \part
    % 	$\left( \frac{3}{5} \cdot \frac{1}{2} - \frac{5}{6} \right) : \frac{1}{3}$
    % 	\vspace{\stretch{1}}

    \part
    $\frac{3}{5} \cdot \left( \frac{1}{2} + \frac{5}{15} \right) : \frac{1}{5}$
    \vspace{\stretch{1}}

    % \part
    % $2 + \frac{3}{3} - \left( \frac{5}{6} + \frac{7}{12} \right) - \frac{1}{3}$
    % \vspace{\stretch{1}}
    % \part 
    % 	$\left( \frac{2}{3} \right)^3 - \left( \frac{2}{3} \right)^{-1}$
    % \part 
    % 	$\frac{40}{18} \cdot \frac{9}{30} - \left( 2 - \frac{3}{2} \right)^{2}$
    % \part 
    % 	$\frac{3}{10} \cdot \left(\frac{1}{7} - \frac{5}{12} \right)^{-1} : \frac{1}{4} \cdot \frac{2}{5}$
\end{parts}

\question[2]
Completa la siguiente tabla. \\

\begin{table}[h!]
    \centering
    %\caption{}
    %\label{tab:my-table}
    \begin{tabular}{|c|c|c|}
        \hline
        \rowcolor[gray]{.9}
        \textbf{Escritura decimal} & \textbf{Escritura P10} & \textbf{Notación científica} \\ \hline
        $25.000.000$               &                        &                              \\ \hline
        % $0,0000043$                &                        &                              \\ \hline
                                   & $29 \cdot 10^{-3}$     &                              \\ \hline
                                %    & $438 \cdot 10^{5}$     &                              \\ \hline
                                   &                        & $3,48 \cdot 10^{-4}$         \\ \hline
                                %    &                        & $1,3 \cdot 10^{5}$           \\ \hline
    \end{tabular}
\end{table}

% \question[2]
% Reduce a índice común y ordena de mayor a menor.
% \begin{parts}
%     \part
%     $\sqrt[4]{4}, \sqrt[8]{8}, \sqrt[6]{6}$
%     \vspace{\stretch{1}}
%     \part
%     $\sqrt[5]{5}, \sqrt{2}, \sqrt[10]{20}$
%     \vspace{\stretch{1}}
% \end{parts}
% \question[3\half]
% Pregunta 2

\newpage
% \question[2]
% Tres personas se han repartido una cantidad de dinero directamente proporcional a los números 6, 3 y 2.
% Si la que menos recibe ha recibido 900\euro{}. ¿qué cantidad total se repartió?
% \vspace{\stretch{1}}

% \question[2]
% Juan, Olivia y Rafa tienen un bar y se reparten las ganancias del mes de forma inversamente proporcional al número de días que han descansado este mes.
% Las ganancias del mes fueron de 4.230\euro{} y los socios descansaron 8, 6 y 10 días respectivamente. ¿Cuánto dinero le corresponde a cada uno?
% \vspace{\stretch{1}}

\question[3]
A tres amigos les han tocado el segundo premio de la lotería de navidad valorado en $125.000\euro{}$
por un décimo que costó $20\euro{}$ al que aportaron 8\euro{}, 7\euro{} y 5\euro{} respectivamente,
¿cómo han de repartir el premio?
\vspace{\stretch{1}}

\endgradingrange{Evaluacion1}

\newpage
%%%%%%%%%%%%%%%%%%%%%%%%%%%%%%%%%%%%%%%%%%  2ª Evaluación  %%%%%%%%%%%%%%%%%%%%%%%%%%%%%%%%%%%%%%%%%%%%%%%%%%%%%%

\section*{2ª Evaluación}

\begingradingrange{Evaluacion2}

% \question[2]
% Si 3 obreros colocan 100 metros cuadrados de suelo en 2 días. ¿Cuántos días tardarán 4 obreros en colocar 1.000 metros cuadrados de suelo?
% \vspace{\stretch{1}}

\question[3]
Cuatro empleados de una tienda de moda tardan 8 días en coser 6 vestidos.
¿Cuánto tiempo tardarían en coser 24 vestidos si se duplica la plantilla?
\vspace{\stretch{1}}

\question[3]
¿Qué interés producen 14.000\euro{} al 3 \% anual durante 5 años?
\vspace{\stretch{1}}

\newpage
\question[3]
Resuelve las siguientes ecuaciones.
\begin{parts}
    \part
    $3x-1 = 2$
    \vspace{\stretch{2}}

    % \part
    % $2x = 3-3x$
    % \vspace{\stretch{2}}
   
    \part
    $x+2(x+1) = 2(3x-1)$
    \vspace{\stretch{2}}

    \part
    $x^2-7x+6=0$
    \vspace{\stretch{3}}

    % \part
    % $(x-2)^2 = 81$
    % \vspace{\stretch{3}}
\end{parts}

\question[1]
Si $P(x)=x^4 - 3 x^3 + 2x^2 - 5x -5$, evalúa.
\begin{parts}
    \part
    $P\left(-1\right)$
    \vspace{\stretch{1}}
    \part
    $P\left(0\right)$
    \vspace{\stretch{1}}
    \part
    $P\left(\frac{1}{2}\right)$
    \vspace{\stretch{1}}
\end{parts}

\endgradingrange{Evaluacion2}

\newpage
%%%%%%%%%%%%%%%%%%%%%%%%%%%%%%%%%%%%%%%%%% 3ª Evaluación  %%%%%%%%%%%%%%%%%%%%%%%%%%%%%%%%%%%%%%%%%%%%%%%%%%%%%%
\section*{3ª Evaluación}

\begingradingrange{Evaluacion3}

% \question[2]
% Resuelve el siguiente sistema de ecuaciones por el método gráfico.
% \begin{flushleft}
%     $\begin{cases}
%             \nonumber
%             x + y  = 2 \\
%             \nonumber
%             3x - y = 2
%         \end{cases}$
% \end{flushleft}

% \begin{figure}[h]
%     \begin{tikzpicture}[scale=1]
%         \tkzInit[xmax=6,ymax=6,xmin=-6,ymin=-6]
%         \tkzGrid
%         \tkzAxeXY
%     \end{tikzpicture}
% \end{figure}

% \newpage
\question[4]
Siete chocolatinas y cinco refrescos cuestan 30\euro{}. Tres chocolatinas y un refresco valen 10,80\euro{}.
¿Cuánto costarán once chocolatinas y nueve refrescos?
\vspace{\stretch{1}}

\newpage
\question[6]
El número de libros solicitados en una biblioteca ha sido:
\begin{table}[h!]
    \centering
    \begin{tabular}{|c|c|c|c|c|c|c|c|}
        \hline
        \cellcolor[gray]{0.8}Libros ($x_i$)   & 1 & 2  & 3 & 4 & 5 & 6 \\
        \hline
        \cellcolor[gray]{0.8}Usuarios ($f_i$) & 8 & 12 & 9 & 6 & 3 & 2 \\
        \hline
    \end{tabular}
\end{table}

\begin{parts}
    \part
    Calcula la tabla de frecuencias relativas y acumuladas.
    \vspace{\stretch{2}}

    \part
    %Media = 2,75
    Calcula la media, la mediana y moda.
    \begin{equation*}
        \bar{x}=\frac{\sum{x_i f_i}}{N} = \sum{x_i h_i}
    \end{equation*}
    \begin{equation*}
        N = \sum{f_i}
    \end{equation*}
    \vspace{\stretch{1}}

    \newpage
    \part
    %Varianza = 1,9375, Desviación típica = 1,39
    Calcula la varianza y la desviación típica.
    \begin{equation*}
        s^2=\frac{\sum{({x_i - \bar{x}})^2} f_i}{N} = \frac{\sum{{x_i}^2 f_i}}{N} - \bar{x}^2 = \sum{{x_i}^2 h_i} - \bar{x}^2
    \end{equation*}

    \begin{equation*}
        s=\sqrt{\frac{\sum{({x_i - \bar{x}})^2} f_i}{N}} = \sqrt{\frac{\sum{{x_i}^2 f_i}}{N} - \bar{x}^2} = \sqrt{\sum{{x_i}^2 h_i} - \bar{x}^2}
    \end{equation*}
    \vspace{\stretch{1}}

    \part
    Calcula el coeficiente de variación.
    % CV = 0,506
    \begin{equation*}
        CV = \frac{s}{\bar{x}}
    \end{equation*}
    \vspace{\stretch{1}}

    \part
    El año anterior se solicitaron una media de $2,4$ libros con una desviación típica de $1,5$.
    Indica, justificando la respuesta, que año se han solicitados más libros, en cual ha se ha producido más diferencia de solicitudes entre solicitantes y que media es más representativa sobre el número de libros que se solicitan en la biblioteca.
    \vspace{\stretch{1}}
\end{parts}
\endgradingrange{Evaluacion3}

\end{questions}
\end{document}