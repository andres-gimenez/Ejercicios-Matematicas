%%%%%%%%%%%%%%%%%%%%%%%%%%%
\newcommand{\documentName} { Examen Recuperación - 2ª Evaluación }
\newcommand{\documentContent} { Convocatoria ordinaria } 
\newcommand{\waterMark} { 2ª Evaluación } 
%%%%%%%%%%%%%%%%%%%%%%%%%%%

% Configuración del documento.
\newcommand{\schoolSubject} { Matemáticas 3º ESO - Recuperación}
\newcommand{\school} { IES La Serna }
\newcommand{\academicPeriod} { Curso 2020/2021 }


\newcommand{\autor} { Andrés Giménez Muñoz }
\newcommand{\emailAuthor} { agimenezmunoz@ieslaserna.com }
\newcommand{\autorSing}{ Profesores: Andrés } 
\renewcommand{\schoolSubject} { Examen Matemáticas 2º ESO  }
\renewcommand{\school} { IES José de Churriguera  }
\renewcommand{\academicPeriod} { Curso 2022/2023 }

\renewcommand{\autor} { Andrés Giménez Muñoz }
\renewcommand{\emailAuthor} { andresprofemates@outlook.es }
\renewcommand{\autorSing}{ Profesor: Andrés } 

%%%%%%%%%%%%%%%%%%%%%%%%%%%
% Exam configuration
%\pointsdroppedatright   %% No mostrar la puntuación
\pointsinrightmargin % Para poner las puntuaciones a la derecha. Se puede cambiar. Si se comenta, sale a la izquierda.
\extrawidth{-1.5cm} %Un poquito más de margen por si ponemos textos largos.
\marginpointname{ \emph{\points}}

%% Si se comenta no aparecerán los espacios de la solución.
%\nocancelspace

%% Esto es de la clase exam. Si dejamos sin comentar \printanswers, se mostraran las soluciones. 
%% Si la comentamos y dejamos sin comentar \noprintanswers, pues no se muestran las soluciones.
%\printanswers
%\noprintanswers

%%%%%%%%%%%%%%%%%%%%%%%%%%%

% \usepackage{tikz}
% \usetikzlibrary{arrows}

% \useslantedhalf

% \usepackage{titling}
% \usepackage{float}

\begin{document}

\StudentData
% \GradeTableHeader

\setcounter{question}{0}

\center{2ª Evaluación}
\begin{center}
    \partialgradetable{Evaluacion2}[h][questions]
\end{center}

\justifying

%%%%%%%%%%%%%%%%%%%%%%%%%%%%%%%%%%%%%%%%%%  2ª Evaluación  %%%%%%%%%%%%%%%%%%%%%%%%%%%%%%%%%%%%%%%%%%%%%%%%%%%%%%
\begin{questions}

\begingradingrange{Evaluacion2}

\question[2]
Si 3 obreros colocan 100 metros cuadrados de suelo en 2 días. ¿Cuántos días tardarán 4 obreros en colocar 1.000 metros cuadrados de suelo?
\vspace{\stretch{1}}

\question[2]
Cuatro empleados de una tienda de moda tardan 8 días en coser 6 vestidos.
¿Cuánto tiempo tardarían en coser 24 vestidos si se duplica la plantilla?
\vspace{\stretch{1}}

\question[2]
¿Qué interés producen 14.000\euro{} al 3 \% anual durante 5 años?
\vspace{\stretch{1}}

\newpage
\question[3]
Resuelve las siguientes ecuaciones.
\begin{parts}
    \part
    $3x-1 = 2$
    \vspace{\stretch{2}}
    \part
    $2x = 3-3x$
    \vspace{\stretch{2}}
    \part
    $x+2(x+1) = 2(3x-1)$
    \vspace{\stretch{2}}

    \newpage

    \part
    $x^2-7x+6=0$
    \vspace{\stretch{3}}

    \part
    $(x-2)^2 = 81$
    \vspace{\stretch{3}}
\end{parts}

\question[1]
Si $P(x)=x^4 - 3 x^3 + 2x^2 - 5x -5$, evalúa.
\begin{parts}
    \part
    $P\left(-1\right)$
    \vspace{\stretch{1}}
    \part
    $P\left(0\right)$
    \vspace{\stretch{1}}
    \part
    $P\left(\frac{1}{2}\right)$
    \vspace{\stretch{1}}
\end{parts}

\endgradingrange{Evaluacion2}

\end{questions}
\end{document}