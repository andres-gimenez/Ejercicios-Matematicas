%%%%%%%%%%%%%%%%%%%%%%%%%%%
\newcommand{\documentName} { Examen Recuperación - 1ª Evaluación  }
\newcommand{\documentContent} { Convocatoria ordinaria } 
\newcommand{\waterMark} { 1ª Evaluación } 
%%%%%%%%%%%%%%%%%%%%%%%%%%%

% Configuración del documento.
\newcommand{\schoolSubject} { Matemáticas 3º ESO - Recuperación}
\newcommand{\school} { IES La Serna }
\newcommand{\academicPeriod} { Curso 2020/2021 }


\newcommand{\autor} { Andrés Giménez Muñoz }
\newcommand{\emailAuthor} { agimenezmunoz@ieslaserna.com }
\newcommand{\autorSing}{ Profesores: Andrés } 
\renewcommand{\schoolSubject} { Examen Matemáticas 2º ESO  }
\renewcommand{\school} { IES José de Churriguera  }
\renewcommand{\academicPeriod} { Curso 2022/2023 }

\renewcommand{\autor} { Andrés Giménez Muñoz }
\renewcommand{\emailAuthor} { andresprofemates@outlook.es }
\renewcommand{\autorSing}{ Profesor: Andrés } 

%%%%%%%%%%%%%%%%%%%%%%%%%%%
% Exam configuration
%\pointsdroppedatright   %% No mostrar la puntuación
\pointsinrightmargin % Para poner las puntuaciones a la derecha. Se puede cambiar. Si se comenta, sale a la izquierda.
\extrawidth{-1.5cm} %Un poquito más de margen por si ponemos textos largos.
\marginpointname{ \emph{\points}}

%% Si se comenta no aparecerán los espacios de la solución.
%\nocancelspace

%% Esto es de la clase exam. Si dejamos sin comentar \printanswers, se mostraran las soluciones. 
%% Si la comentamos y dejamos sin comentar \noprintanswers, pues no se muestran las soluciones.
%\printanswers
%\noprintanswers

%%%%%%%%%%%%%%%%%%%%%%%%%%%

% \usepackage{tikz}
% \usetikzlibrary{arrows}

% \useslantedhalf

% \usepackage{titling}
% \usepackage{float}

\begin{document}

\StudentData
% \GradeTableHeader

\setcounter{question}{0}

\center{1ª Evaluación}
\begin{center}
    \partialgradetable{Evaluacion1}[h][questions]
\end{center}

\justifying

%%%%%%%%%%%%%%%%%%%%%%%%%%%%%%%%%%%%%%%%%%  1ª Evaluación  %%%%%%%%%%%%%%%%%%%%%%%%%%%%%%%%%%%%%%%%%%%%%%%%%%%%%%
\begin{questions}
%\setcounter{question}{0}

\begingradingrange{Evaluacion1}
\setcounter{question}{0}

\question[2]
Calcula el valor de las siguientes potencias de fracciones y expresa el resultado en forma de fracción irreducible.
\begin{parts}
    \part
    $\left( \frac{1}{9} \right)^{-1}$
    \vspace{\stretch{1}}

    \part
    $\left( - \frac{3}{4} \right)^{-4}$
    \vspace{\stretch{1}}

    \part
    $\left( - \frac{5}{8} \right)^{-3}$
    \vspace{\stretch{1}}
    
    \part
    $\left( \frac{3}{5} \right)^{4}$
    \vspace{\stretch{1}}
\end{parts}

\newpage

\question[2]
Realiza las siguientes operaciones y expresa el resultado en forma de fracción irreducible.
\begin{parts}
    \part
    $\frac{3}{5} \cdot \frac{5}{4} \cdot \frac{4}{12} \cdot \frac{12}{3} \cdot \frac{3}{2} \cdot \frac{2}{7} : \frac{3}{7} $
    \vspace{\stretch{1}}
    % \part
    % 	$\frac{1}{25} + \frac{3}{5} - \frac{6}{9} : \frac{14}{12}$
    % 	\vspace{\stretch{1}}
    % \part
    % 	$\left( \frac{3}{5} \cdot \frac{1}{2} - \frac{5}{6} \right) : \frac{1}{3}$
    % 	\vspace{\stretch{1}}

    \part
    $\frac{3}{5} \cdot \left( \frac{1}{2} + \frac{5}{15} \right) : \frac{1}{5}$
    \vspace{\stretch{1}}

    \part
    $2 + \frac{3}{3} - \left( \frac{5}{6} + \frac{7}{12} \right) - \frac{1}{3}$
    \vspace{\stretch{1}}
    % \part 
    % 	$\left( \frac{2}{3} \right)^3 - \left( \frac{2}{3} \right)^{-1}$
    % \part 
    % 	$\frac{40}{18} \cdot \frac{9}{30} - \left( 2 - \frac{3}{2} \right)^{2}$
    % \part 
    % 	$\frac{3}{10} \cdot \left(\frac{1}{7} - \frac{5}{12} \right)^{-1} : \frac{1}{4} \cdot \frac{2}{5}$
\end{parts}

\question[2]
Completa la siguiente tabla. \\

\begin{table}[h!]
    \centering
    %\caption{}
    %\label{tab:my-table}
    \begin{tabular}{|c|c|c|}
        \hline
        \rowcolor[gray]{.9}
        \textbf{Escritura decimal} & \textbf{Escritura P10} & \textbf{Notación científica} \\ \hline
        $25.000.000$               &                        &                              \\ \hline
        $0,0000043$                &                        &                              \\ \hline
                                   & $29 \cdot 10^{-3}$     &                              \\ \hline
                                   & $438 \cdot 10^{5}$     &                              \\ \hline
                                   &                        & $3,48 \cdot 10^{-4}$         \\ \hline
                                   &                        & $1,3 \cdot 10^{5}$           \\ \hline
    \end{tabular}
\end{table}

% \question[2]
% Reduce a índice común y ordena de mayor a menor.
% \begin{parts}
%     \part
%     $\sqrt[4]{4}, \sqrt[8]{8}, \sqrt[6]{6}$
%     \vspace{\stretch{1}}
%     \part
%     $\sqrt[5]{5}, \sqrt{2}, \sqrt[10]{20}$
%     \vspace{\stretch{1}}
% \end{parts}
% \question[3\half]
% Pregunta 2

\newpage
\question[2]
Tres personas se han repartido una cantidad de dinero directamente proporcional a los números 6, 3 y 2.
Si la que menos recibe ha recibido 900\euro{}. ¿qué cantidad total se repartió?
\vspace{\stretch{1}}

% \question[2]
% Juan, Olivia y Rafa tienen un bar y se reparten las ganancias del mes de forma inversamente proporcional al número de días que han descansado este mes.
% Las ganancias del mes fueron de 4.230\euro{} y los socios descansaron 8, 6 y 10 días respectivamente. ¿Cuánto dinero le corresponde a cada uno?
% \vspace{\stretch{1}}

\question[2]
A tres amigos les han tocado el segundo premio de la lotería de navidad valorado en $125.000\euro{}$
por un décimo que costó $20\euro{}$ al que aportaron 8\euro{}, 7\euro{} y 5\euro{} respectivamente,
¿cómo han de repartir el premio?
\vspace{\stretch{1}}

\endgradingrange{Evaluacion1}
\end{questions}
\end{document}