%%%%%%%%%%%%%%%%%%%%%%%%%%%
\newcommand{\documentName} { Examen 3ª evaluación }
\newcommand{\documentContent} { Probabilidad } 
\newcommand{\waterMark} { Modelo A } 
%%%%%%%%%%%%%%%%%%%%%%%%%%%

% Configuración del documento.
\newcommand{\schoolSubject} { Matemáticas 3º ESO - Recuperación}
\newcommand{\school} { IES La Serna }
\newcommand{\academicPeriod} { Curso 2020/2021 }


\newcommand{\autor} { Andrés Giménez Muñoz }
\newcommand{\emailAuthor} { agimenezmunoz@ieslaserna.com }
\newcommand{\autorSing}{ Profesores: Andrés } 
\renewcommand{\schoolSubject} { Examen Matemáticas 2º ESO  }
\renewcommand{\school} { IES José de Churriguera  }
\renewcommand{\academicPeriod} { Curso 2022/2023 }

\renewcommand{\autor} { Andrés Giménez Muñoz }
\renewcommand{\emailAuthor} { andresprofemates@outlook.es }
\renewcommand{\autorSing}{ Profesor: Andrés } 

%%%%%%%%%%%%%%%%%%%%%%%%%%%
% Exam configuration
%\pointsdroppedatright   %% No mostrar la puntuación
\pointsinrightmargin % Para poner las puntuaciones a la derecha. Se puede cambiar. Si se comenta, sale a la izquierda.
\extrawidth{-1.5cm} %Un poquito más de margen por si ponemos textos largos.
\marginpointname{ \emph{\points}}

%% Si se comenta no aparecerán los espacios de la solución.
%\nocancelspace

%% Esto es de la clase exam. Si dejamos sin comentar \printanswers, se mostraran las soluciones. 
%% Si la comentamos y dejamos sin comentar \noprintanswers, pues no se muestran las soluciones.
%\printanswers
%\noprintanswers

%%%%%%%%%%%%%%%%%%%%%%%%%%%

% \usepackage{tikz}
% \usetikzlibrary{arrows}

\begin{document}

	\StudentData
	\GradeTableHeader

    \justifying

	\begin{questions}
		\setcounter{question}{0}

        \question[3]
        Indica la expresión algebraica que corresponda a las siguientes frases
        \begin{parts}
            \part
            ¿Cual es la problabilidad de un suceso imposible? \\ \\
            \begin{oneparcheckboxes}
                \CorrectChoice $0$
                \choice $1$
                \choice $100\%$
            \end{oneparcheckboxes}
            \\
            \part
            ¿Cual es la problabilidad de un suceso seguro? \\ \\
            \begin{oneparcheckboxes}
                \choice $0$
                \CorrectChoice $1$
                \choice $50\%$
            \end{oneparcheckboxes}
            \part
            ¿Cual de estos valores puede representar una probabilidad? \\ \\
            \begin{oneparcheckboxes}
                \choice $-1$
                \CorrectChoice $0,3$
                \choice $100$
            \end{oneparcheckboxes}
            \\
            \part
            Medir cuanto mide una mesa y obtener su resultado es un suceso: \\ \\
            \begin{oneparcheckboxes}
                \choice Aleatorio
                \CorrectChoice Determinista
                \choice Pasado
            \end{oneparcheckboxes}
            \\
            \part
            Tirar un dado y mirar si el resultado es par es un suceso: \\ \\
            \begin{oneparcheckboxes}
                \choice Determinista
                \CorrectChoice Aleatorio
                \choice Aislado
            \end{oneparcheckboxes}
            \\
        \end{parts}

        \question[3]
		En una fiesta de Navidad se sortean 3 regalos por clase, si en tu clase hay 29 alumnos. 
		¿Qué probabilidad tienes de que te toque un regalo?
		\vspace{\stretch{1}}

        \newpage
        \question[4]
        En una bolsa hay 4 bolas rojas, 3 verdes y 2 azules. si se saca al azar una bola de la bolsa, calcula las probabilidades de que:
        \begin{parts}
            \part
            La bola sea verde.
            \vspace{\stretch{1}}
            \part
            La bola no sea roja.
            \vspace{\stretch{1}}
            \part
            La bola sea verde o azul.
            \vspace{\stretch{1}}
            \part
            La bola no sea ni roja ni azul.
            \vspace{\stretch{1}}
        \end{parts}

	\end{questions}
\end{document}