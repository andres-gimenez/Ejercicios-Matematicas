% Configuración del documento.
\renewcommand{\schoolSubject} { Examen Matemáticas 2º ESO  }
\renewcommand{\school} { IES José de Churriguera  }
\renewcommand{\academicPeriod} { Curso 2022/2023 }

\renewcommand{\autor} { Andrés Giménez Muñoz }
\renewcommand{\emailAuthor} { andresprofemates@outlook.es }
\renewcommand{\autorSing}{ Profesor: Andrés } 
\newcommand{\schoolSubject} { Matemáticas 3º ESO - Recuperación}
\newcommand{\school} { IES La Serna }
\newcommand{\academicPeriod} { Curso 2020/2021 }


\newcommand{\autor} { Andrés Giménez Muñoz }
\newcommand{\emailAuthor} { agimenezmunoz@ieslaserna.com }
\newcommand{\autorSing}{ Profesores: Andrés } 

%%%%%%%%%%%%%%%%%%%%%%%%%%%
\newcommand{\numeroHoja} { 2º Evaluación }
\newcommand{\nombreHoja} { Recuperación }
\renewcommand{\waterMark} { Modelo B } 
%%%%%%%%%%%%%%%%%%%%%%%%%%%


%%%%%%%%%%%%%%%%%%%%%%%%%%%
% Exam configuration
%\pointsdroppedatright   %% No mostrar la puntuación
\pointsinrightmargin % Para poner las puntuaciones a la derecha. Se puede cambiar. Si se comenta, sale a la izquierda.

%% Si se comenta no aparecerán los espacios de la solución.
%\nocancelspace

%% Esto es de la clase exam. Si dejamos sin comentar \printanswers, se mostraran las soluciones. 
%% Si la comentamos y dejamos sin comentar \noprintanswers, pues no se muestran las soluciones.
%\printanswers
%\noprintanswers

%%%%%%%%%%%%%%%%%%%%%%%%%%%

\begin{document}

	\StudentData
	\GradeTableHeader

	\begin{questions}

		\question[2]
		Si 3 obreros colocan 100 metros cuadrados de suelo en 2 días. ¿Cuántos días tardarán 4 obreros en colocar 1.000 metros cuadrados de suelo?
		\vspace{\stretch{1}}

		\question[2]
		Cuatro empleados de una tienda de moda tardan 8 días en coser 6 vestidos.
		¿Cuánto tiempo tardarían en coser 24 vestidos si se duplica la plantilla?
		\vspace{\stretch{1}}

		\newpage

		\question[1]
		Calcula el 6\% de 100\euro{}, 1.200\euro{} y 354,07\euro{}.
		\vspace{\stretch{1}}

		\question[2]
		¿Qué interés producen 14.000\euro{} al 3 \% anual durante 5 años?
		\vspace{\stretch{1}}

		\newpage
        \StudentData
        \question[3]
            Resuelve las siguientes ecuaciones.
            \begin{parts}
                \part
                    $3x-1 = 2$
                    \vspace{\stretch{2}}
                \part 
                    $2x = 3-3x$
                    \vspace{\stretch{2}}
                \part
                    $x+2(x+1) = 2(3x-1)$
                    \vspace{\stretch{2}}

			\newpage

				\part
                    $x^2-7x+6=0$
                    \vspace{\stretch{3}}

                \part
                    $(x-2)^2 = 81$
                    \vspace{\stretch{3}}
            \end{parts}

	\end{questions}
\end{document}