%%%%%%%%%%%%%%%%%%%%%%%%%%%
\newcommand{\numeroHoja} { Hoja 2.2 }
\newcommand{\nombreHoja} { Proporcionalidad compuesta }
%%%%%%%%%%%%%%%%%%%%%%%%%%%

% Configuración del documento.
\renewcommand{\schoolSubject} { Examen Matemáticas 2º ESO  }
\renewcommand{\school} { IES José de Churriguera  }
\renewcommand{\academicPeriod} { Curso 2022/2023 }

\renewcommand{\autor} { Andrés Giménez Muñoz }
\renewcommand{\emailAuthor} { andresprofemates@outlook.es }
\renewcommand{\autorSing}{ Profesor: Andrés } 
\newcommand{\schoolSubject} { Matemáticas 3º ESO - Recuperación}
\newcommand{\school} { IES La Serna }
\newcommand{\academicPeriod} { Curso 2020/2021 }


\newcommand{\autor} { Andrés Giménez Muñoz }
\newcommand{\emailAuthor} { agimenezmunoz@ieslaserna.com }
\newcommand{\autorSing}{ Profesores: Andrés } 

%%%%%%%%%%%%%%%%%%%%%%%%%%%
% Exam configuration
\pointsdroppedatright   %% No mostrar la puntuación

%% Si se comenta no aparecerán los espacios de la solución.
%\nocancelspace

%% Esto es de la clase exam. Si dejamos sin comentar \printanswers, se mostraran las soluciones. 
%% Si la comentamos y dejamos sin comentar \noprintanswers, pues no se muestran las soluciones.
%\printanswers
%\noprintanswers

%%%%%%%%%%%%%%%%%%%%%%%%%%%

\begin{document}

    %%\StudentData

    \begin{questions}

        \question
		Una receta para cocinar trufas de chocolate, nos dice que para 30 unidades se requieren 200 gramos de chocolate negro de buena calidad, 120 gramos de nata líquida para montar,
		media vaina de vainilla y 30 gramos de mantequilla. ¿Qué cantidad de ingredientes necesitaremos para cocinar 1400 unidades?

        \question
        Para transportar 640 kilogramos de mercancía a 87 kilómetro de distancia,
        se han gastado 2.700\euro. 
        ¿Cuánto costará transportar 1.218 kilogramos de la misma mercancía a 320 kilómetros?

        \question
        Un hombre que camina durante 7 días a razón de 8 horas diarias, ha recorrido 225 kilómetros. 
        ¿Cuántos habrá recorrido otro que camina 12 días durante 7 horas diarias?      

        \question
        Si una mujer tiene un niño en 9 meses. ¿Cuánto tardaran 9 mujeres en tener un niño?
        \\
        \small Nota: Obviamente este ejercicio no hay que resolverlo. Este es un ejemplo que se pone en las escuelas de negocio para hacer reflexionar que hay problemas que no se resuelven con una regla de tres.

    \end{questions}
\end{document}