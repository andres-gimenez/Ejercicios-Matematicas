%%%%%%%%%%%%%%%%%%%%%%%%%%%
\newcommand{\documentName} { Examen 3ª evaluación }
\newcommand{\documentContent} { Recuperación } 
\newcommand{\waterMark} { Modelo A } 
%%%%%%%%%%%%%%%%%%%%%%%%%%%

% Configuración del documento.
\newcommand{\schoolSubject} { Matemáticas 3º ESO - Recuperación}
\newcommand{\school} { IES La Serna }
\newcommand{\academicPeriod} { Curso 2020/2021 }


\newcommand{\autor} { Andrés Giménez Muñoz }
\newcommand{\emailAuthor} { agimenezmunoz@ieslaserna.com }
\newcommand{\autorSing}{ Profesores: Andrés } 
\renewcommand{\schoolSubject} { Examen Matemáticas 2º ESO  }
\renewcommand{\school} { IES José de Churriguera  }
\renewcommand{\academicPeriod} { Curso 2022/2023 }

\renewcommand{\autor} { Andrés Giménez Muñoz }
\renewcommand{\emailAuthor} { andresprofemates@outlook.es }
\renewcommand{\autorSing}{ Profesor: Andrés } 

%%%%%%%%%%%%%%%%%%%%%%%%%%%
% Exam configuration
%\pointsdroppedatright   %% No mostrar la puntuación
\pointsinrightmargin % Para poner las puntuaciones a la derecha. Se puede cambiar. Si se comenta, sale a la izquierda.
\extrawidth{-1.5cm} %Un poquito más de margen por si ponemos textos largos.
\marginpointname{ \emph{\points}}

%% Si se comenta no aparecerán los espacios de la solución.
%\nocancelspace

%% Esto es de la clase exam. Si dejamos sin comentar \printanswers, se mostraran las soluciones. 
%% Si la comentamos y dejamos sin comentar \noprintanswers, pues no se muestran las soluciones.
%\printanswers
%\noprintanswers

%%%%%%%%%%%%%%%%%%%%%%%%%%%

\begin{document}

\StudentData
\GradeTableHeader

\justifying

\begin{questions}
    \setcounter{question}{0}

    \question[2]
    Resuelve el siguiente sistema de ecuaciones por el método gráfico.
    \begin{flushleft}
        $\begin{cases}
                \nonumber
                x + y  = 2 \\
                \nonumber
                3x - y = 2
            \end{cases}$
    \end{flushleft}

    \begin{figure}[h]
        \begin{tikzpicture}[scale=1]
            \tkzInit[xmax=6,ymax=6,xmin=-6,ymin=-6]
            \tkzGrid
            \tkzAxeXY
        \end{tikzpicture}
    \end{figure}

    \newpage
    \question[3]
    Resuelve los siguientes sistemas de ecuaciones
    \begin{parts}
        \part
        \begin{flushleft}
            $\begin{cases}
                    \nonumber
                    x - 5y  = -8 \\
                    \nonumber
                    x + 3y = 0
                \end{cases}$
        \end{flushleft}
        \vspace{\stretch{1}}

        \part
        \begin{flushleft}
            $\begin{cases}
                    \nonumber
                    3x - 4y = -5 \\
                    \nonumber
                    5x - 2y = 9
                \end{cases}$
        \end{flushleft}
        \vspace{\stretch{1}}
    \end{parts}

    \newpage
    \question[5]
    El número de libros solicitados en una biblioteca ha sido:
    \begin{table}[h!]
        \centering
        \begin{tabular}{|c|c|c|c|c|c|c|c|}
            \hline
            \cellcolor[gray]{0.8}Libros ($x_i$) & 1 & 2 & 3 & 4 & 5 & 6 \\
            \hline
            \cellcolor[gray]{0.8}Usuarios ($f_i$) & 8 & 12 & 9 & 6 & 3 & 2  \\
            \hline
        \end{tabular}
    \end{table}

    \begin{parts}
        \part
        Calcula la tabla de frecuencias relativas y acumuladas.
        \vspace{\stretch{2}}

        \part
        %Media = 2,75
        Calcula la media, la mediana y moda.
        \begin{equation*}
            \bar{x}=\frac{\sum{x_i f_i}}{N} = \sum{x_i h_i}
        \end{equation*}
        \begin{equation*}
            N = \sum{f_i}
        \end{equation*}
        \vspace{\stretch{1}}

        \newpage
        \part
        %Varianza = 1,9375, Desviación típica = 1,39
        Calcula la varianza y la desviación típica.
        \begin{equation*}
            s^2=\frac{\sum{({x_i - \bar{x}})^2} f_i}{N} = \frac{\sum{{x_i}^2 f_i}}{N} - \bar{x}^2 = \sum{{x_i}^2 h_i} - \bar{x}^2
        \end{equation*}

        \begin{equation*}
            s=\sqrt{\frac{\sum{({x_i - \bar{x}})^2} f_i}{N}} = \sqrt{\frac{\sum{{x_i}^2 f_i}}{N} - \bar{x}^2} = \sqrt{\sum{{x_i}^2 h_i} - \bar{x}^2}
        \end{equation*}
        \vspace{\stretch{1}}
       
        \part
        Calcula el coeficiente de variación.
        % CV = 0,506
        \begin{equation*}
            CV = \frac{s}{\bar{x}}
        \end{equation*}
        \vspace{\stretch{1}}

        \part
        El año anterior se solicitaron una media de $2,4$ libros con una desviación típica de $1,5$. 
        Indica, justificando la respuesta, que año se han solicitados más libros, en cual ha se ha producido más diferencia de solicitudes entre solicitantes y que media es más representativa sobre el número de libros que se solicitan en la biblioteca.
        \vspace{\stretch{1}}
    \end{parts}
\end{questions}

\end{document}