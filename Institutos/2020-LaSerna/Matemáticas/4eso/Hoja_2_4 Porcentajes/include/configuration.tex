\documentclass[addpoints,spanish, 12pt,a4paper,cancelspace]{exam}
    
%\usepackage[paperheight=5.8in,paperwidth=8.27in,bindingoffset=0in,left=0.8in,right=1in, top=0.7in,bottom=1in,headsep=.5\baselineskip]{geometry}
%\usepackage[bindingoffset=0in,left=0.8in,right=0.8in, top=0.7in,bottom=1in,headsep=.5\baselineskip]{geometry}
%\usepackage[left=2cm, right=2cm, bottom=0.85cm, top=2cm]{geometry}
\usepackage[T1]{fontenc}
\usepackage[utf8]{inputenc}
\usepackage{textcomp}
\usepackage[gen]{eurosym}
\usepackage{xcolor}
\usepackage{multicol}

\usepackage[normalem]{ulem}
\renewcommand\ULthickness{0.5pt}   %%---> For changing thickness of underline
\setlength\ULdepth{1.5ex}          %\maxdimen ---> For changing depth of underline
\renewcommand{\baselinestretch}{1}
\usepackage{fancybox} % 
\usepackage{graphicx} % Paquete necesario para incluir imágenes, cambiarles el tamaño, etc.
\usepackage{enumitem} % Para poder configurar las listas
\everymath{\displaystyle} % Esto es para que las expresiones se vean... grandes, que resulta diferente de si las queremos entre líneas.

\usepackage{amssymb}
\usepackage{amsmath}
\usepackage{lmodern}
\usepackage{mathtools, nccmath}

\definecolor{azul}{rgb}{0.17,0.40, 0.68}

% Idioma y codificación de texto
\PassOptionsToPackage{T1}{fontenc} 
\usepackage{fontenc} 
\usepackage[utf8]{inputenc}
% Cargar babel y configurar para español
\usepackage[spanish,es-lcroman, es-tabla, es-noshorthands]{babel}

\author{Andrés Giménez Muñoz}
\newcommand{\fecha} { Septiembre 2020 }
\newcommand{\titulo} { { Examen } }
\newcommand{\autor} { Andrés Giménez Muñoz }
\newcommand{\autorSing}{ Andrés }
\newcommand{\emailAuthor} { agimenezmunoz@ieslaserna.com }

\usepackage[pdfborder={0 0 0}, colorlinks=true, linkcolor=black, urlcolor=black, citecolor=black, pdfpagemode=UseNone]{hyperref}
\hypersetup{
pdfauthor = \autor~(\emailAuthor),
pdftitle = {\titulo},
pdfsubject= {},
pdfcreator = {},
pdfproducer = {},
pdfkeywords = {}
}

\usepackage{background}
\backgroundsetup{
  position=current page.east,
  angle=90,
  color=gray,
  %nodeanchor=east,
  hshift=-260mm,
  vshift=5mm,
  opacity=1,
  scale=0.5,
  contents={Profesor: \autorSing}
}

%%%%%%%%%%%%%%%%%%%%%%%%%%%%%%%%%%%%%%%%%%%%%%%%%%%%%%%%%%%%%%%%%%%%%%%%%%%%%%%%%%

    %% Esto es de la clase exam. Si dejamos sin comentar \printanswers, se mostraran las soluciones. 
    %% Si la comentamos y dejamos sin comentar \noprintanswers, pues no se muestran las soluciones.
    %\printanswers
    %\noprintanswers

    %% Si se comenta no aparecerán los espacios de la solución.
    %\nocancelspace

    %%%%%%%%%%%%%%%%%%%%%%%%%%%%%%%%%%%%%%%%%%%%%%%%%%%%%%%%%%%%%%%%%%%%%%%%%%%%%%%%%%
  
    %%%%%%% Datos de estudiante
    \newcommand{\StudentData}{
        \raggedright
        \begin{tabular}{ccc}
            \begin{minipage}{0.15\linewidth}
                \includegraphics[height=2.75cm]{./include/logo}
            \end{minipage} &
            \begin{minipage} {0.75\linewidth}
                {\LARGE \textbf{Bla bla bla bal bla}} \\
                Examen de Matemáticas \\
                Apellidos: \underline{\hspace{5cm}} \hspace{0.1cm} Nombre: \enspace{\hrulefill} 
                \flushright Grupo: \underline{4ºA\hspace{0.5cm}} \hspace{0.1cm} Fecha: \underline{\hspace{2cm}} 
            \end{minipage} &
        \end{tabular}
    }

    \newcommand{\StudentDataDos}{
        \begin{minipage}[c]{6in}
                Examen de Matemáticas \\
                %Apellidos: \enspace\hrulefill \hspace{0.5cm} Nombre: \enspace\hrulefill \\
                Apellidos y Nombre: \enspace\hrulefill \\
                Fecha: \enspace\hrulefill \hspace{1cm} Grupo: \enspace\hrulefill \\
            \end{minipage}
    }

    \newcommand{\WithoutCalculator}{
        \begin{center}
            \fbox{\fbox{\parbox{5.5in}{\centering
            Estos ejercicios están destinados a ejercitar la destreza en la aritmética. No uséis la calculadora para realizarlos. En ninguno de ellos debéis obtener decimales.}}}
        \end{center}
    }

    %%%%%%%%%%%%%%%%%%%%%%%%%%%%%%%%%%%%%%%%%%%%%%%%%%%%%%%%%%%%%%%%%%%%%%%%%%%%%%%%%%
    %%%% Cosas a configurar de la clase EXAM %%%%

    \pagestyle{headandfoot}
    \runningheadrule
    \extraheadheight{1cm}

    \firstpagefooter{}{}{\small Página \thepage\ de \numpages}
    \runningfooter{}{}{\small Página \thepage\ de \numpages}

    \pointpoints{punto}{puntos}
    \bonuspointpoints{punto extra}{puntos extra}
    \hqword{Pregunta}
    \hpword{Puntos}
    \hsword{Calificación}
    \renewcommand{\solutiontitle}{\noindent\textbf{Solución:}\par\noindent}
    \pointformat{(\emph{\thepoints})}
    \bonuspointformat{(\emph{\thepoints})}
    %\pointsinrightmargin % Para poner las puntuaciones a la derecha. Se puede cambiar. Si se comenta, sale a la izquierda.   

    %\extrawidth{-1.8cm} %Un poquito más de margen por si ponemos textos largos.
    \marginpointname{ \emph{\points}}
    %\bracketedpoints

    \firstpageheader 
    {\curso  \\ \small \colegio \\ \small \anoCurso}
    {}
    {\numeroHoja \\ \nombreHoja \\}

    \runningheader
    {\curso  \\ \small \colegio \\ \small \anoCurso}
    {}
    {\numeroHoja \\ \nombreHoja \\}

    \headrule
    \footrule