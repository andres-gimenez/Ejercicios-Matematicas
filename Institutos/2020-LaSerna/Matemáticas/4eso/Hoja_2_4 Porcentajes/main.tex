%%%%%%%%%%%%%%%%%%%%%%%%%%%
\newcommand{\numeroHoja} { Hoja 2.4 }
\newcommand{\nombreHoja} { Porcentajes }
%%%%%%%%%%%%%%%%%%%%%%%%%%%

% Configuración del documento.
\renewcommand{\schoolSubject} { Examen Matemáticas 2º ESO  }
\renewcommand{\school} { IES José de Churriguera  }
\renewcommand{\academicPeriod} { Curso 2022/2023 }

\renewcommand{\autor} { Andrés Giménez Muñoz }
\renewcommand{\emailAuthor} { andresprofemates@outlook.es }
\renewcommand{\autorSing}{ Profesor: Andrés } 
\newcommand{\schoolSubject} { Matemáticas 3º ESO - Recuperación}
\newcommand{\school} { IES La Serna }
\newcommand{\academicPeriod} { Curso 2020/2021 }


\newcommand{\autor} { Andrés Giménez Muñoz }
\newcommand{\emailAuthor} { agimenezmunoz@ieslaserna.com }
\newcommand{\autorSing}{ Profesores: Andrés } 

%%%%%%%%%%%%%%%%%%%%%%%%%%%
% Exam configuration
\pointsdroppedatright   %% No mostrar la puntuación

%% Si se comenta no aparecerán los espacios de la solución.
%\nocancelspace

%% Esto es de la clase exam. Si dejamos sin comentar \printanswers, se mostraran las soluciones. 
%% Si la comentamos y dejamos sin comentar \noprintanswers, pues no se muestran las soluciones.
\printanswers
%\noprintanswers

%%%%%%%%%%%%%%%%%%%%%%%%%%%

\begin{document}

    %%\StudentData

    \begin{questions}

        \setcounter{question}{0}
	
		\question
		Calcula el IVA de los siguientes precios al tipo general del $21\%$, tipo reducido al $10\%$ y superreducido al $4\%$.
		\begin{parts}
			\part
				$12,5\euro{}$
				\begin{solution}
					IVA (21\%) $ = 12,5\euro{} \cdot 21\% = 12,5\euro{} \cdot \frac{21}{100} = 12,5\euro{} \cdot 0,21\ = \boxed{2,63 \euro{}}$ \\ \\
					IVA (10\%) $= 12,5\euro{} \cdot 10\% = 12,5\euro{} \cdot \frac{10}{100} = 12,5\euro{} \cdot 0,10\ = \boxed{1,25 \euro{}}$ \\ \\
					IVA (4\%) $= 12,5\euro{} \cdot 4\% = 12,5\euro{} \cdot \frac{4}{100} = 12,5\euro{} \cdot 0,04\ = \boxed{0,50 \euro{}}$
				\end{solution}
			\part
				$143,34\euro{}$
				\begin{solution}
					IVA (21\%) $= 143,34\euro{} \cdot 21\% = 143,34\euro{} \cdot \frac{21}{100} = 143,34\euro{} \cdot 0,21\ = \boxed{30,10 \euro{}}$ \\ \\
					IVA (10\%) $= 143,34\euro{} \cdot 10\% = 143,34\euro{} \cdot \frac{10}{100} = 143,34\euro{} \cdot 0,10\ = \boxed{14,33 \euro{}}$ \\ \\
					IVA (4\%) $= 143,34\euro{} \cdot 4\% = 143,34\euro{} \cdot \frac{4}{100} = 143,34\euro{} \cdot 0,04\ = \boxed{5,73 \euro{}}$
				\end{solution}
			\part
				$1.342,32\euro{}$
				\begin{solution}
					IVA (21\%) $= 1.342,32\euro{} \cdot 21\% = 1.342,32\euro{} \cdot \frac{21}{100} = 1.342,32\euro{} \cdot 0,21\ = \boxed{281,88 \euro{}}$ \\ \\
					IVA (10\%) $= 1.342,32\euro{} \cdot 10\% = 1.342,32\euro{} \cdot \frac{10}{100} = 1.342,32\euro{} \cdot 0,10\ = \boxed{134,23 \euro{}}$ \\ \\
					IVA (4\%) $= 1.342,32\euro{} \cdot 4\% = 1.342,32\euro{} \cdot \frac{4}{100} = 1.342,32\euro{} \cdot 0,04\ = \boxed{53.69 \euro{}}$
				\end{solution}
			\part
				$23.432,12\euro{}$
				\begin{solution}
					IVA (21\%) $= 23.432,12\euro{} \cdot 21\% = 23.432,12\euro{} \cdot \frac{21}{100} = 23.432,12\euro{} \cdot 0,21\ = \boxed{4.920,74 \euro{}}$ \\ \\
					IVA (10\%) $= 23.432,12\euro{} \cdot 10\% = 23.432,12\euro{} \cdot \frac{10}{100} = 23.432,12\euro{} \cdot 0,10\ = \boxed{2,343,21 \euro{}}$ \\ \\
					IVA (4\%) $= 23.432,12\euro{} \cdot 4\% = 23.432,12\euro{} \cdot \frac{4}{100} = 23.432,12\euro{} \cdot 0,04\ = \boxed{937,28 \euro{}}$
				\end{solution}
			\part
				$203.234,43\euro{}$
				\begin{solution}
					IVA (21\%) $= 203.234,43\euro{} \cdot 21\% = 203.234,43\euro{} \cdot \frac{21}{100} = 203.234,43\euro{} \cdot 0,21\ = \boxed{42.679,23 \euro{}}$ \\ \\
					IVA (10\%) $= 203.234,43\euro{} \cdot 10\% = 203.234,43\euro{} \cdot \frac{10}{100} = 203.234,43\euro{} \cdot 0,10\ = \boxed{20.323,44 \euro{}}$ \\ \\
					IVA (4\%) $= 203.234,43\euro{} \cdot 4\% = 203.234,43\euro{} \cdot \frac{4}{100} = 203.234,43\euro{} \cdot 0,04\ = \boxed{8.129,37 \euro{}}$
				\end{solution}
		\end{parts}

		\question
			El precio de un paquete de 100 mascarillas higiénicas es de 25,75\euro{} más IVA. 
			¿Cuánto nos ahorramos si nos bajan el IVA del tipo general del $21\%$ al superreducido del $4\%$?
			\begin{solution}
				\textbf{Opción 1.} Calculamos el precio con el IVA incluido.

				IVA (21\%) $= 25,75\euro{} + 21\% = 25,75\euro{} \cdot (1 + \frac{21}{100}) = 25,75\euro{} \cdot 1,21\ = \boxed{31,16 \euro{}}$ \\ \\
				IVA (4\%) $= 25,75\euro{} + 4\% = 25,75\euro{} \cdot (1 + \frac{4}{100}) = 25,75\euro{} \cdot 1,04\ = \boxed{26,78 \euro{}}$ \\ \\
				Diferencia $= 31,16 \euro{} - 26,78 \euro{} = \boxed{4,37 \euro{}}$ \\ \\

				\textbf{Opción 2.} Calculamos solo el IVA.

				IVA (21\%) $= 25,75\euro{} \cdot 21\% = 25,75\euro{} \cdot \frac{21}{100} = 25,75\euro{} \cdot 1,21\ = \boxed{5,41 \euro{}}$ \\ \\
				IVA (4\%) $= 25,75\euro{} \cdot 4\% = 25,75\euro{} \cdot \frac{4}{100} = 25,75\euro{} \cdot 1,04\ = \boxed{1,03 \euro{}}$ \\ \\
				Diferencia $= 5,41 \euro{} - 1,03 \euro{} = \boxed{4,38 \euro{}}$ \\ \\

				* Deveriamos obtener lo mismo, salvo problemas de redondeo.

			\end{solution}
		\question
		En una tienda de regalos, quieren poner en las etiquetas cantidades exactas con el precio del IVA incluido. 
        ¿Qué precio tienen que tener los productos para que al aplicarle el IVA general del $21\%$ el precio más IVA que han de aparecer en las etiquetas tenga los siguientes valores?
        
		\begin{parts}
            \begin{solutionorbox}
                ~\\ Sea $x =$ Precio sin IVA
                \\ Sea $y =$ Precio con IVA
				\begin{gather*}
					y = x \cdot (1 + \frac{21}{100}) = x \cdot 1,21 \\	
					x = \frac {y}{1 + \frac{21}{100}} = \frac{y}{1,21}
				\end{gather*}
			\end{solutionorbox}
			\part
				$1\euro{}$
				\begin{solution}
					Precio sin IVA $= \frac {1\euro{}}{1 + \frac{21}{100}} = \frac{1\euro{}}{1,21} = \boxed{0,83 \euro{}} $
				\end{solution}
			\part
				$10\euro{}$
				\begin{solution}
					Precio sin IVA $ = \frac {10\euro{}}{1 + \frac{21}{100}} = \frac{10\euro{}}{1,21} = \boxed{8,26 \euro{}} $
				\end{solution}
			\part
				$25\euro{}$
				\begin{solution}
					Precio sin IVA $ = \frac {25\euro{}}{1 + \frac{21}{100}} = \frac{25\euro{}}{1,21} = \boxed{20,66 \euro{}} $
				\end{solution}
			\part
				$100\euro{}$
				\begin{solution}
					Precio sin IVA $ = \frac {100\euro{}}{1 + \frac{21}{100}} = \frac{100\euro{}}{1,21} = \boxed{82,64 \euro{}} $
				\end{solution}
			\part
				$200\euro{}$
				\begin{solution}
					Precio sin IVA $ = \frac {200\euro{}}{1 + \frac{21}{100}} = \frac{200\euro{}}{1,21} = \boxed{165,29 \euro{}} $
				\end{solution}
		\end{parts}
    \end{questions}
\end{document}