%%%%%%%%%%%%%%%%%%%%%%%%%%%
\newcommand{\documentName} { Didactica }
\newcommand{\documentContent} { Probabilidad (Resuelta)} 
\newcommand{\waterMark} { } 
%%%%%%%%%%%%%%%%%%%%%%%%%%%

% Configuración del documento.
\newcommand{\schoolSubject} { Matemáticas 3º ESO - Recuperación}
\newcommand{\school} { IES La Serna }
\newcommand{\academicPeriod} { Curso 2020/2021 }


\newcommand{\autor} { Andrés Giménez Muñoz }
\newcommand{\emailAuthor} { agimenezmunoz@ieslaserna.com }
\newcommand{\autorSing}{ Profesores: Andrés } 
\renewcommand{\schoolSubject} { Examen Matemáticas 2º ESO  }
\renewcommand{\school} { IES José de Churriguera  }
\renewcommand{\academicPeriod} { Curso 2022/2023 }

\renewcommand{\autor} { Andrés Giménez Muñoz }
\renewcommand{\emailAuthor} { andresprofemates@outlook.es }
\renewcommand{\autorSing}{ Profesor: Andrés } 

%%%%%%%%%%%%%%%%%%%%%%%%%%%
% Exam configuration
%\pointsdroppedatright   %% No mostrar la puntuación
\pointsinrightmargin % Para poner las puntuaciones a la derecha. Se puede cambiar. Si se comenta, sale a la izquierda.
\extrawidth{-1.5cm} %Un poquito más de margen por si ponemos textos largos.
\marginpointname{ \emph{\points}}

%% Si se comenta no aparecerán los espacios de la solución.
%\nocancelspace

%% Esto es de la clase exam. Si dejamos sin comentar \printanswers, se mostraran las soluciones. 
%% Si la comentamos y dejamos sin comentar \noprintanswers, pues no se muestran las soluciones.
\printanswers
%\noprintanswers

%%%%%%%%%%%%%%%%%%%%%%%%%%%

% \usepackage{tikz}
% \usetikzlibrary{arrows}

\begin{document}

	\StudentData
	\GradeTableHeader

    \justifying

	\begin{questions}
		\setcounter{question}{0}

		\question[1]
        ¿Cuántos son cinco más siete?
        \tiny *Malayo.
        \normalsize \\
        \begin{oneparcheckboxes}
            \choice Uno.
            \choice Tres.
            \CorrectChoice Doce.
        \end{oneparcheckboxes}

        \question[1]
        ¿Cuantas patas tiene un gato?
        \tiny *Bosnio.
        \normalsize \\

        \begin{oneparcheckboxes}
            \choice Siete.
            \choice Tres.
            \CorrectChoice Cuatro.
        \end{oneparcheckboxes}

        \question[1]
        ¿El número tres es primo?
        \tiny *Ruso.
        \normalsize \\

        \begin{oneparcheckboxes}
            \CorrectChoice Si.
            \choice No.
            \choice No lo se.
        \end{oneparcheckboxes}

        \question[1]
        ¿Ocho menos cinco?
        \tiny *Chino.
        \normalsize \\

        \begin{oneparcheckboxes}
            \choice Cuatro.
            \CorrectChoice Tres.
            \choice Catorce.
        \end{oneparcheckboxes}

        \question[1]
        Tienda donde se vendía aceite, vinagre, legumbres secas, bacalao
        \tiny *Español.
        \normalsize \\

        \begin{oneparcheckboxes}
            \choice Hocino. \\
            \tiny *Instrumento corvo de hierro acerado con mango de madera (más pequeño que la hoz).
            \normalsize \\
            \CorrectChoice Abacería. \\
            \choice Obradiza. \\
            \tiny *Jornada de trabajo que los vecinos echaban para arreglar caminos o calles.
            \normalsize
        \end{oneparcheckboxes}

        \question[1]
        ¿Cúal es la raíz cuadrada de cuatro?
        \tiny *Coreano.
        \normalsize \\

        \begin{oneparcheckboxes}
            \choice Cinco.
            \choice Siete.
            \CorrectChoice Dos.
        \end{oneparcheckboxes}

        \question[1]
        ¿Cuántas trompas tiene un elefante?

        \begin{oneparcheckboxes}
            \CorrectChoice Una.
            \choice Cinco.
            \choice Ventiocho.
        \end{oneparcheckboxes}

        \question[1]
        ¿Cuántos picos tiene un pollo al nacer?

        \begin{oneparcheckboxes}
            \CorrectChoice Uno
            \choice Catorce
            \choice Trenta y tres
        \end{oneparcheckboxes}

        \question[1]
        ¿Cual es un planeta?

        \begin{oneparcheckboxes}
            \CorrectChoice Mercurio
            \choice El Sol
            \choice La Luna
        \end{oneparcheckboxes}

        \question[1]
        Nombre de un animal mamífero.

        \begin{oneparcheckboxes}
            \choice Delfín
            \choice Aguila
            \CorrectChoice Perro
        \end{oneparcheckboxes}
       
	\end{questions}
\end{document}