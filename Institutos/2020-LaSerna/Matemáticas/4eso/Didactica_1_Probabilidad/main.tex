%%%%%%%%%%%%%%%%%%%%%%%%%%%
\newcommand{\documentName} { Didactica }
\newcommand{\documentContent} { Probabilidad } 
\newcommand{\waterMark} { } 
%%%%%%%%%%%%%%%%%%%%%%%%%%%

% Configuración del documento.
\newcommand{\schoolSubject} { Matemáticas 3º ESO - Recuperación}
\newcommand{\school} { IES La Serna }
\newcommand{\academicPeriod} { Curso 2020/2021 }


\newcommand{\autor} { Andrés Giménez Muñoz }
\newcommand{\emailAuthor} { agimenezmunoz@ieslaserna.com }
\newcommand{\autorSing}{ Profesores: Andrés } 
\renewcommand{\schoolSubject} { Examen Matemáticas 2º ESO  }
\renewcommand{\school} { IES José de Churriguera  }
\renewcommand{\academicPeriod} { Curso 2022/2023 }

\renewcommand{\autor} { Andrés Giménez Muñoz }
\renewcommand{\emailAuthor} { andresprofemates@outlook.es }
\renewcommand{\autorSing}{ Profesor: Andrés } 

%%%%%%%%%%%%%%%%%%%%%%%%%%%
% Exam configuration
%\pointsdroppedatright   %% No mostrar la puntuación
\pointsinrightmargin % Para poner las puntuaciones a la derecha. Se puede cambiar. Si se comenta, sale a la izquierda.
\extrawidth{-1.5cm} %Un poquito más de margen por si ponemos textos largos.
\marginpointname{ \emph{\points}}

%% Si se comenta no aparecerán los espacios de la solución.
%\nocancelspace

%% Esto es de la clase exam. Si dejamos sin comentar \printanswers, se mostraran las soluciones. 
%% Si la comentamos y dejamos sin comentar \noprintanswers, pues no se muestran las soluciones.
%\printanswers
%\noprintanswers

\usepackage{ctex}

%%%%%%%%%%%%%%%%%%%%%%%%%%%

\begin{document}

	\StudentData
	\GradeTableHeader

    \justifying

	\begin{questions}
		\setcounter{question}{0}

		% \question[1]
        % അഞ്ച് പ്ലസ് ഏഴ് എത്രയാണ്? 
        
        % \begin{oneparcheckboxes}
        %     \choice ഒന്ന് 
        %     \choice മൂന്ന്
        %     \CorrectChoice പന്ത്രണ്ട്
        % \end{oneparcheckboxes}

        \question[1]
        Koliko nogu mačka ima?

        \begin{oneparcheckboxes}
            \choice Sedam.
            \choice Tri.
            \CorrectChoice Četiri.
        \end{oneparcheckboxes}

        % \question[1]
        % Номер три - кузен?

        % \begin{oneparcheckboxes}
        %     \CorrectChoice да.
        %     \choice Нет.
        %     \choice Я не знаю.
        % \end{oneparcheckboxes}

        % \question[1]
        % 八减五?

        % \begin{oneparcheckboxes}
        %     \choice 四.
        %     \CorrectChoice 三.
        %     \choice 十四.
        % \end{oneparcheckboxes}

        \question[1]
        Tienda donde se vendía aceite, vinagre, legumbres secas, bacalao

        \begin{oneparcheckboxes}
            \choice Hocino. \\
            \CorrectChoice Abacería. \\
            \choice Obradiza. \\
        \end{oneparcheckboxes}

        % \question[1]
        % 네 개의 제곱근은 무엇입니까?

        % \begin{oneparcheckboxes}
        %     \choice 다섯 
        %     \choice 일곱
        %     \CorrectChoice 두
        % \end{oneparcheckboxes}

        % \question[1]
        % हाथी के पास कितनी सूंड होती है? 

        % \begin{oneparcheckboxes}
        %     \CorrectChoice Una.
        %     \choice Cinco.
        %     \choice Ventiocho.
        % \end{oneparcheckboxes}

        \question[1]
        ¿Cuántos picos tiene un pollo al nacer?

        \begin{oneparcheckboxes}
            \CorrectChoice Uno
            \choice Catorce
            \choice Trenta y tres
        \end{oneparcheckboxes}

        \question[1]
        ¿Cual es un planeta?

        \begin{oneparcheckboxes}
            \CorrectChoice Mercurio
            \choice El Sol
            \choice La Luna
        \end{oneparcheckboxes}

        \question[1]
        Nombre de un animal mamífero.

        \begin{oneparcheckboxes}
            \choice Delfín
            \choice Aguila
            \CorrectChoice Perro
        \end{oneparcheckboxes}
       
	\end{questions}
\end{document}