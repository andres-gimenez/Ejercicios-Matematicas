%%%%%%%%%%%%%%%%%%%%%%%%%%%
\newcommand{\numeroHoja} { Hoja 3.05 }
\newcommand{\nombreHoja} { Estadística }
%%%%%%%%%%%%%%%%%%%%%%%%%%%

% Configuración del documento.
\renewcommand{\schoolSubject} { Examen Matemáticas 2º ESO  }
\renewcommand{\school} { IES José de Churriguera  }
\renewcommand{\academicPeriod} { Curso 2022/2023 }

\renewcommand{\autor} { Andrés Giménez Muñoz }
\renewcommand{\emailAuthor} { andresprofemates@outlook.es }
\renewcommand{\autorSing}{ Profesor: Andrés } 
\newcommand{\schoolSubject} { Matemáticas 3º ESO - Recuperación}
\newcommand{\school} { IES La Serna }
\newcommand{\academicPeriod} { Curso 2020/2021 }


\newcommand{\autor} { Andrés Giménez Muñoz }
\newcommand{\emailAuthor} { agimenezmunoz@ieslaserna.com }
\newcommand{\autorSing}{ Profesores: Andrés } 

%%%%%%%%%%%%%%%%%%%%%%%%%%%
% Exam configuration
\pointsdroppedatright   %% No mostrar la puntuación

%% Si se comenta no aparecerán los espacios de la solución.
%\nocancelspace

%% Esto es de la clase exam. Si dejamos sin comentar \printanswers, se mostraran las soluciones. 
%% Si la comentamos y dejamos sin comentar \noprintanswers, pues no se muestran las soluciones.
\printanswers
%\noprintanswers

%%%%%%%%%%%%%%%%%%%%%%%%%%%

\usepackage{tkz-euclide}

\begin{document}
    \begin{questions}
        \question
        El propietario de una instalación mista solar-eólica está realizando un estudio del volumen de energía que es capaz de producir la instalación.
        Para ello, mide dicha energía a lo largo de un total de N=16 días que considera suficientemente representativos. 
        La energía (en kilovatios hora, kWh) producida en dichos días por instalaciones se encuentra recogida en la siguiente tabla:

    
        \begin{table}[ht]
            \centering
            \begin{tabular}{|l|l|l|l|l|l|l|l|l|}
            \hline
            Central de generación 1 \\ \hline
            
            Generación solar            & 13,1   & 10,5  & 4,1   & 14,8 & 19,5 & 11,9 & 18  & 8,6  \\ \hline
            Generación eólica           & 8,5    & 14,3  & 24,7  & 4    & 2,3  & 6,4  & 3,6 & 9,2  \\ \hline
            \hline
            Central de generación 2 \\ \hline
           
            Generación solar            & 5,7    & 15,9  & 11,2  & 6,8  & 14,2 & 8,2  & 2,6 & 9,7  \\ \hline
            Generación eólica           & 13,5   & 1,4   & 7,6   & 12,8 & 10,3 & 16,5 & 21,4 & 10,9  \\ \hline
            \end{tabular}
        \end{table}

        \begin{parts}
            
            \part
            Calcula la media y la mediana de todas las muestras(cada centra y cada tecnologia). \\
            \begin{equation*}
                \bar{x}=\frac{\sum{x_i f_i}}{N} = \sum{x_i h_i}
            \end{equation*}
            \part
            Calcula la varianza y desviación típica de ambas muestras. \\
            \begin{equation*}
                s^2=\frac{\sum{({x_i - \bar{x}})^2} f_i}{N} = \frac{\sum{{x_i}^2 f_i}}{N} - \bar{x}^2 = \sum{{x_i}^2 h_i} - \bar{x}^2
            \end{equation*}

            \begin{equation*}
                s=\sqrt{\frac{\sum{({x_i - \bar{x}})^2} f_i}{N}} = \sqrt{\frac{\sum{{x_i}^2 f_i}}{N} - \bar{x}^2} = \sqrt{\sum{{x_i}^2 h_i} - \bar{x}^2}
            \end{equation*}
            \part
            Calcula el coeficiente de variación.
            \begin{equation*}
                CV = \frac{s}{\bar{x}}
            \end{equation*}

            \part
            Expón las conclusiones que has obtenido del análisis estadístico de las dos muestras.

        \end{parts}

    \end{questions}    

\end{document}