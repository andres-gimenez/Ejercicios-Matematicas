%%%%%%%%%%%%%%%%%%%%%%%%%%%
\newcommand{\numeroHoja} { Hoja 2.10 }
\newcommand{\nombreHoja} { Problemas con polinomios y ecuaciones }
%%%%%%%%%%%%%%%%%%%%%%%%%%%

% Configuración del documento.
\renewcommand{\schoolSubject} { Examen Matemáticas 2º ESO  }
\renewcommand{\school} { IES José de Churriguera  }
\renewcommand{\academicPeriod} { Curso 2022/2023 }

\renewcommand{\autor} { Andrés Giménez Muñoz }
\renewcommand{\emailAuthor} { andresprofemates@outlook.es }
\renewcommand{\autorSing}{ Profesor: Andrés }
\newcommand{\schoolSubject} { Matemáticas 3º ESO - Recuperación}
\newcommand{\school} { IES La Serna }
\newcommand{\academicPeriod} { Curso 2020/2021 }


\newcommand{\autor} { Andrés Giménez Muñoz }
\newcommand{\emailAuthor} { agimenezmunoz@ieslaserna.com }
\newcommand{\autorSing}{ Profesores: Andrés }

%\usepackage{enumerate}

%%%%%%%%%%%%%%%%%%%%%%%%%%%
% Exam configuration
\pointsdroppedatright   %% No mostrar la puntuación

%% Si se comenta no aparecerán los espacios de la solución.
%\nocancelspace

%% Esto es de la clase exam. Si dejamos sin comentar \printanswers, se mostraran las soluciones. 
%% Si la comentamos y dejamos sin comentar \noprintanswers, pues no se muestran las soluciones.
\printanswers
%\noprintanswers

%%%%%%%%%%%%%%%%%%%%%%%%%%%

\begin{document}

    %%\StudentData

    \begin{questions}
        \question 
        Un campo de fútbol mide 30 metros más de largo que de ancho y su área es de $7.000 m^2$, halla sus dimensiones.

        \question
        Una cancha de baloncesto mide 13 metros más de largo que de ancho y su área es de $420 m^2$, halla sus dimensiones.

        \question
        Una piscina olímpica mide 50 metros de larga por 25 metros de ancha y 3 metros de profundidad. 
        \begin{parts}
            \part
            Calcula el polinomio que representa el volumen de agua de la piscina en función del nivel de esta.
            \part
            Calcula el volumen de agua cuando tenemos medio metro de profundidad.
            \part
            Calcula el volumen de agua cuando tenemos dos metros de profundidad.
            \part
            Calcula el volumen de agua cuando la piscina está llena.
        \end{parts}

        \question
        Un depósito de agua con base circular tiene un diámetro de 14 metros, sabemos que se ha estado vertiendo agua durante 25 horas a razón de $5 m^3/h$.
        \begin{parts}
            \part
            ¿Qué profundidad tiene? \\
            \begin{minipage}[t]{\linewidth}
                \centering
                \includegraphics[width=7cm]{DepositoAgua}
                %\captionof{figure}{Figure of Q.\ref{label:a}}
                \label{label:q1}
            \end{minipage}

            \part
            Mientras se estaba llenando el depósito, llego un camión de bomberos y se llevó 5.000 litros de agua. 
            Corrige las cuentas del apartado anterior para saber la profundidad del depósito.\\
            {\footnotesize (*) $1 m^3$ = $1.000$ litros} \\
            \begin{minipage}[t]{\linewidth}
                \centering
                \includegraphics[width=7cm]{CamionBomberos}
                %\captionof{figure}{Figure of Q.\ref{label:a}}
                \label{label:q1}
            \end{minipage}
        \end{parts}

        \newpage
        \question
        Para operar con el IVA podemos utilizar los polinomios, en función del precio del producto:
        \begin{enumerate} [label=(\roman*)]
            \item
            Calcular el IVA, $P(x) = \frac{21}{100}x$.
            \item
            Precio más IVA, $Q(x) = \left(1 + \frac{21}{100}\right)x$
            \item
            Precio menos el IVA, $R(x) = \left(1 - \frac{21}{100}\right)x$
        \end{enumerate}
        \begin{parts}
            \part
            Obtén los polinomios para dado el precio de venta final obtener el precio del producto.
            \part
            Obtén los polinomios que nos del IVA y el precio más IVA de un producto al que se le ha rebajado 2\euro{} antes de aplicarle el IVA.
        \end{parts}
    \end{questions}
\end{document}