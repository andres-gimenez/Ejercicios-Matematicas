%%%%%%%%%%%%%%%%%%%%%%%%%%%
\newcommand{\numeroHoja} { Hoja 3.04 }
\newcommand{\nombreHoja} { Estadística }
%%%%%%%%%%%%%%%%%%%%%%%%%%%

% Configuración del documento.
\renewcommand{\schoolSubject} { Examen Matemáticas 2º ESO  }
\renewcommand{\school} { IES José de Churriguera  }
\renewcommand{\academicPeriod} { Curso 2022/2023 }

\renewcommand{\autor} { Andrés Giménez Muñoz }
\renewcommand{\emailAuthor} { andresprofemates@outlook.es }
\renewcommand{\autorSing}{ Profesor: Andrés } 
\newcommand{\schoolSubject} { Matemáticas 3º ESO - Recuperación}
\newcommand{\school} { IES La Serna }
\newcommand{\academicPeriod} { Curso 2020/2021 }


\newcommand{\autor} { Andrés Giménez Muñoz }
\newcommand{\emailAuthor} { agimenezmunoz@ieslaserna.com }
\newcommand{\autorSing}{ Profesores: Andrés } 

%%%%%%%%%%%%%%%%%%%%%%%%%%%
% Exam configuration
\pointsdroppedatright   %% No mostrar la puntuación

%% Si se comenta no aparecerán los espacios de la solución.
%\nocancelspace

%% Esto es de la clase exam. Si dejamos sin comentar \printanswers, se mostraran las soluciones. 
%% Si la comentamos y dejamos sin comentar \noprintanswers, pues no se muestran las soluciones.
\printanswers
%\noprintanswers

%%%%%%%%%%%%%%%%%%%%%%%%%%%

\usepackage{tkz-euclide}

\begin{document} 
    \begin{questions}
        \question
        En una encuesta realizada entre adolescentes de 15 años sobre sus hábitos en el uso de Internet 
        se les pregunto el número de horas que pasan conectados a estas, obteniendo los siguientes datos para España y para toda la OCDE.
    
        \begin{table}[ht]
            \centering
            \begin{tabular}{|l|l|l|}
            \hline
                                        & España & OCDE \\ \hline
            \hline                  
            Nada en absoluto            & 2 \%     & 3 \%   \\ \hline
            Menos de 1 hora             & 13 \%    & 16 \%  \\ \hline
            Entre 1 y 2 hora            & 18 \%    & 21 \%  \\ \hline
            Entre 2 y 4 horas           & 26 \%    & 27 \%  \\ \hline
            Entre 4 y 6 horas           & 18 \%    & 16 \%  \\ \hline
            Más de 6 horas              & 22 \%    & 16 \%  \\ \hline
            \end{tabular}
        \end{table}

        \begin{parts}
            
            \part
            Dibuja los histogramas comparando ambas muestras.

            \part
            Calcula la media de las dos muestras.
           
            \part
            Calcula la varianza y desviación típica de ambas muestras.

            \part
            Expón las conclusiones que has obtenido del análisis estadístico de las dos muestras.

        \end{parts}

    \end{questions}    

\end{document}