%%%%%%%%%%%%%%%%%%%%%%%%%%%
\newcommand{\numeroHoja} { Hoja 3.2 }
\newcommand{\nombreHoja} { Problemas sistemas de ecuaciones }
%%%%%%%%%%%%%%%%%%%%%%%%%%%

% Configuración del documento.
\renewcommand{\schoolSubject} { Examen Matemáticas 2º ESO  }
\renewcommand{\school} { IES José de Churriguera  }
\renewcommand{\academicPeriod} { Curso 2022/2023 }

\renewcommand{\autor} { Andrés Giménez Muñoz }
\renewcommand{\emailAuthor} { andresprofemates@outlook.es }
\renewcommand{\autorSing}{ Profesor: Andrés } 
\newcommand{\schoolSubject} { Matemáticas 3º ESO - Recuperación}
\newcommand{\school} { IES La Serna }
\newcommand{\academicPeriod} { Curso 2020/2021 }


\newcommand{\autor} { Andrés Giménez Muñoz }
\newcommand{\emailAuthor} { agimenezmunoz@ieslaserna.com }
\newcommand{\autorSing}{ Profesores: Andrés } 

%%%%%%%%%%%%%%%%%%%%%%%%%%%
% Exam configuration
\pointsdroppedatright   %% No mostrar la puntuación

%% Si se comenta no aparecerán los espacios de la solución.
%\nocancelspace

%% Esto es de la clase exam. Si dejamos sin comentar \printanswers, se mostraran las soluciones. 
%% Si la comentamos y dejamos sin comentar \noprintanswers, pues no se muestran las soluciones.
\printanswers
%\noprintanswers

%%%%%%%%%%%%%%%%%%%%%%%%%%%

\usepackage{tkz-euclide}

\begin{document}
    \begin{questions}
        \question
        %Copiado del libro
        El patrocinador de un equipo deportivo juvenil se ha gastado 735\euro{} en la compra del equipo de los 15 jugadores. 
        Si una camiseta cuesta 3\euro{} más que un pantalón ¿Cuánto cuesta cada prenda?

        \begin{solution}
            \begin{minipage}[t]{0.95\linewidth}
                \centering
                \includegraphics[width=\textwidth]{sol1}
            \end{minipage}
        \end{solution}

        \question
        %Copiado del libro
        Un examen de matemáticas consta de diez cuestions. Por cada una bien resuelta te dan 10 puntos y por cada una mal te quitan 3 puntos.
        Si Ana contestó a todas las cuestiones y obtuvo 61 puntos. ¿qué cantidad de respuestas correctas obtuvo?

        \begin{solution}
            \begin{minipage}[t]{0.95\linewidth}
                \centering
                \includegraphics[width=\textwidth]{sol2}
            \end{minipage}
        \end{solution}

        \question
        %Copiado del libro
        Ana, Bea, Clara y Dunia se quieren pesar de dos en dos. 
        Ana y Bea pesan 88kg, Bea y Clara pesan 91kg, Clara y Dunia pesan 86kg. 
        En ese momento, Dunia dice que no hace falta hacer más pesadas.
        \begin{parts}
            \part
            ¿Cuánto pesan Ana y Dunia juntas?
            \part
            Si cada pesaje cuesta 300\euro{} ¿Cuánto dinero se ahorran?
        \end{parts}

        \begin{solution}
            \begin{minipage}[t]{0.95\linewidth}
                \centering
                \includegraphics[width=\textwidth]{sol3}
            \end{minipage}
        \end{solution}

        \question
        %Copiado del libro
        En un hotel hay habitaciones dobles y triples. En total, hay 43 habitaciones y 105 camas. 
        La habitación doble cuesta 30\euro{} por noche, y la triple, 40\euro{} por noche.
        \begin{parts}
            \part
            ¿Cuánto se recauda el día que el hotel está completo?
            \part
            Si han recaudado 1280\euro{} y están todas las habitaciones dobles ocupadas, ¿cuántas triples quedan libres?
        \end{parts}

        \begin{solution}
            \begin{minipage}[t]{0.95\linewidth}
                \centering
                \includegraphics[width=\textwidth]{sol4}
            \end{minipage}
        \end{solution}

        \question
        %Copiado del libro
        Siete chocolatinas y cinco refrescos cuestan 30\euro{}. Tres chocolatinas y un refresco valen 10,80\euro{}. 
        ¿Cuánto costarán once chocolatinas y nueve refrescos?

        \begin{solution}
            \begin{minipage}[t]{0.95\linewidth}
                \centering
                \includegraphics[width=\textwidth]{sol5}
            \end{minipage}
        \end{solution}

        % \question
        % Marcos, entre pitos y flautas se ha gastado 1000\euro. Si el precio de las flautas es 5 veces el precio de los pitos.
        % \begin{parts}
        %     \part
        %     Con estos datos ¿Podemos saber cuánto cuesta cada instrumento?
        %     \part
        %     % Cuidado que tiene trampa. Hay que calcular primero cuánto cuesta
        %     Si sabemos que ha comprado instrumentos para 400 parsonas. 
        %     ¿Se puede resolver el problema?
        %     ¿Cuánto cuestan cada instrumento musical?  
        % \end{parts}

        
    \end{questions}    

\end{document}