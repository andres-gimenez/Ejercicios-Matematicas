% Configuración del documento.
\renewcommand{\schoolSubject} { Examen Matemáticas 2º ESO  }
\renewcommand{\school} { IES José de Churriguera  }
\renewcommand{\academicPeriod} { Curso 2022/2023 }

\renewcommand{\autor} { Andrés Giménez Muñoz }
\renewcommand{\emailAuthor} { andresprofemates@outlook.es }
\renewcommand{\autorSing}{ Profesor: Andrés } 
\newcommand{\schoolSubject} { Matemáticas 3º ESO - Recuperación}
\newcommand{\school} { IES La Serna }
\newcommand{\academicPeriod} { Curso 2020/2021 }


\newcommand{\autor} { Andrés Giménez Muñoz }
\newcommand{\emailAuthor} { agimenezmunoz@ieslaserna.com }
\newcommand{\autorSing}{ Profesores: Andrés } 

%%%%%%%%%%%%%%%%%%%%%%%%%%%
\renewcommand{\numeroHoja} { 3º Evaluación }
\renewcommand{\nombreHoja} { Sistemas de ecuaciones y funciones } 
\renewcommand{\waterMark} { Modelo B } 
%%%%%%%%%%%%%%%%%%%%%%%%%%%

%%%%%%%%%%%%%%%%%%%%%%%%%%%
% Exam configuration
%\pointsdroppedatright   %% No mostrar la puntuación
\pointsinrightmargin % Para poner las puntuaciones a la derecha. Se puede cambiar. Si se comenta, sale a la izquierda.
\extrawidth{-1.5cm} %Un poquito más de margen por si ponemos textos largos.
\marginpointname{ \emph{\points}}

%% Si se comenta no aparecerán los espacios de la solución.
%\nocancelspace

%% Esto es de la clase exam. Si dejamos sin comentar \printanswers, se mostraran las soluciones. 
%% Si la comentamos y dejamos sin comentar \noprintanswers, pues no se muestran las soluciones.
%\printanswers
%\noprintanswers

%%%%%%%%%%%%%%%%%%%%%%%%%%%

% \usepackage{tikz}
% \usetikzlibrary{arrows}

\begin{document}

	\StudentData
	\GradeTableHeader

    \justifying

	\begin{questions}
		\setcounter{question}{0}

		\question[2]
        Resuelve el siguiente sistema de ecuaciones por el método gráfico.
            \begin{flushleft}
                $\begin{cases}
                    \nonumber
                    x - y  = 2 \\
                    \nonumber
                    3x + y = 2 
                \end{cases}$
            \end{flushleft}

            \begin{figure}[h]
                \begin{tikzpicture}[scale=1]
                    \tkzInit[xmax=6,ymax=6,xmin=-6,ymin=-6]
                    \tkzGrid
                    \tkzAxeXY
                \end{tikzpicture}
            \end{figure}

            \newpage
            \question[3]
            Resuelve los siguientes sistemas de ecuaciones
            \begin{parts}
                \part
                \begin{flushleft}
                    $\begin{cases}
                        \nonumber
                        x + y  = 9 \\
                        \nonumber
                        2x - y = -3 
                    \end{cases}$
                \end{flushleft}
                \vspace{\stretch{1}}

                \part
                \begin{flushleft}
                    $\begin{cases}
                        \nonumber
                        x - 5y = 4 \\
                        \nonumber
                        3x - y = 2 
                    \end{cases}$
                \end{flushleft}
                \vspace{\stretch{1}}

                \part
                \begin{flushleft}
                    $\begin{cases}
                        \nonumber
                        -2(3x+2)+4y+6=0 \\
                        \nonumber
                        4(x-2)+6y=2
                    \end{cases}$
                \end{flushleft}
                \vspace{\stretch{1}}
            \end{parts}
            
        \newpage
        \question[2]
        En un espectáculo, las entradas de tres adultos y dos niños cuestan 26 \euro{}, y las de cuatro adultos y seis niños, 48 \euro{}.
        ¿Cuánto vale una entrada de adulto?
        \vspace{\stretch{1}}

        \newpage
        \question[3]
            Un banco lanza al mercado un plan de inversión cuya rentabilidad R(x), en euros, 
            viene dada en función de la cantidad invertida, x en euros, por medio de la expresión: \\
            $R(x) = -0,001x^2 + 0,4x + 3,5 $ \\

            \begin{parts}
                \part
                    Completa la tabla de rentabilidad
                \begin{table}[h]
                    \centering
                    \begin{tabular}{|
                    >{\columncolor[HTML]{CBCEFB}}c |c|c|c|c|c|c|c|}
                    \hline
                    Inversión    & 50 & 100 & 150 & 200 & 250 \\ \hline
                    Rentabilidad &    &    &    &     &       \\ \hline
                    \end{tabular}
                \end{table}
                \vspace{\stretch{1}}

                \part
                Dibuja la función de rentabilidad en función de la inversión de acuerdo a los valores obtenidos en el apartado anterior.                    
                \begin{figure}[h]
                    \begin{tikzpicture}[scale=0.8]
                        \begin{axis}[
                            axis lines=middle,
                            axis line style={->},
                            x label style={at={(axis description cs:0.5,-0.01)},anchor=north},
                            y label style={at={(axis description cs:-0.06,.5)},rotate=90,anchor=south},
                            xlabel={Inversión},ylabel={Rentabilidad},
                            grid, width=17cm, xmax=300,ymax=50,xmin=0,ymin=-20,
                            %axis lines=middle
                            ]
                        \end{axis}
                    \end{tikzpicture}
                \end{figure}
                
                \part
                ¿Qué cantidad interesa invertir para que la rentabilidad sea máxima?
                \vspace{\stretch{1}}
            \end{parts}

	\end{questions}
\end{document}