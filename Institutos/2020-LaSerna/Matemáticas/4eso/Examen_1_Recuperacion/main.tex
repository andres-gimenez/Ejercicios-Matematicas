%%%%%%%%%%%%%%%%%%%%%%%%%%%
\newcommand{\numeroHoja} { 1º Evaluación }
\newcommand{\nombreHoja} { Recuperación }
%%%%%%%%%%%%%%%%%%%%%%%%%%%

% Configuración del documento.
\renewcommand{\schoolSubject} { Examen Matemáticas 2º ESO  }
\renewcommand{\school} { IES José de Churriguera  }
\renewcommand{\academicPeriod} { Curso 2022/2023 }

\renewcommand{\autor} { Andrés Giménez Muñoz }
\renewcommand{\emailAuthor} { andresprofemates@outlook.es }
\renewcommand{\autorSing}{ Profesor: Andrés } 
\newcommand{\schoolSubject} { Matemáticas 3º ESO - Recuperación}
\newcommand{\school} { IES La Serna }
\newcommand{\academicPeriod} { Curso 2020/2021 }


\newcommand{\autor} { Andrés Giménez Muñoz }
\newcommand{\emailAuthor} { agimenezmunoz@ieslaserna.com }
\newcommand{\autorSing}{ Profesores: Andrés } 

%%%%%%%%%%%%%%%%%%%%%%%%%%%
% Exam configuration
%\pointsdroppedatright   %% No mostrar la puntuación
\pointsinrightmargin % Para poner las puntuaciones a la derecha. Se puede cambiar. Si se comenta, sale a la izquierda.
\extrawidth{-1.5cm} %Un poquito más de margen por si ponemos textos largos.
\marginpointname{ \emph{\points}}

%% Si se comenta no aparecerán los espacios de la solución.
%\nocancelspace

%% Esto es de la clase exam. Si dejamos sin comentar \printanswers, se mostraran las soluciones. 
%% Si la comentamos y dejamos sin comentar \noprintanswers, pues no se muestran las soluciones.
%\printanswers
%\noprintanswers

%%%%%%%%%%%%%%%%%%%%%%%%%%%

\begin{document}

	\StudentData
	\GradeTableHeader

	\begin{questions}
		\setcounter{question}{0}
		% \question[2]
		% Juan reparte 240\euro{} entre sus tres sobrinos, de 15, 16 y 17 años, 
		% de forma directamente proporcional a su edad. ¿Cuánto dinero recibe cada uno?
		% \vspace{\stretch{1}}

		\question[1]
		Calcula el valor de las siguientes potencias de fracciones.
		\begin{parts}
            % \part
			% $\left( \frac{1}{9} \right)^{-1}$
			\vspace{\stretch{1}}
			\part
			$\left( - \frac{3}{4} \right)^{-4}$
			\vspace{\stretch{1}}
			\part
			$\left( - \frac{5}{8} \right)^{-3}$
			\vspace{\stretch{1}}
			% \part
			% $\left( \frac{3}{5} \right)^{4}$
			% \vspace{\stretch{1}}
		\end{parts}

		% \question[1]
		% Racionaliza y simplifica.
		% \begin{parts}
		% 	\part
		% 	$\frac{3}{\sqrt{6}}$
		% 	\vspace{\stretch{1}}
		% 	\part
		% 	$\frac{10}{\sqrt[3]{5}}$
		% 	\vspace{\stretch{1}}
		% 	\part
		% 	$\frac{\sqrt{3}}{\sqrt[3]{12}}$
		% 	\part
		% 	$\frac{\sqrt{2}}{\sqrt[7]{4^3}}$
		% \end{parts}

		\question[1]
		Realiza las siguientes operaciones y expresa el resultado en forma de fracción irreducible.
		\begin{parts}
            \part
				$\frac{3}{5} \cdot \frac{1}{4} \cdot \frac{5}{6} : \frac{4}{3} $
				\vspace{\stretch{1}}
			% \part
			% 	$\frac{1}{25} + \frac{3}{5} - \frac{6}{9} : \frac{14}{12}$
			% 	\vspace{\stretch{1}}
			% \part
			% 	$\left( \frac{3}{5} \cdot \frac{1}{2} - \frac{5}{6} \right) : \frac{1}{3}$
			% 	\vspace{\stretch{1}}
			\part 
				$\frac{3}{5} \cdot \left( \frac{1}{2} + \frac{5}{13} \right) : \frac{1}{5}$
				\vspace{\stretch{1}}
			% \part 
			% 	$2 + \frac{3}{5} - \left( \frac{5}{6} + \frac{7}{12} \right) - \frac{1}{3}$
			% \part 
			% 	$\left( \frac{2}{3} \right)^3 - \left( \frac{2}{3} \right)^{-1}$
			% \part 
			% 	$\frac{40}{18} \cdot \frac{9}{30} - \left( 2 - \frac{3}{2} \right)^{2}$
			% \part 
			% 	$\frac{3}{10} \cdot \left(\frac{1}{7} - \frac{5}{12} \right)^{-1} : \frac{1}{4} \cdot \frac{2}{5}$
		\end{parts}

		\question[1]
		Realiza las siguientes operaciones y expresa el resultado en forma de fracción irreducible:
		\begin{parts}
			\part
			$\left(\frac{4}{7} \right)^{-5} : \left(\frac{4}{7} \right)^{4}$
			\vspace{\stretch{1}}
			\part
			$\left(\frac{3^2 \cdot 3^{-5}}{3^{-2}} \right)$
			\vspace{\stretch{1}}
		\end{parts}
		\vspace{\stretch{1}}

		\question[1]
		Simplifica los siguientes radicales 
		\begin{parts}
            \part
				$\sqrt[3]{8}$
				\vspace{\stretch{1}}
			% \part
			% 	$\sqrt[6]{27}$
			% 	\vspace{\stretch{1}}
			\part
				$\sqrt[3]{8.000}$
				\vspace{\stretch{1}}
			% \part
			% 	$\sqrt[12]{8 a^6 b^9}$
			% 	\vspace{\stretch{1}}
		\end{parts}

		% \question[2]
		% Calcula el valor de la incógnita en cada una de las relaciones de proporcionalidad:
		% \begin{parts}
		% 	\setcounter{partno}{0}
		% 	\part $\frac{3}{x} = \frac{1}{3}$
		% 	% \part $\frac{2}{x} = \frac{7}{5}$
		% 	% \part $\frac{x}{8} = \frac{2}{3}$
		% 	% \part $\frac{6}{x} = 2$
		% 	% \part $\frac{4}{3} = \frac{x}{6}$

		% 	% \part $\frac{1}{2} = \frac{4}{x}$
		% 	\part $\frac{2x}{3} = 4$
		% 	%\part $\frac{5}{2x} = \frac{1}{10}$
		% 	% \part Divide $v^6+1$ by $v-1$.
		% 	% \begin{solution}
		% 	% 	Solution
		% 	% \end{solution}
		% \end{parts}

		\newpage

		\question[2]
		Si medio kilo de clavos cuesta 0,8\euro. ¿Qué peso me darán por 0,60\euro?
		\vspace{\stretch{1}}

		\question[2]
		A tres amigos les han tocado el segundo premio de la lotería de navidad valorado en $125.000\euro{}$ 
		por un décimo que costó $20\euro{}$ al que aportaron 8\euro{}, 7\euro{} y 5\euro{} respectivamente,
		¿cómo han de repartir el premio?
		\vspace{\stretch{1}}

		\question[2]
		Tres personas se han repartido una cantidad de dinero directamente proporcional a los números 6, 3 y 2. 
		Si la que menos recibe ha recibido 900\euro{}. ¿qué cantidad total se repartió?
		\vspace{\stretch{1}}
	\end{questions}
\end{document}