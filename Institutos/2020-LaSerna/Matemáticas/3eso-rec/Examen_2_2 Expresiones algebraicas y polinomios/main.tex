%%%%%%%%%%%%%%%%%%%%%%%%%%%
\newcommand{\numeroHoja} { 2º Evaluación }
\newcommand{\nombreHoja} { Expresiones algebraicas y polinomios } 
%%%%%%%%%%%%%%%%%%%%%%%%%%%

% Configuración del documento.
\renewcommand{\schoolSubject} { Examen Matemáticas 2º ESO  }
\renewcommand{\school} { IES José de Churriguera  }
\renewcommand{\academicPeriod} { Curso 2022/2023 }

\renewcommand{\autor} { Andrés Giménez Muñoz }
\renewcommand{\emailAuthor} { andresprofemates@outlook.es }
\renewcommand{\autorSing}{ Profesor: Andrés } 
\newcommand{\schoolSubject} { Matemáticas 3º ESO - Recuperación}
\newcommand{\school} { IES La Serna }
\newcommand{\academicPeriod} { Curso 2020/2021 }


\newcommand{\autor} { Andrés Giménez Muñoz }
\newcommand{\emailAuthor} { agimenezmunoz@ieslaserna.com }
\newcommand{\autorSing}{ Profesores: Andrés } 

%%%%%%%%%%%%%%%%%%%%%%%%%%%
% Exam configuration
%\pointsdroppedatright   %% No mostrar la puntuación
\pointsinrightmargin % Para poner las puntuaciones a la derecha. Se puede cambiar. Si se comenta, sale a la izquierda.
\extrawidth{-1.5cm} %Un poquito más de margen por si ponemos textos largos.
\marginpointname{ \emph{\points}}

%% Si se comenta no aparecerán los espacios de la solución.
%\nocancelspace

%% Esto es de la clase exam. Si dejamos sin comentar \printanswers, se mostraran las soluciones. 
%% Si la comentamos y dejamos sin comentar \noprintanswers, pues no se muestran las soluciones.
%\printanswers
%\noprintanswers

%%%%%%%%%%%%%%%%%%%%%%%%%%%

% \usepackage{tikz}
% \usetikzlibrary{arrows}

\begin{document}

	\StudentData
	\GradeTableHeader

    \justifying

	\begin{questions}
		\setcounter{question}{0}

		\question[1]
        Selecciona la expresión algebraica que corresponda a las siguientes frases
        \begin{parts}
            \part
            Un número par: \\ \\
            \begin{oneparcheckboxes}
                \CorrectChoice $2n$
                \choice $3n$
                \choice $n^2$
                \choice $n^2 - 2$
            \end{oneparcheckboxes}
            \\
            \part
            Un número impar: \\ \\
            \begin{oneparcheckboxes}
                \choice $2n$
                \choice $3n$
                \choice $(n-1)^2$
                \CorrectChoice $2n - 1 $
            \end{oneparcheckboxes}
            \\
            \part
            El tripe de un número, menos cinco: \\ \\
            \begin{oneparcheckboxes}
                \choice $3n$
                \choice $5n$
                \choice $3(n-5)$
                \CorrectChoice $3n-5$
            \end{oneparcheckboxes}
            \\
            \part
            Raiz cuadrada del doble de un número: \\ \\
            \begin{oneparcheckboxes}
                \choice $\sqrt{n + 2}$
                \choice $2 \sqrt{n}$
                \CorrectChoice $\sqrt{2n}$
                \choice $\sqrt{n^2}$
            \end{oneparcheckboxes}
            \\
        \end{parts}

        \question[2]
        Calcula las siguientes operaciones con monomios
        \begin{parts}
            \part
                $2x^7+4x^7$
                \vspace{\stretch{1}}
            \part
                $2x^2 + \frac{3}{2}x^2$
                \vspace{\stretch{1}}
            \part
                $x + 5x - 3x$
                \vspace{\stretch{1}}
            \part
                $3x^3 \cdot 2x^7$
                \vspace{\stretch{1}}
            \part
                $7x^2 \cdot \frac{1}{7}x^3$
                \vspace{\stretch{1}}
        \end{parts}

        \newpage

        \question[2]
        Si $P(x)=2 x^2 - x - 5$, evalúa.
        \begin{parts}
            \part
                $P\left(-\frac{1}{2}\right)$
                \vspace{\stretch{1}}
            \part
                $P\left(-1\right)$
                \vspace{\stretch{1}}
            \part
                $P\left(0\right)$
                \vspace{\stretch{1}}
            \part
                $P\left(\frac{1}{3}\right)$
                \vspace{\stretch{1}}
        \end{parts}

        \question[2]
        Sea $P(x)=x^4 + 2x^3 -x^2 -2x +2$, $Q(x)=3x^4-3x^2 - 2x+1$, calcula.
        \begin{parts}
            \part
                $P(x) + Q(X)$
                \vspace{\stretch{2}}
            \part
                $3 P(x)$
                \vspace{\stretch{2}}

            \part
                $3 P(x) + Q(x)$
                \vspace{\stretch{2}}
        \end{parts}

        \question[3]
        Sea $P(x)=x^2 -2x +2$, $Q(x)=2x+3$, calcula.
        \begin{parts}
            \part
                $P(x) \cdot Q(x)$
                \vspace{\stretch{4}}
            \part
                $P(x) \cdot 2 Q(x)$
                \vspace{\stretch{4}}
        \end{parts}

	\end{questions}
\end{document}