%%%%%%%%%%%%%%%%%%%%%%%%%%%
\newcommand{\numeroHoja} { 2º Evaluación }
\newcommand{\nombreHoja} { Potencias y raíces } 
%%%%%%%%%%%%%%%%%%%%%%%%%%%

% Configuración del documento.
\renewcommand{\schoolSubject} { Examen Matemáticas 2º ESO  }
\renewcommand{\school} { IES José de Churriguera  }
\renewcommand{\academicPeriod} { Curso 2022/2023 }

\renewcommand{\autor} { Andrés Giménez Muñoz }
\renewcommand{\emailAuthor} { andresprofemates@outlook.es }
\renewcommand{\autorSing}{ Profesor: Andrés } 
\newcommand{\schoolSubject} { Matemáticas 3º ESO - Recuperación}
\newcommand{\school} { IES La Serna }
\newcommand{\academicPeriod} { Curso 2020/2021 }


\newcommand{\autor} { Andrés Giménez Muñoz }
\newcommand{\emailAuthor} { agimenezmunoz@ieslaserna.com }
\newcommand{\autorSing}{ Profesores: Andrés } 

%%%%%%%%%%%%%%%%%%%%%%%%%%%
% Exam configuration
%\pointsdroppedatright   %% No mostrar la puntuación
\pointsinrightmargin % Para poner las puntuaciones a la derecha. Se puede cambiar. Si se comenta, sale a la izquierda.
\extrawidth{-1.5cm} %Un poquito más de margen por si ponemos textos largos.
\marginpointname{ \emph{\points}}

%% Si se comenta no aparecerán los espacios de la solución.
%\nocancelspace

%% Esto es de la clase exam. Si dejamos sin comentar \printanswers, se mostraran las soluciones. 
%% Si la comentamos y dejamos sin comentar \noprintanswers, pues no se muestran las soluciones.
%\printanswers
%\noprintanswers

%%%%%%%%%%%%%%%%%%%%%%%%%%%

% \usepackage{tikz}
% \usetikzlibrary{arrows}

\begin{document}

	\StudentData
	\GradeTableHeader

    \justifying

    \begin{center}
		\fbox{\fbox{\parbox{6.5in}{\centering
		\begin{multicols}{4}
			\begin{align*}
			 	8 &= 2^3 \\
                16 &= 2^4 \\
                128 &= 2^7 \\
                1024 &= 2^{10}
			\end{align*}
			\begin{align*}
                9 &= 3^2 \\
                12 &= 2^2 \cdot 3 \\
                25 &= 5^2			
			\end{align*}
			\begin{align*}
                27 &= 3^3 \\
                49 &= 7^2 \\
                75 &= 3 \cdot 5^2 \\
                81 &= 3^4
			\end{align*}
            \begin{align*}
                2^8 &= 2 \cdot 2^7 \\
                3^3 &= 3 \cdot 3^2 \\
                10^2 &= 100
			\end{align*}
		\end{multicols}
		}}}
	\end{center}

	\begin{questions}
		\setcounter{question}{0}

		\question[1]
        Expresa los productos en forma de potencia.
        \begin{parts}
            \part
            $2 \cdot 2 \cdot 2 \cdot 2 $
            \vspace{\stretch{1}}
            \part
            $5 \cdot 5 \cdot 5 \cdot 5 \cdot 5 \cdot 5 $
            \vspace{\stretch{1}}
            \part
            $(-3) \cdot (-3) \cdot (-3) $
            \vspace{\stretch{1}}
            \part
            $7 \cdot 7 \cdot 7 \cdot 7 \cdot 7$
            \vspace{\stretch{1}}
		\end{parts}

		\question[2]
        Calcula las siguiente potencias.
        \begin{parts}
            \part
                $2^8 $ \\
                \tiny{*{Recuerda que }$2^8 = 2 \cdot 2^7$ y ayudate de la tabla.}
                \normalsize
                \vspace{\stretch{1}}
			\part
                $5^2$
                \vspace{\stretch{1}}
            \part
				$3^3$ \\
				\tiny{*{Recuerda que }$3^3 = 3^2 \cdot 3$ y ayudate de la tabla.}
				\normalsize
				\vspace{\stretch{1}}
			\part
                $(-3)^3$
                \vspace{\stretch{1}}
            \part
                $(-2)^3$ \\
                \tiny{*{Recuerda que }$2^3 = 2 \cdot 2^2$ y ayudate de la tabla.}
                \normalsize
                \vspace{\stretch{1}}
            \part
                $(-1)^{345}$
                \vspace{\stretch{1}}
		\end{parts}

		\question[1]
        Expresa como una única potencia.
        \begin{parts}
            \part
                $2^6 \cdot 5^6 \cdot 7^6$
                \vspace{\stretch{1}}
            \part
                $21^8 : 3^8$
                \vspace{\stretch{1}}
            \part
                $12^5 : 4^5 : 3^5$
                \vspace{\stretch{1}}
        \end{parts}

        \newpage

        \question[3]
        Calcula
        \begin{parts}
            \part 
                $\frac{16 \cdot 81 \cdot 25}{12 \cdot 300}$
                \vspace{\stretch{1}}
            \part 
                $\frac{16^3 \cdot 12^5}{8^5 \cdot 9^2}$
                \vspace{\stretch{1}}

            \part 
                $\frac{9^4 \cdot 27^3}{81^4}$
                \vspace{\stretch{1}}
        \end{parts}

        \question[1]
        Calcula las siguientes raices
        \begin{parts}
            \part
                $\sqrt{16}$
                \vspace{\stretch{1}}
            \part
                $\sqrt{49}$
                \vspace{\stretch{1}}
            \part
                $\sqrt{1024}$
                \vspace{\stretch{1}}
        \end{parts}

		\question[2]
		Fernando tuvo 2 hijos. Cada uno de sus hijos tubo 2 hijos y cada uno de estos tuvo 2 hijos. ¿Cuántos nietos tuvo Fernando?
        \vspace{\stretch{5}}
	\end{questions}
\end{document}