%%%%%%%%%%%%%%%%%%%%%%%%%%%
\newcommand{\documentName} { Examen 3ª evaluación }
\newcommand{\documentContent} { Recuperación } 
\newcommand{\waterMark} { } 
%%%%%%%%%%%%%%%%%%%%%%%%%%%

% Configuración del documento.
\newcommand{\schoolSubject} { Matemáticas 3º ESO - Recuperación}
\newcommand{\school} { IES La Serna }
\newcommand{\academicPeriod} { Curso 2020/2021 }


\newcommand{\autor} { Andrés Giménez Muñoz }
\newcommand{\emailAuthor} { agimenezmunoz@ieslaserna.com }
\newcommand{\autorSing}{ Profesores: Andrés } 
\renewcommand{\schoolSubject} { Examen Matemáticas 2º ESO  }
\renewcommand{\school} { IES José de Churriguera  }
\renewcommand{\academicPeriod} { Curso 2022/2023 }

\renewcommand{\autor} { Andrés Giménez Muñoz }
\renewcommand{\emailAuthor} { andresprofemates@outlook.es }
\renewcommand{\autorSing}{ Profesor: Andrés } 

%%%%%%%%%%%%%%%%%%%%%%%%%%%
% Exam configuration
%\pointsdroppedatright   %% No mostrar la puntuación
\pointsinrightmargin % Para poner las puntuaciones a la derecha. Se puede cambiar. Si se comenta, sale a la izquierda.
\extrawidth{-1.5cm} %Un poquito más de margen por si ponemos textos largos.
\marginpointname{ \emph{\points}}

%% Si se comenta no aparecerán los espacios de la solución.
%\nocancelspace

%% Esto es de la clase exam. Si dejamos sin comentar \printanswers, se mostraran las soluciones. 
%% Si la comentamos y dejamos sin comentar \noprintanswers, pues no se muestran las soluciones.
%\printanswers
%\noprintanswers

%%%%%%%%%%%%%%%%%%%%%%%%%%%

% \usepackage{tikz}
% \usetikzlibrary{arrows}

\begin{document}

\StudentData
\GradeTableHeader

\justifying

\begin{questions}
	\setcounter{question}{0}

	\question[2]
	Resuelve las siguientes ecuaciones
	\begin{parts}
		\part
		$3(x-1)+3=6$
		\vspace{\stretch{1}}

		\part
		$\frac{5x-4}{6}+\frac{4x+3}{3}=-1$
		\vspace{\stretch{1}}
	\end{parts}

	\newpage
	\question[2]
	Resuelve las siguientes ecuaciones de 2º grado.
	\begin{parts}
		\part
		% x=-2, x=4
		$x^2-2x-8=0$
		\vspace{\stretch{1}}

		\part
		% x= 5, x=-2
		$x^2-3x-10=0$
		\vspace{\stretch{1}}
	\end{parts}

	\newpage
	\question[2]
	Resuelve el siguiente sistema de ecuaciones por el método gráfico.
	\begin{flushleft}
		$\begin{cases}
				\nonumber
				x - y  = 2 \\
				\nonumber
				3x + y = 2
			\end{cases}$
	\end{flushleft}

	\begin{figure}[h]
		\begin{tikzpicture}[scale=1]
			\tkzInit[xmax=6,ymax=6,xmin=-6,ymin=-6]
			\tkzGrid
			\tkzAxeXY
		\end{tikzpicture}
	\end{figure}

	\newpage
	\question[2]
	Resuelve los siguientes sistemas de ecuaciones
	\begin{parts}
		\part
		\begin{flushleft}
			$\begin{cases}
					\nonumber
					x + y  = 9 \\
					\nonumber
					2x - y = -3
				\end{cases}$
		\end{flushleft}
		\vspace{\stretch{1}}

		\part
		\begin{flushleft}
			$\begin{cases}
					\nonumber
					5x - y = 4 \\
					\nonumber
					3x - y = 2
				\end{cases}$
		\end{flushleft}
		\vspace{\stretch{1}}

		% \part
		% \begin{flushleft}
		%     $\begin{cases}
		%         \nonumber
		%         3x + 5y = 4\\
		%         \nonumber
		%         3x - y = 2
		%     \end{cases}$
		% \end{flushleft}
		% \vspace{\stretch{1}}
	\end{parts}

	\newpage
	\question[2]
	Hemos comprado 3 yogures de frutas y 2 naturales por 8 \euro{} y, ayer, 2 de frutas y 5 naturales por 
		9 \euro{}. Determinar el precio de cada tipo de yogur.
	\vspace{\stretch{1}}

\end{questions}
\end{document}