%%%%%%%%%%%%%%%%%%%%%%%%%%%%%%%%%%%%%%%%%%%%%%%%%%%%%%%%%%%
%% IES La Serna - 3º ESO -Recuperación.
%% 2º Evaluación
%% Hoja de ejercicios: 1
%%
%% Repaso
%%
%%%%%%%%%%%%%%%%%%%%%%%%%%%%%%%%%%%%%%%%%%%%%%%%%%%%%%%%%%%

% Configuración del documento.
\renewcommand{\schoolSubject} { Examen Matemáticas 2º ESO  }
\renewcommand{\school} { IES José de Churriguera  }
\renewcommand{\academicPeriod} { Curso 2022/2023 }

\renewcommand{\autor} { Andrés Giménez Muñoz }
\renewcommand{\emailAuthor} { andresprofemates@outlook.es }
\renewcommand{\autorSing}{ Profesor: Andrés } 

%%%%%%%%%%%%%%%%%%%%%%%%%%%
% Exam configuration
\pointsdroppedatright   %% No mostrar la puntuación

%% Si se comenta no aparecerán los espacios de la solución.
%\nocancelspace

%% Esto es de la clase exam. Si dejamos sin comentar \printanswers, se mostraran las soluciones. 
%% Si la comentamos y dejamos sin comentar \noprintanswers, pues no se muestran las soluciones.
\printanswers
%\noprintanswers

\firstpageheader 
    {Matemáticas Académicas 3ºESO Recuperacion \\ \small IES La Serna \\}
    {}
    {Hoja 2.1 \small Curso 2020/2021 \\ Repaso} 

\runningheader
    {Matemáticas Académicas 3ºESO Recuperacion \\ \small IES La Serna}
    {}
    {Hoja 2.1 \small Curso 2020/2021 \\ Repaso} 

%%%%%%%%%%%%%%%%%%%%%%%%%%%

\begin{document}

    \SolutionHeader{}

    \WithoutCalculator{ 

    \begin{questions}
        \question
        Resuelve las siguientes operaciones escribiendo el proceso el proceso de resolución paso a paso.
        \begin{parts}
            \part
                $\frac{3}{4} - \frac{1}{3} - \frac{2}{12} + \frac{5}{6}$

                \begin{solution}
                    $m.c.m(3, 4,6,12) = 2^2 \cdot 3 = 12$ \\ \\
                    $\frac{3}{4} - \frac{1}{3} - \frac{2}{12} + \frac{5}{6} = \frac{9 - 4 - 2 + 10}{12} = \frac{13}{12}$
                \end{solution}
            \part
                $\left( 4 + \frac{3}{4}\right) - \left(3 + \frac{2}{3} \right)$

                \begin{solution}
                    $m.c.m(3, 4) = 2^2 \cdot 3 = 12$ \\ \\
                    $\left( 4 + \frac{3}{4}\right) - \left(3 + \frac{2}{3} \right) = \left( \frac{48}{12} + \frac{9}{12} \right) - \left( \frac{36}{12} + \frac{8}{12} \right) = \frac{57}{12} - \frac{44}{12} = \frac{13}{12}$
                \end{solution}
        \end{parts}
       
        \question
        Resuelve las siguientes operaciones y simplifica el resultado.
        \begin{parts}
            \part
                $\frac{5}{6} \cdot \frac{2}{3} $
            \part
                $\frac{2}{15} : \frac{2}{3} $
        \end{parts}

        \question
        Halla la fracción irreducible de cada una de estas fracciones.
        \begin{parts}
            \part
                $\frac{50}{70}$
            \part
                $\frac{36}{40}$
        \end{parts}

        \question
        Ordena de menor a mayor las siguientes fracciones reduciéndolas previamente a común denominador. \\

        $\frac{3}{4}$, $\frac{7}{9}$, $\frac{5}{12}$, $\frac{5}{18}$ \\

        % \question
        % Simplifica las siguientes potencias.
        % \begin{parts}
        %     \part
        %         $3^3 \cdot 3^4 \cdot 3$
        %     \part
        %         $5^7 : 5^3$
        %     \part
        %         $\left( 5^3 \right)^4$
        %     \part
        %         $\left( 5 \cdot 2 \cdot 3 \right)^4$
        % \end{parts}
    \end{questions}
\end{document}