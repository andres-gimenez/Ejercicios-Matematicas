% Configuración del documento.
\renewcommand{\schoolSubject} { Examen Matemáticas 2º ESO  }
\renewcommand{\school} { IES José de Churriguera  }
\renewcommand{\academicPeriod} { Curso 2022/2023 }

\renewcommand{\autor} { Andrés Giménez Muñoz }
\renewcommand{\emailAuthor} { andresprofemates@outlook.es }
\renewcommand{\autorSing}{ Profesor: Andrés } 
\newcommand{\schoolSubject} { Matemáticas 3º ESO - Recuperación}
\newcommand{\school} { IES La Serna }
\newcommand{\academicPeriod} { Curso 2020/2021 }


\newcommand{\autor} { Andrés Giménez Muñoz }
\newcommand{\emailAuthor} { agimenezmunoz@ieslaserna.com }
\newcommand{\autorSing}{ Profesores: Andrés }

%%%%%%%%%%%%%%%%%%%%%%%%%%%
\newcommand{\numeroHoja} { 2º Evaluación }
\newcommand{\nombreHoja} { Expresiones algebraicas y polinomios } 
\renewcommand{\waterMark} { Modelo: A } 
%%%%%%%%%%%%%%%%%%%%%%%%%%%

%%%%%%%%%%%%%%%%%%%%%%%%%%%
% Exam configuration
%\pointsdroppedatright   %% No mostrar la puntuación
\pointsinrightmargin % Para poner las puntuaciones a la derecha. Se puede cambiar. Si se comenta, sale a la izquierda.
\extrawidth{-1.5cm} %Un poquito más de margen por si ponemos textos largos.
\marginpointname{ \emph{\points}}

%% Si se comenta no aparecerán los espacios de la solución.
%\nocancelspace

%% Esto es de la clase exam. Si dejamos sin comentar \printanswers, se mostraran las soluciones. 
%% Si la comentamos y dejamos sin comentar \noprintanswers, pues no se muestran las soluciones.
%\printanswers
%\noprintanswers

%%%%%%%%%%%%%%%%%%%%%%%%%%%

% \usepackage{tikz}
% \usetikzlibrary{arrows}

\begin{document}

	\StudentData
	\GradeTableHeader

    \justifying

	\begin{questions}
		\setcounter{question}{0}

		\question[1]
        Indica la expresión algebraica que corresponda a las siguientes frases
        \begin{parts}
            \part
            Un número par: \\ \\
            \begin{oneparcheckboxes}
                \CorrectChoice $2n$
                \choice $3n$
                \choice $n^2$
                \choice $n^2 - 2$
            \end{oneparcheckboxes}
            \\
            \part
            Un número impar: \\ \\
            \begin{oneparcheckboxes}
                \choice $2n$
                \choice $3n$
                \choice $(n-1)^2$
                \CorrectChoice $2n - 1 $
            \end{oneparcheckboxes}
            \\
            \part
            La suma de dos números consecutivos: \\ \\
            \begin{oneparcheckboxes}
                \choice $n + 2$
                \choice $n + n$
                \choice $n + (n + 1)$
                \CorrectChoice $2n + 1$
            \end{oneparcheckboxes}
            \\
            \part 
            La mitad de un número:
            \vspace{\stretch{1}}
            \part
            El triple de un número más dos unidades:
            \vspace{\stretch{1}}
            \part 
            Un número entero, el anterior y el siguiente:
            \vspace{\stretch{1}}
        \end{parts}

        \question[1]
        Calcula las siguientes operaciones con monomios
        \begin{parts}
            \part
                $-\frac{3}{5}x^4 + x^4$
                \vspace{\stretch{1}}
            \part
                $\frac{2}{5}x^{256} + \frac{3}{5}x^{256}$
                \vspace{\stretch{1}}
            \part
                $x + 5x - 3x$
                \vspace{\stretch{1}}
            \part
                $\frac{2x}{3} - \frac{1x}{4}$
                \vspace{\stretch{1}}

            \part
                $-\frac{3x^2}{5} \cdot \frac{2x^3}{3}$
                \vspace{\stretch{1}}
            \part
                $\left(-x^3\right)^4 $
                \vspace{\stretch{1}}
        \end{parts}

        \newpage

        \question[2]
        Si $P(x)=x^4 + x^3 - 2x^2 -x-5$, evalúa.
        \begin{parts}
            \part
                $P\left(-1\right)$
                \vspace{\stretch{1}}
            \part
                $P\left(0\right)$
                \vspace{\stretch{1}}
            \part
                $P\left(\frac{1}{2}\right)$
                \vspace{\stretch{1}}                
        \end{parts}

        \question[2]
        Sea $P(x)=x^5+3x^3-2x+2$, $Q(x)=3x^4-3x^2-2x+1$, $R(x)=x^2 -2x +2$ calcula.
        \begin{parts}
            \part
                $P(x) - Q(x)$
                \vspace{\stretch{3}}
            % \part
            %     $4 Q(x)$
            %     \vspace{\stretch{2}}

            % \part
            %     $P(x) + 4 Q(x)$
            %     \vspace{\stretch{2}}

            \part
                $Q(x) \cdot R(x)$
                \vspace{\stretch{3}}
            % \part
            %     $P(x) \cdot 2 Q(x)$
            %     \vspace{\stretch{4}}
        \end{parts}

        \newpage
        \StudentData
        \question[2]
            Realiza la siguiente división \\
            $\left(\polynomial[reciprocal]{6,-1,-25,15}\right) : \left( \polynomial[reciprocal]{2,3,-4} \right)$

            \vspace{\stretch{3}}

            \begin{solution}
                \nopagebreak
                $$\polylongdiv[style=D]{6x^3-x^2-25x+15}{2x^2+3x-4}$$
                $$\boxed{\polynomial[reciprocal]{6,-1,-25,15} = \left( \polynomial[reciprocal]{2,3,-4} \right) \left( \polynomial[reciprocal]{3,-5} \right) + \left( \polynomial[reciprocal]{2,-5} \right)}$$
            \end{solution}

        \question[2]
            Realiza la siguiente división por el método de Ruffini \\
            $\left(\polynomial[reciprocal]{2,-3,2,-3}\right) : \left( \polynomial[reciprocal]{1, 2} \right)$
            
            \vspace{\stretch{3}}

            \begin{solution}
                $$\polyhornerscheme[x=1,resultbottomrule=true,resultleftrule=true]{2x^3-3x^2+2x-3}$$

                $$\boxed{\polynomial[reciprocal]{2,-3,2,-3} = \left( \polynomial[reciprocal]{1,2} \right) \left( \polynomial[reciprocal]{2, -1, 1} \right) - 2}$$
            \end{solution}
        
        \newpage
        \question[2]
        Una casa está rodeada por un jardín de acuerdo al siguiente plano.
        \\
        \vspace{0.3cm}
        \begin{minipage}{\linewidth}
            \centering
            \includegraphics[width=9cm]{Plano01}
        \end{minipage}
        \begin{parts}
            \part
            Halla un polinomio que nos permita calcular el área del jardín.
            \vspace{\stretch{5}}
            \part
            Calcula el área del jardín para $x = 50$
            \vspace{\stretch{2}}
            \part
            Tiene sentido calcular el área para $x = 10$ ¿Por qué?
            \vspace{\stretch{3}}
        \end{parts} 
	\end{questions}
\end{document}