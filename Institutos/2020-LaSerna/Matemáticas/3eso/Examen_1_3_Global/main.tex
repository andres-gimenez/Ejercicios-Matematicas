\documentclass[addpoints,spanish, 12pt,a4paper,cancelspace]{./include/gexam}

%%%%%%%%%%%%%%%%%%%%%%%%%%%
\renewcommand{\numeroHoja} { 1º Evaluación }
\renewcommand{\nombreHoja} { Global } 
%%%%%%%%%%%%%%%%%%%%%%%%%%%

% Configuración del documento.
\renewcommand{\schoolSubject} { Examen Matemáticas 2º ESO  }
\renewcommand{\school} { IES José de Churriguera  }
\renewcommand{\academicPeriod} { Curso 2022/2023 }

\renewcommand{\autor} { Andrés Giménez Muñoz }
\renewcommand{\emailAuthor} { andresprofemates@outlook.es }
\renewcommand{\autorSing}{ Profesor: Andrés } 

%%%%%%%%%%%%%%%%%%%%%%%%%%%
% Exam configuration
%\pointsdroppedatright   %% No mostrar la puntuación
\pointsinrightmargin % Para poner las puntuaciones a la derecha. Se puede cambiar. Si se comenta, sale a la izquierda.
\extrawidth{-1.5cm} %Un poquito más de margen por si ponemos textos largos.
\marginpointname{ \emph{\points}}

%% Si se comenta no aparecerán los espacios de la solución.
%\nocancelspace

%% Esto es de la clase exam. Si dejamos sin comentar \printanswers, se mostraran las soluciones. 
%% Si la comentamos y dejamos sin comentar \noprintanswers, pues no se muestran las soluciones.
%\printanswers
%\noprintanswers

%%%%%%%%%%%%%%%%%%%%%%%%%%%

\usepackage{tikz}
\usetikzlibrary{arrows}
% \usetikzlibrary{positioning,decorations.pathmorphing}
% \newcommand{\Interval}[4]{%
%     \tikz{%
%         \coordinate [label={center:$#1$},label=below:$\rule{0pt}{.35cm}#2$] (a) at (0,0); 
%         \coordinate [label={center:$#3$},label=below:$\rule{0pt}{.35cm}#4$] (b) at (1.6,0); 
%         \draw[-{latex},] decorate {(a)--(b)} (-.4,0)--(2,0);
%     }
% }
% \newcommand{\insertinterval}[1]{\rule{0in}{.7cm}\parbox{2.5cm}{#1}}

% \usepackage{pst-plot}
% \newcommand{\Intervala}{
% 	\begin{pspicture}(-3,-.5)(3,.5)
% 		\psaxes[yAxis=false,ticksize=0 -4pt]{<->}(0,0)(-3,-1)(3,1)
% 		\psline[linecolor=blue,arrowscale=1.25]{o-*}(.5,0)(2,0)
% 	\end{pspicture}
% }

\begin{document}

	\StudentData
	\GradeTableHeader
	
	\justifying

	\begin{questions}
		\setcounter{question}{0}
		\question
		Se ha hecho un estudio estadístico sobre la duración de una marca de pilas. Los resultados, en horas, de una muestra de 15 pilas han sido:
		\begin{table}[h]
			\centering
			\begin{tabular}{ccccc}
			20 & 22 & 25 & 21 & 22 \\
			21 & 23 & 21 & 24 & 21 \\
			23 & 22 & 20 & 22 & 24 
			\end{tabular}
		\end{table}
		\begin{parts}
			\part[\half]
			Define la variable estadística. ¿De qué tipo es? \\
			¿Tiene sentido agrupar los datos en intervalos? Justifica tu respuesta.
			\part[1]
			Completa la tabla de frecuencias relativas y acumuladas.
			\part[1]
			Dibuja el diagrama de barras y polígono de frecuencia.
			\part[\half]
			¿Cuál ha sido la duración media de las pilas? ¿Y la moda?
		\end{parts}
		\vspace{\stretch{1}}
		\newpage

		\question
		Representa en la recta real:
		\begin{center}
			$\frac{4}{6}$ y $\frac{12}{7} $
		\end{center}
		\newpage

		\question[1]
		Clasifica los siguientes números como naturales($\mathbb{N}$), enteros($\mathbb{Z}$), racionales($\mathbb{Q}$), o reales($\mathbb{R}$).
		\begin{parts}
			\part
			$7$
			\part
			$2,14$
			\part
			$\frac{3}{7}$
			\part
			$\sqrt{7}$
			\part
			$\pi$
			\part
			$\sqrt{4}$
		\end{parts}

		\question[2]
		Completa la tabla de intervalos y semirrectas

		\begin{table}[h]
			\centering
			\renewcommand{\arraystretch}{2}
			%\begin{tabular}{|c|c|c|}

			\begin{tabular}{|p{5cm}|p{3cm}|m{5cm}|}
				\hline
				\rowcolor[gray]{.9}
				\textbf{Representación gráfica} & \textbf{Intervalos} & \textbf{Definición matemática} \\ \hline
				& \multicolumn{1}{c|}{$[-1, 3)$}     &                       \\ [0.4cm] \hline
				\center \tikz\draw [*-o] (0,0) node[pos=2, below] {$-2$} -- +(3,0) node[pos=1, below] {$4$}; &          &                       \\ [0.4cm] \hline 
				           &         & \multicolumn{1}{c|}{$\left\{x | x > 0 \right\}$}  \\ [0.4cm] \hline
				\center \tikz\draw [<-*] (0,0) node[pos=2, below] {$-\infty$} -- +(3,0) node[pos=1, below] {$3$};&          &                      \\ [0.4cm] \hline
			\end{tabular}
		\end{table}

		\question[2]
		\begin{parts}
			\part
			$\frac{3}{10} - \frac{5}{8} + \frac{1}{6}$
			\part
			$\frac{2}{5} - \left( \frac{5}{8} + \frac{3}{4} \right)$
			\part
			$\left( 2 - \frac{4}{7} \right) : \frac{3}{14}$
			\part
			$\frac{3}{24} \cdot \left( \frac{2}{11} + \frac{18}{22} \right)$
		\end{parts}

		\question[1]
		\begin{parts}
			\part
			$\left(\frac{3}{4} \right)^{-3} : \left(\frac{3}{4} \right)^{2}$
			\part
			$\left(\frac{2^5 \cdot 2^{-7}}{2^{-4}} \right)$
		\end{parts}
	\end{questions}
\end{document}