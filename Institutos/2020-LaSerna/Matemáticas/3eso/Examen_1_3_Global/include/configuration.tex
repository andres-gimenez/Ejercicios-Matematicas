\documentclass[addpoints,spanish, 12pt,a4paper,cancelspace]{exam}

%\usepackage[paperheight=5.8in,paperwidth=8.27in,bindingoffset=0in,left=0.8in,right=1in, top=0.7in,bottom=1in,headsep=.5\baselineskip]{geometry}
%\usepackage[bindingoffset=0in,left=0.8in,right=0.8in, top=0.7in,bottom=1in,headsep=.5\baselineskip]{geometry}
\usepackage[left=1.2cm, right=2.5cm, bottom=2cm, top=2cm]{geometry}
\usepackage[T1]{fontenc}
\usepackage[utf8]{inputenc}
\usepackage{textcomp}
\usepackage[gen]{eurosym}
\usepackage{xcolor}
\usepackage{multicol}

\usepackage[normalem]{ulem}
\renewcommand\ULthickness{0.5pt}   %%---> For changing thickness of underline
\setlength\ULdepth{1.5ex}          %\maxdimen ---> For changing depth of underline
\renewcommand{\baselinestretch}{1}
\usepackage{fancybox} % 
\usepackage{graphicx} % Paquete necesario para incluir imágenes, cambiarles el tamaño, etc.
\usepackage{enumitem} % Para poder configurar las listas
\everymath{\displaystyle} % Esto es para que las expresiones se vean... grandes, que resulta diferente de si las queremos entre líneas.

\usepackage{colortbl}
\definecolor{azul}{rgb}{0.17,0.40, 0.68}
\newcommand{\gray}{\rowcolor[gray]{.90}}

\usepackage{amssymb}
\usepackage{lmodern}
\usepackage{mathtools, nccmath}

\usepackage{tikz}
\usepackage{pgfplots}

% Idioma y codificación de texto
\PassOptionsToPackage{T1}{fontenc}
\usepackage{fontenc}
\usepackage[utf8]{inputenc}
% Cargar babel y configurar para español
\usepackage[spanish,es-lcroman, es-tabla, es-noshorthands]{babel}

\usepackage{ragged2e}

\author{Andrés Giménez Muñoz}
\newcommand{\fecha} { Septiembre 2020 }
\newcommand{\titulo} { { Examen } }
\newcommand{\autor} { Andrés Giménez Muñoz }
\newcommand{\autorSing}{ Andrés }
\newcommand{\emailAuthor} { agimenezmunoz@ieslaserna.com }

\usepackage[pdfborder={0 0 0}, colorlinks=true, linkcolor=black, urlcolor=black, citecolor=black, pdfpagemode=UseNone]{hyperref}
\hypersetup{
	pdfauthor = \autor~(\emailAuthor),
	pdftitle = {\titulo},
	pdfsubject= {},
	pdfcreator = {},
	pdfproducer = {},
	pdfkeywords = {}
}

\usepackage{background}
\backgroundsetup{
	position=current page.east,
	angle=90,
	color=gray,
	%nodeanchor=east,
	hshift=-260mm,
	vshift=5mm,
	opacity=1,
	scale=0.5,
	contents={Profesor: \autorSing}
}

%%%%%%%%%%%%%%%%%%%%%%%%%%%%%%%%%%%%%%%%%%%%%%%%%%%%%%%%%%%%%%%%%%%%%%%%%%%%%%%%%%

%% Esto es de la clase exam. Si dejamos sin comentar \printanswers, se mostraran las soluciones. 
%% Si la comentamos y dejamos sin comentar \noprintanswers, pues no se muestran las soluciones.
%\printanswers
%\noprintanswers

%% Si se comenta no aparecerán los espacios de la solución.
%\nocancelspace

%%%%%%%%%%%%%%%%%%%%%%%%%%%%%%%%%%%%%%%%%%%%%%%%%%%%%%%%%%%%%%%%%%%%%%%%%%%%%%%%%%

%%%%%%% Datos de estudiante
\newcommand{\StudentData}{
	\ifprintanswers
	\else
		\raggedright
		\begin{tabular}{cc}
			\begin{minipage}{0.15\linewidth}
				\includegraphics[height=1.75cm]{./include/logo}
			\end{minipage} &
			\begin{minipage} {0.80\linewidth}
				Apellidos: \underline{\hspace{5cm}} \hspace{0.1cm} Nombre: \enspace{\hrulefill}
				\flushright Grupo: \underline{\hspace{0.5cm}} \hspace{0.1cm} Fecha: \underline{\hspace{2cm}}  \\
			\end{minipage}
		\end{tabular}
	\fi
}

\newcommand{\GradeTableHeader}{
	\ifprintanswers
		%Examen resulto
	\else
		%\StudentData

		%%%%%%%%%%%%%%%%%%%%%%%%%%%%%%%%%%%%%%%%%
		% Tabla para anotar la calificación
		%%%%%%%%%%%%%%%%%%%%%%%%%%%%%%%%%%%%%%%%%
		\begin{center}
			%\resizebox{\textwidth}{!}{\gradetable[h][questions]} % Esto es por si la tabla sale muy grande, para ajustarla al ancho
			\gradetable[h][questions]
		\end{center}
		\vspace{0.1in} % Espacio vertical
	\fi
}

%%%%%%%%%%%%%%%%%%%%%%%%%%%%%%%%%%%%%%%%%%%%%%%%%%%%%%%%%%%%%%%%%%%%%%%%%%%%%%%%%%
%%%% Cosas a configurar de la clase EXAM %%%%

\pagestyle{headandfoot}
\runningheadrule
\extraheadheight{1cm}

\firstpagefooter{}{}{\small Página \thepage\ de \numpages}
\runningfooter{}{}{\small Página \thepage\ de \numpages}

\renewcommand*\half{\!\hspace{0.6mm},5}
\pointpoints{punto}{puntos}
\bonuspointpoints{punto extra}{puntos extra}
\hqword{Pregunta}
\hpword{Puntos}
\hsword{Calificación}
\renewcommand{\solutiontitle}{\noindent\textbf{Solución:}\par\noindent}
\pointformat{(\emph{\thepoints})}
\bonuspointformat{(\emph{\thepoints})}
%\pointsinrightmargin % Para poner las puntuaciones a la derecha. Se puede cambiar. Si se comenta, sale a la izquierda.   

%\extrawidth{-1.8cm} %Un poquito más de margen por si ponemos textos largos.
\marginpointname{ \emph{\points}}
%\bracketedpoints

\headrule
\footrule

\firstpageheader
{\curso  \\ \small \colegio \\ \small \anoCurso}
{}
{\numeroHoja \\ \nombreHoja \\}

\runningheader
{\curso{}  \\ \small \colegio \\ \small \anoCurso}
{}
{\numeroHoja{} \\ \nombreHoja{} \\}