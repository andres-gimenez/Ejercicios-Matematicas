%%%%%%%%%%%%%%%%%%%%%%%%%%%

\newcommand{\numeroHoja} { Hoja 2.1 }
\newcommand{\nombreHoja} { Repaso }
%%%%%%%%%%%%%%%%%%%%%%%%%%%

% Configuración del documento.
\renewcommand{\schoolSubject} { Examen Matemáticas 2º ESO  }
\renewcommand{\school} { IES José de Churriguera  }
\renewcommand{\academicPeriod} { Curso 2022/2023 }

\renewcommand{\autor} { Andrés Giménez Muñoz }
\renewcommand{\emailAuthor} { andresprofemates@outlook.es }
\renewcommand{\autorSing}{ Profesor: Andrés } 
\newcommand{\schoolSubject} { Matemáticas 3º ESO - Recuperación}
\newcommand{\school} { IES La Serna }
\newcommand{\academicPeriod} { Curso 2020/2021 }


\newcommand{\autor} { Andrés Giménez Muñoz }
\newcommand{\emailAuthor} { agimenezmunoz@ieslaserna.com }
\newcommand{\autorSing}{ Profesores: Andrés } 

%%%%%%%%%%%%%%%%%%%%%%%%%%%
% Exam configuration
\pointsdroppedatright   %% No mostrar la puntuación

%% Si se comenta no aparecerán los espacios de la solución.
%\nocancelspace

%% Esto es de la clase exam. Si dejamos sin comentar \printanswers, se mostraran las soluciones. 
%% Si la comentamos y dejamos sin comentar \noprintanswers, pues no se muestran las soluciones.
%\printanswers
%\noprintanswers

%%%%%%%%%%%%%%%%%%%%%%%%%%%

%%%%%%%%%%%%%%%%%%%%%%%%%%%

\begin{document}

    %%\StudentData

    \WithoutCalculator

    \begin{questions}
        \question
        Resuelve las siguientes operaciones escribiendo el proceso el proceso de resolución paso a paso:
        \begin{parts}
            \part
                $\frac{3}{4} - \frac{1}{3} - \frac{2}{12} + \frac{5}{6}$
                \begin{solution}
                    $m.c.m(3, 4,6,12) = 2^2 \cdot 3$ \\ \\
                    $\frac{3}{4} - \frac{1}{3} - \frac{2}{12} + \frac{5}{6} = \frac{9}{12} - \frac{4}{12} - \frac{2}{12} + \frac{10}{12}$ 
                \end{solution}
            \part
                $\left( 4 + \frac{3}{4}\right) - \left(3 + \frac{2}{3} \right)$
                \begin{solution}
                    $m.c.m(3, 4) = 12$ \\ \\
                    $\left( 4 + \frac{3}{4}\right) - \left(3 + \frac{2}{3} \right) = \left( \frac{48}{12} + \frac{9}{12} \right) - \left( \frac{36}{12} + \frac{8}{12} \right) = \frac{57}{12} + \frac{44}{12} = \frac{13}{12}$        
                \end{solution}
            \part
                $\frac{5}{6} \cdot \frac{2}{3} $
            \part
                $\frac{2}{15} : \frac{2}{3} $
            \part
                $\left( \frac{7}{5} - \frac{1}{2} \right) : \left( 1 - \frac{3}{10} \right) $
            \part
                $\frac{5}{8} \cdot \left[ \frac{17}{4} - 3 \cdot \left( 2 -  \frac{2}{3} \right) \right] $
        \end{parts}

        \question
        Halla la fracción irreducible de cada una de estas fracciones:
        \begin{parts}
            \part
                $\frac{250}{70}$
            \part
                $\frac{360}{400}$
        \end{parts}

        \question
        Ordena de menor a mayor las siguientes fracciones reduciéndolas previamente a común denominador:

            $\frac{30}{40}$, $\frac{70}{90}$, $\frac{50}{120}$, $\frac{50}{180}$

        % \question
        % Simplifica las siguientes potencias
        % \begin{parts}
        %     \part
        %         $3^3 \cdot 3^4 \cdot 3$
        %     \part
        %         $5^7 : 5^3$
        %     \part
        %         $\left( 5^3 \right)^4$
        %     \part
        %         $\left( 5 \cdot 2 \cdot 3 \right)^4$
        %     \part
        %         $\left( 3^4 \right)^4$
        %     \part
        %         $\left[ \left( 5^3 \right)^4 \right] ^2$
        %     \part
        %         $\left( 8^2 \right)^3$
        %     \part
        %         $\left( 9^3 \right)^2$
        %     \part
        %         $2^5 \cdot 2^4 \cdot 2$
        %     \part
        %         $2^7 : 2^6$
        %     \part
        %         $\left( 2^2 \right)^4$
        %     \part
        %         $\left( 4 \cdot 2 \cdot 3 \right)^4$
        %     \part
        %         $\left( 2^5 \right)^4$
        %     \part
        %         $\left[\left( 2^3 \right)^4 \right]^0$
        %     \part
        %         $\left( 27 ^2 \right)^5$
        %     \part
        %     $\left( 4^3 \right)^2$
        % \end{parts}
    \end{questions}
\end{document}