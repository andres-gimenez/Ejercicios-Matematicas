
%%%%%%%%%%%%%%%%%%%%%%%%%%%

\newcommand{\numeroHoja} { Hoja 3.1 }
\newcommand{\nombreHoja} { Raíces de polinomios }
%%%%%%%%%%%%%%%%%%%%%%%%%%%

% Configuración del documento.
\renewcommand{\schoolSubject} { Examen Matemáticas 2º ESO  }
\renewcommand{\school} { IES José de Churriguera  }
\renewcommand{\academicPeriod} { Curso 2022/2023 }

\renewcommand{\autor} { Andrés Giménez Muñoz }
\renewcommand{\emailAuthor} { andresprofemates@outlook.es }
\renewcommand{\autorSing}{ Profesor: Andrés } 
\newcommand{\schoolSubject} { Matemáticas 3º ESO - Recuperación}
\newcommand{\school} { IES La Serna }
\newcommand{\academicPeriod} { Curso 2020/2021 }


\newcommand{\autor} { Andrés Giménez Muñoz }
\newcommand{\emailAuthor} { agimenezmunoz@ieslaserna.com }
\newcommand{\autorSing}{ Profesores: Andrés } 

%%%%%%%%%%%%%%%%%%%%%%%%%%%
% Exam configuration
\pointsdroppedatright   %% No mostrar la puntuación

%% Si se comenta no aparecerán los espacios de la solución.
%\nocancelspace

%% Esto es de la clase exam. Si dejamos sin comentar \printanswers, se mostraran las soluciones. 
%% Si la comentamos y dejamos sin comentar \noprintanswers, pues no se muestran las soluciones.
\printanswers
%\noprintanswers

%%%%%%%%%%%%%%%%%%%%%%%%%%%

\begin{document}

    \begin{questions}
        \question
        Comprueba si son raíces de los polinomios dados los valores que se indican. \\
        \scriptsize{*{Recuerda que una raíz de un polinomio son los valores que lo hacen cero.}}
        \normalsize
        \begin{parts}
            \part
            % Copiada del libro.
            $P(x)=x^2-2x \quad\mathit{para}\quad x=2$
            \begin{sample} Evaluamos el polinomio en $x=2$. \\
               $P(2)=2^2-2 \cdot 2 = 4 - 4 = \boxed{0}$ \hspace{4mm} Luego, es una raíz.
            \end{sample}
            \part
            % Copiada del libro.
            $P(x)=x^2-4x \quad\mathit{para}\quad x=2$
            \begin{solution}
                $P(2)= 2^2-4\cdot2 = 4-8 = \boxed{-4}$ \hspace{4mm} No es una raíz.
            \end{solution}
            \part
            % Copiada del libro.
            $P(x)=3x^3+8x^2-5x-6  \quad\mathit{para}\quad  x=-3$
            \begin{solution}
                \begin{multline*}
                   P(-3)= 3(-3)^3+8(-3)^2-5(-3)-6 = 3 (-27) + 8 \cdot 9 + 15 - 6 = \\ 
                    = -81 + 72 + 15 - 6 = \boxed{0} 
                \end{multline*}    
                \hspace{4mm} Sí es una raíz.
            \end{solution}
            \part
            % Copiada del libro.
            $P(x)=x^5-x^3-2x+2 \quad\mathit{para}\quad x=1$
            \begin{solution}
                $P(2)= 1^5-1^3-2\cdot1+2 = 1-1-2+2 = \boxed{0}$ \hspace{4mm} Sí una raíz.
            \end{solution}
        \end{parts}

        \question
        Sin realizar la división, calcula el valor del resto.  \\
        \scriptsize{*{Utiliza el teorema del resto.}}
        \normalsize
        \begin{parts}
            \part
                % Copiada del libro.
                $(3x^2-4x+3):(x-1)$
                \begin{sample}[Teorema del resto]
                    El teorema del resto nos dice que al evaluar el polinomio en 1, obtenemos el resto. \\* \\*
                    $P(1)= 3 \cdot 1^2 - 4 \cdot 1 + 3 = \boxed{2}$ \\* \\*
                    Comprueba que el resto de la división es 2.
                 \end{sample}
            \part
                % Copiada del libro.
                $(2x^3-x^2-2x+8):(x+1)$
                \begin{solution}
                    Utilizando el teorema del resto, evaluamos el polinomio \\* \\*
                    $P(x)=2x^3-x^2-2x+8$ en $x=-1$ \\* \\*
                    $P(-1)= 2(-1)^3-(-1)^2-2(-1) +8 = -2 -1+2 + 8 = \boxed{7} $
                \end{solution}
            \part
                % Copiada del libro.
                $(x^4-3x^3-x+10):(x-2)$
                \begin{solution}
                    Utilizando el teorema del resto, evaluamos el polinomio \\* \\*
                    $P(x)=x^4-3x^3-x+10$ en $x=2$ \\* \\*
                    $P(2)= (2)^4-3(2)^3-2+10 = 16 - 3\cdot 8 - 2 + 10 = 16 - 24 - 2 + 10  = \boxed{0} $
                \end{solution}
        \end{parts}
    \end{questions}
\end{document}