%%%%%%%%%%%%%%%%%%%%%%%%%%%
\newcommand{\numeroHoja} { 2º Evaluación }
\newcommand{\nombreHoja} { Potencias y raíces } 
%%%%%%%%%%%%%%%%%%%%%%%%%%%

% Configuración del documento.
\renewcommand{\schoolSubject} { Examen Matemáticas 2º ESO  }
\renewcommand{\school} { IES José de Churriguera  }
\renewcommand{\academicPeriod} { Curso 2022/2023 }

\renewcommand{\autor} { Andrés Giménez Muñoz }
\renewcommand{\emailAuthor} { andresprofemates@outlook.es }
\renewcommand{\autorSing}{ Profesor: Andrés } 
\newcommand{\schoolSubject} { Matemáticas 3º ESO - Recuperación}
\newcommand{\school} { IES La Serna }
\newcommand{\academicPeriod} { Curso 2020/2021 }


\newcommand{\autor} { Andrés Giménez Muñoz }
\newcommand{\emailAuthor} { agimenezmunoz@ieslaserna.com }
\newcommand{\autorSing}{ Profesores: Andrés } 

%%%%%%%%%%%%%%%%%%%%%%%%%%%
% Exam configuration
%\pointsdroppedatright   %% No mostrar la puntuación
\pointsinrightmargin % Para poner las puntuaciones a la derecha. Se puede cambiar. Si se comenta, sale a la izquierda.
\extrawidth{-1.5cm} %Un poquito más de margen por si ponemos textos largos.
\marginpointname{ \emph{\points}}

%% Si se comenta no aparecerán los espacios de la solución.
%\nocancelspace

%% Esto es de la clase exam. Si dejamos sin comentar \printanswers, se mostraran las soluciones. 
%% Si la comentamos y dejamos sin comentar \noprintanswers, pues no se muestran las soluciones.
%\printanswers
%\noprintanswers

%%%%%%%%%%%%%%%%%%%%%%%%%%%

% \usepackage{tikz}
% \usetikzlibrary{arrows}

\begin{document}

	\StudentData
	\GradeTableHeader
	
	\justifying

	\begin{center}
		\fbox{\fbox{\parbox{6.5in}{\centering
		\begin{multicols}{4}
			\begin{align*}
				4 &= 2^2 \\
				8 &= 2^3 \\
				16 &= 2^4 \\
				32 &= 2^5 \\
				64 &= 2^6
			\end{align*}
			\begin{align*}
				128 &= 2^7 \\
				256 &= 2^8 \\
				512 &= 2^9 \\
				1024 &= 2^{10}
			\end{align*}
			\begin{align*}
				9 &= 3^2 \\
				25 &= 5^2 \\
				75 &= 3 \cdot 5^2 \\
				168 &= 2^3 \cdot 3 \cdot 7  \\
				169 &= 13^2	
			\end{align*}
			\begin{align*}
				216 &= 2^3 \cdot 3^3 \\
				1080 &= 2^4 \cdot 3^3 \cdot 5 \\
				100 &= 10^2 \\
				10.000 &= 10^4 \\
				0,00001 &= 10^{-5}
			\end{align*}
		\end{multicols}
		}}}

	\end{center}

	\begin{questions}
		\setcounter{question}{0}
		\question[1]
		Expresa como una única potencia.
		\begin{parts}
		 \part
			 $5^3 \cdot 5^8 \cdot 5^2 = $
			%  \vspace{\stretch{1}}
		 \part
			 $25^3 : 5^3 = $
			%  \vspace{\stretch{1}}
		 \part
			 $12^5 : 4^5 : 3^5 = $
			%  \vspace{\stretch{1}}
		\end{parts}

		\question[2]
        Simplifica
        \begin{parts}
            \part
				$\frac{(4^2 + 3^2) \cdot 2^6}{(5-1)^3 \cdot (4+1)} = $
				\vspace{\stretch{5}}
            \part
				$\left(\frac{3}{4} \right)^4 \cdot \left(\frac{9}{25}\right)^{-2} = $
				\vspace{\stretch{5}}
			\part
				$\left(\frac{5}{2} + \frac{3}{4} \right)^2 - \left(\frac{1}{4}\right)^2 = $
				\vspace{\stretch{5}}
        \end{parts}

		\newpage

		\question[1]
        Expresa como potencia de exponente fraccionario si es posible
		\begin{parts}
			\begin{multicols}{3}
            \part
				$\sqrt[3]{7} = $
				% \vspace{\stretch{1}}
            \part
				$\sqrt[5]{-7^2} = $
				% \vspace{\stretch{1}}
			\part
				$\sqrt[5]{\frac{2}{3}} = $
				% \vspace{\stretch{1}}
			\end{multicols}
		\end{parts}

        \question[1]
        Escribe las siguientes potencias de exponente fraccionario como radicales.
		\begin{parts}
			\begin{multicols}{3}
            \part
				$11^{\frac{3}{5}} = $
				% \vspace{\stretch{1}}
            \part
				$7^{- \frac{5}{3}} = $
				% \vspace{\stretch{1}}
			\part
				$\left(\frac{3}{2}\right)^{\frac{3}{2}} = $
				% \vspace{\stretch{1}}
			\end{multicols}
        \end{parts}

		\question[3]
		Simplifica los siguientes radicales, extrayendo fuera lo que se pueda.
		\begin{parts}
			\begin{multicols}{2}
			\part
				$\sqrt{12} = $ \\
            \part
				$\sqrt{75} = $ \\
            \part
				$\sqrt[3]{-8} = $  \\				
            \part
				$\sqrt[3]{64} = $ \\
	        \part
				$\sqrt{128} = $ \\
	        \part
				$\sqrt{-1} = $ \\
	        \part
				$\sqrt{1} = $ \\
			\part
				$1^{23} = $ \\
            \part
				$\sqrt[4]{0} = $ \\
            \part
				$\sqrt[5]{-1024} = $ \\
            \part
				$\sqrt[3]{\frac{1}{216}} = $ \\
            \part
				$\sqrt[4]{10.000} = $ \\
			\part
				$\sqrt{1080} = $ \\
            \part
				$\sqrt[5]{0,00032} = $ \\
			\end{multicols}
		\end{parts}

		\question[2]
		Un terreno cuadrado tiene una superficie de $900 m^2$. ¿Cuántos metros lineales de alambre se necesitan para cercarlo?
		\vspace{\stretch{1}}
		
	\end{questions}
\end{document}