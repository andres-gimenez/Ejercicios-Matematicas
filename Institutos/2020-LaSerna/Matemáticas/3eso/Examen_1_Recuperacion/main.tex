\documentclass[addpoints,spanish, 12pt,a4paper,cancelspace]{./include/gexam}

%%%%%%%%%%%%%%%%%%%%%%%%%%%
\renewcommand{\numeroHoja} { 1º Evaluación }
\renewcommand{\nombreHoja} { Recuperación } 
%%%%%%%%%%%%%%%%%%%%%%%%%%%

% Configuración del documento.
\renewcommand{\schoolSubject} { Examen Matemáticas 2º ESO  }
\renewcommand{\school} { IES José de Churriguera  }
\renewcommand{\academicPeriod} { Curso 2022/2023 }

\renewcommand{\autor} { Andrés Giménez Muñoz }
\renewcommand{\emailAuthor} { andresprofemates@outlook.es }
\renewcommand{\autorSing}{ Profesor: Andrés } 

%%%%%%%%%%%%%%%%%%%%%%%%%%%
% Exam configuration
%\pointsdroppedatright   %% No mostrar la puntuación
\pointsinrightmargin % Para poner las puntuaciones a la derecha. Se puede cambiar. Si se comenta, sale a la izquierda.
\extrawidth{-1.5cm} %Un poquito más de margen por si ponemos textos largos.
\marginpointname{ \emph{\points}}

%% Si se comenta no aparecerán los espacios de la solución.
%\nocancelspace

%% Esto es de la clase exam. Si dejamos sin comentar \printanswers, se mostraran las soluciones. 
%% Si la comentamos y dejamos sin comentar \noprintanswers, pues no se muestran las soluciones.
%\printanswers
%\noprintanswers

%%%%%%%%%%%%%%%%%%%%%%%%%%%

\usepackage{tikz}
\usetikzlibrary{arrows}
% \usetikzlibrary{positioning,decorations.pathmorphing}
% \newcommand{\Interval}[4]{%
%     \tikz{%
%         \coordinate [label={center:$#1$},label=below:$\rule{0pt}{.35cm}#2$] (a) at (0,0); 
%         \coordinate [label={center:$#3$},label=below:$\rule{0pt}{.35cm}#4$] (b) at (1.6,0); 
%         \draw[-{latex},] decorate {(a)--(b)} (-.4,0)--(2,0);
%     }
% }
% \newcommand{\insertinterval}[1]{\rule{0in}{.7cm}\parbox{2.5cm}{#1}}

% \usepackage{pst-plot}
% \newcommand{\Intervala}{
% 	\begin{pspicture}(-3,-.5)(3,.5)
% 		\psaxes[yAxis=false,ticksize=0 -4pt]{<->}(0,0)(-3,-1)(3,1)
% 		\psline[linecolor=blue,arrowscale=1.25]{o-*}(.5,0)(2,0)
% 	\end{pspicture}
% }

\begin{document}

	\StudentData
	\GradeTableHeader
	
	\justifying

	\begin{questions}
		\setcounter{question}{0}
		\question[2]
		Realiza las siguientes operaciones y expresa el resultado en forma de fracción irreducible:
		\begin{parts}
			\part
				$\frac{3}{5} \cdot \frac{1}{2} - \frac{5}{6} : \frac{1}{3}$
				\vspace{\stretch{1}}
			\part
				$\frac{3}{5} \cdot \left(\frac{1}{2} - \frac{5}{6} : \frac{1}{3} \right)$		
				\vspace{\stretch{1}}
			\part
				$\left( 4 + \frac{3}{4}\right) \cdot \left(3 + \frac{2}{3} \right)$
				\vspace{\stretch{1}}
			\part
				$\left( \frac{7}{5} - \frac{1}{2} \right) : \left( 1 - \frac{3}{10} \right) $
				\vspace{\stretch{1}}
		\end{parts}

		\question[2]
		Realiza las siguientes operaciones y expresa el resultado en forma de fracción irreducible:
		\begin{parts}
			\part
			$\left(\frac{4}{7} \right)^{-5} : \left(\frac{4}{7} \right)^{4}$
			\vspace{\stretch{1}}
			\part
			$\left(\frac{3^2 \cdot 3^{-5}}{3^{-2}} \right)$
			\vspace{\stretch{1}}
		\end{parts}
		\vspace{\stretch{1}}

		\newpage

		\question[2]
		Al mediodía me he comido la mitad de una tortilla de patatas. A la hora de la merienda, Ana ha tomado un tercio de la tortilla original, y para cenar, 
		Luis se ha tomado tres cuartas partes de lo que quedaba. ¿Qué proporción de la tortilla queda al final del día? 
		Representa con dibujos cada paso a seguir.
		\vspace{\stretch{1}}

	
		\question[2]
		En una fiesta de Navidad se sortean 3 regalos por clase, si en tu clase hay 29 alumnos. 
		¿Qué probabilidad tienes de que te toque un regalo?
		\vspace{\stretch{1}}

		\question[2]
		En un club deportivo juvenil admiten socios con edades entre 12 y 18 años. La distribución de las edades es:
		\begin{table}[h!]
			\centering
			\begin{tabular}{|c|c|c|c|c|c|c|c|}
			\hline
			\cellcolor[gray]{0.8}Edad & 12 & 13 & 14 & 15 & 16 & 17 & 18 \\
			\hline
			\cellcolor[gray]{0.8}$f_i$ & 4 & 6 & 12 & 16 & 14 & 8 & 4 \\
			\hline
			\end{tabular}
		\end{table}
		\\
		Calcula la moda, la media aritmética y la mediana.
		\vspace{\stretch{1}}
	\end{questions}
\end{document}