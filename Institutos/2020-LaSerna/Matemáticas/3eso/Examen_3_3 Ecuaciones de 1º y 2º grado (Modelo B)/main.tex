% Configuración del documento.
\renewcommand{\schoolSubject} { Examen Matemáticas 2º ESO  }
\renewcommand{\school} { IES José de Churriguera  }
\renewcommand{\academicPeriod} { Curso 2022/2023 }

\renewcommand{\autor} { Andrés Giménez Muñoz }
\renewcommand{\emailAuthor} { andresprofemates@outlook.es }
\renewcommand{\autorSing}{ Profesor: Andrés } 
\newcommand{\schoolSubject} { Matemáticas 3º ESO - Recuperación}
\newcommand{\school} { IES La Serna }
\newcommand{\academicPeriod} { Curso 2020/2021 }


\newcommand{\autor} { Andrés Giménez Muñoz }
\newcommand{\emailAuthor} { agimenezmunoz@ieslaserna.com }
\newcommand{\autorSing}{ Profesores: Andrés }

%%%%%%%%%%%%%%%%%%%%%%%%%%%
\newcommand{\numeroHoja} { 3º Evaluación }
\newcommand{\nombreHoja} { Ecuaciones } 
\renewcommand{\waterMark} { Modelo: B } 
%%%%%%%%%%%%%%%%%%%%%%%%%%%

%%%%%%%%%%%%%%%%%%%%%%%%%%%
% Exam configuration
%\pointsdroppedatright   %% No mostrar la puntuación
\pointsinrightmargin % Para poner las puntuaciones a la derecha. Se puede cambiar. Si se comenta, sale a la izquierda.
\extrawidth{-1.5cm} %Un poquito más de margen por si ponemos textos largos.
\marginpointname{ \emph{\points}}

%% Si se comenta no aparecerán los espacios de la solución.
%\nocancelspace

%% Esto es de la clase exam. Si dejamos sin comentar \printanswers, se mostraran las soluciones. 
%% Si la comentamos y dejamos sin comentar \noprintanswers, pues no se muestran las soluciones.
%\printanswers
%\noprintanswers

%%%%%%%%%%%%%%%%%%%%%%%%%%%

% \usepackage{tikz}
% \usetikzlibrary{arrows}

\begin{document}

	\StudentData
	\GradeTableHeader

    \justifying

	\begin{questions}
		\setcounter{question}{0}

        \question[1]
        Comprueba, sin resolver, si la solución de cada ecuación es correcta.
        \begin{parts}
            \part
            $2x^2+5x-3 = 0$, $\boxed{x=\frac{1}{2}}$ y $\boxed{x=-3}$
            \vspace{\stretch{1}}

            \part
            $2x^2+\frac{7}{2}x+\frac{3}{2} = 0$, $\boxed{x=-1}$ y $\boxed{x=-\frac{3}{4}}$
            \vspace{\stretch{1}}
		\end{parts}

        \question[2]
		Resuelve las siguientes ecuaciones.
        \begin{parts}
            \part
            $2x+\frac{3}{2}x+8=\frac{3x}{5}-1$
            \vspace{\stretch{2}}

            \part
            $\frac{x-2}{5}-\frac{3-2x}{5}=6-\frac{5x}{5}$
            \vspace{\stretch{2}}
        \end{parts}

        \newpage
        \question[2]
        Resuelve las siguientes ecuaciones de 2º grado.
        \begin{parts}
            \part
            % Solución x=-2;x=7
            $x^2 = 5 x + 14$
            \vspace{\stretch{1}}

            \part
            % Solución x=7;x=1
            $(x-4)^2-9 =0$
            \vspace{\stretch{1}}
	    \end{parts}

        \newpage
        \question[3]
		Resuelve las siguientes ecuaciones utilizando la regla de Ruffini.
        \begin{parts}
            \part
            % Solucion x = -1, 1, 2 
            $x^3-2 x^2-x+2=0$
            \vspace{\stretch{1}}

            \part
            % Solución: x = 1, 2, 2
            $ x^3 - 5 x^2 + 8 x - 4=0$
            \vspace{\stretch{1}}

            % \part
            % % Solución: x = 2, 2, 3 
            % $ x^3 - 7 x^2 + 16 x - 12=0$
            % % \vspace{\stretch{1}}
	    \end{parts}

        \newpage
        \question[2]
		Resuelve las siguientes ecuaciones.
        \begin{parts}
            \part
            $x^4-29x^2+100=0$
            \vspace{\stretch{1}}

            \part
            $(x+3)(x-2)(x-52)(x^2+7x-8)=0$
            \vspace{\stretch{2}}
	    \end{parts}
	\end{questions}
\end{document}