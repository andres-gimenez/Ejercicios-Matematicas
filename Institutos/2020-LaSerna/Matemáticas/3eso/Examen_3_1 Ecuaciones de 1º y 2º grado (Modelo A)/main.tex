% Configuración del documento.
\renewcommand{\schoolSubject} { Examen Matemáticas 2º ESO  }
\renewcommand{\school} { IES José de Churriguera  }
\renewcommand{\academicPeriod} { Curso 2022/2023 }

\renewcommand{\autor} { Andrés Giménez Muñoz }
\renewcommand{\emailAuthor} { andresprofemates@outlook.es }
\renewcommand{\autorSing}{ Profesor: Andrés } 
\newcommand{\schoolSubject} { Matemáticas 3º ESO - Recuperación}
\newcommand{\school} { IES La Serna }
\newcommand{\academicPeriod} { Curso 2020/2021 }


\newcommand{\autor} { Andrés Giménez Muñoz }
\newcommand{\emailAuthor} { agimenezmunoz@ieslaserna.com }
\newcommand{\autorSing}{ Profesores: Andrés }

%%%%%%%%%%%%%%%%%%%%%%%%%%%
\newcommand{\numeroHoja} { 3º Evaluación }
\newcommand{\nombreHoja} { Ecuaciones de 1º y 2º grado } 
\renewcommand{\waterMark} { Modelo: A } 
%%%%%%%%%%%%%%%%%%%%%%%%%%%

%%%%%%%%%%%%%%%%%%%%%%%%%%%
% Exam configuration
%\pointsdroppedatright   %% No mostrar la puntuación
\pointsinrightmargin % Para poner las puntuaciones a la derecha. Se puede cambiar. Si se comenta, sale a la izquierda.
\extrawidth{-1.5cm} %Un poquito más de margen por si ponemos textos largos.
\marginpointname{ \emph{\points}}

%% Si se comenta no aparecerán los espacios de la solución.
%\nocancelspace

%% Esto es de la clase exam. Si dejamos sin comentar \printanswers, se mostraran las soluciones. 
%% Si la comentamos y dejamos sin comentar \noprintanswers, pues no se muestran las soluciones.
%\printanswers
%\noprintanswers

%%%%%%%%%%%%%%%%%%%%%%%%%%%

% \usepackage{tikz}
% \usetikzlibrary{arrows}

\begin{document}

	\StudentData
	\GradeTableHeader

    \justifying

	\begin{questions}
		\setcounter{question}{0}

        \question[1]
        Comprueba, sin resolver, si la solución de cada ecuación es correcta.
        \begin{parts}
            \part
            $\frac{x-3}{2}-\frac{5x-10}{7} = \frac{-1}{2}$, $\boxed{x=2}$
            \vspace{\stretch{1}}
            
            \part
            $\frac{2x-3}{4}-\frac{x}{5} = -\frac{33}{20}$, $\boxed{x=-3}$
            \vspace{\stretch{1}}

            % \part
            % $2x^2-3x-2 = 0$, $x=-\frac{1}{2}$ y $x=2$
            % \vspace{\stretch{1}}

            \part
            $x^3-2x^2-x+2=0$, $\boxed{x=-1}$, $\boxed{x=1}$ y $\boxed{x=2} $
            \vspace{\stretch{1}}
		\end{parts}

        \newpage
        \question[2]
		Resuelve las siguientes ecuaciones de 1º grado.
        \begin{parts}
            \part
            $2x-3 = 5-3x$
            \vspace{\stretch{1}}

            \part
            $3x-2=x+5$
            \vspace{\stretch{1}}

            \part
            $7x-5=x+4-6x$
            \vspace{\stretch{1}}

            \part
            $6-x=9-2x$
            \vspace{\stretch{1}}

            \part
            $3(3x-5)+3(2x-6)=42$
            \vspace{\stretch{1}}

            \part
            $-(-2x-1)+(x+3)=x+16$
            \vspace{\stretch{1}}
	    \end{parts}

        \newpage
        \question[3]
		Resuelve las siguientes ecuaciones de 2º grado.
        \begin{parts}
            \part
            $2x^2-3x-2=0$
            \vspace{\stretch{1}}

            \part
            $3x^2-\frac{1}{2}x-188=0$
            \vspace{\stretch{1}}

            \part
            $3x^2-48=0$
            \vspace{\stretch{1}}

            \part
            $-x^2+7x=0$
            \vspace{\stretch{1}}

            \part
            $(2x-3)(x+5)-2x(x-3)=5$
            \vspace{\stretch{1}}

        \end{parts} 

        \newpage
        \question[2]
		Resuelve las siguientes ecuaciones.
        \begin{parts}
            \part
            $(5x-2x)(x+4)=6x-2x^2$
            \vspace{\stretch{1}}

            \part
            $\left(x-\frac{1}{2}\right)-3\left(x+1\right)=2\left(x+\frac{1}{4}\right)$
            \vspace{\stretch{1}}
            \part
            $x-\frac{6(1-x)+3(4-2x)}{5}=3\left(x+\frac{1}{6}\right)-2$
            \vspace{\stretch{1}}
        \end{parts} 

        \question[2]
        Indica si tiene solución y cuantas, las siguientes ecuaciones
        \begin{parts}
            \part
            $x+3-5x=-4x+7$
            \vspace{\stretch{1}}

            \part
            $3x-7=-4x+2-9+7x$
            \vspace{\stretch{1}}

            \part
            $(x+3)(x-5)=x^2-2x-15$
            \vspace{\stretch{1}}
        \end{parts} 

	\end{questions}
\end{document}