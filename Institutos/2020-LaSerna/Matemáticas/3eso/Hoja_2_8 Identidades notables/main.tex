%%%%%%%%%%%%%%%%%%%%%%%%%%%
\newcommand{\numeroHoja} { Hoja 2.8 }
\newcommand{\nombreHoja} { Identidades notables }
%%%%%%%%%%%%%%%%%%%%%%%%%%%

% Configuración del documento.
\renewcommand{\schoolSubject} { Examen Matemáticas 2º ESO  }
\renewcommand{\school} { IES José de Churriguera  }
\renewcommand{\academicPeriod} { Curso 2022/2023 }

\renewcommand{\autor} { Andrés Giménez Muñoz }
\renewcommand{\emailAuthor} { andresprofemates@outlook.es }
\renewcommand{\autorSing}{ Profesor: Andrés } 
\newcommand{\schoolSubject} { Matemáticas 3º ESO - Recuperación}
\newcommand{\school} { IES La Serna }
\newcommand{\academicPeriod} { Curso 2020/2021 }


\newcommand{\autor} { Andrés Giménez Muñoz }
\newcommand{\emailAuthor} { agimenezmunoz@ieslaserna.com }
\newcommand{\autorSing}{ Profesores: Andrés } 

%%%%%%%%%%%%%%%%%%%%%%%%%%%
% Exam configuration
\pointsdroppedatright   %% No mostrar la puntuación

%% Si se comenta no aparecerán los espacios de la solución.
%\nocancelspace

%% Esto es de la clase exam. Si dejamos sin comentar \printanswers, se mostraran las soluciones. 
%% Si la comentamos y dejamos sin comentar \noprintanswers, pues no se muestran las soluciones.
\printanswers
%\noprintanswers

%%%%%%%%%%%%%%%%%%%%%%%%%%%

\begin{document}

    %%\StudentData

    \begin{questions}
        \question
        Demuestra que el cuadrado de una suma es:
            \begin{equation*}
                (a+b)^2 = a^2 + b^2 + 2ab
            \end{equation*}
            \begin{solution}
                \begin{multline*}
                    (a+b)^2 = (a+b)(a+b) = a^2 +ab + ba + b^2 =  \cdots \\
                        \cdots = a^2 + b^2 + ab + ba = a^2 + b^2 + ab + ab = a^2 + b^2 + 2ab
                \end{multline*}
            \end{solution}
        \question
        Demuestra que el cuadrado de una resta es:
            \begin{equation*}
                (a-b)^2 = a^2 + b^2 - 2ab
            \end{equation*}
            \begin{solution}
                \begin{multline*}
                    (a-b)^2 = (a-b)(a-b) = a^2 - ab - ba + b^2 =  \cdots \\
                        \cdots = a^2 + b^2 - ab - ba = a^2 + b^2 - ab - ab = a^2 + b^2 - 2ab
                \end{multline*}
            \end{solution}
        \question
        Demuestra que suma por diferencia es la diferencia de cuadrados:
            \begin{equation*}
                (a+b)(a-b) = a^2-b^2 
            \end{equation*}
            \begin{solution}
                \begin{equation*}
                    (a+b)(a-b) = a^2 + ab - ba - b^2 = a^2 + ab - ab - b^2 = a^2 - b^2                    
                \end{equation*}
            \end{solution}
        \question
        Desarrolla las siguientes expresiones algebraicas utilizando las igualdades notables:
        \begin{parts}
            \part
                $(x^3 + 2x)^2$
                \begin{solution}
                    {\footnotesize (*) Usamos $(a+b)^2 = a^2 + b^2 + 2ab$}
                    \begin{equation*}
                        (x^3 + 2x)^2 = \left(x^3\right)^2 + (2x)^2 + 2(x^3)(2x) = x^6 + 4x^2 + 4x^4 = \boxed{x^6 + 4x^4 + 4x^2}
                    \end{equation*}
                \end{solution}
            \part
                $(2x^3-5)^2$
                \begin{solution}
                    {\footnotesize (*) Usamos $(a-b)^2 = a^2 + b^2 - 2ab$}
                    \begin{equation*}
                        (2x^3-5)^2 = \left(2x^3\right)^2 + 5^2 - 2(2x^3)5 = 4x^6 + 25 - 20x^3 = \boxed{4x^6 - 20x^3 + 25 }
                    \end{equation*}
                \end{solution}
            \part
                $(5x-6)(5x+6)$
                \begin{solution}
                    {\footnotesize (*) Usamos $(a+b)(a-b) = a^2-b^2$ }
                    \begin{equation*}
                        (5x-6)(5x+6) = \left(5x\right)^2 - \left(6\right)^2 = 5^2 \cdot x^2 - 36 = \boxed{25x^2 - 36 }
                    \end{equation*}
                \end{solution}
            \part
                $(xy+2x^2y^2)^2$
                \begin{solution}
                    {\footnotesize (*) Usamos $(a+b)^2 = a^2 + b^2 + 2ab$}
                    \begin{multline*}
                        (xy+2x^2y^2)^2 = \left(xy\right)^2 + \left(2x^2y^2\right)^2 - 2 \left(xy\right) \left(2x^2y^2\right)  =  \cdots \\
                        \cdots = x^2 y^2 + 4 x^4 y^4 - 4 x^3 y^3  = \boxed{ 4 x^4 y^4 - 4 x^3 y^3 + x^2 y^2 }
                    \end{multline*}
                \end{solution}
            \part
                $\left(\frac{2xy}{3}-\frac{y^2}{6}\right)\left(\frac{2xy}{3}-\frac{y^2}{6}\right)$
                \begin{solution}
                    {\footnotesize (*) Usamos $(a-b)^2 = a^2 + b^2 - 2ab$}
                    \begin{multline*}
                        \left(\frac{2xy}{3}-\frac{y^2}{6}\right)\left(\frac{2xy}{3}-\frac{y^2}{6}\right) = \left(\frac{2xy}{3}-\frac{y^2}{6}\right)^2 = \cdots \\
                        \cdots =
                        \left(\frac{2xy}{3}\right)^2 + \left(\frac{y^2}{6}\right)^2 - 2 \left(\frac{2xy}{3}\right) \left(\frac{y^2}{6}\right) =
                        \frac{4}{9} x^2 y^2 + \frac{1}{36} y^4 - \frac{2}{9} x y^3 = \cdots \\
                        \cdots = \boxed{ \frac{1}{36} y^4 - \frac{2}{9} x y^3 + \frac{4}{9} x^2 y^2}
                    \end{multline*}
                \end{solution}
            
                \newpage
            \part
                $\left(\frac{2xy}{3}+\frac{y^2}{6}\right)\left(\frac{2xy}{3}-\frac{y^2}{6}\right)$
                \begin{solution}
                    {\footnotesize (*) Usamos $(a+b)(a-b) = a^2-b^2$ }
                    \begin{equation*}
                        \left(\frac{2xy}{3}+\frac{y^2}{6}\right)\left(\frac{2xy}{3}-\frac{y^2}{6}\right) =
                        \left(\frac{2xy}{3}\right)^2 - \left(\frac{y^2}{6}\right)^2 = \boxed{\frac{4}{9} x^2 y ^2 - \frac{1}{36} y^4}
                    \end{equation*}
                \end{solution}
            \part
                $(2-5x^2y^2)^2$
                \begin{solution}
                    {\footnotesize (*) Usamos $(a-b)^2 = a^2 + b^2 - 2ab$}
                    \begin{multline*}
                        \left(2-5x^2y^2\right)^2 = 2^2 + \left({5x^2y^2}\right)^2 - 2 \cdot 2 \cdot \left(5x^2y^2\right) = \cdots \\
                        \cdots = 4 + 25x^4y^4 - 20 x^2y^2 = \boxed{25x^4y^4 - 20 x^2y^2 + 4}
                    \end{multline*}
                \end{solution}
            \part
                $2{\left(\frac{2}{3}x+\frac{xy}{4}\right)}^2$
                \begin{solution}
                    {\footnotesize (*) Usamos $(a+b)^2 = a^2 + b^2 + 2ab$}
                    \begin{multline*}
                        2{\left(\frac{2}{3}x+\frac{xy}{4}\right)}^2 = 2{\left(\left(\frac{2}{3}x\right)^2 + \left(\frac{xy}{4}\right)^2 + 2 \left(\frac{2}{3}x\right) \left(\frac{xy}{4}\right) \right) }  = \cdots \\
                        \cdots = 2{\left( \frac{4}{9}x^2 + \frac{x^2y^2}{16} + \frac{4}{3\cdot4} x^2y \right)} =
                        2{\left( \frac{4}{9}x^2 + \frac{1}{16} x^2y^2 + \frac{1}{3} x^2y \right)} = \cdots \\
                        \cdots = \left( \frac{8}{9}x^2 + \frac{2}{16} x^2y^2 + \frac{2}{3} x^2y \right) = \boxed{\left(\frac{1}{8} x^2y^2 + \frac{2}{3} x^2y + \frac{8}{9}x^2  \right)}
                    \end{multline*}
                \end{solution}
        \end{parts}

        \newpage

        \question
        Opera y simplifica la siguiente expresiónes:
        \begin{parts}
            \part
                $4{\left(3x-1\right)}^2 - 4{\left(3x+1\right)}^2 + 2{\left(2x-1\right)}{\left(2x+1\right)}$
                \begin{solution}
                    \begin{multline*}
                        4{\left(3x-1\right)}^2 - 4{\left(3x+1\right)}^2 + 2{\left(2x-1\right)}{\left(2x+1\right)} =  \cdots \\
                        \cdots = 4{\left(\left(3x\right)^2 +1^2 - 2 \cdot \left(3x\right) \cdot 1 \right)} - 4\left(\left(3x\right)^2 +1^2 + 2 \cdot \left(3x\right) \cdot 1\right) + 2 \left(\left(2x\right)^2 - 1^2\right) \cdots \\
                        \cdots = 4{\left(\left(\left(3x\right)^2 + 1 - 6x \right) -\left(\left(3x\right)^2 + 1 + 6x \right)  \right)} + 2 \left(4x^2 - 1\right) = \cdots \\
                        \cdots = 4{\left(\left(3x\right)^2 + 1 - 6x - \left(3x\right)^2 - 1 - 6x \right)} + 8x^2 - 2 = \cdots \\
                        \cdots = 4{\left(-12x\right) + 8x^2 -2} = -48x + 8x^2 -2 = \boxed{8x^2 -48x - 2}
                    \end{multline*}
                \end{solution}
            \part
                ${\left(5x-2\right)}^2 + {\left(3x + 2\right)}^2 + 2{\left(5x-2\right)}{\left(3x + 2\right)}$
                \begin{solution}
                    \begin{multline*}
                        {\left(5x-2\right)}^2 + {\left(3x + 2\right)}^2 + 2{\left(5x-2\right)}{\left(3x + 2\right)} =  \cdots \\
                        \cdots = \left(\left(5x-2\right) + \left(3x + 2\right)\right)^2 = \left(5x-2 + 3x + 2\right)^2 = \left(8x\right)^2 = \boxed{64x^2}
                    \end{multline*}
                \end{solution}
        \end{parts}
    \end{questions}
\end{document}