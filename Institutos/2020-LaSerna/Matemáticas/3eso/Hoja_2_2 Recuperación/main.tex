%%%%%%%%%%%%%%%%%%%%%%%%%%%

\newcommand{\numeroHoja} { Hoja 2.2 Recuperación }
\newcommand{\nombreHoja} { 2º Evaluación }
%%%%%%%%%%%%%%%%%%%%%%%%%%%

% Configuración del documento.
\renewcommand{\schoolSubject} { Examen Matemáticas 2º ESO  }
\renewcommand{\school} { IES José de Churriguera  }
\renewcommand{\academicPeriod} { Curso 2022/2023 }

\renewcommand{\autor} { Andrés Giménez Muñoz }
\renewcommand{\emailAuthor} { andresprofemates@outlook.es }
\renewcommand{\autorSing}{ Profesor: Andrés } 
\newcommand{\schoolSubject} { Matemáticas 3º ESO - Recuperación}
\newcommand{\school} { IES La Serna }
\newcommand{\academicPeriod} { Curso 2020/2021 }


\newcommand{\autor} { Andrés Giménez Muñoz }
\newcommand{\emailAuthor} { agimenezmunoz@ieslaserna.com }
\newcommand{\autorSing}{ Profesores: Andrés } 

%%%%%%%%%%%%%%%%%%%%%%%%%%%
% Exam configuration
\pointsdroppedatright   %% No mostrar la puntuación

%% Si se comenta no aparecerán los espacios de la solución.
%\nocancelspace

%% Esto es de la clase exam. Si dejamos sin comentar \printanswers, se mostraran las soluciones. 
%% Si la comentamos y dejamos sin comentar \noprintanswers, pues no se muestran las soluciones.
%\printanswers
%\noprintanswers

%%%%%%%%%%%%%%%%%%%%%%%%%%%

\begin{document}

    \begin{questions}
        \question
        Si $P(x)=x^5 + 2x^3 - 3 x^2 - 5$, evalua.
        \begin{parts}
            \part
                $P(-2)$
            \part
                $P(0)$
            \part
                $P\left(\frac{1}{3}\right)$
            \part
            $P\left(\frac{2}{5}\right)$
        \end{parts}
       
        \question
        Si $P(x)=x^3 - 2$, $Q(x)=2x^3-3x^2+x-1$, calcula.
        \begin{parts}
            \part
                $P(x) + Q(X)$
            \part
                $2 P(x) + 3 Q(x)$
        \end{parts}

        \question
        Si $P(x)=2x-3x^2$, $Q(x)=2x^2 - 3$, calcula.
        \begin{parts}
            \part
                $P(x) \cdot Q(X)$
            \part
                $2 P(x) \cdot 4 Q(x)$
        \end{parts}

        \question
        Realiza las siguientes divisiones de polinomios

        \begin{parts}
            \part
            $\left(\polynomial[reciprocal]{3,-4,5}\right) : \left( \polynomial[reciprocal]{1,-1} \right)$

            \part 
            $\left(\polynomial[reciprocal]{4,0,-10,3,-1,6}\right) : \left( \polynomial[reciprocal]{1,-2+1} \right)$

        \end{parts}

        \question
        Realiza las siguientes divisiones de polinomios por el método de Ruffini
        \begin{parts}
            \part
                $\left(\polynomial[reciprocal]{4,0,-10,3}\right) : \left( \polynomial[reciprocal]{1,-3} \right)$
            \part
                $\left(\polynomial[reciprocal]{1,0,0,0,0,-32}\right) : \left( \polynomial[reciprocal]{1,-2} \right)$
        \end{parts}
    \end{questions}
\end{document}