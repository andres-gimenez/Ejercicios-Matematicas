%%%%%%%%%%%%%%%%%%%%%%%%%%%
\newcommand{\documentName} { Examen 1ª evaluación }
\newcommand{\documentContent} { Sistema de numeración } 
\newcommand{\waterMark} { Modelo A } 
%%%%%%%%%%%%%%%%%%%%%%%%%%%

% Configuración del documento.
\newcommand{\schoolSubject} { Matemáticas 3º ESO - Recuperación}
\newcommand{\school} { IES La Serna }
\newcommand{\academicPeriod} { Curso 2020/2021 }


\newcommand{\autor} { Andrés Giménez Muñoz }
\newcommand{\emailAuthor} { agimenezmunoz@ieslaserna.com }
\newcommand{\autorSing}{ Profesores: Andrés } 
\renewcommand{\schoolSubject} { Examen Matemáticas 2º ESO  }
\renewcommand{\school} { IES José de Churriguera  }
\renewcommand{\academicPeriod} { Curso 2022/2023 }

\renewcommand{\autor} { Andrés Giménez Muñoz }
\renewcommand{\emailAuthor} { andresprofemates@outlook.es }
\renewcommand{\autorSing}{ Profesor: Andrés } 

%%%%%%%%%%%%%%%%%%%%%%%%%%%
% Exam configuration
%\pointsdroppedatright   %% No mostrar la puntuación
\pointsinrightmargin % Para poner las puntuaciones a la derecha. Se puede cambiar. Si se comenta, sale a la izquierda.
\extrawidth{-1.5cm} %Un poquito más de margen por si ponemos textos largos.
\marginpointname{ \emph{\points}}

%% Si se comenta no aparecerán los espacios de la solución.
%\nocancelspace

%% Esto es de la clase exam. Si dejamos sin comentar \printanswers, se mostraran las soluciones. 
%% Si la comentamos y dejamos sin comentar \noprintanswers, pues no se muestran las soluciones.
%\printanswers
%\noprintanswers

%%%%%%%%%%%%%%%%%%%%%%%%%%%

\begin{document}

\StudentData
\GradeTableHeader

\justifying

\begin{center}
    \fbox{\fbox{\parbox{6.5in}{
                Aunque la motivación de estudiar distintos sistemas de numeración dentro de una asignatura de redes informáticas
                es entender el funcionamiento interno de los protocolos de red, véase por ejemplo tramas Ethernet o direccionamiento IP.
                Recuerda que te encuentras ante un examen de matemáticas, destinado a evaluar tus conocimientos teóricos, para que puedas
                aplicarlos más adelante en problemas prácticos.
                \vspace{0.2cm}
                \par
                A la hora de resolver los apartados de los problemas ten en cuenta los siguientes puntos:
                \begin{itemize}
                    \item No se permite la utilización de ayudas externas, como ordenador, móvil, calculadora, ...
                    \item Cualquier desarrollo adicional que necesites realizar, utiliza el margen de la página o la última hoja.
                    \item Justifica todos los problemas, por muy simples que sean, un resultado sin su desarrollo o explicación de como se ha obtenido no se considerara resuelto.
                          En la mayoría de los casos se pueden explicar en una línea.
                    \item A la hora de evaluar cada ejercicio se tendrá más en cuenta la comprensión del contenido teórico que la obtención del resultado correcto.
                          No obstante, para obtener la máxima puntuación de cada ejercicio tendrán que ser ambos correctos.
                    \item En algunos ejercicios se pueden utilizar el resultado de ejercicios anteriores para su resolución.
                          Si utilizas resultados de otro apartado o desarrollos en la última página, pon una referencia a ellos.
                \end{itemize}

                \par
                Para la resolución de los problemas puedes utilizar la siguiente ayuda:
                \begin{multicols}{4}
                    \begin{align*}
                        0,5  & = 2^{-1} \\
                        0,25  & = 2^{-2} \\
                        0,125 & = 2^{-3} \\
                        0,0625 & = 2^{-4} \\
                        0,03125 & = 2^{-5}
                    \end{align*}
                    \begin{align*}
                        4  & = 2^2 \\
                        8  & = 2^3 \\
                        16 & = 2^4 \\
                        32 & = 2^5 \\
                        64 & = 2^6
                    \end{align*}
                    \begin{align*}
                        128  & = 2^7    \\
                        256  & = 2^8    \\
                        512  & = 2^9    \\
                        1024 & = 2^{10} \\
                        2048 & = 2^{11}
                    \end{align*}
                    \begin{align*}
                        16^2 & = 256        \\
                        16^3 & = 4096       \\
                        16^4 & = 65.536     \\
                        16^5 & = 1.048.576  \\
                        16^6 & = 16.777.216
                    \end{align*}
                \end{multicols}
            }
        }}
\end{center}

\newpage

\begin{questions}
    % \setcounter{question}{0}

    \question[2]
    Convertir los siguientes números en notación binarios a sus equivalentes decimal:
    \begin{parts}
        \part
        $(1100)_{2}$
        \vspace{\stretch{1}}
        \part
        $(11100000)_{2}$
        \vspace{\stretch{3}}
        \part
        $(11100001)_{2}$
        \vspace{\stretch{1}}
        \part
        $(11101100)_{2}$
        \vspace{\stretch{1}}
        \part
        $(111101100)_{2}$
        \vspace{\stretch{1}}
        \part
        $(11111111)_{2}$
        \vspace{\stretch{1}}
        \part
        $(100000000)_{2}$
        \vspace{\stretch{1}}
        \part
        % 492,375 
        $(111101100,011)_{2}$
        \vspace{\stretch{3}}
    \end{parts}

    \newpage

    \question[2]
    Convertir los siguientes números en notación decimal a sus equivalentes binaria:
    \begin{parts}
        \part
        $(0)_{10}$
        \vspace{\stretch{1}}
        \part
        $(255)_{10}$
        \vspace{\stretch{1}}
        \part
        $(1100)_{10}$
        \vspace{\stretch{2}}
        \part
        $(2552)_{10}$
        \vspace{\stretch{2}}
        \part
        $(64)_{10}$
        \vspace{\stretch{1}}
        \part
        $(16)_{10}$
        \vspace{\stretch{1}}
        \part
        % 1101001,110
        $(105,75)_{10}$
        \vspace{\stretch{3}}
    \end{parts}

    \newpage

    \question[2]
    Convertir los siguientes números en notación hexadecimal a sus equivalentes binario:
    \begin{parts}
        \part
        $(C)_{16}$
        \vspace{\stretch{1}}
        \part
        $(H)_{16}$
        \vspace{\stretch{1}}
        \part
        $(102)_{16}$
        \vspace{\stretch{1}}
        \part
        $(2010)_{16}$
        \vspace{\stretch{1}}
        \part
        $(A2)_{16}$
        \vspace{\stretch{1}}
        \part
        $(B3)_{16}$
        \vspace{\stretch{1}}
        \part
        $(10000)_{16}$
        \vspace{\stretch{1}}
        \part
        $(A0007)_{16}$
        \vspace{\stretch{1}}
    \end{parts}

    \newpage

    \question[2]
    Calcula las siguientes cantidades binarias en su representación octal y hexadecimal.

    \begin{parts}
        \part
        0000 0110 0000 0111
        \vspace{\stretch{1}}
        \part
        0000 1101 0000 0110
        \vspace{\stretch{1}}
        \part
        1001 0000 0000 1010
        \vspace{\stretch{1}}
        \part
        1100 1111 0011 0000 1101 0000 0111 0101
        \vspace{\stretch{1}}
    \end{parts}

    \newpage

    \newpage

    \question[2]
    Realiza las siguientes operaciones y expresa el resultado en forma hexadecimal.
    \begin{parts}
        \part
        $(11010111011001110101)_2+(111100011001001)_2$
        \vspace{\stretch{1}}
        \part
        $(AB0A)_{16}+(18)_{10}$
        \vspace{\stretch{1}}
        \part
        $(AB0A)_{16}+(18)_{16}$
        \vspace{\stretch{1}}
        \part
        $(A0AAC32)_{16}+(1011)_2$
        \vspace{\stretch{1}}
        \part
        $(A1ACC32)_{16}+(11011011)_2$
        \vspace{\stretch{1}}
    \end{parts}

    \newpage

    \fbox{\fbox{\parbox{6.5in}{
                Espacio para operaciones adicionales.
                \begin{itemize}
                    \item
                          Debéis entregar esta hoja junto al resto del examen.
                    \item
                          No olvidéis poner referencias desde cada ejercicio cuando se realice uso de estas.
                \end{itemize}
            }}}

\end{questions}

\end{document}