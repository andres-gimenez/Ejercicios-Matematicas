%%%%%%%%%%%%%%%%%%%%%%%%%%%
\newcommand{\documentName} { Examen 1ª evaluación }
\newcommand{\documentContent} { Examen ordinario } 
\newcommand{\waterMark} { Modelo A } 
%%%%%%%%%%%%%%%%%%%%%%%%%%%

% Configuración del documento.
\newcommand{\schoolSubject} { Matemáticas 3º ESO - Recuperación}
\newcommand{\school} { IES La Serna }
\newcommand{\academicPeriod} { Curso 2020/2021 }


\newcommand{\autor} { Andrés Giménez Muñoz }
\newcommand{\emailAuthor} { agimenezmunoz@ieslaserna.com }
\newcommand{\autorSing}{ Profesores: Andrés } 
\renewcommand{\schoolSubject} { Examen Matemáticas 2º ESO  }
\renewcommand{\school} { IES José de Churriguera  }
\renewcommand{\academicPeriod} { Curso 2022/2023 }

\renewcommand{\autor} { Andrés Giménez Muñoz }
\renewcommand{\emailAuthor} { andresprofemates@outlook.es }
\renewcommand{\autorSing}{ Profesor: Andrés } 

%%%%%%%%%%%%%%%%%%%%%%%%%%%
% Exam configuration
%\pointsdroppedatright   %% No mostrar la puntuación
\pointsinrightmargin % Para poner las puntuaciones a la derecha. Se puede cambiar. Si se comenta, sale a la izquierda.
\extrawidth{-1.5cm} %Un poquito más de margen por si ponemos textos largos.
\marginpointname{ \emph{\points}}

%% Si se comenta no aparecerán los espacios de la solución.
\nocancelspace

%% Esto es de la clase exam. Si dejamos sin comentar \printanswers, se mostraran las soluciones. 
%% Si la comentamos y dejamos sin comentar \noprintanswers, pues no se muestran las soluciones.
% \printanswers
%\noprintanswers

%%%%%%%%%%%%%%%%%%%%%%%%%%%

\begin{document}

\StudentData
\GradeTableHeader

\justifying

\begin{questions}
    \setcounter{question}{0}

    \question[4]
    Indica las afirmaciones más apropiadas
    \begin{center}
        \fbox{\fbox{\parbox{6.5in}{
                    La forma de valor el resultado las preguntas tipo test será:

                    \begin{equation*} \label{eqn}
                        Nota = \left( correctas - \frac{incorrecta}{2} \right) \cdot \left( \frac{nota\:máxima}{preguntas}  \right)
                    \end{equation*}

                    \vspace{0.2cm}
                    \par
                    Cuando contestes a cada pregunta del cuestionario ten presente:
                    \begin{itemize}
                        \item Lee atentamente las preguntas y las contestaciones, tienes tiempo de sobra para resolver el examen.
                        \item Los fallos restan puntos, si no estás seguro de la respuesta no la contestes.
                        \item Aunque en algunas preguntas pueda parecer que existe más de una respuesta correcta, tienes que dar una respuesta profesional que indique que dominas los conceptos teóricos.
                    \end{itemize}
                }}}
    \end{center}

    \begin{parts}
        \PartNoBreak{
            Un conmutador o switch es un elemento de red perteneciente a \dots
            \newline
            \begin{oneparcheckboxes}
                \CorrectChoice 
                \dots la capa de enlace a datos.
                \newline
                \choice 
                \dots la capa de red.
                \newline
                \choice 
                \dots la capa de transporte.
                \newline
            \end{oneparcheckboxes}
        }
        
        \PartNoBreak{
            Entendemos por Hosts como \dots
            \newline
            \begin{oneparcheckboxes}
                \CorrectChoice 
                \dots los nodos de origen y de destino dentro de un canal de comunicación.
                \newline
                \choice 
                \dots cada dispositivo electrónico de una red de telecomunicaciones.
                \newline
                \choice 
                \dots cada elemento de interconexión de una red de comunicaciones.
                \newline
            \end{oneparcheckboxes}
        }

        \PartNoBreak{
            Un enlace punto a punto puede ser denominado dúplex si \dots
            \newline
            \begin{oneparcheckboxes}
                \choice 
                \dots puede trasmitir y recibir información en ambas direcciones simultáneamente.
                \newline
                \choice 
                \dots envía información por un canal y lo recibe por otro.
                \newline
                \CorrectChoice 
                \dots puede transmitir y recibir información simultáneamente
                \newline
            \end{oneparcheckboxes}
        }

        \PartNoBreak{
            Un proveedor de servicios de Internet, llamado ISP (Internet service provide) es una empresa que brinda conexión a Internet a sus clientes. 
            En la actualidad para ofrecer conexión de fibra óptica a sus clientes instalan \dots
            \newline
            \begin{oneparcheckboxes}
                \choice 
                \dots un ONT (optical network terminal) en sus instalaciones y un  en el OLT (Optical line terminal) en el cliente,
                conectándolos por el protocolo PPPoE (Point-To-Point over ethernet). 
                \newline
                \choice \dots un router de fibra que permite conectar los dispositivos tanto por ethernet como redes WiFi mediante protocolo TCP/IP. 
                \newline
                \CorrectChoice 
                \dots un OLT (Optical line terminal) en sus instalaciones y un ONT (optical network terminal) en el cliente,
                conectándolos por el protocolo PPPoE (Point-To-Point over ethernet).  
                \newline
            \end{oneparcheckboxes}
        }

        \PartNoBreak{
            El canal de comunicación es \dots
            \newline
            \begin{oneparcheckboxes}
                \choice 
                \dots una numeración que se utiliza para dividir la comunicación en diversos contenidos.
                \newline
                \CorrectChoice 
                \dots el medio por el que circula la información.
                \newline
                \choice 
                \dots un conjunto de frecuencias por las que se trasmite datos.
                \newline
            \end{oneparcheckboxes}
        }

        \PartNoBreak{
            El ruido es \dots
            \newline
            \begin{oneparcheckboxes}
                \choice 
                \dots perturbaciones sonoras que interrumpen la comunicación.
                \newline
                \choice 
                \dots es la sensación auditiva y molesta.
                \newline
                \CorrectChoice 
                \dots cualquier perturbación que dificulte la comunicación.
                \newline
            \end{oneparcheckboxes}
        }

        \PartNoBreak{
            Dentro del modelo OSI el protocolo IP pertenece a \dots
            \newline
            \begin{oneparcheckboxes}
                \choice 
                \dots la capa de trasporte.
                \newline
                \CorrectChoice 
                \dots la capa de red.
                \newline
                \choice 
                \dots la capa de enlace de datos.
                \newline
            \end{oneparcheckboxes}
        }

        \PartNoBreak{
            Dentro del modelo OSI el protocolo ethernet pertenece a \dots
            \newline
            \begin{oneparcheckboxes}
                \CorrectChoice 
                \dots la capa de enlace de datos.
                \newline
                \choice 
                \dots la capa de trasporte.
                \newline
                \choice 
                \dots la capa de red.
                \newline
            \end{oneparcheckboxes}
        }

        \PartNoBreak{
            La multiplexación es un mecanismo por el que \dots
            \newline
            \begin{oneparcheckboxes}
                \choice 
                \dots nos permite utilizar varios circuitos de comunicación para formar una sola línea.
                \newline
                \CorrectChoice 
                \dots nos permite dirigir dos o más señales por el mismo medio.  
                \newline
                \choice 
                \dots nos permite encaminar la comunicación en la dirección que deseamos.
                \newline
            \end{oneparcheckboxes}
        }

        \PartNoBreak{
            ¿Qué función realiza la ICANN en la asignación de direcciones IP?
            \newline
            \begin{oneparcheckboxes}
                \choice La ICANN concede rangos de direcciones IP a los proveedores de servicios de Internet,
                que a su vez estos los reparten entre sus clientes.
                \newline
                \CorrectChoice La ICANN delega los recursos de Internet a los RIRs,
                y a su vez los RIRs siguen sus políticas regionales para una
                posterior subdelegación de recursos a sus clientes, que
                incluyen Proveedores de servicios y organizaciones para uso
                propio. 
                \newline
                \choice La ICANN es una organización privada, sin ánimo de lucro que, entre otras competencias,
                se encarga de mantener y gestionar los servidores troncales DHCP que reparten de forma automática las direcciones IP públicas.
                \newline
            \end{oneparcheckboxes}
        }

        \PartNoBreak{
            Un puente o bridge es un \dots
            \newline
            \begin{oneparcheckboxes}
                \choice 
                \dots un dispositivo de red que permite conectar redes a gran distancia.
                \newline
                \CorrectChoice 
                \dots un dispositivo de red que opera en la capa de enlace a datos conectando dos segmentos de red.
                \newline
                \choice 
                \dots un dispositivo de red que permite pasar sobre un protocolo no acto para la comunicación
                \newline
            \end{oneparcheckboxes}
        }

        \PartNoBreak{
            Un enrutador o router es un elemento de red perteneciente a \dots
            \newline
            \begin{oneparcheckboxes}
                \choice 
                \dots la capa de transporte.
                \newline
                \CorrectChoice 
                \dots la capa de red.
                \newline
                \choice 
                \dots la capa de enlace a datos.
                \newline
            \end{oneparcheckboxes}
        }

        \PartNoBreak{
            Entendemos por dirección MAC a \dots
            \newline
            \begin{oneparcheckboxes}
                \choice 
                \dots una dirección única para cada Host.
                \newline
                \CorrectChoice 
                \dots una dirección única para cada interfaz de red.
                \newline
                \choice 
                \dots una dirección que debemos configurar para acceder a la red.
                \newline
            \end{oneparcheckboxes}
        }

        \PartNoBreak{
            El protocolo IP \dots
            \newline
            \begin{oneparcheckboxes}
                \CorrectChoice 
                \dots no tiene un control de errores, con lo que se puede considerar no fiable.
                \newline
                \choice 
                \dots mantiene un control de errores.
                \newline
                \choice 
                \dots nos asegura que toda la información que enviamos llegará a su destinatario.
                \newline
            \end{oneparcheckboxes}
        }

        \PartNoBreak{
            Una dirección IP pública \dots
            \newline
            \begin{oneparcheckboxes}
                \CorrectChoice 
                \dots es visible desde cualquier Host conectado a Internet.
                \newline
                \choice 
                \dots es visibles solamente desde otra dirección pública.
                \newline
                \choice 
                \dots son direcciones que se pueden usar libremente por cualquier usuario.
                \newline
            \end{oneparcheckboxes}
        }

        \PartNoBreak{
            Una dirección IP privada \dots
            \newline
            \begin{oneparcheckboxes}
                \choice 
                \dots solo se puede utilizar por organizaciones autorizadas para utilizarlas.
                \newline
                \CorrectChoice 
                \dots solo es visible desde su red local.
                \newline
                \choice 
                \dots son direcciones internas de Internet utilizadas para los routers.
                \newline
            \end{oneparcheckboxes}
        }

        \PartNoBreak{
            Los conectores de los cables de par trenzados utilizados por Ethernet se denominan.
            \newline
            \begin{oneparcheckboxes}
                \choice 
                    RJ-11
                \newline
                \choice 
                    RJ-12
                \newline
                \CorrectChoice 
                    RJ-45
                \newline
            \end{oneparcheckboxes}
        }
    \end{parts}

    \newpage
    \question[1] 
        Calcula la dirección de red y broadcast para las siguientes configuraciones de red. \\
        {\small Muestra los cálculos que has realizado.}
    \begin{parts}
        \part
        IP: 82.213.249.244 \\
        Mask: 255.255.255.252 
        \begin{solution}[2cm]
            Red: 82.213.249.244 / 30 \\
            Broadcas: 82.213.249.247 \\
        \end{solution}
        \vspace{\stretch{1}}

        \part
        IP: 82.213.249.244 \\
        Mask: 255.255.252.0
        \begin{solution}[2cm]
            Red: 82.213.248.00 / 22 \\
            Broadcas: 82.213.251.255
        \end{solution}
        \vspace{\stretch{1}}
        
        \part
        IP: 16.35.23.12 \\
        Mask: 255.255.255.248
        \begin{solution}[2cm]
            Red: 16.35.23.8 / 29 \\
            Broadcas: 16.35.23.15
        \end{solution}
        \vspace{\stretch{1}}

        \part
        IP: 16.35.23.12 \\
        Mask: 255.255.192.0
        \begin{solution}[2cm]
            Red: 16.35.0.0 / 18 \\
            Broadcas: 16.35.63.255
        \end{solution}
        \vspace{\stretch{1}}

        \part
        IP: 1200:235:43::325a / 64 
        \begin{solution}[0cm]
            Red: 1200:235:43:: / 64 \\
            Rango de hosts:  1200:235:43::1 \\ - 1200:235:43::ffff:ffff:ffff:ffff \\
        \end{solution}
        \vspace{\stretch{2}}

    \end{parts}

    \newpage

    \question[1] 
    Indica si los siguientes IPs perteneces a la misma red.\\
    {\small Muestra los cálculos que has realizado.}

    \begin{parts}
        \part
        IP1: 192.168.0.25 \\
        IP2: 192.168.0.38 \\
        Mask: 255.255.255.0
        \begin{solution}[0cm]
            Son válidas y perteneces a la misma red.
        \end{solution}
        \vspace{\stretch{1}}

        \part
        IP1: 25.12.23.60 \\
        IP2: 25.12.23.70 \\
        Mask: 255.255.255.224
        \begin{solution}[0cm]
            Son válidas y perteneces a redes distintas.
        \end{solution}
        \vspace{\stretch{1}}

        \part
        IP1: 2000::32ed / 125 \\
        IP2: 2000::32fd / 125 \\
        \begin{solution}[0cm]
            Son válidas y perteneces a redes distintas.
        \end{solution}
        \vspace{\stretch{1}}
   
    \end{parts}

    \newpage
    \question[2] Configura el siguiente esquema de red, para que puedan conectarse todos los hosts.

    \begin{center}
        \centering
        \includegraphics[width=1\textwidth]{red03}
    \end{center}

    \begin{center}
        \begin{tabular}{|l|l|l|l|}
            \hline
            \multirow{7}{*}{Router01} & \multicolumn{3}{l|}{\phantom{00000.00000.00000.00000}\phantom{00000.00000.00000.00000}\phantom{00000.0000.00000.00000}} \\ 
                                    & \multicolumn{3}{l|}{} \\ 
                                    & \multicolumn{3}{l|}{} \\ 
                                    & \multicolumn{3}{l|}{} \\ 
                                    & \multicolumn{3}{l|}{} \\ 
                                    & \multicolumn{3}{l|}{} \\ 
                                    & \multicolumn{3}{l|}{} \\  \hline
            \multirow{7}{*}{Router02} & \multicolumn{3}{l|}{} \\ 
                                    & \multicolumn{3}{l|}{} \\ 
                                    & \multicolumn{3}{l|}{} \\ 
                                    & \multicolumn{3}{l|}{} \\ 
                                    & \multicolumn{3}{l|}{} \\ 
                                    & \multicolumn{3}{l|}{} \\ 
                                    & \multicolumn{3}{l|}{} \\ \hline
            \multirow{7}{*}{Router03} & \multicolumn{3}{l|}{} \\ 
                                    & \multicolumn{3}{l|}{} \\ 
                                    & \multicolumn{3}{l|}{} \\ 
                                    & \multicolumn{3}{l|}{} \\ 
                                    & \multicolumn{3}{l|}{} \\ 
                                    & \multicolumn{3}{l|}{} \\ 
                                    & \multicolumn{3}{l|}{} \\  \hline
        \end{tabular}
    \end{center}
    \begin{center}
        \begin{tabular}{|l|l|l|l|}
            \hline
            Host                       & IP                 & Mask            & Gateway       \\ \hline
            \hline
            Server01                   & \phantom{0000.0000.0000.0000}     &  \phantom{0000.0000.0000.0000}  & \phantom{0000.0000.0000.0000}  \\ \hline
            PC01                       &      &                 &                \\ \hline
            PC02                       &      &                 &                \\ \hline
            PC03                       &      &                 &                \\ \hline
            PC04                       &      &                 &                \\ \hline
            PC05                       &      &                 &                \\ \hline
            PC06                       &      &                 &                \\ \hline
            Destop02                   &      &                 &                \\ \hline
            Printer02                  &      &                 &                \\ \hline
            \end{tabular}
        \end{center}
    \newpage

    \question[2] De acuerdo a la siguiente configuración de red, 
        recibes una incidencia en la que desde contabilidad pueden imprimir, 
        pero desde el almacén no, busca cuál es el problema y propón una solución (modificando lo menos posible).
        Identifica cuantas redes lógicas hay y cuáles son. 

    \begin{center}
        \centering
        \includegraphics[width=1\textwidth]{red04}
    \end{center}

    \begin{center}
        \begin{tabular}{|l|l|l|l|}
        
        \hline
        \multirow{2}{*}{Router01} & Interface 1 & \multicolumn{2}{l|}{ 192.168.56.128/29 } \\ \cline{2-4}
                                  & Interface 2 & \multicolumn{2}{l|}{ 192.168.56.137/29 } \\ \cline{2-4}
        \hline
        \hline
        Host                       & IP                 & Mask            & Gateway       \\ \hline
        \hline
        Server01                   & 192.168.56.129 & 255.255.255.248 & 192.168.56.128 \\ \hline
        Destop01                   & 192.168.56.130 & 255.255.255.248 & 192.168.56.128 \\ \hline
        PC01                       & 192.168.56.131 & 255.255.255.248 & 192.168.56.128 \\ \hline
        PC02                       & 192.168.56.132 & 255.255.255.248 & 192.168.56.128 \\ \hline
        PC03                       & 192.168.56.133 & 255.255.255.248 & 192.168.56.128 \\ \hline
        PC04                       & 192.168.56.134 & 255.255.255.248 & 192.168.56.128 \\ \hline
        \hline
        Destop02                   & 192.168.56.138 & 255.255.255.248 & 192.168.56.137 \\ \hline
        PC05                       & 192.168.56.139 & 255.255.255.248 & 192.168.56.137 \\ \hline
        PC06                       & 192.168.56.140 & 255.255.255.248 & 192.168.56.137 \\ \hline
        PC07                       & 192.168.56.141 & 255.255.255.248 & 192.168.56.137 \\ \hline
        Printer02                  & 192.168.56.142 & 255.255.255.248 & 192.168.56.137 \\ \hline
        \end{tabular}
    \end{center}

    \begin{solution}[2cm]
        La IP de Router01-Inteface 1  es una dirección de red, con lo que no se puede usar.
    \end{solution}
    \newpage
    \phantom{00}
\end{questions}

\end{document}