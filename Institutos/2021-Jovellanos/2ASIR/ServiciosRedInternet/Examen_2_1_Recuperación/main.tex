%%%%%%%%%%%%%%%%%%%%%%%%%%%
\newcommand{\documentName} { Examen 2ª evaluación }
\newcommand{\documentContent} { Examen recuperación } 
\newcommand{\waterMark} { } 
%%%%%%%%%%%%%%%%%%%%%%%%%%%

% Configuración del documento.
\newcommand{\schoolSubject} { Matemáticas 3º ESO - Recuperación}
\newcommand{\school} { IES La Serna }
\newcommand{\academicPeriod} { Curso 2020/2021 }


\newcommand{\autor} { Andrés Giménez Muñoz }
\newcommand{\emailAuthor} { agimenezmunoz@ieslaserna.com }
\newcommand{\autorSing}{ Profesores: Andrés } 
\renewcommand{\schoolSubject} { Examen Matemáticas 2º ESO  }
\renewcommand{\school} { IES José de Churriguera  }
\renewcommand{\academicPeriod} { Curso 2022/2023 }

\renewcommand{\autor} { Andrés Giménez Muñoz }
\renewcommand{\emailAuthor} { andresprofemates@outlook.es }
\renewcommand{\autorSing}{ Profesor: Andrés } 

%%%%%%%%%%%%%%%%%%%%%%%%%%%
% Exam configuration
%\pointsdroppedatright   %% No mostrar la puntuación
\pointsinrightmargin % Para poner las puntuaciones a la derecha. Se puede cambiar. Si se comenta, sale a la izquierda.
\extrawidth{-1.5cm} %Un poquito más de margen por si ponemos textos largos.
\marginpointname{ \emph{\points}}

%% Si se comenta no aparecerán los espacios de la solución.
%\nocancelspace

%% Esto es de la clase exam. Si dejamos sin comentar \printanswers, se mostraran las soluciones. 
%% Si la comentamos y dejamos sin comentar \noprintanswers, pues no se muestran las soluciones.
\printanswers
%\noprintanswers

%%%%%%%%%%%%%%%%%%%%%%%%%%%

\begin{document}

\StudentData
% \GradeTableHeader

\justifying

\begin{center}
    \fbox{\fbox{\parbox{6.5in}{
                La forma de valor el resultado del examen será:

                \begin{equation*} \label{eqn}
                    \text{Nota} = \left( \text{correctas} - \frac{\text{incorrecta}}{2} \right) \cdot \left( \frac{\text{nota máxima}}{\text{preguntas}}  \right)
                \end{equation*}

                \vspace{0.2cm}
                \par
                Cuando contestes a cada pregunta del cuestionario ten presente:
                \begin{itemize}
                    \item Lee atentamente las preguntas y las contestaciones, tienes tiempo de sobra para resolver el examen.
                    \item Los fallos restan puntos, si no estás seguro de la respuesta no la contestes.
                    \item Aunque en algunas preguntas pueda parecer que existe más de una respuesta correcta, tienes que dar una respuesta profesional que indique que dominas los conceptos teóricos.
                \end{itemize}
            }}}
\end{center}

\begin{questions}
    \setcounter{question}{0}
    
    \QuestionNoBreak{}
    El formato svg es un formato \dots
    \newline
    \nolinebreak{
        \begin{oneparcheckboxes}
            \CorrectChoice{}
            \dots vectorial de imágenes, utilizado en aplicaciones responsive (aquellas que se adaptan al tamaño del dispositivo).
            \newline
            \choice{}
            \dots formato de video sin perdida, muy utilizado para componer video.
            \newline
            \choice{}
            \dots un formato de codificación de audio, ya en desuso.
            \newline
        \end{oneparcheckboxes}
    }
    
    \QuestionNoBreak{}
    Si hablamos de fps, estamos hablando de \dots
    \newline
    \nolinebreak{
        \begin{oneparcheckboxes}
            \choice{}
            \dots un formato de audio sin perdida.
            \newline
            \choice{}
            \dots un formato de video, ya en desuso.
            \newline
            \CorrectChoice{}
            \dots los fotogramas (o frames) por segundo a la que se ha de reproducir un video.
            \newline
        \end{oneparcheckboxes}
    }

    \QuestionNoBreak{}
    El TFTP se refiere a\dots
    \newline
    \nolinebreak{
        \begin{oneparcheckboxes}
            \choice{}
            \dots una variante de FTP que implementa seguridad mediante formato SSL.
            \newline
            \choice{}
            \dots una variante de FTP para trasmitir grandes volúmenes de datos.
            \newline
            \CorrectChoice{}
            \dots una variante más simple de FTP, sin seguridad.
            \newline
        \end{oneparcheckboxes}
    }

    \QuestionNoBreak{}
    El protocolo FTP \dots
    \newline
    \nolinebreak{
        \begin{oneparcheckboxes}
            \choice{}
            \dots no permite crear directorios en el servidor remoto.
            \newline
            \choice{}
            \dots permite crear directorios en el servidor remoto, solo si nos conectamos en modo activo.
            \newline
            \CorrectChoice{}
            \dots permite crear directorios en el servidor remoto.
            \newline
        \end{oneparcheckboxes}
    }

    \QuestionNoBreak{}
    Si una conexión HTTP devuelve un código 400 \dots
    \newline
    \nolinebreak{
        \begin{oneparcheckboxes}
            \choice{}
            \dots el servidor no está bien configurado, por ejemplo, podría deberse a un error de acceso a la base de datos.
            \newline
            \choice{}
            \dots es un código que utiliza el programador para comunicar el navegador con el servidor.
            \newline
            \CorrectChoice{}
            \dots indica que el cliente ha cometido un error.
            \newline
        \end{oneparcheckboxes}
    }

    \QuestionNoBreak{
		En el cuerpo de un correo electrónico podemos encontrar los campos \dots
		\newline
		\nolinebreak{
			\begin{oneparcheckboxes}
				\CorrectChoice
				\dots FROM que identifica él creó el correo,
				TO que identifica a los destinatarios,
				CC que identifica a destinatario que recibirá una copia y
				CCO que identifica a destinatario que recibirá una copia de forma oculta.
				\newline
				\choice
				\dots FROM que identifica él creó el correo,
				TO que identifica a los destinatarios,
				CC que identifica a destinatario que recibirá una copia y
				RETURN que identifica el nombre de usuario al que se le debe contestar.
				\newline
				\choice
				\dots FROM que identifica él creó el correo,
				TO que identifica a los destinatarios,
				RETURN que identifica el nombre de usuario al que se le debe contestar y
				HOST que identifica el nombre de la máquina de origen del correo.
				\newline
			\end{oneparcheckboxes}
		}
	}

	\QuestionNoBreak{
		El motivo por el que los editores de correo electrónico no descarga las imágenes por defecto es \dots
		\newline
		\nolinebreak{
			\begin{oneparcheckboxes}
				\choice
				\dots por motivos de reducir el tráfico de red.
				\newline
                \CorrectChoice
				\dots por motivos de privacidad.
				\newline
				\choice
				\dots para desmotivar a los usuarios a que muestren imágenes en sus correos.
				\newline
			\end{oneparcheckboxes}
		}
	}

	\QuestionNoBreak{
		El protocolo IMAP \dots
		\newline
		\nolinebreak{
			\begin{oneparcheckboxes}
				\CorrectChoice
				\dots nos permite consultar el correo del servidor, manteniéndolo en este.
				\newline
				\choice
				\dots nos permite descargar el correo del servidor, borrándolo de este.
				\newline
				\choice
				\dots nos permite enviar correo a un cliente.
				\newline
			\end{oneparcheckboxes}
		}
	}

	\QuestionNoBreak{
		El protocolo SMTP
		\newline
		\nolinebreak{
			\begin{oneparcheckboxes}
				\choice
				\dots nos permite enviar y recibir correos electrónicos
				\newline
                \CorrectChoice
				\dots utiliza los registros MX del servidor DNS
				\newline
				\choice
				\dots está obsoleto, ya que no mantiene los correos en el servidor.
				\newline
			\end{oneparcheckboxes}
		}
	}
    
    \QuestionNoBreak{}
    Cuando se realiza una petición a un servidor HTTP \dots
    \newline
    \nolinebreak{
        \begin{oneparcheckboxes}
            \CorrectChoice{}
            \dots podemos utilizar los métodos GET, HEAD, POST y PUT, entre otros.
            \newline
            \choice{}
            \dots hay que preceder la llamada con http.
            \newline
            \choice{}
            \dots tenemos que utilizar un navegador o browser.
            \newline
        \end{oneparcheckboxes}
    }

    \QuestionNoBreak{}
    Si una conexión HTTP devuelve un código 200 podemos interpretar que\dots
    \newline
    \nolinebreak{
        \begin{oneparcheckboxes}
            \choice{}
            \dots se ha utilizado el puerto 200.
            \newline
            \CorrectChoice{}
            \dots la conexión se ha realizado con éxito.
            \newline
            \choice{}
            \dots debemos revisar el fichero de log para ver a que tipo de error se refiere.
            \newline
        \end{oneparcheckboxes}
    }

    \QuestionNoBreak{}
    Si una conexión HTTP devuelve un código 301\dots
    \newline
    \nolinebreak{
        \begin{oneparcheckboxes}
            \choice{}
            \dots indica que el servidor no está bien configurado, por ejemplo, podría deberse a un error de acceso a la base de datos.
            \newline
            \CorrectChoice{}
            \dots indica que la información ha sido movida a otro lugar.
            \newline
            \choice{}
            \dots indica que la comunicación se ha realizado correctamente.
            \newline
        \end{oneparcheckboxes}
    }

    \QuestionNoBreak{}
    Cuando proporcionamos un certificado digital a un servidor Web \dots
    \newline
    \nolinebreak{
        \begin{oneparcheckboxes}
            \CorrectChoice{}
            \dots nos estamos identificando, para que el cliente pueda tener certeza de quien sirve esos datos.
            \newline
            \choice{}
            \dots evitamos que el navegador o browser nos bloquee la página.
            \newline
            \choice{}
            \dots evitamos que personas no autorizadas puedan ver el contenido de nuestra Web.
            \newline
        \end{oneparcheckboxes}
    }
    
    \QuestionNoBreak{}
    Con un proxy inverso\dots
    \newline
    \nolinebreak{
        \begin{oneparcheckboxes}
            \choice{}
            \dots podemos navegar en Internet sin una conexión directa.
            \newline
            \choice{}
            \dots podemos acceder de forma inversa a un servidor HTTP.
            \newline
            \CorrectChoice{}
            \dots podemos disponer de varios servidores HTTP, con tecnologías distintas y los clientes los interpreten como un único servidor.
            \newline
        \end{oneparcheckboxes}
    }
    
    \QuestionNoBreak{}
    En una conexión FTP\dots
    \newline
    \nolinebreak{
        \begin{oneparcheckboxes}
            \choice{}
            \dots con el modo pasivo, al copiar un fichero en modo texto entre una máquina Unix y otra Windows cambiará los saltos de línea.
            \newline
            \CorrectChoice{}
            \dots el modo activo utiliza los puertos 21 para conexión de control y 20 para transmitir datos, mientras que en modo pasivo utiliza el 21 para conexión de control y se negocia un puerto distinto entre cliente y servidor.
            \newline
            \choice{}
            \dots el modo pasivo utiliza los puertos 21 para conexión de control y 20 para transmitir datos, mientras que en modo activo utiliza el 21 para conexión de control y se negocia un puerto distinto entre cliente y servidor.
            \newline
        \end{oneparcheckboxes}
    }

    \QuestionNoBreak{}
    Un navegador identifica el tipo de archivo que se ha descargado \dots
    \newline
    \nolinebreak{
        \begin{oneparcheckboxes}
            \choice{}
            \dots analizando el contenido.
            \newline
            \CorrectChoice{}
            \dots por su tipo MIME.
            \newline
            \choice{}
            \dots por la extensión del nombre del archivo.
            \newline
        \end{oneparcheckboxes}
    }

    \QuestionNoBreak{}
    Para conocer el tipo MIME de un fichero\dots
    \newline
    \nolinebreak{
        \begin{oneparcheckboxes}
            \CorrectChoice{}
            \dots debemos buscar el campo \emph{content-type} de la cabecera que devuelve el protocolo HTTP.
            \newline
            \choice{}
            \dots debemos buscarlo en una lista de referencias dentro del navegador o browser.
            \newline
            \choice{}
            \dots debemos mirar la extensión del nombre del fichero.
            \newline
        \end{oneparcheckboxes}
    }
    
    \QuestionNoBreak{}
    El protocolo HTTP \dots
    \newline
    \nolinebreak{
        \begin{oneparcheckboxes}
            \choice{}
            \dots no permite requerir autentificación por parte del servidor al cliente.
            \newline
            \CorrectChoice{}
            \dots permite requerir autentificación por parte del servidor al cliente.
            \newline
            \choice{}
            \dots no permite requerir autentificación, la identificación hay que realizar la a través de otros mecanismos ajenos al protocolo.
            \newline
        \end{oneparcheckboxes}
    }
    
    \QuestionNoBreak{}
    El protocolo HTTP \dots
    \newline
    \nolinebreak{
        \begin{oneparcheckboxes}
            \CorrectChoice{}
            \dots utiliza los puertos 80 y 443 por defecto, aunque se puede utilizar otro puerto.
            \newline
            \choice{}
            \dots solo puede utilizar los puertos 80 y 443.
            \newline
            \choice{}
            \dots utiliza los puertos que son negociados entre el cliente y servidor, no obstante, se puede indicar que puerto se quiere que utilice indicándolo en la URL.
            \newline
        \end{oneparcheckboxes}
    }

    \QuestionNoBreak{}
    El protocolo HTTP fue creado para \dots
    \newline
    \nolinebreak{
        \begin{oneparcheckboxes}
            \choice{}
            \dots crear páginas en Internet, para la difusión de noticias.
            \newline
            \choice{}
            \dots crear aplicaciones que funcionen en un navegador, también llamados browser.
            \newline
            \CorrectChoice{}
            \dots facilitar el acceso a ficheros a través de una red de internet.
            \newline
        \end{oneparcheckboxes}
    }
   
    \QuestionNoBreak{}
    HTML es \dots
    \newline
    \nolinebreak{
        \begin{oneparcheckboxes}
            \CorrectChoice{}
            \dots un lenguaje de marcas que nos permite dar formato a documentos y enlazarlos a través de hipervínculos.
            \newline
            \choice{}
            \dots la extensión que ha de tener una página Web para que la pueda interpretar el navegador, también llamdo browser.
            \newline
            \choice{}
            \dots el lenguaje de Internet.
            \newline
        \end{oneparcheckboxes}
    }
    
    \QuestionNoBreak{}
    Con un proxy inverso\dots
    \newline
    \nolinebreak{
        \begin{oneparcheckboxes}
            \choice{}
            \dots podemos controlar mejor los accesos a base de datos.
            \newline
            \CorrectChoice{}
            \dots podemos distribuir la carga de un servidor Web y controla ataques DDoS.
            \newline
            \choice{}
            \dots podemos liberar espacio en memoria.
            \newline
        \end{oneparcheckboxes}
    }
    
    \QuestionNoBreak{}
    El protocolo FTP utiliza \dots
    \newline
    \nolinebreak{
        \begin{oneparcheckboxes}
            \CorrectChoice{}
            \dots los puertos 20 y 21 por defecto.
            \newline
            \choice{}
            \dots los puertos 80 y 433 por defecto.
            \newline
            \choice{}
            \dots los puertos 30 y 31 por defecto.
            \newline
        \end{oneparcheckboxes}
    }
    
    \QuestionNoBreak{}
    Podemos acceder a un servidor ftp por medio de\dots
    \newline
    \nolinebreak{
        \begin{oneparcheckboxes}
            \CorrectChoice{}
            \dots línea de comando con comandos ftp o wget, aplicaciones específicas de ftp como filezilla o winscp y navegadores como Crome o Edge.
            \newline
            \choice{}
            \dots línea de comando con comandos ftp o wget, aplicaciones específicas de ftp como filezilla.
            \newline
            \choice{}
            \dots línea de comando con comandos ftp o wget.
            \newline
        \end{oneparcheckboxes}
    }

    \QuestionNoBreak{}
    El protocolo FTP \dots
    \newline
    \nolinebreak{
        \begin{oneparcheckboxes}
            \choice{}
            \dots es un sistema de almacenamiento de información.
            \newline
            \choice{}
            \dots utiliza el puerto 31 por defecto.
            \newline
            \CorrectChoice{}
            \dots esta basado en arquitectura cliente-servidor.
            \newline
        \end{oneparcheckboxes}
    }
    
    \QuestionNoBreak{}
    El encargado de codificar las señales de video y audio \dots
    \newline
    \nolinebreak{
        \begin{oneparcheckboxes}
            \choice{}
            \dots es el driver de audio y video.
            \newline
            \choice{}
            \dots es el software de visualización de audio o vídeo, pudiendo ser aplicaciones independientes o embebidas en un navegador o browser.
            \newline
            \CorrectChoice{}
            \dots es un software denominado códec. 
            \newline
        \end{oneparcheckboxes}
    }
    
    \QuestionNoBreak{}
    Cuando hablamos de un formato de video o audio con perdida nos referimos \dots
    \newline
    \nolinebreak{
        \begin{oneparcheckboxes}
            \choice{}
            \dots en que no nos garantiza la integridad de la trasmisión.
            \newline
            \CorrectChoice{}
            \dots en que el video o audio reproducido no mantienen la misma calidad que el original.
            \newline
            \choice{}
            \dots en que no nos garantiza la integridad de los datos en el tiempo y por eso debemos tener backup de estos ficheros.
            \newline
        \end{oneparcheckboxes}
    }
    
    \QuestionNoBreak{}
    El objetivo de utilizar formatos de imagen, video o audio con pérdida es \dots
    \newline
    \nolinebreak{
        \begin{oneparcheckboxes}
            \CorrectChoice{}
            \dots reducir el tamaño de los ficheros y el tiempo de descodificación.
            \newline
            \choice{}
            \dots reducir el tamaño de los ficheros.
            \newline
            \choice{}
            \dots reducir el tiempo de descodificación.
            \newline
        \end{oneparcheckboxes}
    }

    \QuestionNoBreak{}
    La profundidad de pixel en un formato de imagen o video se refiere\dots
    \newline
    \nolinebreak{
        \begin{oneparcheckboxes}
            \choice{}
            \dots al número de pixel que componen una imagen.
            \newline
            \CorrectChoice{}
            \dots al número de colores que se pueden definir por pixel.
            \newline
            \choice{}
            \dots al tamaño del pixel medido en puntos por pulgas.
            \newline
        \end{oneparcheckboxes}
    }

    \QuestionNoBreak{}
    Cuando hablamos de los ppp de una imagen nos referimos a \dots
    \newline
    \nolinebreak{
        \begin{oneparcheckboxes}
            \choice{}
            \dots la perdida que tiene al ser comprimida, expresada en partes por puntos.
            \newline
            \CorrectChoice{}
            \dots la densidad de puntos de la imagen, expresada en puntos por pulgada.
            \newline
            \choice{}
            \dots las partes en las que está descompuesta, expresada en pulgadas por partes.
            \newline
        \end{oneparcheckboxes}
    }
\end{questions}

\end{document}