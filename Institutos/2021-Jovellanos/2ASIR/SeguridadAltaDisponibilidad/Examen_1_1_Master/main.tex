%%%%%%%%%%%%%%%%%%%%%%%%%%%
\newcommand{\documentName} { Examen 1ª evaluación }
\newcommand{\documentContent} { Examen conocimientos } 
\newcommand{\waterMark} { Modelo Master } 
%%%%%%%%%%%%%%%%%%%%%%%%%%%

% Configuración del documento.
\newcommand{\schoolSubject} { Matemáticas 3º ESO - Recuperación}
\newcommand{\school} { IES La Serna }
\newcommand{\academicPeriod} { Curso 2020/2021 }


\newcommand{\autor} { Andrés Giménez Muñoz }
\newcommand{\emailAuthor} { agimenezmunoz@ieslaserna.com }
\newcommand{\autorSing}{ Profesores: Andrés } 
\renewcommand{\schoolSubject} { Examen Matemáticas 2º ESO  }
\renewcommand{\school} { IES José de Churriguera  }
\renewcommand{\academicPeriod} { Curso 2022/2023 }

\renewcommand{\autor} { Andrés Giménez Muñoz }
\renewcommand{\emailAuthor} { andresprofemates@outlook.es }
\renewcommand{\autorSing}{ Profesor: Andrés } 

%%%%%%%%%%%%%%%%%%%%%%%%%%%
% Exam configuration
%\pointsdroppedatright   %% No mostrar la puntuación
\pointsinrightmargin % Para poner las puntuaciones a la derecha. Se puede cambiar. Si se comenta, sale a la izquierda.
\extrawidth{-1.5cm} %Un poquito más de margen por si ponemos textos largos.
\marginpointname{ \emph{\points}}

%% Si se comenta no aparecerán los espacios de la solución.
%\nocancelspace

%% Esto es de la clase exam. Si dejamos sin comentar \printanswers, se mostraran las soluciones. 
%% Si la comentamos y dejamos sin comentar \noprintanswers, pues no se muestran las soluciones.
%\printanswers
%\noprintanswers

%%%%%%%%%%%%%%%%%%%%%%%%%%%

\begin{document}

\StudentData
% \GradeTableHeader

\justifying

\begin{center}
    \fbox{\fbox{\parbox{6.5in}{
                La forma de valor el resultado del examen será:
                
                \begin{equation*} \label{eqn}
                    Nota = \left( correctas - \frac{incorrecta}{2} \right) \cdot \left( \frac{nota\:máxima}{preguntas}  \right)
                \end{equation*}

                \vspace{0.2cm}
                \par
                Cuando contestes a cada pregunta del cuestionario ten presente:
                \begin{itemize}
                    \item Lee atentamente las preguntas y las contestaciones, tienes tiempo de sobra para resolver el examen.
                    \item Los fallos restan puntos, si no estás seguro de la respuesta no la contestes.
                    \item Aunque en algunas preguntas pueda parecer que existe más de una respuesta correcta, tienes que dar una respuesta profesional que indique que dominas los conceptos teóricos.
                \end{itemize}
        }}}
\end{center}

\begin{questions}
    \setcounter{question}{0}

    \question
    Dentro de los aspectos básicos de la seguridad informática, nos tenemos que centrar \dots
    \newline
    \nolinebreak{
        \begin{oneparcheckboxes}
            \CorrectChoice
            \dots en la confidencialidad, la integridad y la disponibilidad. 
            \newline
            \choice
            \dots en las vulnerabilidades del sistema, para evitar que un hacker acceda a nuestros servicios.
            \newline
            \choice
            \dots en que el software malicioso nos robe datos a los usuarios.
            \newline
        \end{oneparcheckboxes}
    }

    \question
    Respecto al origen de los ataques que podemos recibir dentro de una organización, a nivel informático nos tenemos que centrar \dots
    \newline
    \nolinebreak{
        \begin{oneparcheckboxes}
            \CorrectChoice
            \dots tanto en los ataques internos como externos.
            \newline
            \choice
            \dots principalmente en los ataques externos, ya que es la principal fuente de amenazas.
            \newline
            \choice
            \dots en las vulnerabilidades de los sistemas operativos, para que un virus no los pueda infectar.
            \newline
        \end{oneparcheckboxes}
    }

    \question
    La incorrecta manipulación de los datos por parte de los usuarios es \dots
    \newline
    \nolinebreak{
        \begin{oneparcheckboxes}
            \CorrectChoice
            \dots parte de la seguridad informática, y un responsable de seguridad debe poner los medios necesarios para que esto no ocurra.
            \newline
            \choice
            \dots únicamente responsabilidad del usuario y un responsable de seguridad no debe involucrarse en estos aspectos.
            \newline
            \choice
            \dots siempre debido a la falta de formación del usuario y se ha de formar al usuario para que no cometa errores.
            \newline
        \end{oneparcheckboxes}
    }

    \question
    Un responsable de seguridad informática debe \dots
    \newline
    \nolinebreak{
        \begin{oneparcheckboxes}
            \CorrectChoice
            \dots investigar y analizar las técnicas de ataques en ciberseguridad.
            \newline
            \choice
            \dots centrarse únicamente en tener los sistemas actualizados para evitar las últimas vulnerabilidades.
            \newline
            \choice
            \dots centrarse en el uso que hacen los usuarios de los sistemas, para que no creen nuevas vulnerabilidades.
            \newline
        \end{oneparcheckboxes}
    }

    \question
    La alta disponibilidad se refiere a que \dots
    \newline
    \nolinebreak{
        \begin{oneparcheckboxes}
            \CorrectChoice
            \dots las aplicaciones y datos se encuentren operativas para los usuarios autorizados en todo momento y sin interrupciones.
            \newline
            \choice
            \dots los servicios respondan a la mayor brevedad posible.
            \newline
            \choice
            \dots se le dé prioridad en los servicios a los usuarios con más alta responsabilidad.
            \newline
        \end{oneparcheckboxes}
    }

    \question
    La seguridad \dots
    \newline
    \nolinebreak{
        \begin{oneparcheckboxes}
            \CorrectChoice
            \dots es un problema integral que afecta a toda la organización.
            \newline
            \choice
            \dots es un problema que se centra en una persona o departamento responsable de la seguridad.
            \newline
            \choice
            \dots se afronta atendiendo una serie de reglas fijas dentro de una organización.
            \newline
        \end{oneparcheckboxes}
    }

    \question
    Las auditorías informáticas \dots
    \newline
    \nolinebreak{
        \begin{oneparcheckboxes}
            \CorrectChoice
            \dots consiste en evaluar la eficiencia y eficacia de una organización con el fin realizar acciones de mejora.
            \newline
            \choice
            \dots consiste en solucionar los problemas de seguridad que puede tener una organización.
            \newline
            \choice
            \dots consisten en buscar los errores que ha cometido el responsable de seguridad de una organización.
            \newline
        \end{oneparcheckboxes}
    }

    \question
    Las auditorías \dots
    \newline
    \nolinebreak{
        \begin{oneparcheckboxes}
            \CorrectChoice
            \dots las puede realizar personal interno o externo a una organización, dependiendo de si son auditorías internas o externas.
            \newline
            \choice
            \dots siempre han de realizarlas personal altamente cualificado y con experiencia en aspectos de ciberseguridad.
            \newline
            \choice
            \dots es obligatorio que las realicen profesionales certificados en seguridad informática.
            \newline
        \end{oneparcheckboxes}
    }

    \question
    Una auditoria \dots
    \newline
    \nolinebreak{
        \begin{oneparcheckboxes}
            \CorrectChoice
            \dots se puede tener de manera global y parcial, dependiendo de nuestros recursos.
            \newline
            \choice
            \dots ha de realizarse siempre de forma global, para no dejarnos ninguna vulnerabilidad del sistema al descubierto.
            \newline
            \choice
            \dots siempre se ha de hacer de manera meticulosa, mirando de forma parcial punto por punto todos los posibles puntos de vulnerabilidad del sistema.
            \newline
        \end{oneparcheckboxes}
    }

    \question
    La seguridad pasiva \dots
    \newline
    \nolinebreak{
        \begin{oneparcheckboxes}
            \CorrectChoice
            \dots intenta minimizar el impacto y efectos realizados por robos y accidentes.
            \newline
            \choice
            \dots es un término que  se refiere a la actitud de una organización cuando no tiene ningún tipo de medida relacionada con la seguridad informática.
            \newline
            \choice
            \dots es un término que  se refiere a la actitud de una organización cuando tiene una actitud muy lenta ante ataques informáticos.
            \newline
        \end{oneparcheckboxes}
    }

    \question
    Para identificar a una persona, se puede hacer por algo que se posee, algo que se conoce o algo que se es. 
    Un sistema de acceso se considera eficiente si se verifican \dots
    \newline
    \nolinebreak{
        \begin{oneparcheckboxes}
            \choice
            \dots uno de los parámetros.
            \newline
            \CorrectChoice
            \dots dos de los parámetros.
            \newline
            \choice
            \dots los tres parámetros.
            \newline
        \end{oneparcheckboxes}
    }

    \question
    Los armarios o Rack, están destinados a almacenar electrónico, informático y de comunicaciones. 
    A la hora de determinar su capacidad de espacio nos tenemos que fijar en \dots
    \newline
    \nolinebreak{
        \begin{oneparcheckboxes}
            \CorrectChoice
            \dots El número de unidades, denominadas U.
            \newline
            \choice
            \dots El tamaño en pulgadas que tiene de alto.
            \newline
            \choice
            \dots El número de tomas de red de las que dispone, ya que esto determinará el número de aparatos que podemos conectar a la red.
            \newline
        \end{oneparcheckboxes}
    }

    \question
    Un armario Rack \dots
    \newline
    \nolinebreak{
        \begin{oneparcheckboxes}
            \CorrectChoice
            \dots aporta seguridad pasiva.
            \newline
            \choice
            \dots no aporta ningún tipo de seguridad informática.
            \newline
            \choice
            \dots únicamente es útil para centralizar las conexiones de red.
            \newline
        \end{oneparcheckboxes}
    }

    \question
    Un sistema de alimentación ininterrumpido \dots
    \newline
    \nolinebreak{
        \begin{oneparcheckboxes}
            \CorrectChoice
            \dots nos aporta seguridad antes subidas y bajadas de tensión eléctrica.
            \newline
            \choice
            \dots únicamente nos da seguridad antes apagones en el suministro de la red eléctrica.
            \newline
            \choice
            \dots está destinado únicamente a grandes centros de datos.
            \newline
        \end{oneparcheckboxes}
    }

    \question
    Desde el punto de vista de la eficiencia \dots
    \newline
    \nolinebreak{
        \begin{oneparcheckboxes}
            \CorrectChoice
            \dots es necesario que los usuarios sean identificados y autentificados solo una vez.
            \newline
            \choice
            \dots es necesario que los usuarios sean identificados y autentificados cada vez acceden a un recurso de sistema.
            \newline
            \choice
            \dots es necesario que sean identificados una vez y autentificados cada vez acceden a un recurso de sistema.
            \newline
        \end{oneparcheckboxes}
    }

    \question
    Se denomina identificación \dots
    \newline
    \nolinebreak{
        \begin{oneparcheckboxes}
            \CorrectChoice
            \dots al momento en le que un usuario se da a conocer en el sistema.
            \newline
            \choice
            \dots al momento en el que un usuario accede por primera vez al sistema.
            \newline
            \choice
            \dots al momento en el que un usuario introduce su usuario y contraseña en un sistema.
            \newline
        \end{oneparcheckboxes}
    }

    \question
    Desde el punto de vista de seguridad denomina autentificación \dots
    \newline
    \nolinebreak{
        \begin{oneparcheckboxes}
            \CorrectChoice
            \dots a la verificación que realiza el sistema sobre un intento de identificación.
            \newline
            \choice
            \dots a la comprobación que realiza el sistema sobre los accesos de entrada de una persona.
            \newline
            \choice
            \dots a la comprobación de que los datos introducidos por el usuario son auténticos.
            \newline
        \end{oneparcheckboxes}
    }

    \question
    La identificación y autentificación lógica se aplica \dots
    \newline
    \nolinebreak{
        \begin{oneparcheckboxes}
            \CorrectChoice
            \dots tanto a personas como cualquier tipo de dispositivo con acceso a un sistema.
            \newline
            \choice
            \dots únicamente una persona física, ya que son estas los usuarios de los servicios informáticos
            \newline
            \choice
            \dots a el dispositivo, ya que la persona necesita un dispositivo electrónico para acceder a un sistema informático.
            \newline
        \end{oneparcheckboxes}
    }

    \question
    Un ataque por fuerza bruta consiste \dots
    \newline
    \nolinebreak{
        \begin{oneparcheckboxes}
            \CorrectChoice
            \dots en intentar recuperar una clave probar todas las combinaciones posibles hasta obtener aquella que permite el acceso.
            \newline
            \choice
            \dots utilizar medios físicos y violentos para acceder a un sistema informático.
            \newline
            \choice
            \dots intentar acceder a un sistema informático con pocos conocimientos técnicos.
            \newline
        \end{oneparcheckboxes}
    }

    \question
    Un ataque de diccionario consiste\dots
    \newline
    \nolinebreak{
        \begin{oneparcheckboxes}
            \CorrectChoice
            \dots en intentar recuperar una clave a partir de una lista de palabras de un diccionario o conjunto de palabras claves.
            \newline
            \choice
            \dots un método criptográfico basado en el orden alfabético de las letras tal como aparecen en un diccionario.
            \newline
            \choice
            \dots buscar en un diccionario el significado de ciertas palabras claves que tienen que ver con el cifrado de la información.
            \newline
        \end{oneparcheckboxes}
    }

    \question
    El codigo pin de una tarjeta inteligente (o smart card)  \dots
    \newline
    \nolinebreak{
        \begin{oneparcheckboxes}
            \CorrectChoice
            \dots no es ninguna clave de cifrado, únicamente sirve para proteger la clave pública y privada almacenada en la tarjeta.
            \newline
            \choice
            \dots es la semilla que utiliza la tarjeta para generar las claves.
            \newline
            \choice
            \dots es la clave pública utilizada para el cifrado de mensajes.
            \newline
        \end{oneparcheckboxes}
    }

    \question
    Si hablamos de sha-2 estamos hablando de \dots
    \newline
    \nolinebreak{
        \begin{oneparcheckboxes}
            \CorrectChoice
            \dots un algoritmo de generación de hash.
            \newline
            \choice
            \dots un protocolo de seguridad criptográfico.
            \newline
            \choice
            \dots un sistema de seguridad basado en clave pública y privada.
            \newline
        \end{oneparcheckboxes}
    }

    \question
    Triple DES se refiere \dots
    \newline
    \nolinebreak{
        \begin{oneparcheckboxes}
            \CorrectChoice
            \dots un algoritmo de criptografía simétrica.
            \newline
            \choice
            \dots un algoritmo de criptografía asimétrica.
            \newline
            \choice
            \dots una práctica consistente en aplicar tres veces cualquier algoritmo criptográfico.
            \newline
        \end{oneparcheckboxes}
    }

    \question
    Las políticas de usuarios y grupos \dots
    \newline
    \nolinebreak{
        \begin{oneparcheckboxes}
            \CorrectChoice
            \dots se han de definir a nivel organizativo tendiendo al puesto del usuario y su sensibilidad al acceso a los datos.
            \newline
            \choice
            \dots únicamente son útiles para facilitar la gestión de seguridad al responsable de los sistemas informáticos.
            \newline
            \choice
            \dots las define el responsable de los sistemas informáticos para facilitar sus tareas diarias.
            \newline
        \end{oneparcheckboxes}
    }

    \question
    El malware se refiere \dots
    \newline
    \nolinebreak{
        \begin{oneparcheckboxes}
            \CorrectChoice
            \dots a todo software destinado a no dar un servicio para el usuario del equipo, incluyendo software que realizan bromas.
            \newline
            \choice
            \dots únicamente se refiere a software malicioso con fines delictivos, como por ejemplo robar información.
            \newline
            \choice
            \dots a software que no realiza correctamente la funcionalidad para la que ha sido realizado.
            \newline
        \end{oneparcheckboxes}
    }

    \question
    Para fortalecer un algoritmo criptográfico debemos \dots
    \newline
    \nolinebreak{
        \begin{oneparcheckboxes}
            \CorrectChoice
            \dots aumentar el tamaño de la clave.
            \newline
            \choice
            \dots aumentar el número de veces que procesamos el mensaje a través de dicho algoritmo.
            \newline
            \choice
            \dots cerciorarnos que la vigencia de la fecha de caducidad de la clave.
            \newline
        \end{oneparcheckboxes}
    }

    \question
    El descifrado se refiere a \dots
    \newline
    \nolinebreak{
        \begin{oneparcheckboxes}
            \CorrectChoice
            \dots el proceso inverso al cifrado, pasar del texto cifrado al texto original.
            \newline
            \choice
            \dots el proceso por el que se rompe la seguridad de un sistema criptográfico.
            \newline
            \choice
            \dots el momento en el que un sistema de seguridad deja de ser seguro.
            \newline
        \end{oneparcheckboxes}
    }

    \question
    La criptografia híbrida consiste \dots
    \newline
    \nolinebreak{
        \begin{oneparcheckboxes}
            \CorrectChoice
            \dots en utilizar un sistema de cifrado asimétrico para proteger la clave de un sistema de cifrado simétrico.
            \newline
            \choice
            \dots en utilizar un sistema de cifrado simétrico para proteger la clave de un sistema de cifrado asimétrico.
            \newline
            \choice
            \dots las dos respuestas son correctas.
            \newline
        \end{oneparcheckboxes}
    }

    \question
    Si hablamos de SSL y TLS estamos hablando de \dots
    \newline
    \nolinebreak{
        \begin{oneparcheckboxes}
            \CorrectChoice
            \dots un protocolo de encriptación.
            \newline
            \choice
            \dots un algoritmo de encriptación.
            \newline
            \choice
            \dots un sistema para crear certificados digitales.
            \newline
        \end{oneparcheckboxes}
    }

    \question
    Para cifrar un mensaje con encriptación asimétrico \dots
    \newline
    \nolinebreak{
        \begin{oneparcheckboxes}
            \CorrectChoice
            \dots usamos la clave pública para cifrar y la privada para descifrar los mensajes.
            \newline
            \choice
            \dots usamos la clave privada para cifrar y la pública para descifrar los mensajes.
            \newline
            \choice
            \dots podemos utilizar distintos protocolos en el origen y en el destinatario.
            \newline
        \end{oneparcheckboxes}
    }

    \question
    Para firmar un mensaje con encriptación asimétrico \dots
    \newline
    \nolinebreak{
        \begin{oneparcheckboxes}
            \CorrectChoice
            \dots usamos la clave privada para cifrar el hast del documento y la pública para verificar la firma.
            \newline
            \choice
            \dots usamos la clave pública para cifrar el hast del documento y la privada para verificar la firma.
            \newline
            \choice
            \dots podemos utilizar distintos protocolos para firmar y para descifrar el documento.
            \newline
        \end{oneparcheckboxes}
    }

    \question
    En el formato estándar X.509 para distribuir certificados digitales \dots
    \newline
    \nolinebreak{
        \begin{oneparcheckboxes}
            \CorrectChoice
            \dots la extensión .pfx o .p12 contiene la clave pública y la privada y los ficheros con extensión .cer o .crt únicamente la clave pública.
            \newline
            \choice
            \dots la extensión .cer o .crt contiene la clave pública y la privada y los ficheros con extensión .pfx o .p12 únicamente la clave pública.
            \newline
            \choice
            \dots están protegidos por la clave pública para que no se pueda acceder a la clave privada.
            \newline
        \end{oneparcheckboxes}
    }

    \question
    Una entidad emisora de certificados se encarga de certificar la validez y veracidad de los certificados digitales, para ello \dots
    \newline
    \nolinebreak{
        \begin{oneparcheckboxes}
            \CorrectChoice
            \dots firma los certificados con su clave privada, exponiendo su clave pública.
            \newline
            \choice
            \dots los usuarios delegan en ella la tarea de cifrar sus menajes a través de su clave pública.
            \newline
            \choice
            \dots los usuarios delegan en ella la tarea de cifrar sus menajes a través de su clave privada.
            \newline
        \end{oneparcheckboxes}
    }

    \question
    Los sistemas de almacenamiento se pueden clasificar en \dots
    \newline
    \nolinebreak{
        \begin{oneparcheckboxes}
            \CorrectChoice
            \dots DAS acceso director, 
                  NAS acceso a través de una red compartida con otros servicios y 
                  SAN acceso a través de una red dedicada.
            \newline
            \choice
            \dots NAS acceso director, 
                  DAS acceso a través de una red compartida con otros servicios y 
                  SAN acceso a través de una red dedicada.
            \dots.
            \newline
            \choice
            \dots DAS acceso director, 
                  SAN acceso a través de una red compartida con otros servicios y 
                  NAS acceso a través de una red dedicada.
            \dots .
            \newline
        \end{oneparcheckboxes}
    }

    \question
    Un RAID 0 es un sistema de almacenamiento en el que \dots
    \newline
    \nolinebreak{
        \begin{oneparcheckboxes}
            \CorrectChoice
            \dots los datos se almacenan en dos discos alternando los bloques, 
                  con esto se consigue mayor velocidad de acceso, 
                  pero en caso de que falle uno de los discos se pierde toda la información.
            \newline
            \choice
            \dots los datos se almacena la información en dos discos, 
                  se consigue mayor velocidad en lectura, pero no en escritura. 
                  En caso de que uno de los discos falle conservamos toda la información.
            \newline
            \choice
            \dots los datos se almacenan en el mínimo espacio posible.
            \newline
        \end{oneparcheckboxes}
    }

    \question
    Un RAID 5 es un sistema de almacenamiento en el que \dots
    \newline
    \nolinebreak{
        \begin{oneparcheckboxes}
            \CorrectChoice
            \dots los datos se almacenan en un mínimo de 3 discos y
                en caso de que falle un solo disco no perdemos la información.
            \newline
            \choice
            \dots los datos se almacenan en varios discos, 
                  debiendo funcionar al menos 5 discos para no perder la información.
            \newline
            \choice
            \dots los datos se almacenan en exactamente 5 discos y
                  en caso de que falle un solo disco no perdemos la información.
            \newline
        \end{oneparcheckboxes}
    }

    \question
    Si tenemos una infraestructura basada en máquinas virtuales,
    para diseñar una política de copias de seguridad \dots
    \newline
    \nolinebreak{
        \begin{oneparcheckboxes}
            \CorrectChoice
            \dots podemos guardar la copia de las máquinas virtuales como copia de seguridad.
            \newline
            \choice
            \dots debemos realizar una política de copias de seguridad idéntica a si utilizáramos máquinas físicas, 
            ya que las copias de seguridad de máquinas virtuales son problemática.
            \newline
            \choice
            \dots debemos tener en cuenta el sistema operativo de cada máquina virtual. 
            \newline
        \end{oneparcheckboxes}
    }

    \question
    Si nos hablan de una cabina, nos están hablando de \dots
    \newline
    \nolinebreak{
        \begin{oneparcheckboxes}
            \CorrectChoice
            \dots un sistema SAN de almacenamiento de datos.
            \newline
            \choice
            \dots un sistema NAS de almacenamiento de datos.
            \newline
            \choice
            \dots un lugar donde podemos disponer de un terminal telefónico.
            \newline
        \end{oneparcheckboxes}
    }
\end{questions}

\end{document}