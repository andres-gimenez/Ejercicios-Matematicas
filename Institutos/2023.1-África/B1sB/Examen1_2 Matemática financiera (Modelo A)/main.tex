\documentclass[addpoints,spanish, 12pt,a4paper,cancelspace]{./include/gexam}

%%%%%%%%%%%%%%%%%%%%%%%%%%%
\renewcommand{\documentName} { 1ª evaluación 2º examen }
\renewcommand{\documentContent} { Matemáticas financiera } 
\renewcommand{\waterMark} { Modelo A }

% Configuración del documento.
\renewcommand{\schoolSubject} { Examen Matemáticas 2º ESO  }
\renewcommand{\school} { IES José de Churriguera  }
\renewcommand{\academicPeriod} { Curso 2022/2023 }

\renewcommand{\autor} { Andrés Giménez Muñoz }
\renewcommand{\emailAuthor} { andresprofemates@outlook.es }
\renewcommand{\autorSing}{ Profesor: Andrés } 
%%%%%%%%%%%%%%%%%%%%%%%%%%%

%%%%%%%%%%%%%%%%%%%%%%%%%%%
% Exam configuration
%\pointsdroppedatright   %% No mostrar la puntuación
\pointsinrightmargin % Para poner las puntuaciones a la derecha. Se puede cambiar. Si se comenta, sale a la izquierda.
\extrawidth{-1.5cm} %Un poquito más de margen por si ponemos textos largos.
\marginpointname{ \emph{\points}}

%% Si se comenta no aparecerán los espacios de la solución.
%\nocancelspace

%% Esto es de la clase exam. Si dejamos sin comentar \printanswers, se mostraran las soluciones. 
%% Si la comentamos y dejamos sin comentar \noprintanswers, pues no se muestran las soluciones.
%\printanswers
%\noprintanswers

%%%%%%%%%%%%%%%%%%%%%%%%%%%

\begin{document}

\StudentData{}
\GradeTableHeader{}

\justifying{}

\begin{center}
    \fbox{\fbox{\parbox{6.5in}{             
                \begin{itemize}
                    \item Copiar, hablar, levantarse de la silla o molestar a al resto de la clase pueden ser motivos de la retirada del examen que se valorará con un cero.
                    \item Deben aparecer todas las operaciones, no vale solo con indicar el resultado.
                    % \item Se podrán quitar hasta cinco décimas por falta de claridad o rigor en el desarrollo de las respuestas o por una mala presentación.
                    % \item Se valorará que se indiquen las cuentas en línea, realizando las operaciones en el margen.
                    \item Está permitido utilizar la calculadora.
                \end{itemize}
            }}}
\end{center}

\begin{questions}
    \setcounter{question}{0}

    \question[2] Para agilizar los cobros en una taquilla de un teatro, tras aplicar un 4\% de descuento y 21\% de IVA, 
    queremos que se paguen 55\euro{} por entrada en la taquilla. ¿Cuál será el precio real de la entrada? 
    ¿Cuánto será el importe del IVA, teniendo en cuenta que primero se aplica el descuento y luego se aplica el IVA?
    \vspace{\stretch{1}}

    \newpage 
    \question[2] En dos años el precio de un producto se ha triplicado. 
    ¿Cuál ha sido el porcentaje de aumento anual si dicho incremento ha sido el mismo los dos años?
    \newpage 

    \question[3] Calcula el capital final que se obtiene al depositar 7.500\euro{} en una entidad bancaria al 7\% anual durante 10 años, 
    aplicando interés simple e interés compuesto con capitalización semestral. Calcula la diferencia e indica que tipo de interés nos conviene utilizar.
    \newpage 
    
    \question[3] ¿Cuál es el mínimo tiempo que hay que depositar 7.000\euro{} al 7\% anual, con periodo de capitalización trimestral, 
    para que se conviertan en más de 10.000\euro{}?
    

\end{questions}
\end{document}