\documentclass[addpoints,spanish, 12pt,a4paper,cancelspace]{./include/gexam}

%%%%%%%%%%%%%%%%%%%%%%%%%%%
\renewcommand{\documentName} { 1ª evaluación 2º examen }
\renewcommand{\documentContent} {  Polinomios y ecuaciones } 
\renewcommand{\waterMark} { Modelo A } 

% Configuración del documento.
\renewcommand{\schoolSubject} { Examen Matemáticas 2º ESO  }
\renewcommand{\school} { IES José de Churriguera  }
\renewcommand{\academicPeriod} { Curso 2022/2023 }

\renewcommand{\autor} { Andrés Giménez Muñoz }
\renewcommand{\emailAuthor} { andresprofemates@outlook.es }
\renewcommand{\autorSing}{ Profesor: Andrés } 
%%%%%%%%%%%%%%%%%%%%%%%%%%%

%%%%%%%%%%%%%%%%%%%%%%%%%%%
% Exam configuration
%\pointsdroppedatright   %% No mostrar la puntuación
\pointsinrightmargin % Para poner las puntuaciones a la derecha. Se puede cambiar. Si se comenta, sale a la izquierda.
\extrawidth{-1.5cm} %Un poquito más de margen por si ponemos textos largos.
\marginpointname{ \emph{\points}}

%% Si se comenta no aparecerán los espacios de la solución.
%\nocancelspace

%% Esto es de la clase exam. Si dejamos sin comentar \printanswers, se mostraran las soluciones. 
%% Si la comentamos y dejamos sin comentar \noprintanswers, pues no se muestran las soluciones.
%\printanswers
%\noprintanswers

%%%%%%%%%%%%%%%%%%%%%%%%%%%

\begin{document}

\StudentData{}
\GradeTableHeader{}

\justifying{}

\begin{center}
    \fbox{\fbox{\parbox{6.5in}{             
                \begin{itemize}
                    \item Copiar, hablar, levantarse de la silla o molestar a al resto de la clase pueden ser motivos de la retirada del examen que se valorará con un cero.
                    \item Deben aparecer todas las operaciones, no vale solo con indicar el resultado.
                    % \item Se podrán quitar hasta cinco décimas por falta de claridad o rigor en el desarrollo de las respuestas o por una mala presentación.
                    % \item Se valorará que se indiquen las cuentas en línea, realizando las operaciones en el margen.
                    \item Está permitido utilizar la calculadora.
                \end{itemize}
            }}}
\end{center}

\begin{questions}
    \setcounter{question}{0}

    \question[1] Realiza y reduce al máximo la siguiente operación.
        \begin{parts}
            \part $\left(x+1\right) \cdot \left(x-2\right)^2 - \left(x-2\right) \cdot \left(x+1\right)^2 = $
            \vspace{\stretch{1}}

            % \part $$
            % \vspace{\stretch{1}}
        \end{parts}
    \newpage 

    \question[2] Realiza las siguientes operaciones y expresa las de la forma más simple posible.    
        \begin{parts}
            \part ${-1-x \over x+3} + {2x \over x-3}$
            \vspace{\stretch{1}}

            \part ${2 \over x} - {3x \over x+1} + {x^2-4x \over \left(x+1\right)^2}$
            \vspace{\stretch{1}}

        \end{parts}
    
    \question[5] Resuelve las siguientes ecuaciones recuadrando el resultado.
    \begin{parts}
        \part ${x-2 \over 10} - {x+3 \over 4} = {2x- 7 \over 8}$
        \vspace{\stretch{1}}

        \part $\sqrt{x^2 +5} + 6 = 3-3x$
        \vspace{\stretch{1}}

        \part $2^x+2^{2x} = 6$
        \vspace{\stretch{1}}

        \part $\log_3{\left(2x^2-15\right)} = 4$
        \vspace{\stretch{1}}

        \part $x^5 + 2 x^4 + 4 x^3 + 8 x^2 - 5 x - 10 = 0$
        \vspace{\stretch{1}}
    \end{parts}

\end{questions}
\end{document}