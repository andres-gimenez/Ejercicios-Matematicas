\documentclass[addpoints,spanish, 12pt,a4paper,cancelspace]{./include/gexam}

%%%%%%%%%%%%%%%%%%%%%%%%%%%
\renewcommand{\documentName} { 1ª evaluación }
\renewcommand{\documentContent} { Números reales } 
\renewcommand{\waterMark} { \phantom{Modelo A} } 

% Configuración del documento.
\renewcommand{\schoolSubject} { Examen Matemáticas 2º ESO  }
\renewcommand{\school} { IES José de Churriguera  }
\renewcommand{\academicPeriod} { Curso 2022/2023 }

\renewcommand{\autor} { Andrés Giménez Muñoz }
\renewcommand{\emailAuthor} { andresprofemates@outlook.es }
\renewcommand{\autorSing}{ Profesor: Andrés } 
%%%%%%%%%%%%%%%%%%%%%%%%%%%

%%%%%%%%%%%%%%%%%%%%%%%%%%%
% Exam configuration
%\pointsdroppedatright   %% No mostrar la puntuación
\pointsinrightmargin % Para poner las puntuaciones a la derecha. Se puede cambiar. Si se comenta, sale a la izquierda.
\extrawidth{-1.5cm} %Un poquito más de margen por si ponemos textos largos.
\marginpointname{ \emph{\points}}

%% Si se comenta no aparecerán los espacios de la solución.
%\nocancelspace

%% Esto es de la clase exam. Si dejamos sin comentar \printanswers, se mostraran las soluciones. 
%% Si la comentamos y dejamos sin comentar \noprintanswers, pues no se muestran las soluciones.
%\printanswers
%\noprintanswers

%%%%%%%%%%%%%%%%%%%%%%%%%%%

\begin{document}

\StudentData{}
\GradeTableHeader{}

\justifying{}

% \begin{center}
%     \fbox{\fbox{\parbox{6.5in}{             
%                 \begin{itemize}
%                     \item Copiar, hablar, levantarse de la silla o molestar a al resto de la clase pueden ser motivos de la retirada del examen que se valorará con un cero.
%                     \item Deben aparecer todas las operaciones, no vale solo con indicar el resultado.
%                     \item Se podrán quitar hasta cinco décimas por falta de claridad o rigor en el desarrollo de las respuestas o por una mala presentación.
%                     % \item Se valorará que se indiquen las cuentas en línea, realizando las operaciones en el margen.
%                     \item Está permitido utilizar la calculadora.
%                 \end{itemize}
%             }}}
% \end{center}

\begin{questions}
    \setcounter{question}{0}

    \question[2] Simplifica las siguientes expresiones:
        \begin{parts}
            \part $\log (4)$ 
            \vspace{\stretch{1}}

            \part $\log_2{1024}$
            \vspace{\stretch{1}}

            \part $\log_4{1024}$
            \vspace{\stretch{1}}
        \end{parts}

    \newpage 

\end{questions}
\end{document}