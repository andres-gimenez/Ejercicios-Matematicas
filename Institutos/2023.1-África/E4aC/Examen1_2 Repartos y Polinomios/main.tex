\documentclass[addpoints,spanish, 12pt,a4paper,cancelspace]{./include/gexam}

%%%%%%%%%%%%%%%%%%%%%%%%%%%
\renewcommand{\documentName} { 1ª evaluación 3º examen }
\renewcommand{\documentContent} { Repartos y polinomios } 
\renewcommand{\waterMark} { Modelo A } 

% Configuración del documento.
\renewcommand{\schoolSubject} { Examen Matemáticas 2º ESO  }
\renewcommand{\school} { IES José de Churriguera  }
\renewcommand{\academicPeriod} { Curso 2022/2023 }

\renewcommand{\autor} { Andrés Giménez Muñoz }
\renewcommand{\emailAuthor} { andresprofemates@outlook.es }
\renewcommand{\autorSing}{ Profesor: Andrés } 
%%%%%%%%%%%%%%%%%%%%%%%%%%%

%%%%%%%%%%%%%%%%%%%%%%%%%%%
% Exam configuration
%\pointsdroppedatright   %% No mostrar la puntuación
\pointsinrightmargin % Para poner las puntuaciones a la derecha. Se puede cambiar. Si se comenta, sale a la izquierda.
\extrawidth{-1.5cm} %Un poquito más de margen por si ponemos textos largos.
\marginpointname{ \emph{\points}}

%% Si se comenta no aparecerán los espacios de la solución.
%\nocancelspace

%% Esto es de la clase exam. Si dejamos sin comentar \printanswers, se mostraran las soluciones. 
%% Si la comentamos y dejamos sin comentar \noprintanswers, pues no se muestran las soluciones.
%\printanswers
%\noprintanswers

%%%%%%%%%%%%%%%%%%%%%%%%%%%

\begin{document}

\StudentData{}
\GradeTableHeader{}

\justifying{}

\begin{center}
    \fbox{\fbox{\parbox{6.5in}{             
                \begin{itemize}
                    \item Copiar, hablar, levantarse de la silla o molestar a al resto de la clase pueden ser motivos de la retirada del examen que se valorará con un cero.
                    \item Deben aparecer todas las operaciones, no vale solo con indicar el resultado.
                    % \item Se podrán quitar hasta cinco décimas por falta de claridad o rigor en el desarrollo de las respuestas o por una mala presentación.
                    % \item Se valorará que se indiquen las cuentas en línea, realizando las operaciones en el margen.
                    \item Está permitido utilizar la calculadora y se valorará el correcto uso de esta.
                \end{itemize}
            }}}
\end{center}

\begin{questions}
    \setcounter{question}{0}

    \question[2] Calcula el capital final que se obtiene al depositar 7.500\euro{} en una entidad bancaria al 7\% anual durante 10 años, 
    aplicando interés simple e interés compuesto. Calcula la diferencia e indica que tipo de interés nos conviene utilizar
    \newpage 

    \question[2] Si un grifo, dando por minuto 100 litros de agua, llena en 8 horas un pozo; cinco grifos, dando cada uno 40 litros por minuto. 
    ¿En cuántas horas llenará un pozo 6 veces el anterior?
    \vspace{\stretch{1}}

    \question[2] Para agilizar los cobros en una taquilla de un teatro, tras aplicar un 5\% de descuento y 21\% de IVA, 
    queremos que se paguen 50\euro{} por entrada en la taquilla. ¿Cuál será el precio real de la entrada? ¿Cuánto será el importe del IVA? 
    \vspace{\stretch{1}}
    \newpage 

    \question[2] María quiere regalarle una fotografía enmarcada a su abuela, del día que la invitaron a comer. 
    Para ello dispone de una fotografía en formato digital de 1034 pixeles x 799 pixeles, la cual tiene intención de imprimir con papel fotográfico en una impresora a color.
    Al ir a la tienda a comprar un marco para la fotografía, encuentra las siguientes ofertas:
    \begin{enumerate}
        \item Marco de madera de 27 cm x 21 cm, en su parte interior, a 6,5\euro{} la unidad.
        \item Marco metálico de 22 cm x 17 cm, en su parte interior, a 12,65\euro{} la unidad.
        \item Marco de plástico de 23 cm x 17 cm, en su parte interior, a 8,43\euro{} la unidad.
    \end{enumerate}
    ¿Si María quiere comprar el marco que mejor se adapta a la relación de aspecto con la foto, qué marco ha de comprar? \\ \\
    (*Explica la contestación desde un punto de vista matemático, mostrando las cuentas que has utilizado para llegar a la conclusión que defiendes.)

    
    \newpage 
    \question[2] Resuelve los siguientes ejercicios con logaritmos. 
    \begin{parts}
        \part Descompón el valor numérico en factores primos para posteriormente calcula el valor de los siguientes logaritmos. \\
        (*Si se expresa la solución en un solo paso, se considerará que se ha realizado con la calculadora y no se considerará válido.)
            \begin{subparts}
                \subpart $\log_3 27 $
                \vspace{\stretch{1}}

                \subpart $\log_3 \sqrt{27}$
                \vspace{\stretch{1}}

                \subpart $\log_7 \frac{1}{7}$
                \vspace{\stretch{1}}

            \end{subparts}

    \part Expresa las siguientes expresiones como un solo logaritmo: \\ 
    (*La solución hay que expresarla como el logaritmo de una expresión numérica.)

    \begin{subparts}
        \subpart $\log_5 8 + 2 \log_5 7 - \log_5 400$
        \vspace{\stretch{1}}

        \subpart $\log_7 9 - 3 \log_7 5 + \log_7 3$
        \vspace{\stretch{1}}
    \end{subparts}
    \end{parts}

\end{questions}
\end{document}