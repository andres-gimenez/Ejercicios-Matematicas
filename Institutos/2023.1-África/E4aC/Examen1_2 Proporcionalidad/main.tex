\documentclass[addpoints,spanish, 12pt,a4paper,cancelspace]{./include/gexam}

%%%%%%%%%%%%%%%%%%%%%%%%%%%
\renewcommand{\documentName} { 1ª evaluación 2º examen }
\renewcommand{\documentContent} { Proporcionalidad } 
\renewcommand{\waterMark} { Modelo A } 

% Configuración del documento.
\renewcommand{\schoolSubject} { Examen Matemáticas 2º ESO  }
\renewcommand{\school} { IES José de Churriguera  }
\renewcommand{\academicPeriod} { Curso 2022/2023 }

\renewcommand{\autor} { Andrés Giménez Muñoz }
\renewcommand{\emailAuthor} { andresprofemates@outlook.es }
\renewcommand{\autorSing}{ Profesor: Andrés } 
%%%%%%%%%%%%%%%%%%%%%%%%%%%

%%%%%%%%%%%%%%%%%%%%%%%%%%%
% Exam configuration
%\pointsdroppedatright   %% No mostrar la puntuación
\pointsinrightmargin % Para poner las puntuaciones a la derecha. Se puede cambiar. Si se comenta, sale a la izquierda.
\extrawidth{-1.5cm} %Un poquito más de margen por si ponemos textos largos.
\marginpointname{ \emph{\points}}

%% Si se comenta no aparecerán los espacios de la solución.
%\nocancelspace

%% Esto es de la clase exam. Si dejamos sin comentar \printanswers, se mostraran las soluciones. 
%% Si la comentamos y dejamos sin comentar \noprintanswers, pues no se muestran las soluciones.
%\printanswers
%\noprintanswers

%%%%%%%%%%%%%%%%%%%%%%%%%%%

\begin{document}

\StudentData{}
\GradeTableHeader{}

\justifying{}

\begin{center}
    \fbox{\fbox{\parbox{6.5in}{             
                \begin{itemize}
                    \item Copiar, hablar, levantarse de la silla o molestar a al resto de la clase pueden ser motivos de la retirada del examen que se valorará con un cero.
                    \item Deben aparecer todas las operaciones, no vale solo con indicar el resultado.
                    % \item Se podrán quitar hasta cinco décimas por falta de claridad o rigor en el desarrollo de las respuestas o por una mala presentación.
                    % \item Se valorará que se indiquen las cuentas en línea, realizando las operaciones en el margen.
                    \item Está permitido utilizar la calculadora y se valorará el correcto uso de esta.
                \end{itemize}
            }}}
\end{center}

\begin{questions}
    \setcounter{question}{0}

    \question[2] Si un grifo, dando por minuto 100 litros de agua, llena en 8 horas un pozo; cinco grifos, dando cada uno 40 litros por minuto. 
    ¿En cuántas horas llenará un pozo 6 veces el anterior?
    \vspace{\stretch{1}}

    \question[2] Para agilizar los cobros en una taquilla de un teatro, tras aplicar un 5\% de descuento y 21\% de IVA, 
    queremos que se paguen 50\euro{} por entrada en la taquilla. ¿Cuál será el precio real de la entrada? ¿Cuánto será el importe del IVA? 
    \vspace{\stretch{1}}

    \newpage 
    
    \question[2] Calcula el capital final que se obtiene al depositar 7.500\euro{} en una entidad bancaria al 7\% anual durante 10 años, 
    aplicando interés simple e interés compuesto. Calcula la diferencia e indica que tipo de interés nos conviene utilizar
    % \question[2] Opera y simplifica los resultados:
    %     \begin{parts}
    %         \part $\frac{5}{3} + \frac{4}{9} + \frac{4}{6}$
    %         \vspace{\stretch{1}}
           
    %         \part $\frac{15}{18} + \frac{3}{16} - \frac{4}{9}$
    %         \vspace{\stretch{1}}

    %         \part $\frac{5}{21} - \frac{4}{35} + \frac{4}{7}$
    %         \vspace{\stretch{1}}

    %         \part $\frac{1}{17} - \frac{2}{7} + \frac{4}{3}$
    %         \vspace{\stretch{1}}
    %     \end{parts} 
    % \newpage 

    % \question[2] Realiza las siguientes operaciones y simplifica el resultado:
    %     \begin{parts}
    %         % Repasar porque sale muy complejo
    %         \part ${\left(\frac{4}{3} + \frac{3}{5}\right)}^2 - {\left( \frac{3}{4} - \frac{5}{2} \right)}^3 $ 
    %         \vspace{\stretch{1}}

    %         % Repasar porque sale muy complejo
    %         \part ${\left(\frac{1}{4} - \frac{7}{12}\right)}^{-2} \cdot {\left(\frac{1}{3}-\frac{7}{9}\right)}^{-1} + {\left(\frac{2}{3} - 1\right)}^{2}$
    %         \vspace{\stretch{1}}
    %     \end{parts}
    % \newpage 

    % \question[2] Extrae todos los factores posibles, numéricos y literales, de estos radicales:
    %     \begin{parts}
    %         \part $\sqrt[3]{7^5}$ 
    %         \vspace{\stretch{1}}

    %         \part $\sqrt[3]{24}$ 
    %         \vspace{\stretch{1}}

    %         \part $\sqrt{x^2 y}$ 
    %         \vspace{\stretch{1}}

    %         \part $\sqrt[3]{x^6 y^3}$ 
    %         \vspace{\stretch{1}}

    %         \part $\sqrt{864}$ 
    %         \vspace{\stretch{1}}

    %         \part $\sqrt[4]{16x^8y^7z^3}$ 
    %         \vspace{\stretch{1}}
    %     \end{parts}

    % \question[2] Racionaliza y simplifica:
    %     \begin{parts}
    %         \part $\frac{3}{\sqrt{2}}$
    %         \vspace{\stretch{1}}

    %         \part $\frac{1}{2\sqrt{3}}$
    %         \vspace{\stretch{1}}
           
    %         \part $\frac{1}{\sqrt[5]{81}}$
    %         \vspace{\stretch{1}}

    %         % Repasar porque sale muy complejo
    %         \part $\sqrt[]{\frac{\sqrt[3]{a^{10}}}{\sqrt[6]{a^8}}}$
    %         \vspace{\stretch{1}}

    %     \end{parts}

    % \newpage 
    % \question[2] Realiza las siguientes operaciones y racionaliza si es necesario:
    %     \begin{parts}
    %         \part $ \frac{5}{\sqrt{2}} + \frac{3}{\sqrt{3}} $
    %         \vspace{\stretch{1}}

    %         \part $ \sqrt{12} - 3\sqrt{27} + 5 \sqrt{3} + \sqrt{75} $
    %         \vspace{\stretch{1}}
            
    %         % Repasar porque sale muy complejo
    %         \part $ \frac{3}{\sqrt{5}} - \frac{1}{\sqrt{2}} + \frac{3}{\sqrt{10}} $
    %         \vspace{\stretch{1}}

    %         % Repasar porque sale muy complejo
    %         \part $ \frac{4}{\sqrt{2}} + \frac{3}{\sqrt[4]{4}} - \frac{5}{\sqrt[6]{8}}$
    %         \vspace{\stretch{1}}
    %     \end{parts}
    % \newpage

\end{questions}
\end{document}