\documentclass[addpoints,spanish, 12pt,a4paper,cancelspace]{./include/gexam}
 
 %%%%%%%%%%%%%%%%%%%%%%%%%%%
 \renewcommand{\documentName} { 3ª evaluación }
 \renewcommand{\documentContent} { Recuperación } 
 \renewcommand{\waterMark} { } 

 % Configuración del documento.
 \renewcommand{\schoolSubject} { Examen Matemáticas 2º ESO  }
\renewcommand{\school} { IES José de Churriguera  }
\renewcommand{\academicPeriod} { Curso 2022/2023 }

\renewcommand{\autor} { Andrés Giménez Muñoz }
\renewcommand{\emailAuthor} { andresprofemates@outlook.es }
\renewcommand{\autorSing}{ Profesor: Andrés } 
 %%%%%%%%%%%%%%%%%%%%%%%%%%%
  
 % \renewcommand{\thepartno}{\arabic{partno}}
 \usepackage{nicefrac, xfrac}
 \usepackage{amsmath}

 %%%%%%%%%%%%%%%%%%%%%%%%%%%
 % Exam configuration
 %\pointsdroppedatright   %% No mostrar la puntuación
 \pointsinrightmargin{} % Para poner las puntuaciones a la derecha. Se puede cambiar. Si se comenta, sale a la izquierda.
 \extrawidth{-1.5cm} %Un poquito más de margen por si ponemos textos largos.
 \marginpointname{ \emph{\points}}
 
 %% Si se comenta no aparecerán los espacios de la solución.
 %\nocancelspace
 
 %% Esto es de la clase exam. Si dejamos sin comentar \printanswers, se mostraran las soluciones. 
 %% Si la comentamos y dejamos sin comentar \noprintanswers, pues no se muestran las soluciones.
 % \printanswers
 %\noprintanswers
 
 %%%%%%%%%%%%%%%%%%%%%%%%%%%
 
 \begin{document}
 
 \StudentData{}
 \GradeTableHeader{}
 
 \Large
 \justifying{}
 
 \checkboxchar{$\Box$}
 \checkedchar{$\blacksquare$}

 \begin{questions}
    \question[0] Pendiente de realizar
    % \question[3] Completa la siguiente tabla
    % \begin{center}
    %     % \begin{table}[]
    %         \begin{tabular}{|l|l|}
    %             \hline
    %             4º 36' 54'' & Cuatro grados, treinta y seis minutos y cincuenta y cuatro segundos \\ \hline
    %             5º 32' 32'' &                                                                     \\ \hline
    %             40º 54''    &                                                                     \\ \hline
    %                         & Doce grados, veintitrés minutos y cinco segundos                     \\ \hline
    %                         & Tres grados, doce minutos y cincuenta y cuatro segundos             \\ \hline
    %         \end{tabular}
    %     % \end{table}
    % \end{center}

    % \vspace{2cm}

    % \question[4] Suma las siguientes medidas de ángulos y simplifícala.
    % \begin {parts} 
    % \begin{multicols}{2}
    %     \part
    %     \begin{center}
    %         % \begin{table}[]
    %             \begin{tabular}{llllll}
    %                 & 5º &  & 45' &  & 15'' \\
    %             + & 8º &  & 43' &  & 55'' \\ \hline
    %             \end{tabular}
    %         % \end{table}
    %     \end{center}
    %     \vspace{\stretch{1}}
    %     \part
    %     \begin{center}
    %         % \begin{table}[]
    %             \begin{tabular}{llllll}
    %                 & 15º &  & 5' &  & 54'' \\
    %             + & 3º &  & 57' &  & 32'' \\ \hline
    %             \end{tabular}
    %         % \end{table}
    %     \end{center}
    %     \vspace{\stretch{1}}
    % \end{multicols}
    % \end{parts}

    % \newpage 
    % \question[3] Indica el nombre y calcula el área y el perímetro del siguiente polígono.

    % \begin {parts} 
    %     \part   
    %     \begin{minipage}{\linewidth}
    %         % \centering
    %         \includegraphics[width=7cm]{Romboide01}
    %     \end{minipage}
    %     \vspace{\stretch{1}}
    %     \part   
    %     \begin{minipage}{\linewidth}
    %         % \centering
    %         \includegraphics[width=3cm]{Rombo01}
    %     \end{minipage}
    %     \vspace{\stretch{1}}
    %     \part   
    %     \begin{minipage}{\linewidth}
    %         % \centering
    %         \includegraphics[width=5cm]{Rectangulo01}
    %     \end{minipage}
    %     \vspace{\stretch{1}}
    % \end{parts}

    % \newpage
    
\end{questions}
 
\end{document}