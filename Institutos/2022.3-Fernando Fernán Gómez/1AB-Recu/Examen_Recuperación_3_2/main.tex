\documentclass[addpoints,spanish, 12pt,a4paper,cancelspace]{./include/gexam}
 
 %%%%%%%%%%%%%%%%%%%%%%%%%%%
 \renewcommand{\documentName} { 3ª evaluación }
 \renewcommand{\documentContent} { Sistema métrico } 
 \renewcommand{\waterMark} { } 

 % Configuración del documento.
 \renewcommand{\schoolSubject} { Examen Matemáticas 2º ESO  }
\renewcommand{\school} { IES José de Churriguera  }
\renewcommand{\academicPeriod} { Curso 2022/2023 }

\renewcommand{\autor} { Andrés Giménez Muñoz }
\renewcommand{\emailAuthor} { andresprofemates@outlook.es }
\renewcommand{\autorSing}{ Profesor: Andrés } 
 %%%%%%%%%%%%%%%%%%%%%%%%%%%
  
 % \renewcommand{\thepartno}{\arabic{partno}}
 \usepackage{nicefrac, xfrac}
 
 %%%%%%%%%%%%%%%%%%%%%%%%%%%
 % Exam configuration
 %\pointsdroppedatright   %% No mostrar la puntuación
 \pointsinrightmargin{} % Para poner las puntuaciones a la derecha. Se puede cambiar. Si se comenta, sale a la izquierda.
 \extrawidth{-1.5cm} %Un poquito más de margen por si ponemos textos largos.
 \marginpointname{ \emph{\points}}
 
 %% Si se comenta no aparecerán los espacios de la solución.
 %\nocancelspace
 
 %% Esto es de la clase exam. Si dejamos sin comentar \printanswers, se mostraran las soluciones. 
 %% Si la comentamos y dejamos sin comentar \noprintanswers, pues no se muestran las soluciones.
 % \printanswers
 %\noprintanswers
 
 %%%%%%%%%%%%%%%%%%%%%%%%%%%
 
 \begin{document}
 
 \StudentData{}
 \GradeTableHeader{}
 
 \justifying{}
 
 \checkboxchar{$\Box$}
 \checkedchar{$\blacksquare$}

 \begin{questions}
    \question[2] Selecciona las opciones correctas (pueden ser más de una):
    \begin {parts} 
        \part Unidades de longitud \\
        \begin{oneparcheckboxes}
            \choice metro
            \choice kilómetro
            \choice miligramo
            \choice hectolitro 
        \end{oneparcheckboxes}

        \part Unidades de peso \\ \\
        \begin{oneparcheckboxes}
            \choice miligramo
            \choice litro
            \choice metro cubico
            \choice hectogramo 
        \end{oneparcheckboxes}

        \part Unidades de masa
        \begin{oneparcheckboxes}
            \choice miligramo
            \choice metro cubico
            \choice decametro
            \choice quintal métrico 
        \end{oneparcheckboxes}

        \part Unidades de capacidad
        \begin{oneparcheckboxes}
            \choice kilolitro
            \choice milimetro
            \choice metro cuadrado
            \choice decilitro
        \end{oneparcheckboxes}

        \part Unidades de area
        \begin{oneparcheckboxes}
            \choice decimetro
            \choice tonelada metrica
            \choice decimetro cuadrado
            \choice decigramo
        \end{oneparcheckboxes}
    \end{parts}

    % \question Se han tomado las siguientes medidas de temperatura a lo largo de 10 días.
    %     \begin{center}
    %     \begin{tabular}{ |l|l|l|l|l| }
    %         \hline
    %         19ºC & 20ºC & 20ºC & 19ºC & 19ºC \\ \hline
    %         19ºC & 18ºC & 17ºC & 16ºC & 16ºC \\ \hline
    %     \end{tabular}
    %     \end{center}

    %     \begin {parts} 
    %         \part[2]
    %         Completa la siguiente tabla de frecuencia:
    %         \vspace{5mm}
    %         \begin{center}
    %             % \begin{table}[]
    %                 \begin{tabular}{|r|l|l|}
    %                 \hline
    %                 \multicolumn{1}{|l|}{Temperaturas} & Frecuencia absoluta & Frecuencia relativa \\ \hline
    %                 16º                                &                     &                     \\ \hline
    %                 17º                                &                     &                     \\ \hline
    %                 18º                                &                     &                     \\ \hline
    %                 19º                                &                     &                     \\ \hline
    %                 20º                                &                     &                     \\ \hline
    %                 \end{tabular}
    %             % \end{table}
    %         \end{center}
    %         \vspace{5mm}

    %         \part[2] Calcula la media de las temperaturas:
    %         \vspace{\stretch{3}}

    %         \part[1] Calcula la moda de las temperaturas:
    %         \vspace{\stretch{2}}
    %     \end{parts}
    % \newpage{}
    
    % \question Luisa tiene en su estuche 3 bolígrafos negros, 2 azules, 4 rojos y uno verde. 
    %         Si saca uno al azar:
    %     \begin {parts} 
    %         \part[2] ¿Cuál es la probabilidad de sacar un bolígrafo rojo?
    %             \vspace{\stretch{2}}
    %         \part[2] Teniendo en cuenta que solo se puede hacer el examen con bolígrafo azul o negro. 
    %         ¿Cuál es la probabilidad de que pueda hacer el examen con el bolígrafo que ha sacado?
    %             \vspace{\stretch{2}}
    %     \end{parts}
        
    % \question[1] Indica cuál es la probabilidad de:
    % \begin {parts} 
    %     \part Suceso seguro:
    %         \vspace{\stretch{1}}
    %     \part Suceso imposible: 
    %         \vspace{\stretch{1}}
    % \end{parts}
\end{questions}
 
\end{document}