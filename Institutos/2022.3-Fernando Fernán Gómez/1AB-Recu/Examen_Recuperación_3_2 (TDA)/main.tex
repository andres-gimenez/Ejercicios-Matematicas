\documentclass[addpoints,spanish, 12pt,a4paper,cancelspace]{./include/gexam}
 
 %%%%%%%%%%%%%%%%%%%%%%%%%%%
 \renewcommand{\documentName} { 3ª evaluación }
 \renewcommand{\documentContent} { Sistema métrico } 
 \renewcommand{\waterMark} { TDA } 

 % Configuración del documento.
 \renewcommand{\schoolSubject} { Examen Matemáticas 2º ESO  }
\renewcommand{\school} { IES José de Churriguera  }
\renewcommand{\academicPeriod} { Curso 2022/2023 }

\renewcommand{\autor} { Andrés Giménez Muñoz }
\renewcommand{\emailAuthor} { andresprofemates@outlook.es }
\renewcommand{\autorSing}{ Profesor: Andrés } 
 %%%%%%%%%%%%%%%%%%%%%%%%%%%
  
 % \renewcommand{\thepartno}{\arabic{partno}}
 \usepackage{nicefrac, xfrac}
 \usepackage{amsmath}

 %%%%%%%%%%%%%%%%%%%%%%%%%%%
 % Exam configuration
 %\pointsdroppedatright   %% No mostrar la puntuación
 \pointsinrightmargin{} % Para poner las puntuaciones a la derecha. Se puede cambiar. Si se comenta, sale a la izquierda.
 \extrawidth{-1.5cm} %Un poquito más de margen por si ponemos textos largos.
 \marginpointname{ \emph{\points}}
 
 %% Si se comenta no aparecerán los espacios de la solución.
 %\nocancelspace
 
 %% Esto es de la clase exam. Si dejamos sin comentar \printanswers, se mostraran las soluciones. 
 %% Si la comentamos y dejamos sin comentar \noprintanswers, pues no se muestran las soluciones.
 % \printanswers
 %\noprintanswers
 
 %%%%%%%%%%%%%%%%%%%%%%%%%%%
 
 \begin{document}
 
 \StudentData{}
 \GradeTableHeader{}
 
 \Large
 \justifying{}
 
 \checkboxchar{$\Box$}
 \checkedchar{$\blacksquare$}

 \begin{questions}
    \question[2] Selecciona las opciones correctas (en cada caso pueden ser ninguna o más de una):
    \begin {parts} 
        \part Unidades de longitud \\ \\
        \begin{oneparcheckboxes}
            \choice metro
            \choice kilómetro
            \choice miligramo
            \choice hectolitro 
        \end{oneparcheckboxes}
        \\

        \part Unidades de peso \\ \\
        \begin{oneparcheckboxes}
            \choice miligramo
            \choice litro
            \choice metro cubico
            \choice hectogramo 
        \end{oneparcheckboxes}
        \\

        \part Unidades de masa \\ \\
        \begin{oneparcheckboxes}
            \choice miligramo
            \choice metro cubico \\
            \choice decametro
            \choice quintal métrico 
        \end{oneparcheckboxes}
        \\

        \part Unidades de capacidad \\ \\
        \begin{oneparcheckboxes}
            \choice kilolitro
            \choice milimetro  \\
            \choice metro cuadrado
            \choice decilitro
        \end{oneparcheckboxes}
        \\

        \part Unidades de area \\ \\
        \begin{oneparcheckboxes}
            \choice decimetro
            \choice tonelada metrica \\
            \choice decimetro cuadrado
            \choice decigramo
        \end{oneparcheckboxes}
        \\

        \part Unidades de volumen \\ \\
        \begin{oneparcheckboxes}
            \choice decimetro cubico
            \choice metro cuadrado
            \choice gramo
            \choice area
        \end{oneparcheckboxes}
        \\ \\
    \end{parts}

    \newpage

    \question[2] Completa la tabla de los múltiplos y submúltiplos de las unidades de longitud:
    \\
    \begin{center}
    % \begin{table}[]
        \begin{tabular}{|l|l|l|l|}
        \hline
        \multirow{4}{*}{Múltiplos}             & Miriámetro & \multicolumn{1}{r|}{Mam} & \multicolumn{1}{r|}{10.000 m} \\ \cline{2-4} 
                                               &            &     &          \\ \cline{2-4} 
                                               &            &     &          \\ \cline{2-4} 
                                               &            &     &          \\ \hline
        \multicolumn{1}{|r|}{Unidad principal} & Metro      & \multicolumn{1}{r|}{m} &  \multicolumn{1}{r|}{1 m} \\ \hline
        \multirow{3}{*}{Submúltiplos}          &            &     &          \\ \cline{2-4} 
                                               &            &     &          \\ \cline{2-4} 
                                               &            &     &          \\ \hline
        \end{tabular}
        % \end{table}
    \end{center}

    \vspace{5mm}
    
    \question[6] Realiza las siguientes conversiones de unidades:
    \begin {parts} 
        \part $2.000 \text{ m}$ = \fillin[2] $\text{ Km}$
        \vspace{5mm}

        \part $2.000 \text{ cm}$ = \fillin[2]  $\text{ m}$
        \vspace{5mm}

        \part $0,1 \text{ km}$ = \fillin[100]  $\text{ m}$
        \vspace{5mm}

        \part $300 \text{ Hm}$ = \fillin[0,3]  $\text{ Km}$
        \vspace{5mm}

        \part $85 \text{ m}^2$ que mide una vivienda = \fillin[850.000] $\text{ cm}^2$
        \vspace{5mm}

        \part $19,46 \text{ Km}^2$ municio de Humanes de Madrid = \fillin[19.460.000] $\text{ m}^2$
        \vspace{5mm}

        \part $18 \text{ m}^3$ de agua en un piscina = \fillin[1.800.000] $\text{ cm}^3$
        \vspace{5mm}

        \part $25 \text{ Hm}^3$ de agua embalsada en un embalse = \fillin[25.000.000] $\text{ m}^3$
        \vspace{5mm}

        \part $5 \text{ TM}$ que pesa un camión = \fillin[5.000]  $\text{ Kg}$
        \vspace{5mm}

        \part $100 \text{g}$ de de pipas de girasol = \fillin[0,1]  $\text{ Kg}$ 
        \vspace{5mm}

        \part $2 \text{2 l}$ de una botella refresco = \fillin[200]  $\text{ cl}$
        \vspace{5mm}

        \part $30 \text{Hl}$ de un tanque de leche = \fillin[3.000]  $\text{ l}$
        \vspace{5mm}
    \end{parts}
    
\end{questions}
 
\end{document}