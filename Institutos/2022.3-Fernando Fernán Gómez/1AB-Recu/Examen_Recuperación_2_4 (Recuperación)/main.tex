 %%%%%%%%%%%%%%%%%%%%%%%%%%%
 \newcommand{\documentName} { 2ª evaluación }
 \newcommand{\documentContent} { Recuperación } 
 \newcommand{\waterMark} { } 
 %%%%%%%%%%%%%%%%%%%%%%%%%%%
 
 % Configuración del documento.
 \newcommand{\schoolSubject} { Matemáticas 3º ESO - Recuperación}
\newcommand{\school} { IES La Serna }
\newcommand{\academicPeriod} { Curso 2020/2021 }


\newcommand{\autor} { Andrés Giménez Muñoz }
\newcommand{\emailAuthor} { agimenezmunoz@ieslaserna.com }
\newcommand{\autorSing}{ Profesores: Andrés } 
 \renewcommand{\schoolSubject} { Examen Matemáticas 2º ESO  }
\renewcommand{\school} { IES José de Churriguera  }
\renewcommand{\academicPeriod} { Curso 2022/2023 }

\renewcommand{\autor} { Andrés Giménez Muñoz }
\renewcommand{\emailAuthor} { andresprofemates@outlook.es }
\renewcommand{\autorSing}{ Profesor: Andrés } 
 
 % \renewcommand{\thepartno}{\arabic{partno}}
 \usepackage{nicefrac, xfrac}
 
 %%%%%%%%%%%%%%%%%%%%%%%%%%%
 % Exam configuration
 %\pointsdroppedatright   %% No mostrar la puntuación
 \pointsinrightmargin{} % Para poner las puntuaciones a la derecha. Se puede cambiar. Si se comenta, sale a la izquierda.
 \extrawidth{-1.5cm} %Un poquito más de margen por si ponemos textos largos.
 \marginpointname{ \emph{\points}}
 
 %% Si se comenta no aparecerán los espacios de la solución.
 %\nocancelspace
 
 %% Esto es de la clase exam. Si dejamos sin comentar \printanswers, se mostraran las soluciones. 
 %% Si la comentamos y dejamos sin comentar \noprintanswers, pues no se muestran las soluciones.
 % \printanswers
 %\noprintanswers
 
 %%%%%%%%%%%%%%%%%%%%%%%%%%%
 
 \begin{document}
 
 \StudentData{}
 \GradeTableHeader{}
 
 \justifying{}
 
 \begin{questions}
    \question[5] Ordena los siguientes fracciones
        \begin {parts} 
            \part $\left\lbrace -\frac{3}{4}, \frac{4}{5}, \frac{4}{6}, -\frac{3}{2} \right\rbrace$
            \vspace{\stretch{1}}

            \part $\left\lbrace -\frac{1}{3}, \frac{2}{6}, \frac{3}{9} \right\rbrace$
            \vspace{\stretch{1}}

            \part $\left\lbrace -\frac{5}{3}, \frac{5}{2}, \frac{5}{4} \right\rbrace$
            \vspace{\stretch{1}}
        \end{parts}
    \newpage{}
    
    \question[5] Realiza las siguientes operaciones con fracciones:
        \begin {parts} 
            \part $\frac{5}{6} + \frac{4}{15}$
            \vspace{\stretch{1}}
            \part $5 - \frac{1}{2}$
            \vspace{\stretch{1}}
            \part $\frac{2}{3} - \frac{1}{5} + \frac{4}{15}$
            \vspace{\stretch{1}}
        \end{parts}
   
    \end{questions}
 
\end{document}