\documentclass[addpoints,spanish, 12pt,a4paper,cancelspace]{./include/gexam}

 %%%%%%%%%%%%%%%%%%%%%%%%%%%
 \renewcommand{\documentName} { 3ª evaluación }
 \renewcommand{\documentContent} { Lenguaje algebraico y ecuaciones } 
 \renewcommand{\waterMark} { Modelo A } 

 % Configuración del documento.
 \renewcommand{\schoolSubject} { Examen Matemáticas 2º ESO  }
\renewcommand{\school} { IES José de Churriguera  }
\renewcommand{\academicPeriod} { Curso 2022/2023 }

\renewcommand{\autor} { Andrés Giménez Muñoz }
\renewcommand{\emailAuthor} { andresprofemates@outlook.es }
\renewcommand{\autorSing}{ Profesor: Andrés } 
 %%%%%%%%%%%%%%%%%%%%%%%%%%%
 
% \renewcommand{\thepartno}{\arabic{partno}}
%  \renewcommand{\thepartno}{\thecurrentpartno.\arabic{partno}}

% \renewcommand{\partlabel}{(\thequestion.\arabic{partno})}
% \renewcommand{\subpartlabel}{(\thepart.\arabic{subpartno})}

\renewcommand\subpartlabel{(\thesubpart)}
\renewcommand\subpartshook{\renewcommand\makelabel[1]{##1\hfil }} 
 
 %%%%%%%%%%%%%%%%%%%%%%%%%%%
 % Exam configuration
 %\pointsdroppedatright   %% No mostrar la puntuación
 \pointsinrightmargin{} % Para poner las puntuaciones a la derecha. Se puede cambiar. Si se comenta, sale a la izquierda.
 \extrawidth{-1.5cm} %Un poquito más de margen por si ponemos textos largos.
 \marginpointname{ \emph{\points}}
 
 %% Si se comenta no aparecerán los espacios de la solución.
 %\nocancelspace
 
 %% Puntuación a la izquierda.
%  \nopointsinrightmargin 

 %% Esto es de la clase exam. Si dejamos sin comentar \printanswers, se mostraran las soluciones. 
 %% Si la comentamos y dejamos sin comentar \noprintanswers, pues no se muestran las soluciones.
 % \printanswers
 %\noprintanswers
 
 %%%%%%%%%%%%%%%%%%%%%%%%%%%
 
 \begin{document}
 
 \StudentData{}
 \GradeTableHeader{}
 
 \justifying{}

 \justifying

% \begin{center}
%     \fbox{\fbox{\parbox{6.5in}{             
%                 \begin{itemize}
%                     \item Deben aparecer todas las operaciones, no vale solo con indicar el resultado.
%                     \item Se podrán quitar hasta cinco décimas por falta de claridad o rigor en el desarrollo de las respuestas o por una mala presentación.
%                     \item Se valorará que se indiquen las cuentas en línea, realizando las operaciones en el margen.
%                     \item No se puede utilizar la calculadora.
%                 \end{itemize}
%             }}}
% \end{center}
 
 \begin{questions}
    
    \question[1] \begin{tikzpicture}[scale=1]
        \begin{axis}[
                %title={$f(x)=x^2-10$},
                axis lines=middle,
                axis line style={<->},
                legend style={inner ysep=7pt},
                %x label style={at={(axis description cs:0.5,-0.06)},anchor=north},
                %y label style={at={(axis description cs:-0.06,.5)},rotate=90,anchor=south},
                %xlabel={Nº de bombillas},ylabel={Vida en horas},
                no markers,
                black!50,
                grid,
                width=15cm,
                xmax=20,ymax=20,xmin=-20,ymin=-20,
                %axis lines=middle
                % restrict y to domain=-20:20
            ]

            % \addplot[domain=10:15,black] {x-10};

            % \addplot[domain=-5:0,black] {x*x+6*x};
            % \addplot[domain=0:5,black] {x*x-6*x};

            % \addplot[domain=-5:0,black] {-x};
            % \addplot[domain=-9:-5,black] {x*x+15*x+55};
            
            % \addplot[domain=6:15,black] {-5};
            % \draw[dashed] (axis cs:6,6) -- (axis cs:6,-5);
            % \addplot[holdotBlack] coordinates{(-9,1)(5,-5)(15,5)};
            % \addplot[soldotBlack] coordinates{(10,0)};
        \end{axis}
    \end{tikzpicture}

\end{questions}
 
\end{document}