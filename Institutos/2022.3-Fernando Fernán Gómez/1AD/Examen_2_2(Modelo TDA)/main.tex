 %%%%%%%%%%%%%%%%%%%%%%%%%%%
 \newcommand{\documentName} { 2ª evaluación }
 \newcommand{\documentContent} { Fracciones y proporcionalidad } 
 \newcommand{\waterMark} { Modelo TDA } 
 %%%%%%%%%%%%%%%%%%%%%%%%%%%
 
 % Configuración del documento.
 \newcommand{\schoolSubject} { Matemáticas 3º ESO - Recuperación}
\newcommand{\school} { IES La Serna }
\newcommand{\academicPeriod} { Curso 2020/2021 }


\newcommand{\autor} { Andrés Giménez Muñoz }
\newcommand{\emailAuthor} { agimenezmunoz@ieslaserna.com }
\newcommand{\autorSing}{ Profesores: Andrés } 
 \renewcommand{\schoolSubject} { Examen Matemáticas 2º ESO  }
\renewcommand{\school} { IES José de Churriguera  }
\renewcommand{\academicPeriod} { Curso 2022/2023 }

\renewcommand{\autor} { Andrés Giménez Muñoz }
\renewcommand{\emailAuthor} { andresprofemates@outlook.es }
\renewcommand{\autorSing}{ Profesor: Andrés } 
 
 % \renewcommand{\thepartno}{\arabic{partno}}
 \usepackage{nicefrac, xfrac}
 
 %%%%%%%%%%%%%%%%%%%%%%%%%%%
 % Exam configuration
 %\pointsdroppedatright   %% No mostrar la puntuación
 \pointsinrightmargin{} % Para poner las puntuaciones a la derecha. Se puede cambiar. Si se comenta, sale a la izquierda.
 \extrawidth{-1.5cm} %Un poquito más de margen por si ponemos textos largos.
 \marginpointname{ \emph{\points}}
 
 %% Si se comenta no aparecerán los espacios de la solución.
 %\nocancelspace
 
 %% Esto es de la clase exam. Si dejamos sin comentar \printanswers, se mostraran las soluciones. 
 %% Si la comentamos y dejamos sin comentar \noprintanswers, pues no se muestran las soluciones.
 % \printanswers
 %\noprintanswers
 
 %%%%%%%%%%%%%%%%%%%%%%%%%%%
 
 \begin{document}
 
 \Large

 \StudentData{}
 \GradeTableHeader{}
 
 \justifying{}
 
 \begin{questions}
    \question[3] Realiza las siguientes operaciones con fracciones y simplifica el resultado lo máximo posible:
        \\
        (*) Recuerda que con las fracciones no se pueden utilizar decimales: 
        \begin {parts}
            \part $\frac{11}{8} + \frac{5}{6} - \frac{4}{3}$
            \vspace{\stretch{1}}

            \part $\frac{2}{3} \cdot \left(\frac{7}{6} + \frac{11}{15}\right)$
            \vspace{\stretch{1}}

            \newpage{}

            \part $\frac{2}{3} : \left(\frac{1}{6} + \frac{1}{2}\right)$
            \vspace{\stretch{1}}

            \part $\left(\frac{3}{4} \right)^3 \cdot \left(\frac{3}{4} \right)^5 $
            \vspace{\stretch{1}}

            \newpage{}

            \part $\left(\frac{5}{3} \right)^7 : \left[\left(\frac{5}{3} \right)^2\right]^3 $
            \vspace{\stretch{1}}

            \part $\sqrt{\frac{16}{9}}$
            \vspace{\stretch{1}}
        \end{parts}

    \newpage{}

    \question[1] Indica si las siguientes razones con proporcionales:
    \\
    (*) Recuerda que con las razones si se pueden utilizar decimales: 
    \begin {parts}
        \part $\frac{4}{9}$ y $\frac{5}{12}$
        \vspace{\stretch{3}}
        \part $\frac{10}{2}$ y $\frac{15}{3}$
        \vspace{\stretch{3}}
    \end{parts}

    \question[1] Calcula el valor de R para que las siguientes razones sean proporcionales:
    \begin {parts}
        \part $\frac{21}{7} = \frac{R}{2}$
        \vspace{\stretch{3}}

        \part $\frac{4}{6} = \frac{R}{9}$
        \vspace{\stretch{3}}

        \part $\frac{8}{12} = \frac{R}{9}$
        \vspace{\stretch{3}}
    \end{parts}
    \newpage{}
    
    \question[1] Indica si las siguientes magnitudes guardan una relación de proporcionalidad y si esta es directa o inversa.
    \begin {parts}
        \part Peso de pintura con metros cuadrados que podemos pintar con ella.
        \vspace{\stretch{1}}
        % \part Número de ejercicios realizados y nota obtenida en el examen de matemáticas.
        \part Kilogramos de pienso y número de días que puedo alimentar a mi perro.
        \vspace{\stretch{1}}
        \part Al construir un muro de ladrillo, el número de personas trabajando y el tiempo que se tarda en construirlo.
        \vspace{\stretch{1}}
        \part Nota obtenida en el examen de matemáticas con el precio del bocadillo de lomo en la cafetería del instituto.
        \vspace{\stretch{1}}
        \part Velocidad de un automóvil y tiempo en llegar a su destino.
        \vspace{\stretch{1}}
    \end{parts}

    \newpage{}

    \question[2] El precio de 12 fotocopias es de 0,50\euro{}. ¿Cuánto costará hacer 30 fotocopias?
    \\
    (*) Se ha de resolver el problema mediante una regla de tres.
    \vspace{\stretch{1}}
    
    \question[2] Diez barras de pan cuestan 4,75\euro{}. ¿Cuánto costarán 18 barras? ¿Y 24 barras? 
    \\
    (*) Se ha de resolver el problema mediante una regla de tres.
    \vspace{\stretch{1}}

\end{questions}
 
\end{document}