\documentclass[addpoints,spanish, 12pt,a4paper,cancelspace]{./include/gexam}

 %%%%%%%%%%%%%%%%%%%%%%%%%%%
 \renewcommand{\documentName} { 3ª evaluación }
 \renewcommand{\documentContent} { Recuperación } 
 \renewcommand{\waterMark} {  } 

 % Configuración del documento.
 \renewcommand{\schoolSubject} { Examen Matemáticas 2º ESO  }
\renewcommand{\school} { IES José de Churriguera  }
\renewcommand{\academicPeriod} { Curso 2022/2023 }

\renewcommand{\autor} { Andrés Giménez Muñoz }
\renewcommand{\emailAuthor} { andresprofemates@outlook.es }
\renewcommand{\autorSing}{ Profesor: Andrés } 
 %%%%%%%%%%%%%%%%%%%%%%%%%%%
 
% \renewcommand{\thepartno}{\arabic{partno}}
%  \renewcommand{\thepartno}{\thecurrentpartno.\arabic{partno}}

% \renewcommand{\partlabel}{(\thequestion.\arabic{partno})}
% \renewcommand{\subpartlabel}{(\thepart.\arabic{subpartno})}

\renewcommand\subpartlabel{(\thesubpart)}
\renewcommand\subpartshook{\renewcommand\makelabel[1]{##1\hfil }} 
 
 %%%%%%%%%%%%%%%%%%%%%%%%%%%
 % Exam configuration
 %\pointsdroppedatright   %% No mostrar la puntuación
 \pointsinrightmargin{} % Para poner las puntuaciones a la derecha. Se puede cambiar. Si se comenta, sale a la izquierda.
 \extrawidth{-1.5cm} %Un poquito más de margen por si ponemos textos largos.
 \marginpointname{ \emph{\points}}
 
 %% Si se comenta no aparecerán los espacios de la solución.
 %\nocancelspace
 
 %% Puntuación a la izquierda.
%  \nopointsinrightmargin 

 %% Esto es de la clase exam. Si dejamos sin comentar \printanswers, se mostraran las soluciones. 
 %% Si la comentamos y dejamos sin comentar \noprintanswers, pues no se muestran las soluciones.
 % \printanswers
 %\noprintanswers
 %%%%%%%%%%%%%%%%%%%%%%%%%%%
 
 \begin{document}
 
 \StudentData{}
 \GradeTableHeader{}
 
 \justifying{}

 \justifying

\begin{center}
    \fbox{\fbox{\parbox{6.5in}{             
                \begin{itemize}
                    \item Copiar, hablar, levantarse de la silla o molestar a al resto de la clase pueden ser motivos de la retirada del examen que se valorará con un cero.
                    \item Deben aparecer todas las operaciones, no vale solo con indicar el resultado.
                    \item Se valorará que se indiquen las cuentas en línea o columna, realizando las operaciones en el margen.
                    \item No se puede utilizar la calculadora.
                \end{itemize}
            }}}
\end{center}
 
 \begin{questions}

    \question[2] Realiza las siguientes operaciones:
    \begin {parts}
        \part $3x^2+2x^2$
        \vspace{\stretch{1}}
        \part $2x \cdot \left(-7x^5\right)$
        \vspace{\stretch{1}}
        \part $7x^3 \cdot 4x^2$
        \vspace{\stretch{1}}

        \part $2x^2-2x+1 + 2x^2+3x-3$
        \vspace{\stretch{1}}
    \end {parts}

    \newpage

    \question[1] Expresa en lenguaje algebraico:
    \begin {parts}
        \part El cuadrado de un número.
        \vspace{\stretch{1}}

        \part La tercera parte de un número.
        \vspace{\stretch{1}}

        \part Un número menos su cuadrado.
        \vspace{\stretch{1}}

        \part Un número más tres unidades.
        \vspace{\stretch{1}}
    \end {parts}
    
    \question[1] Halla el valor numérico de las siguientes expresiones algebraicas con los valores dados:
    \begin {parts}
        \part $2x^2+x-15$ para $x=3$
        \vspace{\stretch{3}}
    \end {parts}

    \newpage{}

    \question[2] Resuelve las siguientes ecuaciones
    \begin {parts}
        \part $5x-8=3x$
        \vspace{\stretch{1}}
    
        \part $x-2\left(x-2\right) = x-4$
        \vspace{\stretch{1}}

        \part $3x-3=2x+10$
        \vspace{\stretch{1}}
    \end {parts}

    \newpage

    \question[1] En una zapatería ves unos zapatos que cuestan 65\euro{} con una rebaja del 15\%, ¿Cuánto cuestan los zapatos?
    \vspace{\stretch{1}}

    \question[1\half] Tres obreros realizan un trabajo en 5 horas. ¿En cuánto tiempo realizarán dos obreros el mismo trabajo?
    \vspace{\stretch{2}}

    \question[1\half] En el Gran Hotel del Mar, durante el invierno, hay 3 jardineros. 
    Entre todos, riegan y cuidan todos los jardines del hotel en 6 horas. 
    Si durante el verano hay 3 jardineros más, ¿en cuánto tiempo regarán y cuidarán los jardines del hotel entre todos?
    \vspace{\stretch{2}}

\end{questions}
 
\end{document}