\documentclass[addpoints,spanish, 12pt,a4paper,cancelspace]{./include/gexam}

 %%%%%%%%%%%%%%%%%%%%%%%%%%%
 \renewcommand{\documentName} { 3ª evaluación }
 \renewcommand{\documentContent} {  Recuperación } 
 \renewcommand{\waterMark} {  } 

 % Configuración del documento.
 \renewcommand{\schoolSubject} { Examen Matemáticas 2º ESO  }
\renewcommand{\school} { IES José de Churriguera  }
\renewcommand{\academicPeriod} { Curso 2022/2023 }

\renewcommand{\autor} { Andrés Giménez Muñoz }
\renewcommand{\emailAuthor} { andresprofemates@outlook.es }
\renewcommand{\autorSing}{ Profesor: Andrés } 
 %%%%%%%%%%%%%%%%%%%%%%%%%%%
 
% \renewcommand{\thepartno}{\arabic{partno}}
%  \renewcommand{\thepartno}{\thecurrentpartno.\arabic{partno}}

% \renewcommand{\partlabel}{(\thequestion.\arabic{partno})}
% \renewcommand{\subpartlabel}{(\thepart.\arabic{subpartno})}

\renewcommand\subpartlabel{(\thesubpart)}
\renewcommand\subpartshook{\renewcommand\makelabel[1]{##1\hfil }} 
 
 %%%%%%%%%%%%%%%%%%%%%%%%%%%
 % Exam configuration
 %\pointsdroppedatright   %% No mostrar la puntuación
 \pointsinrightmargin{} % Para poner las puntuaciones a la derecha. Se puede cambiar. Si se comenta, sale a la izquierda.
 \extrawidth{-1.5cm} %Un poquito más de margen por si ponemos textos largos.
 \marginpointname{ \emph{\points}}
 
 %% Si se comenta no aparecerán los espacios de la solución.
 %\nocancelspace
 
 %% Puntuación a la izquierda.
%  \nopointsinrightmargin 

 %% Esto es de la clase exam. Si dejamos sin comentar \printanswers, se mostraran las soluciones. 
 %% Si la comentamos y dejamos sin comentar \noprintanswers, pues no se muestran las soluciones.
 % \printanswers
 %\noprintanswers
 
 %%%%%%%%%%%%%%%%%%%%%%%%%%%
 
 \begin{document}
 
 \StudentDataC{}
 \GradeTableHeader{}
 
 \justifying{}

 \justifying

% \begin{center}
%     \fbox{\fbox{\parbox{6.5in}{             
%                 \begin{itemize}
%                     \item Deben aparecer todas las operaciones, no vale solo con indicar el resultado.
%                     \item Se podrán quitar hasta cinco décimas por falta de claridad o rigor en el desarrollo de las respuestas o por una mala presentación.
%                     \item Se valorará que se indiquen las cuentas en línea, realizando las operaciones en el margen.
%                     \item No se puede utilizar la calculadora.
%                 \end{itemize}
%             }}}
% \end{center}
 
 \begin{questions}

    \question[0] Pendiente
    % \question[2] Reduce las siguientes expresiones algebraicas todo lo que se pueda:
    % \begin {parts}
    %     \part $\frac{1}{4}x^2+\frac{1}{8}x^2 = $
    %     \vspace{\stretch{1}}
        
    %     \part $2x^2+3x-2-6x^2+x+7 = $
    %     \vspace{\stretch{1}}
    
    %     \part $5x \cdot 2x^2 = $
    %     \vspace{\stretch{1}}

    %     % \part $3x - 2 \left(5x^2 - 2x +3\right) = $
    %     % \vspace{\stretch{1}}

    % \end {parts}

    % \question[2] Resuelve las siguientes ecuaciones
    % \begin {parts}
    %     \part $2x-5 = 3$
    %     \vspace{\stretch{2}}

    %     \part $3 + 2(x-3)= 1$
    %     \vspace{\stretch{2}}

    %     \part $2x-(3x-4)=x+2$
    %     \vspace{\stretch{2}}
    % \end{parts}

    % \newpage 
    % \question Sobre la siguiente sucesión
    % \\
    % \begin{minipage}{\linewidth}
    %      \centering
    %      \includegraphics[width=12cm]{secuencia01}
    %  \end{minipage}
    % \begin {parts}
    %     \part[1] Escribe los cinco primeros términos de la sucesión numérica que 
    %     representa la cantidad de cerillas en cada posición
    %     \vspace{\stretch{1}}

    %     \part[2] Juan dice que la expresión algebraica que indica el número de 
    %     cerillas en cada posición es $4+3\left(n-1\right)+2$, 
    %     pero Nuria dice que es $n+n+\left(n+1\right)+2$
    %     ¿quién tiene razón? ¿pueden tener razón los dos? 
    %     \\ \\ Reduce las expresiones algebraicas del apartado anterior 
    %     a la forma más sencilla posible y compara las.
    %     \vspace{\stretch{1}}

    %     \part[1] ¿Cuántas cerillas tiene la figura que se encuentra en la posición 120?
    %     \vspace{\stretch{1}}

    %     \part[2] ¿En qué posición se encuentra la figura que tiene 159 cerillas?
    %     \vspace{\stretch{1}}
    % \end {parts}

\end{questions}
 
\end{document}