\documentclass[addpoints,spanish, 12pt,a4paper,cancelspace]{./include/gexam}

 %%%%%%%%%%%%%%%%%%%%%%%%%%%
 \renewcommand{\documentName} { Examen global 3ª evaluación }
 \renewcommand{\documentContent} { Lenguaje algebraico y ecuaciones de 1º grado } 
 \renewcommand{\waterMark} { Modelo A } 

 % Configuración del documento.
 \renewcommand{\schoolSubject} { Examen Matemáticas 2º ESO  }
\renewcommand{\school} { IES José de Churriguera  }
\renewcommand{\academicPeriod} { Curso 2022/2023 }

\renewcommand{\autor} { Andrés Giménez Muñoz }
\renewcommand{\emailAuthor} { andresprofemates@outlook.es }
\renewcommand{\autorSing}{ Profesor: Andrés } 
 %%%%%%%%%%%%%%%%%%%%%%%%%%%
 
% \renewcommand{\thepartno}{\arabic{partno}}
%  \renewcommand{\thepartno}{\thecurrentpartno.\arabic{partno}}

% \renewcommand{\partlabel}{(\thequestion.\arabic{partno})}
% \renewcommand{\subpartlabel}{(\thepart.\arabic{subpartno})}

\renewcommand\subpartlabel{(\thesubpart)}
\renewcommand\subpartshook{\renewcommand\makelabel[1]{##1\hfil }} 
 
 %%%%%%%%%%%%%%%%%%%%%%%%%%%
 % Exam configuration
 %\pointsdroppedatright   %% No mostrar la puntuación
 \pointsinrightmargin{} % Para poner las puntuaciones a la derecha. Se puede cambiar. Si se comenta, sale a la izquierda.
 \extrawidth{-1.5cm} %Un poquito más de margen por si ponemos textos largos.
 \marginpointname{ \emph{\points}}
 
 %% Si se comenta no aparecerán los espacios de la solución.
 %\nocancelspace
 
 %% Puntuación a la izquierda.
%  \nopointsinrightmargin 

 %% Esto es de la clase exam. Si dejamos sin comentar \printanswers, se mostraran las soluciones. 
 %% Si la comentamos y dejamos sin comentar \noprintanswers, pues no se muestran las soluciones.
 % \printanswers
 %\noprintanswers
 
 %%%%%%%%%%%%%%%%%%%%%%%%%%%
 
 \begin{document}
 
 \StudentData{}
 \GradeTableHeader{}
 
 \justifying{}

 \justifying

\begin{center}
    \fbox{\fbox{\parbox{6.5in}{             
                \begin{itemize}
                    \item Deben aparecer todas las operaciones, no vale solo con indicar el resultado.
                    \item Se podrán quitar hasta cinco décimas por falta de claridad o rigor en el desarrollo de las respuestas o por una mala presentación.
                    \item Se valorará que se indiquen las cuentas en línea, realizando las operaciones en el margen.
                    \item No se puede utilizar la calculadora.
                    \item Copiar, hablar, levantarse de la silla o molestar a al resto de la clase pueden ser motivos de la retirada del examen que se valorará con un cero.
                \end{itemize}
            }}}
\end{center}
 
 \begin{questions}
    
    \question[1]  Una moto recorre 30 km en 15 minutos. ¿Cuántos kilómetros recorrerá en 2 horas? 

    \question[2] Calcula el área de un triángulo equilátero de 6 cm de lado.

    % ¿Y cuánto tardará en recorrer 50 km?
    % \vspace{\stretch{1}}

    % \question[1] Diez obreros tardan 2 meses en construir una casa. 
    % ¿Cuántos días tardarían 15 obreros?
    % \vspace{\stretch{1}}

    % \newpage

    % \question[1] Un barco hace la travesía del Atlántico en 5 días si mantiene una velocidad de 22,5 nudos. 
    % ¿Cuál debe ser su velocidad si el tiempo ha de reducirse a 4,5 días?
    % \vspace{\stretch{1}}

    % \question[1] Nueve trabajadores cargan un camión en 2 horas. ¿Cuánto tardan seis trabajadores? 
    % \vspace{\stretch{1}}

    % \question[2] Ayer 2 camiones transportaron una mercancía desde el puerto hasta el almacén. 
    % Hoy 3 camiones, iguales a los de ayer, tendrán que hacer 6 viajes cada uno para transportar la misma cantidad de mercancía del almacén al centro comercial. 
    % ¿Cuántos viajes tuvieron que hacer ayer los camiones?
    % \vspace{\stretch{1}}
   
    % \newpage

    % \question[1] Calcula los siguientes porcentajes:
    % \begin {parts}
    %     \part 3\% de 5.365
    %     \vspace{\stretch{1}}

    %     \part 21\% de 3.000
    %     \vspace{\stretch{1}}

    %     \part 125\% de 3.754
    %     \vspace{\stretch{1}}

    %     \part 23\% de 365
    %     \vspace{\stretch{1}}

    % \end{parts} 

    % \question[2] En un escaparate vemos unas zapatillas que cuestan 72\euro{} más 21\% de IVA con un descuento del 20\%. 
    % ¿Cuánto tenemos que pagar por las zapatillas? 
    % ¿Se paga más o menos que antes de aplicar el IVA y el descuento? 
    % ¿Qué opinas del resultado?
    % \vspace{\stretch{4}}

    % \question[1] El 12\% de los alumnos de secundaria practican el baloncesto y el 27\% 
    % practican el fútbol. Si en el centro hay 600 alumnos, ¿cuántos practican cada deporte?
    % \vspace{\stretch{4}}
\end{questions}
 
\end{document}