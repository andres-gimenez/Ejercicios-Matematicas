\documentclass[addpoints,spanish, 12pt,a4paper,cancelspace]{./include/gexam}

 %%%%%%%%%%%%%%%%%%%%%%%%%%%
 \renewcommand{\documentName} { 3ª evaluación }
 \renewcommand{\documentContent} { Lenguaje algebraico y ecuaciones } 
 \renewcommand{\waterMark} { Modelo A } 

 % Configuración del documento.
 \renewcommand{\schoolSubject} { Examen Matemáticas 2º ESO  }
\renewcommand{\school} { IES José de Churriguera  }
\renewcommand{\academicPeriod} { Curso 2022/2023 }

\renewcommand{\autor} { Andrés Giménez Muñoz }
\renewcommand{\emailAuthor} { andresprofemates@outlook.es }
\renewcommand{\autorSing}{ Profesor: Andrés } 
 %%%%%%%%%%%%%%%%%%%%%%%%%%%
 
% \renewcommand{\thepartno}{\arabic{partno}}
%  \renewcommand{\thepartno}{\thecurrentpartno.\arabic{partno}}

% \renewcommand{\partlabel}{(\thequestion.\arabic{partno})}
% \renewcommand{\subpartlabel}{(\thepart.\arabic{subpartno})}

\renewcommand\subpartlabel{(\thesubpart)}
\renewcommand\subpartshook{\renewcommand\makelabel[1]{##1\hfil }} 
 
 %%%%%%%%%%%%%%%%%%%%%%%%%%%
 % Exam configuration
 %\pointsdroppedatright   %% No mostrar la puntuación
 \pointsinrightmargin{} % Para poner las puntuaciones a la derecha. Se puede cambiar. Si se comenta, sale a la izquierda.
 \extrawidth{-1.5cm} %Un poquito más de margen por si ponemos textos largos.
 \marginpointname{ \emph{\points}}
 
 %% Si se comenta no aparecerán los espacios de la solución.
 %\nocancelspace
 
 %% Puntuación a la izquierda.
%  \nopointsinrightmargin 

 %% Esto es de la clase exam. Si dejamos sin comentar \printanswers, se mostraran las soluciones. 
 %% Si la comentamos y dejamos sin comentar \noprintanswers, pues no se muestran las soluciones.
 % \printanswers
 %\noprintanswers
 
 %%%%%%%%%%%%%%%%%%%%%%%%%%%
 
 \begin{document}
 
 \StudentData{}
 \GradeTableHeader{}
 
 \justifying{}

 \justifying

% \begin{center}
%     \fbox{\fbox{\parbox{6.5in}{             
%                 \begin{itemize}
%                     \item Deben aparecer todas las operaciones, no vale solo con indicar el resultado.
%                     \item Se podrán quitar hasta cinco décimas por falta de claridad o rigor en el desarrollo de las respuestas o por una mala presentación.
%                     \item Se valorará que se indiquen las cuentas en línea, realizando las operaciones en el margen.
%                     \item No se puede utilizar la calculadora.
%                 \end{itemize}
%             }}}
% \end{center}
 
 \begin{questions}
    
    \question[1] Expresa las siguientes frases en lenguaje algebraico:
    \begin {parts}
        \part El doble de un número.
        \vspace{5mm}
        \part El cuadrado de un número.
        \vspace{5mm}
        \part El cuadrado de un número menos su cuarta parte.
        \vspace{5mm}
        \part La mitad de un número menos cuatro.
        \vspace{5mm}
        \part El triple de un número menos su cuarta parte.
        \vspace{5mm}
    \end{parts} 

    \question[2] Al abrir una hucha, Manuel ha contado '$x$' monedas de 20 cént. e '$y$' monedas de 50 cént.
    \begin {parts}
        \part Expresa en lenguaje algebraico:
        \begin {subparts}   
            \subpart El dinero que tiene Manuel en monedas de 20 cént.
            \vspace{5mm}
            \subpart El dinero que tiene Manuel en monedas de 50 cént.
            \vspace{5mm}
            \subpart El dinero total que tiene Manuel.
            \vspace{5mm}
        \end {subparts}
        \part Si Manuel cuenta las monedas y tiene 3 monedas de 20 cént y 7 de 50 cént, indica:
        \begin {subparts}   
            \subpart El valor de '$x$' y de '$y$' del apartado anterior
            \vspace{5mm}
            \subpart El dinero que tiene Manuel en monedas de 20 cént.
            \vspace{5mm}
            \subpart El dinero que tiene Manuel en monedas de 50 cént.
            \vspace{5mm}
            \subpart El dinero total que tiene Manuel.
            \vspace{5mm}
        \end {subparts}
    \end {parts}

    \newpage{}

    \question[2] Realiza las siguientes sumas y restas dejando la expresión algebraica lo más simple posible:
    \begin {parts}
        \part $5x+2x$
        \vspace{\stretch{1}}
        \part $3x^5-5x^5$
        \vspace{\stretch{1}}
        \part $3x+2x-4x-5x+x+3x$
        \vspace{\stretch{1}}
        \part $\frac{1}{3}x^2+\frac{2}{5}x^2$
        \vspace{\stretch{1}}
        \part $3x^2+2x-4x^2+5x-3$
        \vspace{\stretch{1}}
    \end {parts}

    \question[2] Multiplica las siguientes expresiones algebraica: 
    \begin {parts}
        \part $5x \cdot 2x$
        \vspace{\stretch{1}}
        \part $9x \cdot \left(-4x^2\right)$
        \vspace{\stretch{1}}
        \part $5 \cdot 4x^3 $
        \vspace{\stretch{1}}
        \part $5x^2 \cdot 4y^3$
        \vspace{\stretch{1}}
        \part $x^3 \cdot \left(-5x^3y^2\right)$
        \vspace{\stretch{1}}
    \end {parts}

    \newpage{}

    \question[2] Elimina los paréntesis y siplifica la expresión algebraica:
    \begin {parts}
        \part $7x^2-\left(3x^2+2x^2\right)$
        \vspace{\stretch{1}}
        \part $\left(3x+2\right)+\left(5x+4\right)$
        \vspace{\stretch{1}}
        \part $-\left(-3x+2\right)-\left(4x+1\right)$
        \vspace{\stretch{1}}
    \end {parts}

    \question[2] Resuelve las siguientes ecuaciones
    \begin {parts}
        \part $4x+17 = 1$
        \vspace{\stretch{1}}
        \part $-3x + 15 = 18$
        \vspace{\stretch{1}}
        \part $x + 3 = 5$
        \vspace{\stretch{1}}
        \part $2+6x-2x = -8+2x$
        \vspace{\stretch{1}}
    \end {parts}

    % \newpage

    \question[2] Cálcula el valor de la expresión $-2x^2+3x-2$ para los siguientes valores:
    \begin {parts}
        \part $x=2$
        \vspace{\stretch{1}}
        \part $x=-2$
        \vspace{\stretch{1}}
        \part $x=0$
        \vspace{\stretch{1}}
        \part $x=\frac{3}{5}$
        \vspace{\stretch{1}}
    \end{parts}

    \question[2] Realiza las siguientes operaciones:
    \begin {parts}
        \part $5x + 3x$
        \vspace{\stretch{1}}

        \part $5x^2 + 3x$
        \vspace{\stretch{1}}

        \part $5x^2 + 3x^2$
        \vspace{\stretch{1}}

        \part $5x \cdot 3x$
        \vspace{\stretch{1}}

        \part $5x^2 \cdot 3x$
        \vspace{\stretch{1}}
    \end{parts}

    \newpage

    \question[2] Reduce todo lo posible:
    \begin {parts}
        \part $7x^2+\left(3x^2+2x^2\right) = $
        \vspace{\stretch{1}}
        \part $7x^2-\left(3x^2+2x^2\right) = $
        \vspace{\stretch{1}}
        \part $\left(3x^2-2x\right) - \left(5x^2+3x\right) = $
        \vspace{\stretch{1}}
    \end{parts}

    \question[2] Resuelve las siguientes ecuaciones:
    \begin {parts}
        \part $x+2 = 1$
        \vspace{\stretch{1}}
        \part $x-3 = 2$
        \vspace{\stretch{1}}
        \part $3x=-15$
        \vspace{\stretch{1}}
        \part $4x+16=0$
        \vspace{\stretch{1}}
        \part $2x= 1 + 3x$
        \vspace{\stretch{1}}
        \part $3x + 2 = x + 1$
        \vspace{\stretch{1}}


    \end{parts}

    \newpage

    \question[1] Si duplicamos un número y al resultado le restamos 10, obtenemos 18. 
    ¿Qué número buscamos?
    \vspace{\stretch{1}}

    \question[1]
    Manuel guarda 25 euros en su hucha, que supone sumar una cuarta parte del dinero que ya había. ¿Cuánto dinero hay en la hucha?
    \vspace{\stretch{1}}

\end{questions}
 
\end{document}