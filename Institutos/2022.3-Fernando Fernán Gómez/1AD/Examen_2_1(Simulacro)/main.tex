\documentclass[addpoints,spanish, 12pt,a4paper,cancelspace]{./include/gexam}

 %%%%%%%%%%%%%%%%%%%%%%%%%%%
 \renewcommand{\documentName} { (Simulacro)  2ª evaluación }
 \renewcommand{\documentContent} { Fracciones y números decimales } 
 \renewcommand{\waterMark} { Simulacro } 

 % Configuración del documento.
 \renewcommand{\schoolSubject} { Examen Matemáticas 2º ESO  }
\renewcommand{\school} { IES José de Churriguera  }
\renewcommand{\academicPeriod} { Curso 2022/2023 }

\renewcommand{\autor} { Andrés Giménez Muñoz }
\renewcommand{\emailAuthor} { andresprofemates@outlook.es }
\renewcommand{\autorSing}{ Profesor: Andrés } 
 %%%%%%%%%%%%%%%%%%%%%%%%%%%
 
 % \renewcommand{\thepartno}{\arabic{partno}}
 
 %%%%%%%%%%%%%%%%%%%%%%%%%%%
 % Exam configuration
 %\pointsdroppedatright   %% No mostrar la puntuación
 \pointsinrightmargin{} % Para poner las puntuaciones a la derecha. Se puede cambiar. Si se comenta, sale a la izquierda.
 \extrawidth{-1.5cm} %Un poquito más de margen por si ponemos textos largos.
 \marginpointname{ \emph{\points}}
 
 %% Si se comenta no aparecerán los espacios de la solución.
 %\nocancelspace
 
 %% Esto es de la clase exam. Si dejamos sin comentar \printanswers, se mostraran las soluciones. 
 %% Si la comentamos y dejamos sin comentar \noprintanswers, pues no se muestran las soluciones.
 % \printanswers
 %\noprintanswers
 
 %%%%%%%%%%%%%%%%%%%%%%%%%%%
 
 \begin{document}
 
 \StudentData{}
 \GradeTableHeader{}
 
 \justifying{}
 
 \begin{questions}
    \question[2] Realiza las siguientes operaciones con números decimales:
        \begin {parts}
            \part $45,78 \cdot 1,9$
            \vspace{\stretch{1}}
            \part $\left(-1,6\right) \cdot \left(-3,4\right)$
            \vspace{\stretch{1}}
            \part $42 : 1,5$
            \vspace{\stretch{1}}
            \part $819,32+42,93-9.145,13-48,12$
            \vspace{\stretch{1}}
        \end{parts}

    \question[2] Ordena los siguientes números de menor a mayor
        \begin {parts} 
            \part $\left\lbrace 45,22; 2,3442; -3,4; -45,22 \right\rbrace$
            \vspace{\stretch{1}}
            \part $\left\lbrace 5; 0,5; -5; 1,2 \right\rbrace$
            \vspace{\stretch{1}}
            \part $\left\lbrace -\frac{4}{18}, \frac{9}{4}, \frac{5}{3}, -\frac{3}{2} \right\rbrace$
            \vspace{\stretch{1}}
        \end{parts}

    \newpage{}
    
    \question[3] Realiza las siguientes operaciones con fracciones:
        \begin {parts} 
            \part $\frac{5}{6} + \frac{4}{5} - \frac{8}{10}$
            \vspace{\stretch{1}}
            \part $12 - \frac{7}{6}-6+\frac{8}{15}$
            \vspace{\stretch{1}}
            \part $-\left(\frac{2}{3} - \frac{1}{5}\right) - \left(\frac{7}{10} - \frac{1}{2}\right)$
            \vspace{\stretch{1}}
        \end{parts}

    \question[3] Realiza las siguientes operaciones y simplifica el resusltado lo máximo posible.
        \begin {parts} 
            \part $\frac{3}{6} \cdot \frac{6}{5}$
            \vspace{\stretch{1}}
            \part $\frac{12}{7} : \frac{14}{3}$
            \vspace{\stretch{1}}
            \part $\frac{5}{3}+\left( \frac{7}{3} \cdot \frac{4}{5} \right)$
            \vspace{\stretch{1}}
        \end{parts}
    \end{questions}
 
\end{document}