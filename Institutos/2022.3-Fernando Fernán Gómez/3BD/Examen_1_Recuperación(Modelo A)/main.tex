 %%%%%%%%%%%%%%%%%%%%%%%%%%%
 \newcommand{\documentName} { 1ª evaluación }
 \newcommand{\documentContent} { Recuperación } 
 \newcommand{\waterMark} { Modelo A } 
 %%%%%%%%%%%%%%%%%%%%%%%%%%%
 
 % Configuración del documento.
 \newcommand{\schoolSubject} { Matemáticas 3º ESO - Recuperación}
\newcommand{\school} { IES La Serna }
\newcommand{\academicPeriod} { Curso 2020/2021 }


\newcommand{\autor} { Andrés Giménez Muñoz }
\newcommand{\emailAuthor} { agimenezmunoz@ieslaserna.com }
\newcommand{\autorSing}{ Profesores: Andrés } 
 \renewcommand{\schoolSubject} { Examen Matemáticas 2º ESO  }
\renewcommand{\school} { IES José de Churriguera  }
\renewcommand{\academicPeriod} { Curso 2022/2023 }

\renewcommand{\autor} { Andrés Giménez Muñoz }
\renewcommand{\emailAuthor} { andresprofemates@outlook.es }
\renewcommand{\autorSing}{ Profesor: Andrés } 
 
 % \renewcommand{\thepartno}{\arabic{partno}}
 \usepackage{nicefrac, xfrac}
 
 %%%%%%%%%%%%%%%%%%%%%%%%%%%
 % Exam configuration
 %\pointsdroppedatright   %% No mostrar la puntuación
 \pointsinrightmargin{} % Para poner las puntuaciones a la derecha. Se puede cambiar. Si se comenta, sale a la izquierda.
 \extrawidth{-1.5cm} %Un poquito más de margen por si ponemos textos largos.
 \marginpointname{ \emph{\points}}
 
 %% Si se comenta no aparecerán los espacios de la solución.
 %\nocancelspace
 
 %% Esto es de la clase exam. Si dejamos sin comentar \printanswers, se mostraran las soluciones. 
 %% Si la comentamos y dejamos sin comentar \noprintanswers, pues no se muestran las soluciones.
 % \printanswers
 %\noprintanswers
 
 %%%%%%%%%%%%%%%%%%%%%%%%%%%
 
 \begin{document}
 
 \StudentData{}
 \GradeTableHeader{}
 
 \justifying{}
 
 \begin{questions}
    \question[1] Calcula, desarrollando todos los pasos:
    \begin {parts}
        \part $5 \cdot 4^2-3^2:3-3^2-\left(-2\right)^4=$
        \vspace{\stretch{1}}
        \part $2\left[3 \cdot 4 - \left(13- 4 \cdot 2^2 \right) \right] \cdot 2 - 20$
        \vspace{\stretch{1}}
    \end{parts}
 
    \question[1] Calcula, desarrollando todos los pasos: 
    \par
        $\frac{(\frac{3}{5})^2 \cdot (\frac{5}{3})^2 \cdot (\frac{3}{5})^{-3}}
            {\left(\frac{3}{5}\right)^{-4} \cdot \left(\frac{3}{5}\right) \cdot \left[\left(\frac{3}{5}\right)^2 \right]^3}=$
        \vspace{\stretch{1}}
    \newpage{}

    \question[1] Completa el cuadro

    \begin{table}[h]
        \centering
        \renewcommand{\arraystretch}{2}

        \begin{tabular}{|p{5cm}|p{3cm}|m{5cm}|}
            \hline
            \rowcolor[gray]{.9}
            \textbf{Representación gráfica} & \textbf{Intervalos} & \textbf{Definición matemática} \\ \hline
            & \multicolumn{1}{c|}{$[-2, 18)$}     &                       \\ [0.4cm] \hline
            \center \tikz\draw [*-o] (0,0) node[pos=2, below] {$-12$} -- +(3,0) node[pos=1, below] {$-3$}; &          &                       \\ [0.4cm] \hline 
                       &         & \multicolumn{1}{c|}{$\left\{x | x > 0 \right\}$}  \\ [0.4cm] \hline
            \center \tikz\draw [<-*] (0,0) node[pos=2, below] {$-\infty$} -- +(3,0) node[pos=1, below] {$3$};&          &                      \\ [0.4cm] \hline
        \end{tabular}
    \end{table}

    \question[1] Calcula, desarrollando todos los pasos y obteniendo una fracción irreducible:
    \par
    $\left(\frac{5}{3}-\frac{4}{4}\right):\frac{6}{12}+\left(1+\frac{1}{3}-\frac{9}{7}\right)$
    \vspace{\stretch{1}}

    \newpage{}
    \question[3] Resuelve y explica los siguientes problemas
    \begin {parts}
        \part
        Entre tres hermanos deben repartirse 12 \euro{}. 
        El primero se lleva $\sfrac{7}{16}$ del total, el segundo $\sfrac{5}{12}$ del total y el tercero el resto. 
        ¿Qué fracción del total se lleva el 3º? 
        ¿Cuánto dinero se ha llevado cada uno?
        \vspace{\stretch{1}}
        
        \part
        La población de una pequeña ciudad según antiguo censo era de 560 habitantes. 
        Se ha realizado un nuevo censo y los datos informan de un incremento del $10\%$.
        ¿Cuál es la población actual?  
        \vspace{\stretch{1}}

        \part
        Cinco amigos que van de excursión y gastan 850\euro{} en 10 días. ¿Cuánto gastarán 8 personas en 12 días? 
        \vspace{\stretch{1}}

        \part
        Tres estudiantes participan en un concurso para repartirse el premio de 240\euro{} de forma inversamente proporcional al número de errores cometidos.
        El primero alumno cometió 8 errores, el segundo 6 errores y el tercero 3 errores. 
        ¿Cuánto recibirá cada uno? 
        \vspace{\stretch{1}}

    \end {parts}
    
    \newpage{}

    \question[3]
    Siendo $P(x) = 3x^4+3x-2$, $Q(x)=-3x^3-2x^2+x-1$, $R(x)=x^2-1$, calcula:
    \begin {parts}
        \part $Q(-1)$
        \vspace{\stretch{1}}
        \part $P(x) - Q(x)$
        \vspace{\stretch{1}}    
        \part $P(x):R(x)$
        \vspace{\stretch{1}}
        \part $P(x) \cdot Q(x)$
        \vspace{\stretch{1}}
    \end{parts}
 \end{questions}
 
 \end{document}