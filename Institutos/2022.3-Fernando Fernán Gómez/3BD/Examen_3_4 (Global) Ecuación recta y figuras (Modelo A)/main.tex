\documentclass[addpoints,spanish, 12pt,a4paper,cancelspace]{./include/gexam}

%%%%%%%%%%%%%%%%%%%%%%%%%%%
\renewcommand{\documentName} { Global 3ª evaluación }
\renewcommand{\documentContent} { Ecuación de la recta y la parábola, figuras planas y volúmenes } 
\renewcommand{\waterMark} { Modelo A } 

% Configuración del documento.
\renewcommand{\schoolSubject} { Examen Matemáticas 2º ESO  }
\renewcommand{\school} { IES José de Churriguera  }
\renewcommand{\academicPeriod} { Curso 2022/2023 }

\renewcommand{\autor} { Andrés Giménez Muñoz }
\renewcommand{\emailAuthor} { andresprofemates@outlook.es }
\renewcommand{\autorSing}{ Profesor: Andrés } 
%%%%%%%%%%%%%%%%%%%%%%%%%%%

%%%%%%%%%%%%%%%%%%%%%%%%%%%
% Exam configuration
%\pointsdroppedatright   %% No mostrar la puntuación
\pointsinrightmargin % Para poner las puntuaciones a la derecha. Se puede cambiar. Si se comenta, sale a la izquierda.
\extrawidth{-1.5cm} %Un poquito más de margen por si ponemos textos largos.
\marginpointname{ \emph{\points}}

%% Si se comenta no aparecerán los espacios de la solución.
%\nocancelspace

%% Esto es de la clase exam. Si dejamos sin comentar \printanswers, se mostraran las soluciones. 
%% Si la comentamos y dejamos sin comentar \noprintanswers, pues no se muestran las soluciones.
%\printanswers
%\noprintanswers

%%%%%%%%%%%%%%%%%%%%%%%%%%%

\begin{document}

\StudentData
\GradeTableHeader

\justifying

\begin{questions}
    \setcounter{question}{0}

    % \question[1] Una piscina tiene las dimensiones que se ve en la imagen. 
    % Calcula el área que ocupa y el volumen de agua que contiene si tiene $2m$ de profundidad.
    % \\
    % \begin{minipage}{\linewidth}
    %     \centering
    %     \includegraphics[width=7cm]{img01}
    % \end{minipage}
    % \vspace{\stretch{2}}

    % \question[1] Calcula el radio de una circunferencia que tiene una longitud de $75,32m$
    % \vspace{\stretch{1}}

    % \question[1] ?A que altura se encuentra la cometa?
    % \\
    % \begin{minipage}{\linewidth}
    %     \centering
    %     \includegraphics[width=7cm]{img02}
    % \end{minipage}

    % \newpage{}
    
    

\end{questions}
\end{document}