\documentclass[addpoints,spanish, 12pt,a4paper,cancelspace]{./include/gexam}

%%%%%%%%%%%%%%%%%%%%%%%%%%%
\renewcommand{\documentName} { 3ª evaluación }
\renewcommand{\documentContent} { Recuperación } 
\renewcommand{\waterMark} { Modelo A } 

% Configuración del documento.
\renewcommand{\schoolSubject} { Examen Matemáticas 2º ESO  }
\renewcommand{\school} { IES José de Churriguera  }
\renewcommand{\academicPeriod} { Curso 2022/2023 }

\renewcommand{\autor} { Andrés Giménez Muñoz }
\renewcommand{\emailAuthor} { andresprofemates@outlook.es }
\renewcommand{\autorSing}{ Profesor: Andrés } 
%%%%%%%%%%%%%%%%%%%%%%%%%%%

%%%%%%%%%%%%%%%%%%%%%%%%%%%
% Exam configuration
%\pointsdroppedatright   %% No mostrar la puntuación
\pointsinrightmargin % Para poner las puntuaciones a la derecha. Se puede cambiar. Si se comenta, sale a la izquierda.
\extrawidth{-1.5cm} %Un poquito más de margen por si ponemos textos largos.
\marginpointname{ \emph{\points}}

%% Si se comenta no aparecerán los espacios de la solución.
%\nocancelspace

%% Esto es de la clase exam. Si dejamos sin comentar \printanswers, se mostraran las soluciones. 
%% Si la comentamos y dejamos sin comentar \noprintanswers, pues no se muestran las soluciones.
%\printanswers
%\noprintanswers

%%%%%%%%%%%%%%%%%%%%%%%%%%%
\begin{document}

\StudentData
\GradeTableHeader

\justifying

\begin{center}
    \fbox{\fbox{\parbox{6.5in}{  
            Para la valoración de la calificación obtenida en el presente trabajo se tendrán en cuenta los siguientes criterios.
                \begin{itemize}
                    \item Colaboración entre los miembros del equipo (20\%)
                    \item Argumentación y explicación en la resolución del ejercicio. 
                        La resolución del ejercicio ha de ser fácil de entender y se ha de utilizar el lenguaje matemático de forma correcta (30\%)
                    \item Exposición de los conceptos matemáticos que se utilizan. 
                        Indicar el nombre de las figuras geométricas, fórmula de su área o longitud, utilización de simetrías, simplicidad de los cálculos. (30\%)
                    \item El resultado obtenido es correcto (20\%)
                \end{itemize}
            }}}
\end{center}

\begin{questions}
    \setcounter{question}{0}

    \question[0] Pendiente
    

\end{questions}
\end{document}