\documentclass[addpoints,spanish, 12pt,a4paper,cancelspace]{./include/gexam}

%%%%%%%%%%%%%%%%%%%%%%%%%%%
\renewcommand{\documentName} { 1º examen 3ª evaluación }
\renewcommand{\documentContent} { Figuras planas y volúmenes } 
\renewcommand{\waterMark} { Modelo A } 

% Configuración del documento.
\renewcommand{\schoolSubject} { Examen Matemáticas 2º ESO  }
\renewcommand{\school} { IES José de Churriguera  }
\renewcommand{\academicPeriod} { Curso 2022/2023 }

\renewcommand{\autor} { Andrés Giménez Muñoz }
\renewcommand{\emailAuthor} { andresprofemates@outlook.es }
\renewcommand{\autorSing}{ Profesor: Andrés } 
%%%%%%%%%%%%%%%%%%%%%%%%%%%

%%%%%%%%%%%%%%%%%%%%%%%%%%%
% Exam configuration
%\pointsdroppedatright   %% No mostrar la puntuación
\pointsinrightmargin % Para poner las puntuaciones a la derecha. Se puede cambiar. Si se comenta, sale a la izquierda.
\extrawidth{-1.5cm} %Un poquito más de margen por si ponemos textos largos.
\marginpointname{ \emph{\points}}

%% Si se comenta no aparecerán los espacios de la solución.
%\nocancelspace

%% Esto es de la clase exam. Si dejamos sin comentar \printanswers, se mostraran las soluciones. 
%% Si la comentamos y dejamos sin comentar \noprintanswers, pues no se muestran las soluciones.
%\printanswers
%\noprintanswers

%%%%%%%%%%%%%%%%%%%%%%%%%%%



\begin{document}

\StudentData
\GradeTableHeader

\justifying

\begin{questions}
    \setcounter{question}{0}

    \question[2]Calcula 

    % \question[2] Obtén la ecuación de la recta que pasa por los puntos A(4,5) y B(2,6) 
    % expresándola en su forma general y explicita.
    % \vspace{\stretch{1}}

    % \question[2] Indica cuál de los siguientes puntos $A \left(3, -4\right)$, $B\left(2, 7\right)$, $C\left(-3, 6\right)$, $D\left(\frac{1}{5}, \frac{2}{3}\right)$ 
    % pertenecen a la recta $5x + 3y - 3 = 0$, 
    % indicando los cálculos que has realizado para obtener la respuesta.
    % \vspace{\stretch{1}}

    % \newpage

    % \question[2] Dada la recta $2x-3y+3 = 0$ indica cuál es su pendiente y el punto de corte con los ejes.
    % \vspace{\stretch{1}}

    % \question[2] Dada la parábola $y=x^2+x-6$ calcula su vértice, punto de corte con los ejes de coordenadas 
    % e indica si el vértice es un máximo o un mínimo.
    % \vspace{\stretch{1}}

    % \newpage
    % \question[2] Manuel trabaja como repartidor los fines de semana y recibe una cantidad fija de 40\euro{} mensuales, 
    % más 4\euro{} por cada paquete que reparte.
    % \begin{parts}
    %     \part Escribe la función que relaciona el número de paquetes repartidos con el dinero recibido al mes
    %     y representa la gráfica.

    %     \vspace{\stretch{1}}
    %     \begin{figure}[h]
    %         \begin{center}
    %             \begin{tikzpicture}[xscale=1, yscale=1, domain=-5:200]
    %                 \tkzInit[xmax=60,xmin=0,xstep=5, ymax=400,ymin=-5,ystep=40]
    %                 \tkzGrid[color=black!20]
    %                 \tkzAxeXY
    %             \end{tikzpicture}
    %         \end{center}
    %     \end{figure}
    %         \part ¿Cuántos paquetes tiene que repartir Manuel para cobrar en un mes 152\euro{}?
    %         \\
    %         (*) Resuelve de forma algebraica.
    %         \vspace{\stretch{2}}
    %     \end{parts}
    

\end{questions}
\end{document}