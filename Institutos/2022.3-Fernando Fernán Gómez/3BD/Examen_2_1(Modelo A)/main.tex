 %%%%%%%%%%%%%%%%%%%%%%%%%%%
 \newcommand{\documentName} { 2ª evaluación }
 \newcommand{\documentContent} { Ecuaciones } 
 \newcommand{\waterMark} { Modelo A } 
 %%%%%%%%%%%%%%%%%%%%%%%%%%%
 
 % Configuración del documento.
 \newcommand{\schoolSubject} { Matemáticas 3º ESO - Recuperación}
\newcommand{\school} { IES La Serna }
\newcommand{\academicPeriod} { Curso 2020/2021 }


\newcommand{\autor} { Andrés Giménez Muñoz }
\newcommand{\emailAuthor} { agimenezmunoz@ieslaserna.com }
\newcommand{\autorSing}{ Profesores: Andrés } 
 \renewcommand{\schoolSubject} { Examen Matemáticas 2º ESO  }
\renewcommand{\school} { IES José de Churriguera  }
\renewcommand{\academicPeriod} { Curso 2022/2023 }

\renewcommand{\autor} { Andrés Giménez Muñoz }
\renewcommand{\emailAuthor} { andresprofemates@outlook.es }
\renewcommand{\autorSing}{ Profesor: Andrés } 
 
 % \renewcommand{\thepartno}{\arabic{partno}}
 \usepackage{nicefrac, xfrac}
 
 %%%%%%%%%%%%%%%%%%%%%%%%%%%
 % Exam configuration
 %\pointsdroppedatright   %% No mostrar la puntuación
 \pointsinrightmargin{} % Para poner las puntuaciones a la derecha. Se puede cambiar. Si se comenta, sale a la izquierda.
 \extrawidth{-1.5cm} %Un poquito más de margen por si ponemos textos largos.
 \marginpointname{ \emph{\points}}
 
 %% Si se comenta no aparecerán los espacios de la solución.
 %\nocancelspace
 
 %% Esto es de la clase exam. Si dejamos sin comentar \printanswers, se mostraran las soluciones. 
 %% Si la comentamos y dejamos sin comentar \noprintanswers, pues no se muestran las soluciones.
 % \printanswers
 %\noprintanswers
 
 %%%%%%%%%%%%%%%%%%%%%%%%%%%
 
 \begin{document}
 
 \StudentData{}
 \GradeTableHeader{}
 
 \justifying{}
 
 \begin{questions}
    \question[1]
    Comprueba, sin resolver, si la solución de cada ecuación es correcta.
    \begin{parts}
        \part
        $2x^2-3x-2 = 0$, $\boxed{x=-\frac{1}{2}}$ y $\boxed{x=2}$
        \vspace{\stretch{1}}

        \part
        $x^2+\frac{5}{2}x+1 = 0$, $\boxed{x=-\frac{1}{2}}$ y $\boxed{x=-2}$
        \vspace{\stretch{1}}
    \end{parts}

    \newpage
    \question[2]
    Resuelve las siguientes ecuaciones.
    \begin{parts}
        \part
        $(5x-2x)(x+4)=6x-2x^2$
        \vspace{\stretch{1}}

        \part
        $\left(x-\frac{1}{2}\right)-3\left(x+1\right)=2\left(x+\frac{1}{4}\right)$
        \vspace{\stretch{1}}
        \part
        $x-\frac{6(1-x)+3(4-2x)}{5}=3\left(x+\frac{1}{6}\right)-2$
        \vspace{\stretch{1}}
    \end{parts} 

    \question[2]
    Resuelve las siguientes ecuaciones de 2º grado.
    \begin{parts}
        \part
        % Solución x=-2;x=7
        $x^2 = 5 x + 14$
        \vspace{\stretch{1}}

        \part
        % Solución x=7;x=1
        $(x-4)^2-9 =0$
        \vspace{\stretch{1}}
    \end{parts}

    \newpage
        \question[3]
		Resuelve las siguientes ecuaciones utilizando la regla de Ruffini.
        \begin{parts}
            \part
            % Solucion x = -1, 1, 2 
            $x^3-2 x^2-x+2=0$
            \vspace{\stretch{1}}

            \part
            % Solución: x = 1, 2, 2
            $ x^3 - 5 x^2 + 8 x - 4=0$
            \vspace{\stretch{1}}

            % \part
            % % Solución: x = 2, 2, 3 
            % $ x^3 - 7 x^2 + 16 x - 12=0$
            % % \vspace{\stretch{1}}

            \part
            $x^4-29x^2+100=0$
            \vspace{\stretch{1}}

            \part
            $(x+3)(x-2)(x-52)(x^2+7x+30)=0$
            \vspace{\stretch{2}}
	    \end{parts}

 \end{questions}
 
 \end{document}