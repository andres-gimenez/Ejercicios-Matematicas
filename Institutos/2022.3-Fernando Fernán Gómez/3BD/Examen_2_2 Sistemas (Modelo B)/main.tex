\documentclass[addpoints,spanish, 12pt,a4paper,cancelspace]{./include/gexam}

%%%%%%%%%%%%%%%%%%%%%%%%%%%
\renewcommand{\documentName} { Examen 2ª evaluación }
\renewcommand{\documentContent} { Sistemas de ecuaciones } 
\renewcommand{\waterMark} { Modelo B } 

% Configuración del documento.
\renewcommand{\schoolSubject} { Examen Matemáticas 2º ESO  }
\renewcommand{\school} { IES José de Churriguera  }
\renewcommand{\academicPeriod} { Curso 2022/2023 }

\renewcommand{\autor} { Andrés Giménez Muñoz }
\renewcommand{\emailAuthor} { andresprofemates@outlook.es }
\renewcommand{\autorSing}{ Profesor: Andrés } 
%%%%%%%%%%%%%%%%%%%%%%%%%%%

%%%%%%%%%%%%%%%%%%%%%%%%%%%
% Exam configuration
%\pointsdroppedatright   %% No mostrar la puntuación
\pointsinrightmargin % Para poner las puntuaciones a la derecha. Se puede cambiar. Si se comenta, sale a la izquierda.
\extrawidth{-1.5cm} %Un poquito más de margen por si ponemos textos largos.
\marginpointname{ \emph{\points}}

%% Si se comenta no aparecerán los espacios de la solución.
%\nocancelspace

%% Esto es de la clase exam. Si dejamos sin comentar \printanswers, se mostraran las soluciones. 
%% Si la comentamos y dejamos sin comentar \noprintanswers, pues no se muestran las soluciones.
%\printanswers
%\noprintanswers

%%%%%%%%%%%%%%%%%%%%%%%%%%%



\begin{document}

\StudentData
\GradeTableHeader

\justifying

\begin{questions}
    \setcounter{question}{0}

    \question[2]
    Resuelve el siguiente sistema de ecuaciones por el método gráfico.
    \begin{flushleft}
        $\begin{cases}
                \nonumber
                -2x + y = 0 \\
                \nonumber
                3x + 2y  = 14
            \end{cases}$
    \end{flushleft}

    \begin{figure}[h]
        \begin{tikzpicture}[scale=1]
            \tkzInit[xmax=7,ymax=7,xmin=-7,ymin=-7]
            \tkzGrid[color=black!50]
            \tkzAxeXY
        \end{tikzpicture}
    \end{figure}

    \newpage
    \question[4]
    Resuelve los siguientes sistemas de ecuaciones:
    \begin{parts}
        \part
        Por el método de sustitución:
        \begin{flushleft}
            $\begin{cases}
                    \nonumber
                    2x - 3y  = -4 \\
                    \nonumber
                    3x +2y = 7
                \end{cases}$
        \end{flushleft}
        \vspace{\stretch{1}}

        \part
        Por el método de igualación.
        \begin{flushleft}
            $\begin{cases}
                    \nonumber
                    2x + y = 10 \\
                    \nonumber
                    -3x - 2y = -16
                \end{cases}$
        \end{flushleft}
        \vspace{\stretch{1}}

        \part
        Por el método de reducción.
        \begin{flushleft}
            $\begin{cases}
                    \nonumber
                    x + 2y = 6 \\
                    \nonumber
                    x + 3y = 7
                \end{cases}$
        \end{flushleft}
        \vspace{\stretch{1}}
    \end{parts}

    \newpage

    %\question[2\half]
    \question[2]
    En un garaje hay 110 vehículos entre coches y motos y sus ruedas suman 360.
    ¿Cuántas motos y coches hay?
    \vspace{\stretch{1}}

    % \question[2\half]
    \question[2]
    Dos ciclistas avanzan uno hacia el otro por una misma carretera. 
    Sus velocidades son de 20km/h y de 15 km/h. Si les separan 78 km. 
    ¿Cuánto tardarán en encontrarse?
    \vspace{\stretch{1}}

\end{questions}
\end{document}