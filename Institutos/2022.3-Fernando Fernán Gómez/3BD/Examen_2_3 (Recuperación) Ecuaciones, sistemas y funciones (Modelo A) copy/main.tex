%%%%%%%%%%%%%%%%%%%%%%%%%%%
\newcommand{\documentName} { Examen recuperación 2ª evaluación }
\newcommand{\documentContent} { Ecuaciones, sistemas de ecuaciones y funciones } 
\newcommand{\waterMark} { Modelo A} 
%%%%%%%%%%%%%%%%%%%%%%%%%%%

% Configuración del documento.
\newcommand{\schoolSubject} { Matemáticas 3º ESO - Recuperación}
\newcommand{\school} { IES La Serna }
\newcommand{\academicPeriod} { Curso 2020/2021 }


\newcommand{\autor} { Andrés Giménez Muñoz }
\newcommand{\emailAuthor} { agimenezmunoz@ieslaserna.com }
\newcommand{\autorSing}{ Profesores: Andrés } 
\renewcommand{\schoolSubject} { Examen Matemáticas 2º ESO  }
\renewcommand{\school} { IES José de Churriguera  }
\renewcommand{\academicPeriod} { Curso 2022/2023 }

\renewcommand{\autor} { Andrés Giménez Muñoz }
\renewcommand{\emailAuthor} { andresprofemates@outlook.es }
\renewcommand{\autorSing}{ Profesor: Andrés } 

%%%%%%%%%%%%%%%%%%%%%%%%%%%
% Exam configuration
%\pointsdroppedatright   %% No mostrar la puntuación
\pointsinrightmargin % Para poner las puntuaciones a la derecha. Se puede cambiar. Si se comenta, sale a la izquierda.
\extrawidth{-1.5cm} %Un poquito más de margen por si ponemos textos largos.
\marginpointname{ \emph{\points}}

%% Si se comenta no aparecerán los espacios de la solución.
%\nocancelspace

%% Esto es de la clase exam. Si dejamos sin comentar \printanswers, se mostraran las soluciones. 
%% Si la comentamos y dejamos sin comentar \noprintanswers, pues no se muestran las soluciones.
%\printanswers
%\noprintanswers

%%%%%%%%%%%%%%%%%%%%%%%%%%%
\begin{document}

\StudentData
\GradeTableHeader

\justifying

\begin{questions}
    \setcounter{question}{0}

    \question[2]
    % x = - 3; Y = -4
    Resuelve el siguiente sistema de ecuaciones por el método gráfico.
    \begin{flushleft}
        $\begin{cases}
            \nonumber
            x - 2y = 2 \\
            \nonumber
            x - y = 0
        \end{cases}$
    \end{flushleft}

    \begin{figure}[h]
        \begin{tikzpicture}[scale=1]
            \tkzInit[xmax=7,ymax=7,xmin=-7, ymin=-7]
            \tkzGrid[color=black!50]
            \tkzAxeXY
        \end{tikzpicture}
    \end{figure}

    \newpage
    \question[2]
    Resuelve los siguientes sistemas de ecuaciones:
    \begin{parts}
        \part
        % Por el método de sustitución:
        \begin{flushleft}
            $\begin{cases}
                    \nonumber
                    3x - 5y  = 4 \\
                    \nonumber
                    2x + y = 7
                \end{cases}$
        \end{flushleft}
        \vspace{\stretch{1}}

        % \part
        % Por el método de igualación.
        % \begin{flushleft}
        %     $\begin{cases}
        %             \nonumber
        %             -3x - 4y = 5 \\
        %             \nonumber
        %             -2x + 3y = 9
        %         \end{cases}$
        % \end{flushleft}
        % \vspace{\stretch{1}}

        \part
        % Por el método de reducción.
        \begin{flushleft}
            $\begin{cases}
                    \nonumber
                    2x + y = 9 \\
                    \nonumber
                    x - 3y = 1
                \end{cases}$
        \end{flushleft}
        \vspace{\stretch{1}}
    \end{parts}

    \newpage
    \question[2]
    Resuelve las siguientes ecuaciones:
    \begin{parts}
        \part
        % $5 \left(x-3\right) - 2 \left(x-1\right) = 3x - 13$
        $x - 11 = 3x - 5 \left(x - 2\right)$
        \vspace{\stretch{1}}

        \part
        $\frac{x-2}{3}-\frac{3x-5}{3} = -1$
        \vspace{\stretch{1}}

        \part
        $8x-6=2\left[x+4\left(x-1\right)\right]$
        \vspace{\stretch{1}}
    \end{parts}

    \newpage

    \question[2]
    Resuelve las siguientes ecuaciones de segundo grado
    \begin{parts}
        \part 
        $-2x^2 + 15x - 7 = 0$
        \vspace{\stretch{1}}

        \part 
        $25 - x^2 = 0$
        \vspace{\stretch{1}}

        \part 
        $2x^2 - 3x + 1 = 0$
        \vspace{\stretch{1}}

    \end{parts}


    \newpage
    \question[2]
    Dada la función de la gráfica, identifica el dominio, la imagen o recorrido, continuidad y puntos de discontinuidad, intervalos de crecimiento y decrecimiento, máximos y mínimos relativos y absolutos.

    \begin{tikzpicture}[scale=1]
        \begin{axis}[
                %title={$f(x)=x^2-10$},
                axis lines=middle,
                axis line style={<->},
                legend style={inner ysep=7pt},
                %x label style={at={(axis description cs:0.5,-0.06)},anchor=north},
                %y label style={at={(axis description cs:-0.06,.5)},rotate=90,anchor=south},
                %xlabel={Nº de bombillas},ylabel={Vida en horas},
                no markers,
                black!50,
                grid,
                width=15cm,
                xmax=20,ymax=20,xmin=-20,ymin=-20,
                %axis lines=middle
                % restrict y to domain=-20:20
            ]
            \addplot[domain=-15:-5,black] {-x-10};
            \addplot[domain=-5:0,black] {x*x+6*x};
            \addplot[domain=0:6,black] {x};
            \addplot[domain=6:15,black] {-5};
            \draw[dashed] (axis cs:6,6) -- (axis cs:6,-5);
            \addplot[holdotBlack] coordinates{(-15,5)(6,-5)};
            \addplot[soldotBlack] coordinates{(6,6)(15,-5)};
        \end{axis}
    \end{tikzpicture}
    \\
    \begin{table}[h]
        \begin{tabular}{|l|p{10cm}|}   
            \hline
            Dominio                     &  \\ \hline
            Imagen o recorrido          &  \\ \hline
            Continuidad                 &  \\ \hline
            Puntos de discontinuidad    &  \\ \hline
            Intervalos de crecimiento   &  \\ \hline
            Intervalos de decrecimiento &  \\ \hline
            Máximos relativos           &  \\ \hline
            Mínimos relativos           &  \\ \hline
            Máximo absoluto             &  \\ \hline
            Máximo relativo             &  \\ \hline
        \end{tabular}
    \end{table}
\end{questions}
\end{document}