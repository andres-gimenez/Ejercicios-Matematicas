%%%%%%%%%%%%%%%%%%%%%%%%%%%
\newcommand{\documentName} { Trabajo 2ª evaluación }
\newcommand{\documentContent} { Funciones } 
\newcommand{\waterMark} { } 
%%%%%%%%%%%%%%%%%%%%%%%%%%%

% Configuración del documento.
\newcommand{\schoolSubject} { Matemáticas 3º ESO - Recuperación}
\newcommand{\school} { IES La Serna }
\newcommand{\academicPeriod} { Curso 2020/2021 }


\newcommand{\autor} { Andrés Giménez Muñoz }
\newcommand{\emailAuthor} { agimenezmunoz@ieslaserna.com }
\newcommand{\autorSing}{ Profesores: Andrés } 
\renewcommand{\schoolSubject} { Examen Matemáticas 2º ESO  }
\renewcommand{\school} { IES José de Churriguera  }
\renewcommand{\academicPeriod} { Curso 2022/2023 }

\renewcommand{\autor} { Andrés Giménez Muñoz }
\renewcommand{\emailAuthor} { andresprofemates@outlook.es }
\renewcommand{\autorSing}{ Profesor: Andrés } 

\usepackage{hyperref}

%%%%%%%%%%%%%%%%%%%%%%%%%%%
% Exam configuration
%\pointsdroppedatright   %% No mostrar la puntuación
\pointsinrightmargin % Para poner las puntuaciones a la derecha. Se puede cambiar. Si se comenta, sale a la izquierda.
\extrawidth{-1.5cm} %Un poquito más de margen por si ponemos textos largos.
\marginpointname{ \emph{\points}}

%% Si se comenta no aparecerán los espacios de la solución.
%\nocancelspace

%% Esto es de la clase exam. Si dejamos sin comentar \printanswers, se mostraran las soluciones. 
%% Si la comentamos y dejamos sin comentar \noprintanswers, pues no se muestran las soluciones.
%\printanswers
%\noprintanswers

%%%%%%%%%%%%%%%%%%%%%%%%%%%



\begin{document}

% \StudentData
% \GradeTableHeader

\justifying

El presente trabajo está destinado al conocimiento de las principales funciones elementales. 
Para realizarlo, los alumnos se deberán agrupar en grupos de 2 o 3 miembros y mediante la herramienta
\url{https://www.geogebra.org/graphing}.
Se deberán representar cada grupo de funciones en uno o varios ejes de coordenadas cartesiano, 
comparando e indicando las propiedades que se piden en cada apartado.

El trabajo se debe presentar en soporte electrónico, utilizando un editor de texto y entregándolo al profesor en formato pdf.

\begin{questions}
    \setcounter{question}{0}

    \question Funciones lineales (modificación de la pendiente):
    Indicar de cada función: dominio, recorrido, paridad (demostrando en el caso que sea par o impar) y pendiente de las rectas.
    \begin{parts}
        \part
            $f(x)= x$ 
        \part
            $f(x)= 2x$ 
        \part
            $f(x)= \frac{x}{2}$
    \end{parts}

    \question Funciones lineales (modificación del punto de corte con los ejes):
    Indicar de cada función: dominio, recorrido, puntos de corte con los ejes.
    \begin{parts}
        \part
            $f(x)= x$ 
        \part
            $f(x)= x + 1$ 
        \part
            $f(x)= x - 1$
    \end{parts}

    \question Funciones cuadráticas:
    Indicar de cada función: dominio, recorrido, paridad (demostrando en el caso que sea par o impar), 
    puntos de corte con los ejes, extremos relativos y absolutos.
    \begin{parts}
        \part
            $f(x)= x^2$ 
        \part
            $f(x)= (x+1)^2$ 
        \part
            $f(x)= x^2+1$
    \end{parts}

    \question Hipérbola:
    Indicar de cada función: dominio, recorrido y puntos de corte con los ejes.
    \begin{parts}
        \part 
            $f(x)= \frac{1}{x}$  \\
            ¿Qué pasa cuando $x=0$?
        \part
            $f(x)= \frac{1}{x-2}$ \\
            ¿Qué pasa cuando $x=2$?
    \end{parts}

    \question Funciones trigonométricas:
    Indicar de cada función: dominio, recorrido, puntos de corte con los ejes, periodicidad ($2\pi{}$) y extremos relativos y absolutos.
    \begin{parts}
        \part
            $f(x)= sen(x)$ 
        \part
            $f(x)= sen(x+\pi{})$ 
        \part
            $f(x)= 2 \cdot sen(x)$
        \part
            $f(x)= 0,5 \cdot sen(x)$
    \end{parts}
    Observa que las funciones (a) y (b) es la misma desplazada $\pi$ unidades en el eje de abscisas. 
    \\
    Observa que las funciones (a), (b) y (d) pueden representar ondas y la diferencia es la amplitud de estas.

\end{questions}
\end{document}