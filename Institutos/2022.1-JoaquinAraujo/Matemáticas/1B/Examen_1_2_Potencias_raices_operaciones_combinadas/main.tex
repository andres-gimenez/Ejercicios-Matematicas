 %%%%%%%%%%%%%%%%%%%%%%%%%%%
 \newcommand{\documentName} { 1ª evaluación }
 \newcommand{\documentContent} {Expresiones combinadas y divisibilidad} 
 \newcommand{\waterMark} {  } 
 %%%%%%%%%%%%%%%%%%%%%%%%%%%
 
 % Configuración del documento.
 \newcommand{\schoolSubject} { Matemáticas 3º ESO - Recuperación}
\newcommand{\school} { IES La Serna }
\newcommand{\academicPeriod} { Curso 2020/2021 }


\newcommand{\autor} { Andrés Giménez Muñoz }
\newcommand{\emailAuthor} { agimenezmunoz@ieslaserna.com }
\newcommand{\autorSing}{ Profesores: Andrés } 
 \renewcommand{\schoolSubject} { Examen Matemáticas 2º ESO  }
\renewcommand{\school} { IES José de Churriguera  }
\renewcommand{\academicPeriod} { Curso 2022/2023 }

\renewcommand{\autor} { Andrés Giménez Muñoz }
\renewcommand{\emailAuthor} { andresprofemates@outlook.es }
\renewcommand{\autorSing}{ Profesor: Andrés } 
 
 % \renewcommand{\thepartno}{\arabic{partno}}
 \usepackage{xfrac}
 
 %%%%%%%%%%%%%%%%%%%%%%%%%%%
 % Exam configuration
 %\pointsdroppedatright   %% No mostrar la puntuación
 \pointsinrightmargin{} % Para poner las puntuaciones a la derecha. Se puede cambiar. Si se comenta, sale a la izquierda.
 \extrawidth{-1.5cm} %Un poquito más de margen por si ponemos textos largos.
 \marginpointname{ \emph{\points}}
 
 %% Si se comenta no aparecerán los espacios de la solución.
 %\nocancelspace
 
 %% Esto es de la clase exam. Si dejamos sin comentar \printanswers, se mostraran las soluciones. 
 %% Si la comentamos y dejamos sin comentar \noprintanswers, pues no se muestran las soluciones.
 % \printanswers
 %\noprintanswers
 
 %%%%%%%%%%%%%%%%%%%%%%%%%%%
 
 \begin{document}
 
 \StudentData{}
 \GradeTableHeader{}
 
 \justifying{}
 
 \begin{questions}
     \question[2]Escribe como una sola potencia (sin resolver la potencia)
     \begin{parts}
         \part $\left(2^3\right)^5$
         \vspace{\stretch{1}}
         \part $25^{31} \cdot 25^{43}$
         \vspace{\stretch{1}}
         \part $7^6 : 7^2$
         \vspace{\stretch{1}}
         \part $\left(12^4\right)^2 \cdot \left(12^2\right)^4$
         \vspace{\stretch{1}}
     \end{parts}
 
     \question[2]Calcula el valor de las siguientes raíces y su resto
     \begin{parts}
           \part $\sqrt{25}$
           \vspace{\stretch{1}}
           \part $\sqrt{85}$
           \vspace{\stretch{1}}
           \part $\sqrt{121}$
           \vspace{\stretch{1}}
           \part $\sqrt{172}$
           \vspace{\stretch{1}}
     \end{parts}

     \question[2]
     ¿Cuál es el número de monedas que hay en el lado de un cuadrado que se puede formar con 131 monedas?
     ¿Cuántas monedas sobran?
     \vspace{\stretch{5}}
    
     \newpage{}

     \question[2] Resuelve estas operaciones     
        \begin{parts}
            \part $25 : 5 + 5 \cdot 7$
            \vspace{\stretch{2}}
            \part $12 \cdot 4 - 12 : 3$
            \vspace{\stretch{2}}
            \part $3 \cdot \left(14 + 12 - 20 \right) + 7 \cdot 5 - 3 $
            \vspace{\stretch{3}}
            \part $3 \cdot \left(100 -90 \right) + 12 \cdot \left(2 - 4 : 3 \right) $
            \vspace{\stretch{3}}
        \end{parts}

    \newpage{}

    \question[2] Obtén el resultado
    \begin{parts}
       \part $4 \cdot 9 - 2^3 \cdot 3$
       \vspace{\stretch{1}}
       \part $3 \cdot 2 + 3^2 \cdot 5$
       \vspace{\stretch{1}}
       \part $\left(\sqrt{25} + \sqrt{9}\right)\cdot 4 + 5$
       \vspace{\stretch{2}}
       \part $\sqrt{25} + 3^2 \cdot 2 - 2^4 : 4$
       \vspace{\stretch{2}}
   \end{parts}
 \end{questions}
 
 \end{document}