 %%%%%%%%%%%%%%%%%%%%%%%%%%%
\newcommand{\documentName} { Ecuaciones }
\newcommand{\documentContent} { Problemas con sistema de ecuaciones } 
\newcommand{\waterMark} {  } 
%%%%%%%%%%%%%%%%%%%%%%%%%%%

% Configuración del documento.
\newcommand{\schoolSubject} { Matemáticas 3º ESO - Recuperación}
\newcommand{\school} { IES La Serna }
\newcommand{\academicPeriod} { Curso 2020/2021 }


\newcommand{\autor} { Andrés Giménez Muñoz }
\newcommand{\emailAuthor} { agimenezmunoz@ieslaserna.com }
\newcommand{\autorSing}{ Profesores: Andrés } 
\renewcommand{\schoolSubject} { Examen Matemáticas 2º ESO  }
\renewcommand{\school} { IES José de Churriguera  }
\renewcommand{\academicPeriod} { Curso 2022/2023 }

\renewcommand{\autor} { Andrés Giménez Muñoz }
\renewcommand{\emailAuthor} { andresprofemates@outlook.es }
\renewcommand{\autorSing}{ Profesor: Andrés } 

% \renewcommand{\thepartno}{\arabic{partno}}

\usepackage{yhmath}

\usepackage{amsthm}
\theoremstyle{definition}
\newtheorem*{theorem}{Theorem}
\newtheorem{definition}{Definición}
\newtheorem{property}{Propiedad}

%%%%%%%%%%%%%%%%%%%%%%%%%%%
% Exam configuration
%\pointsdroppedatright   %% No mostrar la puntuación
\pointsinrightmargin{} % Para poner las puntuaciones a la derecha. Se puede cambiar. Si se comenta, sale a la izquierda.
\extrawidth{-1.5cm} %Un poquito más de margen por si ponemos textos largos.
\marginpointname{ \emph{\points}}

%% Si se comenta no aparecerán los espacios de la solución.
%\nocancelspace

%% Esto es de la clase exam. Si dejamos sin comentar \printanswers, se mostraran las soluciones. 
%% Si la comentamos y dejamos sin comentar \noprintanswers, pues no se muestran las soluciones.
% \printanswers
%\noprintanswers

%%%%%%%%%%%%%%%%%%%%%%%%%%%

\begin{document}

% \StudentData{}
% \GradeTableHeader{}

\justifying{}

\begin{questions}
    \question
    Dos ciclistas avanzan uno hacia el otro por una misma carretera. Sus velocidades son de 20km/h y de 15 km/h. Si les separan 78 km. ¿Cuánto tardarán en encontrarse?

    \question
    En un garaje hay 110 vehículos entre coches y motos y sus ruedas suman 360. ¿Cuántas motos y coches hay?

    \question
    En una granja hay doble número de gatos que de perros y triple número de gallinas que de perros y gatos juntos. ¿Cuántos gatos, perros y gallinas hay si en total son 96 animales?

    \question
    Una granja tiene cerdos y pavos, en total hay 35 cabezas y 116 patas. ¿Cuántos cerdos y pavos hay?

    \question
    Las tres cuartas partes de la edad del padre de Juan excede en 15 años a la edad de este. Hace cuatro años la edad del padre era el doble que la edad del hijo. Hallar las edades de ambos.

    \question
    Un padre tiene 34 años y su hijo 12. ¿Al cabo de cuántos años la edad del padre será el doble que la del hijo?

    \question
    Se distribuyen 400 bolsas en tres urnas sabiendo que la primera tiene 80 menos que la segunda y esta tiene 60 menos que la tercera, averigua cuántas bolsas tiene cada una.

    \question
    Reparten 390\euro{} entre dos personas de tal modo que la parte de la primera sea igual al doble de la parte de la segunda menos 60.

    \question
    Roberto tiene 18\euro{} en monedas de 20 céntimos y 50 céntimos. 
    ¿Cuántas monedas tiene si hay el doble de monedas de 20 céntimos que de 50 céntimos?

    \question
    Antonio se ha comprado dos pantalones y tres camisas en las rebajas. Los pantalones tenían un $30\%$ de descuento y las camisas un $20\%$.
    El precio original de un pantalón era el doble que el de una camisa, pero con el descuento solo ha pagado 104\euro{}
    ¿Cuánto costaba cada artículo antes de las rebajas?
\end{questions}

\end{document}